\documentclass{article}
\usepackage{hyperlatex}

\setcounter{htmldepth}{1}

\newcommand{\xsubsection}[1]{%
\texonly{\subsection{#1}}%
\htmlonly{\strong{#1}\\}%
}

\newcommand{\evalsto}{%
\texonly{$\rightarrow$}%
\htmlonly{\code{->}}%
}

\newcommand{\cvar}[1]{%
\texonly{{\em{#1}}}%
\htmlonly{\code{\var{#1}}}%
}

%%%%%%%%%%%%%%%% Latex prototypes
\texonly{
\newenvironment{protos}{\list{$\bullet$}
{\leftmargin1.2em\rightmargin0pt\itemsep0pt\parsep0pt\partopsep-2pt}}
{\endlist}

% The following is for prototypes that have return types.
%    (foo int int) -> int

\newcommand{\proto}[3]{\item\noindent\unskip%
\cindex{\code{#1}}%
\hbox{\spaceskip=0.5em\code{({#1}{\it#2\/})} {$\rightarrow$} {\it#3}}}

% TeX (or LaTeX, who can say?) is such a piece of crap.  This has to be put at
% the end of each block of prototypes to avoid a spurious warning.
%\newcommand{\postproto}{\hspace*{\fill}}

\newcommand{\protonoresult}[2]{\item\noindent\unskip%
\hbox{\spaceskip=0.5em\code{(\hbox{#1}{\it#2\/})}}}

% Syntax prototypes

\newcommand{\syntaxprotonoresult}[2]{\item\noindent\unskip%
\hbox{\spaceskip=0.5em\code{(\hbox{#1}{#2})}}\hfill\penalty 0%
\hbox{ }\nobreak\hfill\hbox{\rm syntax}}

\newcommand{\syntaxproto}[3]{\syntaxprotonoresult{#1}{#2}%
\hspace*{24pt}{$\rightarrow$} {\it#3}}

% This can be reduced

\newcommand{\pconstproto}[2]{\noindent\unskip%
\hbox{\spaceskip=0.5em#1}\code\hfill\penalty 0%
\hbox{ }\nobreak\hfill\hbox{\rm #2}}

% Variable prototype
\newcommand{\constproto}[2]{\pconstproto{#1}{#2}}

}
%%%%%%%%%%%%%%%% end of Latex proto definitions

%%%%%%%%%%%%%%%% HTML prototypes
\htmlonly{
\newenvironment{protos}{\begin{itemize}}{\end{itemize}}

% The following is for prototypes that have return types.
%    (foo int int) -> int

\newcommand{\proto}[3]{%
\cindex{\code{#1}}%
\item\noindent\code{({#1}{\var{#2}\/})~-->~{\var{#3}}}}

\newcommand{\protonoresult}[2]{%
\cindex{\code{#1}}%
\item\noindent\code{({#1}{\var{#2}\/})}}

}
%%%%%%%%%%%%%%%% end of HTML proto definitions

\makeindex

\htmltitle{A Quick Guide to Kali}
\htmladdress{kelsey@research.nj.nec.com}

\title{A Quick Guide to Kali}
\author{Richard A. Kelsey \\ NEC Research Institute \\ kelsey@research.nj.nec.com}

\date{}

\begin{document}
\maketitle

\section{Address spaces and servers}

Address-spaces are an abstraction of Unix processes.
An address space is identified by the machine on which it runs and the
 socket on which it listens
 for connections from other address spaces.
New address spaces can be added as a Kali program runs.

All of the procedures described in this section are in structure \code{kali}.

\begin{protos}
\protonoresult{start-server}{}
\end{protos}
This starts a server for the current process, making it an address space.
The socket that the server is listening on is printed out.
Normally the server is started as a separate thread, using \code{spawn}
 from structure \code{threads}.

\begin{example}
> ,open kali threads
> (spawn start-server 'kali-server)
Waiting for connection on port 1228
>
\end{example}

\begin{protos}
\proto{socket-id->address-space}{ machine-name socket}{address-space}
\proto{address-space?}{ thing}{boolean}
\protonoresult{remote-run!}{ address-space procedure arg$_0$ \ldots}
\proto{remote-apply}{ address-space procedure arg$_0$ \ldots}{values}
\end{protos}
\code{Socket-id->address-space} returns the address-space corresponding to
 the process whose server is listening at \cvar{socket} on \cvar{machine-name}.
\cvar{Socket} should be the socket number printed out by the call to
 \code{start-server} that created the address space.

\code{Address-space?} is the predicate for address spaces.

\code{Remote-run!} and \code{remote-apply} transport \cvar{procedure}
 and \cvar{arguments} to \cvar{address-space} and do the application there.
\code{Remote-run!} returns immediately, while \code{remote-apply} blocks until
 \cvar{procedure} returns, and then returns whatever values \cvar{procedure}
 returned.
\cvar{Procedure}, \cvar{arguments}, and \cvar{values}
 are all transmitted by copying, with the exception of proxies and symbols.
Objects are shared within a particular message, including the message that
 send \cvar{procedure} and \cvar{arguments} and the message returning
 \cvar{values}.
Objects are not shared between messages.
\begin{example}
  (let ((x (list 1 2)))
    (remote-apply a1 eq? x x))
    \evalsto \#t

  (let ((x (list 1 2)))
    (eq? x (remote-apply a1 (lambda () x))))
    \evalsto \#f

  (let ((x (list 1)))
    (remote-apply a1 (lambda () (set-car! x 2)))
    (car x))
    \evalsto 1
\end{example}
\noindent If \cvar{address-space} is the local address space, no messages are
 sent and no copying occurs.
For a \code{remote-run!} where \cvar{address-space} is the local address space,
 a separate thread is spawned to do the application of \cvar{procedure} to
 \cvar{arguments}.

There is currently no mechanism for GCing address spaces.
Kali makes socket connections between address spaces only as needed, but once
 made they stay forever.

\section{Proxies}

Proxies are globally-unique, distributed cells.  Every proxy potentially has a
distinct value in every address space.

These procedures are in structure \code{kali}.

\begin{protos}
\proto{make-proxy}{ value}{proxy}
\proto{proxy?}{ thing}{boolean}
\proto{proxy-owner}{ proxy}{address-space}
\proto{proxy-local-ref}{ proxy}{value}
\protonoresult{proxy-local-set!}{ proxy value}
\proto{proxy-remote-ref}{ proxy}{value}
\protonoresult{proxy-remote-set!}{ proxy value}
\proto{any-proxy-value}{ proxy}{value}
\end{protos}
\code{Make-proxy} makes a new proxy, whose value in the current address space
 is \cvar{value}.
Initially the new proxy has no value on other address spaces.
\code{Proxy-owner} returns the address space on which the proxy
 was created.

\code{Proxy-local-ref} and \code{proxy-local-set!} access and set the value of
 the proxy in the current address space.
\code{Proxy-remote-ref} and \code{proxy-remote-set!} do the same for the
 value on the address space on which the proxy was created.
They could be defined as follows:
\begin{example}
  (define proxy-remote-ref
    (lambda (proxy)
      (remote-apply (proxy-owner proxy)
                    proxy-local-ref
                    proxy)))

  (define proxy-remote-set!
    (lambda (proxy value)
      (remote-run! (proxy-owner proxy)
                   proxy-local-set!
                   proxy
                   value)))
\end{example}
\code{Any-proxy-value} returns either the local value of \cvar{proxy}, if there
 is one, or the value on the proxy's owner.

Note that the remote values may be transmitted between nodes and thus may be
 a copy of the original value.
Each proxy is itself a unique global object and is never copied.
\begin{example}
  (let ((x (make-proxy \#f)))
    (eq? x (remote-apply a1 (lambda () x))))
   \evalsto \#t
\end{example}

Typically, a proxy only has a value on the owning address space.
Local values, via \code{Proxy-local-ref} and \code{proxy-local-set!},
 are only used when a per-address-space cell is needed.
An example might be a per-address-space queue of tasks.

A proxy is required whenever a \code{remote-run!} or \code{remote-apply} may
 refer to an object that should not be copied.
This includes lexically bound variables that are \code{set!}.
\begin{example}
  (let* ((call-count 0)
         (f (lambda ()
              (set! call-count (+ 1 call-count)))))

    (remote-apply a1 (lambda () (f) (f) (f)))

    call-count)
   \evalsto 0 \textit{if} \code{a1} \textit{is not the local address space,} \code{3} \textit{if it is.}

  (let* ((call-count (make-proxy 0))
         (f (lambda ()
              (proxy-remote-set!
                call-count
                (+ 1 (proxy-remote-ref call-count))))))

    (remote-apply a1 (lambda () (f) (f) (f)))

    (proxy-remote-ref call-count))
   \evalsto 3
\end{example}

Many system-supplied data structures, including locks, tables, queues,
 placeholders and so forth should be put in proxies if they are used remotely.

The current proxy GC algorithm does not collect proxies that are given values
 on remote nodes.
All other proxies are eventually GC'ed when no longer referenced.

\section{Debugging Kali programs}

Kali programs run in a distributed, multithreaded environment, making debugging
 a non-trivial task.
As described in doc/threads.txt, when any thread raises an
 error, Scheme 48 stops running all of the threads at that command level.
Kali does not extend this between address spaces, so other address spaces will
 keep running as if nothing had happened.
Messages to the stopped address space are buffered until the user restarts
 the stopped command level.

Another difficulty in debugging Kali programs is that redefinitions are not
 propagated between address spaces.
A redefinition is handled as a \code{set!} to the local cell for the variable.
Other address space have their own copies of the cell, which are not updated
automatically.  The following example shows this effect.
\begin{example}
> (define (f) 10)
> (define (g) (f))
> (g)
10
> (remote-apply a1 g)
10
> (define (f) 20)
> (g)
20
> (remote-apply a1 g)
10
\end{example}
The remote application of \code{g} gets the original value of \code{f},
 not the new one.
The remote \code{f} can be updated by hand.
\begin{example}
> (remote-run! a1 (lambda (x) (set! f x)) f)
> (remote-apply a1 g)
20
\end{example}
Note that the argument to \code{remote-run!} is evaluated in the local address
 space, and so gets the new value of \code{f}.
Doing
\begin{example}
(remote-run! a1 (lambda () (set! f f)))
\end{example}
would have had no effect.
Both occurrences of \code{f} would refer to the binding on the remote
 address space.
When in doubt it is best to restart the program from scratch.

The following procedure is useful in debugging multi-threaded programs.
\begin{protos}
\protonoresult{debug-message}{ element$_0$ \ldots}
\end{protos}
\code{Debug-message} prints the elements to `\code{stderr}', followed by a
 newline.
The only types of values that \code{debug-message} prints in full are small
 integers (fixnums), strings, characters, symbols, boolean, and the empty list.
Values of other types are abbreviated as follows.

\begin{tabular}{ll}
 pair       &   \code{(...)}\\
 vector     &   \code{\#(...)}\\
 procedure  &   \code{\#\{procedure\}}\\
 record     &   \code{\#\{<name of record type>\}}\\
 all others &   \code{???}\\
\end{tabular}

The great thing about \code{debug-message} is that it bypasses Scheme~48's
 I/O and thread handling.  The message appears immediately, with no delays
 or errors.

\code{Debug-message} is exported by the structure \code{debug-messages}.

\section{Code sharing between address spaces}

In Kali, Scheme code in one address space is treated as distinct from code
in all other address spaces.  If the same file is loaded into two different
address spaces, each will have its own copy.  If these two address spaces
both run that code in a third address space, that third space will get two
copies of the code.  To avoid duplication, it is a good idea to load a
file into only one address space.

The same lack of sharing occurs if, after a file is loaded, an image is
dumped and then used to start two address spaces.  The two spaces are each
considered to have loaded the file.  To circumvent this,
each Kali image contains a table of loaded code that can be shared between
address spaces, assuming that all spaces are using the same table.  Address
spaces started with different tables are assumed to have nothing in common,
and all code needed for remote evaluation must be copied to the remote
 address space.
The following procedure, from the structure \code{address-spaces}, can be used
 to rebuild the shared-code table after loading additional code.

\begin{protos}
\protonoresult{initialize-shared-address-space!}{}
\end{protos}
This creates a table containing all existing code and and proxies.
This table is shared between all address spaces that are started from the
 current image.
Code loaded before the call to \code{initialize-shared-address-space!} will
 not be copied between address spaces.

\end{document}
