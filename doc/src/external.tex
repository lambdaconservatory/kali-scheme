\chapter{Mixing Scheme 48 and C}
\label{external-chapter}

This chapter describes an interface for calling C functions
 from Scheme, calling Scheme functions from C, and allocating
 storage in the Scheme heap..
Scheme~48 manages stub functions in C that
 negotiate between the calling conventions of Scheme and C and the
 memory allocation policies of both worlds.
No stub generator is available yet, but writing stubs is a straightforward task.

\section{Available facilities}
\label{sec:facilities}

The following facilities are available for interfacing between
 Scheme~48 and C:
%
\begin{itemize}
\item Scheme code can call C functions.
\item The external interface provides full introspection for all
  Scheme objects.  External code may inspect, modify, and allocate
  Scheme objects arbitrarily.
\item External code may raise exceptions back to Scheme~48 to
  signal errors.
\item External code may call back into Scheme.  Scheme~48
  correctly unrolls the process stack on non-local exits.
\item External modules may register bindings of names to values with a 
  central registry accessible from
  Scheme.  Conversely, Scheme code can register shared
  bindings for access by C code.
\end{itemize}
%
%This section has three parts: the first describes how bindings are
% moved from Scheme to C and vice versa, the second tells how to call
% C functions from Scheme, and the third covers the C interface
% to Scheme objects, including calling Scheme procedures, using the
% Scheme heap, and so forth.

\subsection{Scheme structures}

The structure \code{external-calls} has 
 most of the Scheme functions described here.
The others are in 
 \code{dynamic-externals}, which has the functions for dynamic loading and
 name lookup from
Section~\ref{dynamic-externals},
 and \code{shared-bindings}, which has the additional shared-binding functions
 described in
section~\ref{more-shared-bindings}.

\subsection{C naming conventions}

The names of all of Scheme~48's visible C bindings begin 
 with `\code{s48\_}' (for procedures and variables) or 
 `\code{S48\_}' (for macros).
Whenever a C name is derived from a Scheme identifier, we
 replace `\code{-}' with `\code{\_}' and convert letters to lowercase
 for procedures and uppercase for macros.
A final `\code{?}'  converted to `\code{\_p}' (`\code{\_P}' in C macro names).
A final `\code{!}' is dropped.
Thus the C macro for Scheme's \code{pair?} is \code{S48\_PAIR\_P} and
 the one for \code{set-car!} is \code{S48\_SET\_CAR}.
Procedures and macros that do not check the types of their arguments
 have `\code{unsafe}' in their names.

All of the C functions and macros described have prototypes or definitions
 in the file \code{c/scheme48.h}.
The C type for Scheme values is defined there to be \code{s48\_value}.

\subsection{Garbage collection}

Scheme~48 uses a copying garbage collector.
The collector must be able to locate all references
 to objects allocated in the Scheme~48 heap in order to ensure that
 storage is not reclaimed prematurely and to update references to objects
 moved by the collector.
The garbage collector may run whenever an object is allocated in the heap.
C variables whose values are Scheme~48 objects and which are live across
 heap allocation calls need to be registered with
 the
garbage collector.  See section~\ref{gc} for more information.

\section{Shared bindings}
\label{sec:shared-bindings}

Shared bindings are the means by which named values are shared between Scheme
 code and C code.
There are two separate tables of shared bindings, one for values defined in
 Scheme and accessed from C and the other for values going the other way.
Shared bindings actually bind names to cells, to allow a name to be looked
 up before it has been assigned.
This is necessary because C initialization code may be run before or after
 the corresponding Scheme code, depending on whether the Scheme code is in
 the resumed image or is run in the current session.

\subsection{Exporting Scheme values to C}

\begin{protos}
\proto{define-exported-binding}{ name value}{shared-binding}
\end{protos}

\begin{protos}
\cproto{s48\_value s48\_get\_imported\_binding(char *name)}
\cproto{s48\_value S48\_SHARED\_BINDING\_REF(s48\_value shared\_binding)}
\end{protos}

\noindent\code{Define-exported-binding} makes \cvar{value} available to C code
 under as \cvar{name} which must be a \cvar{string}, creating a new shared
 binding if necessary.
The C function \code{s48\_get\_imported\_binding} returns the shared binding
 defined for \code{name}, again creating it if necessary.
The C macro \code{S48\_SHARED\_BINDING\_REF} dereferences a shared binding,
 returning its current value.

\subsection{Exporting C values to Scheme}

\begin{protos}
\cproto{void s48\_define\_exported\_binding(char *name, s48\_value v)}
\end{protos}

\begin{protos}
\proto{lookup-imported-binding}{ string}{shared-binding}
\proto{shared-binding-ref}{ shared-binding}{value}
\end{protos}

\noindent These are used to define shared bindings from C and to access them
 from Scheme.
Again, if a name is looked up before it has been defined, a new binding is
 created for it.

The common case of exporting a C function to Scheme can be done using
 the macro \code{S48\_EXPORT\_FUNCTION(\cvar{name})}.
This expands into
\begin{example}
s48\_define\_exported\_binding("\cvar{name}",
                               s48\_enter\_pointer(\cvar{name}))
\end{example}

\noindent which boxes the function into a Scheme byte vector and then
 exports it.
Note that \code{s48\_enter\_pointer} allocates space in the Scheme heap
 and might trigger a 
garbage collection; see Section~\ref{gc}.

\begin{protos}
\syntaxprotonoresult{import-definition}{ \cvar{name}}
\syntaxprotonoresultnoindex{import-definition}{ \cvar{name c-name}}
\end{protos}
These macros simplify importing definitions from C to Scheme.
They expand into

\code{(define \cvar{name} (lookup-imported-binding \cvar{c-name}))}

\noindent{}where \cvar{c-name} is as supplied for the second form.
For the first form \cvar{c-name} is derived from \cvar{name} by
 replacing `\code{-}' with `\code{\_}' and converting letters to lowercase.
For example, \code{(import-definition my-foo)} expands into

\code{(define my-foo (lookup-imported-binding "my\_foo"))}

\subsection{Complete shared binding interface}
\label{more-shared-bindings}

There are a number of other Scheme functions related to shared bindings;
 these are in the structure \code{shared-bindings}.

\begin{protos}
\proto{shared-binding?}{ x}{boolean}
\proto{shared-binding-name}{ shared-binding}{string}
\proto{shared-binding-is-import?}{ shared-binding}{boolean}
\protonoresultnoindex{shared-binding-set!}{ shared-binding value}\mainschindex{shared-binding-set"!}
\protonoresult{define-imported-binding}{ string value}
\protonoresult{lookup-exported-binding}{ string}
\protonoresult{undefine-imported-binding}{ string}{}
\protonoresult{undefine-exported-binding}{ string}{}
\end{protos}

\noindent\code{Shared-binding?} is the predicate for shared-bindings.
\code{Shared-binding-name} returns the name of a binding.
\code{Shared-binding-is-import?} is true if the binding was defined from C.
\code{Shared-binding-set!} changes the value of a binding.
\code{Define-imported-binding} and \code{lookup-exported-binding} are
 Scheme versions of \code{s48\_define\_exported\_binding}
 and \code{s48\_lookup\_imported\_binding}.
The two \code{undefine-} procedures remove bindings from the two tables.
They do nothing if the name is not found in the table.

The following C macros correspond to the Scheme functions above.

\begin{protos}
\cproto{int\ \ \ \ \ \ \ S48\_SHARED\_BINDING\_P(x)}
\cproto{int\ \ \ \ \ \ \ S48\_SHARED\_BINDING\_IS\_IMPORT\_P(s48\_value s\_b)}
\cproto{s48\_value S48\_SHARED\_BINDING\_NAME(s48\_value s\_b)}
\cproto{void\ \ \ \ \ \ S48\_SHARED\_BINDING\_SET(s48\_value s\_b, s48\_value v)}
\end{protos}

\section{Calling C functions from Scheme}
\label{sec:external-call}

There are three different ways to call C functions from Scheme, depending on
 how the C function was obtained.

\begin{protos}
\proto{call-imported-binding}{ binding arg$_0$ \ldots}{value}
\end{protos}
\noindent
Each of these applies its first argument, a C function, to the rest of
 the arguments.
For \code{call-imported-binding} the function argument must be an
 imported binding.

For all of these, the C function is passed the \cvar{arg$_i$} values
 and the value returned is that returned by C procedure.
No automatic representation conversion occurs for either arguments or
 return values.
Up to twelve arguments may be passed.
There is no method supplied for returning multiple values to
 Scheme from C (or vice versa) (mainly because C does not have multiple return
 values).

Keyboard interrupts that occur during a call to a C function are ignored
 until the function returns to Scheme (this is clearly a
 problem; we are working on a solution).

\begin{protos}
\syntaxprotonoresult{import-lambda-definition}
{ \cvar{name} (\cvar{formal} \ldots)}
\syntaxprotonoresultnoindex{import-lambda-definition}
{ \cvar{name} (\cvar{formal} \ldots)\ \cvar{c-name}}
\end{protos}
\noindent{}These macros simplify importing functions from C.
They define \cvar{name} to be a function with the given formals that
 applies those formals to the corresponding C binding.
\cvar{C-name}, if supplied, should be a string.
These expand into

\begin{example}
(define temp (lookup-imported-binding \cvar{c-name}))
(define \cvar{name}
  (lambda (\cvar{formal} \ldots)
    (call-imported-binding temp \cvar{formal} \ldots)))
\end{example}

\noindent{}
If \cvar{c-name} is not supplied, it is derived from \cvar{name} by converting
 all letters to lowercase and replacing `\code{-}' with `\code{\_}'.

\section{Dynamic loading}
\label{dynamic-externals}

External code can be loaded into a running Scheme~48---at least on
most variants of Unix and on Windows.  The required Scheme functions
are in the structure \code{load-dynamic-externals}.

To be suitable for dynamic loading, the externals code must reside in
a shared object.  The shared object must define a function:
%
\begin{protos}
  \cproto{void s48\_on\_load(void)}
\end{protos}
%
The \code{s48\_on\_load} is run upon loading the shared objects.  It
typically contains invocations of \code{S48\_EXPORT\_FUNCTION} to
make the functionality defined by the shared object known to
Scheme~48. 

The shared object may also define either or both of the following
functions:
%
\begin{protos}
   \cproto{void s48\_on\_unload(void)}
   \cproto{void s48\_on\_reload(void)}
\end{protos}
Scheme~48 calls \code{s48\_on\_unload} just before it unloads the
shared object.  If \code{s48\_on\_reload} is present, Scheme~48 calls
it when it loads the shared object for the second time, or some new
version thereof.  If it is not present, Scheme~48 calls
\code{s48\_on\_load} instead.  (More on that later.)

For Linux, the following commands compile \code{foo.c} into a file
\code{foo.so} that can be loaded dynamically.
\begin{example}
\% gcc -c -o foo.o foo.c
\% ld -shared -o foo.so foo.o
\end{example}
%
The following procedures provide the basic functionality for loading
shared objects containing dynamic externals:
%
\begin{protos}
\proto{load-dynamic-externals}{ string plete?
  rrepeat? rresume?}{dynamic-externals}
\proto{unload-dynamic-externals}{ string}{ dynamic-externals}
\protonoresult{reload-dynamic-externals}{ dynamic-externals}
\end{protos}
%
\code{Load-dynamic-externals} loads the named shared objects.  The
\cvar{plete?} argument determines whether Scheme~48 appends the
OS-specific suffix (typically \code{.so} for Unix, and \code{.dll} for
Windows) to the name.  The \cvar{rrepeat?}  argument determines how
\code{load-dynamic-externals} behaves if it is called again with the
same argument: If this is true, it reloads the shared object (and
calls its \code{s48\_on\_unload} on unloading if present, and, after
reloading, \code{s48\_on\_reload} if present or \code{s48\_on\_load}
if not), otherwise, it will not do anything.  The \cvar{rresume?}
argument determines if an image subsequently dumped will try to load
the shared object again automatically.  (The shared objects will be
loaded before any record resumers run.)  \code{Load-dynamic-externals}
returns a handle identifying the shared object just loaded.

\code{Unload-dynamic-externals} unloads the shared object associated
with the handle passed as its argument, previously calling its
\code{s48\_on\_unload} function if present.  Note that this invalidates
all external bindings associated with the shared object; referring to
any of them will probably crash the program.

\code{Reload-dynamic-externals} will reload the shared object named by
its argument and call its \code{s48\_on\_unload} function before
unloading, and, after reloading, \code{s48\_on\_reload} if present or
\code{s48\_on\_load} if not.

\begin{protos}
\proto{import-dynamic-externals}{ string}{dynamic-externals}  
\end{protos}
%
This procedure represents the expected most usage for loading
dynamic-externals.  It is best explained by its definition:
%
\begin{example}
(define (import-dynamic-externals name)
  (load-dynamic-externals name #t #f #t))
\end{example}

\section{Accessing Scheme data from C}
\label{sec:scheme-data}

The C header file \code{scheme48.h} provides
 access to Scheme~48 data structures.
The type \code{s48\_value} is used for Scheme values.
When the type of a value is known, such as the integer returned
 by \code{vector-length} or the boolean returned by \code{pair?},
 the corresponding C procedure returns a C value of the appropriate
 type, and not a \code{s48\_value}.
Predicates return \code{1} for true and \code{0} for false.

\subsection{Constants}
\label{sec:constants}

The following macros denote Scheme constants:
%
\begin{itemize}
\item \code{S48\_FALSE} is \verb|#f|.
\item \code{S48\_TRUE} is \verb|#t|.
\item \code{S48\_NULL} is the empty list.
\item \code{S48\_UNSPECIFIC} is a value used for functions which have no
  meaningful return value
 (in Scheme~48 this value returned by the nullary procedure \code{unspecific}
 in the structure \code{util}).
\item \code{S48\_EOF} is the end-of-file object
 (in Scheme~48 this value is returned by the nullary procedure \code{eof-object}
 in the structure \code{i/o-internal}).
\end{itemize}

\subsection{Converting values}

The following macros and functions convert values between Scheme and C
 representations.
The `extract' ones convert from Scheme to C and the `enter's go the other
 way.

\begin{protos}
\cproto{int \ \ \ \ \ \ S48\_EXTRACT\_BOOLEAN(s48\_value)}
\cproto{long\ \ \ \ \ \ s48\_extract\_char(s48\_value)}
\cproto{char * \ \ \ s48\_extract\_string(s48\_value)}
\cproto{char * \ \ \ s48\_extract\_byte\_vector(s48\_value)}
\cproto{long \ \ \ \ \ s48\_extract\_integer(s48\_value)}
\cproto{double \ \ \ s48\_extract\_double(s48\_value)}
\cproto{s48\_value S48\_ENTER\_BOOLEAN(int)}
\cproto{s48\_value s48\_enter\_char(long)}
\cgcproto{s48\_value s48\_enter\_byte\_vector(char *, long)}
\cgcproto{s48\_value s48\_enter\_integer(long)}
\cgcproto{s48\_value s48\_enter\_double(double)}
\end{protos}

\noindent{}\code{S48\_EXTRACT\_BOOLEAN} is false if its argument is
 \code{\#f} and true otherwise.
 \code{S48\_ENTER\_BOOLEAN} is \code{\#f} if its argument is zero
  and \code{\#t} otherwise.

The \code{s48\_extract\_char} function extracts the scalar value from
a Scheme character as a C \code{long}.  Conversely,
\code{s48\_enter\_char} creates a Scheme character from a scalar
value.  (Note that ASCII values are also scalar values.)

 The \code{s48\_extract\_byte\_vector} function returns a
 pointer to the actual
 storage used by the byte vector.
 These pointers are valid only until the next GC; see Section~\ref{gc}.

The second argument to \code{s48\_enter\_byte\_vector} is the length of
 byte vector.

\code{s48\_enter\_integer()} needs to allocate storage when
 its argument is too large to fit in a Scheme~48 fixnum.
In cases where the number is known to fit within a fixnum (currently 30 bits
 including the sign), the following procedures can be used.
These have the disadvantage of only having a limited range, but
 the advantage of never causing a garbage collection.
\code{S48\_FIXNUM\_P} is a macro that true if its argument is a fixnum
 and false otherwise.

\begin{protos}
\cproto{int \ \ \ \ \ \ S48\_TRUE\_P(s48\_value)}
\cproto{int \ \ \ \ \ \ S48\_FALSE\_P(s48\_value)}
\end{protos}

\noindent \code{S48\_TRUE\_P} is true if its argument is \code{S48\_TRUE}
 and \code{S48\_FALSE\_P} is true if its argument is \code{S48\_FALSE}.

\begin{protos}
\cproto{int \ \ \ \ \ \ S48\_FIXNUM\_P(s48\_value)}
\cproto{long \ \ \ \ \ s48\_extract\_fixnum(s48\_value)}
\cproto{s48\_value s48\_enter\_fixnum(long)}
\cproto{long \ \ \ \ \ S48\_MAX\_FIXNUM\_VALUE}
\cproto{long \ \ \ \ \ S48\_MIN\_FIXNUM\_VALUE}
\end{protos}

\noindent An error is signalled if \code{s48\_extract\_fixnum}'s argument
 is not a fixnum or if the argument to \code{s48\_enter\_fixnum} is less than
 \code{S48\_MIN\_FIXNUM\_VALUE} or greater than \code{S48\_MAX\_FIXNUM\_VALUE}
 ($-2^{29}$ and $2^{29}-1$ in the current system).

\begin{protos}
\cgcproto{s48\_value s48\_enter\_string\_latin\_1(char*);}
\cgcproto{s48\_value s48\_enter\_string\_latin\_1\_n(char*, long);}
\cproto{void\ \ \ \ \ \ s48\_copy\_latin\_1\_to\_string(char*, s48\_value);}
\cproto{void\ \ \ \ \ \ s48\_copy\_latin\_1\_to\_string\_n(char*, long, s48\_value);}
\cproto{void\ \ \ \ \ \ s48\_copy\_string\_to\_latin\_1(s48\_value, char*);}
\cproto{void\ \ \ \ \ \ s48\_copy\_string\_to\_latin\_1\_n(s48\_value, long, long, char*);}
\cgcproto{s48\_value s48\_enter\_string\_utf\_8(char*);}
\cgcproto{s48\_value s48\_enter\_string\_utf\_8\_n(char*, long);}
\cproto{long\ \ \ \ \ \ s48\_string\_utf\_8\_length(s48\_value);}
\cproto{long\ \ \ \ \ \ s48\_string\_utf\_8\_length\_n(s48\_value, long, long);}
\cproto{void\ \ \ \ \ \ s48\_copy\_string\_to\_utf\_8(s48\_value, char*);}
\cproto{void\ \ \ \ \ \ s48\_copy\_string\_to\_utf\_8\_n(s48\_value, long, long, char*);}
\end{protos}
%
The \code{s48\_enter\_string\_latin\_1} function creates a Scheme
string, initializing its contents from its NUL-terminated,
Latin-1-encoded argument.  The \code{s48\_enter\_string\_latin\_1\_n}
function does the same, but allows specifying the length explicitly---no NUL
terminator is necessary.

The \code{s48\_copy\_latin\_1\_to\_string} function copies
Latin-1-encoded characters from its first NUL-terminated argument to
the Scheme string that is its second argument.  The
\code{s48\_copy\_latin\_1\_to\_string\_n} does the same, but allows
specifying the number of characters explicitly.  The
\code{s48\_copy\_string\_to\_latin\_1} function converts the
characters of the Scheme string specified as the first argument into
Latin-1 and writes them into the string specified as the second
argument.  (Note that it does not NUL-terminate the result.)  The
\code{s48\_copy\_string\_to\_latin\_1\_n} function does the same, but
allows specifying a starting index and a character count into the
source string.

The \code{s48\_enter\_string\_utf\_8} function creates a Scheme
string, initializing its contents from its NUL-terminated,
UTF-8-encoded argument.  The \code{s48\_enter\_string\_utf\_8\_n}
function does the same, but allows specifying the length
explicitly---no NUL terminator is necessary.

The \code{s48\_string\_utf\_8\_length} function computes the length
that the UTF-8 encoding of its argument (a Scheme string) would
occupy, not including NUL termination.  The
\code{s48\_string\_utf\_8\_length} function does the same, but allows
specifying a starting index and a count into the input string.

The \code{s48\_copy\_string\_to\_utf\_8} function converts the
characters of the Scheme string specified as the first argument into
UTF-8 and writes them into the string specified as the second
argument.  (Note that it does not NUL-terminate the result.)  The
\code{s48\_copy\_string\_to\_utf\_8\_n} function does the same, but
allows specifying a starting index and a character count into the
source string.

\subsection{C versions of Scheme procedures}

The following macros and procedures are C versions of Scheme procedures.
The names were derived by replacing `\code{-}' with `\code{\_}',
 `\code{?}' with `\code{\_P}', and dropping `\code{!}.

\begin{protos}
\cproto{int \ \ \ \ \ \ S48\_EQ\_P(s48\_value, s48\_VALUE)}
\cproto{int \ \ \ \ \ \ S48\_CHAR\_P(s48\_value)}
\end{protos}
\begin{protos}
\cproto{int \ \ \ \ \ \ S48\_PAIR\_P(s48\_value)}
\cproto{s48\_value S48\_CAR(s48\_value)}
\cproto{s48\_value S48\_CDR(s48\_value)}
\cproto{void \ \ \ \ \ S48\_SET\_CAR(s48\_value, s48\_value)}
\cproto{void \ \ \ \ \ S48\_SET\_CDR(s48\_value, s48\_value)}
\cgcproto{s48\_value s48\_cons(s48\_value, s48\_value)}
\cproto{long \ \ \ \ \ s48\_length(s48\_value)} 
\end{protos}
\begin{protos}
\cproto{int \ \ \ \ \ \ S48\_VECTOR\_P(s48\_value)} 
\cproto{long \ \ \ \ \ S48\_VECTOR\_LENGTH(s48\_value)} 
\cproto{s48\_value S48\_VECTOR\_REF(s48\_value, long)} 
\cproto{void \ \ \ \ \ S48\_VECTOR\_SET(s48\_value, long, s48\_value)} 
\cgcproto{s48\_value s48\_make\_vector(long, s48\_value)}
\end{protos}
\begin{protos}
\cproto{int \ \ \ \ \ \ S48\_STRING\_P(s48\_value)} 
\cproto{long \ \ \ \ \ S48\_STRING\_LENGTH(s48\_value)} 
\cproto{long \ \ \ \ \ S48\_STRING\_REF(s48\_value, long)} 
\cproto{void \ \ \ \ \ S48\_STRING\_SET(s48\_value, long, long)} 
\cgcproto{s48\_value s48\_make\_string(long, char)}
\end{protos}
\begin{protos}
\cproto{int \ \ \ \ \ \ S48\_SYMBOL\_P(s48\_value)} 
\cproto{s48\_value s48\_SYMBOL\_TO\_STRING(s48\_value)} 
\end{protos}
\begin{protos}
\cproto{int \ \ \ \ \ \ S48\_BYTE\_VECTOR\_P(s48\_value)} 
\cproto{long \ \ \ \ \ S48\_BYTE\_VECTOR\_LENGTH(s48\_value)} 
\cproto{char \ \ \ \ \ S48\_BYTE\_VECTOR\_REF(s48\_value, long)} 
\cproto{void \ \ \ \ \ S48\_BYTE\_VECTOR\_SET(s48\_value, long, int)} 
\cgcproto{s48\_value s48\_make\_byte\_vector(long, int)}
\end{protos}

\section{Calling Scheme functions from C}
\label{sec:external-callback}

External code that has been called from Scheme can call back to Scheme
 procedures using the following function.

\begin{protos}
\cproto{s48\_value s48\_call\_scheme(s48\_value p, long nargs, \ldots)}
\end{protos}
\noindent{}This calls the Scheme procedure \code{p} on \code{nargs}
 arguments, which are passed as additional arguments to \code{s48\_call\_scheme}.
There may be at most twelve arguments.
The value returned by the Scheme procedure is returned by the C procedure.
Invoking any Scheme procedure may potentially cause a garbage collection.

There are some complications that occur when mixing calls from C to Scheme
 with continuations and threads.
C only supports downward continuations (via \code{longjmp()}).
Scheme continuations that capture a portion of the C stack have to follow the
 same restriction.
For example, suppose Scheme procedure \code{s0} captures continuation \code{a}
 and then calls C procedure \code{c0}, which in turn calls Scheme procedure
 \code{s1}.
Procedure \code{s1} can safely call the continuation \code{a}, because that
 is a downward use.
When \code{a} is called Scheme~48 will remove the portion of the C stack used
 by the call to \code{c0}.
On the other hand, if \code{s1} captures a continuation, that continuation
 cannot be used from \code{s0}, because by the time control returns to
 \code{s0} the C stack used by \code{c0} will no longer be valid.
An attempt to invoke an upward continuation that is closed over a portion
 of the C stack will raise an exception.

In Scheme~48 threads are implemented using continuations, so the downward
 restriction applies to them as well.
An attempt to return from Scheme to C at a time when the appropriate
 C frame is not on top of the C stack will cause the current thread to
 block until the frame is available.
For example, suppose thread \code{t0} calls a C procedure which calls back
 to Scheme, at which point control switches to thread \code{t1}, which also
 calls C and then back to Scheme.
At this point both \code{t0} and \code{t1} have active calls to C on the
 C stack, with \code{t1}'s C frame above \code{t0}'s.
If thread \code{t0} attempts to return from Scheme to C it will block,
 as its frame is not accessible.
Once \code{t1} has returned to C and from there to Scheme, \code{t0} will
 be able to resume.
The return to Scheme is required because context switches can only occur while
 Scheme code is running.
\code{T0} will also be able to resume if \code{t1} uses a continuation to
 throw past its call to C.

\section{Interacting with the Scheme heap}
\label{sec:heap-allocation}
\label{gc}

Scheme~48 uses a copying, precise garbage collector.
Any procedure that allocates objects within the Scheme~48 heap may trigger
 a garbage collection.
Variables bound to values in the Scheme~48 heap need to be registered with
 the garbage collector so that the value will be retained and so that the
 variables will be updated if the garbage collector moves the object.
The garbage collector has no facility for updating pointers to the interiors
 of objects, so such pointers, for example the ones returned by
 \code{s48\_extract\_byte\_vector}, will likely become invalid when a garbage collection
 occurs.

\subsection{Registering objects with the GC}
\label{sec:gc-register}

A set of macros are used to manage the registration of local variables with the
 garbage collector.

\begin{protos}
\cproto{S48\_DECLARE\_GC\_PROTECT($n$)}
\cproto{void S48\_GC\_PROTECT\_$n$(s48\_value$_1$, $\ldots$, s48\_value$_n$)}
\cproto{void S48\_GC\_UNPROTECT()}
\end{protos}

\code{S48\_DECLARE\_GC\_PROTECT($n$)}, where  $1\leq n\leq 9$, allocates
 storage for registering $n$ variables.
% JAR says: what is a block?  (How to describe it? -RK)
At most one use of \code{S48\_DECLARE\_GC\_PROTECT} may occur in a
 block.
\code{S48\_GC\_PROTECT\_$n$($v_1$, $\ldots$, $v_n$)} registers the
 $n$ variables (l-values) with the garbage collector.
It must be within scope of a \code{S48\_DECLARE\_GC\_PROTECT($n$)}
 and be before any code which can cause a GC.
\code{S48\_GC\_UNPROTECT} removes the block's protected variables from
 the garbage collector's list.
It must be called at the end of the block after 
  any code which may cause a garbage collection.
Omitting any of the three may cause serious and
 hard-to-debug problems.
Notably, the garbage collector may relocate an object and
 invalidate \code{s48\_value} variables which are not protected.

A \code{gc-protection-mismatch} exception is raised if, when a C
 procedure returns to Scheme, the calls
 to \code{S48\_GC\_PROTECT()} have not been matched by an equal number of
 calls to \code{S48\_GC\_UNPROTECT()}.

Global variables may also be registered with the garbage collector.

\begin{protos}
\cproto{void * S48\_GC\_PROTECT\_GLOBAL(\cvar{value})}
\cproto{void S48\_GC\_UNPROTECT\_GLOBAL(void * handle)}
\end{protos}

\noindent{}\code{S48\_GC\_PROTECT\_GLOBAL} permanently registers the
variable \cvar{value} (an l-value of type \code{s48\_value}) with the
garbage collector.  It returns a handle pointer for use as an argument
to \code{S48\_GC\_UNPROTECT\_GLOBAL}, which unregisters the variable
again.

\subsection{Keeping C data structures in the Scheme heap}
\label{sec:external-data}

C data structures can be kept in the Scheme heap by embedding them
 inside byte vectors.
The following macros can be used to create and access embedded C objects.

\begin{protos}
\cgcproto{s48\_value S48\_MAKE\_VALUE(type)}
\cgcproto{s48\_value S48\_MAKE\_SIZED\_VALUE(size)}
\cproto{type \ \ \ \ \ S48\_EXTRACT\_VALUE(s48\_value, type)}
\cproto{long \ \ \ \ \ S48\_VALUE\_SIZE(s48\_value)}
\cproto{type * \ \ \ S48\_EXTRACT\_VALUE\_POINTER(s48\_value, type)}
\cproto{void \ \ \ \ \ S48\_SET\_VALUE(s48\_value, type, value)}
\end{protos}

\noindent{}
\code{S48\_MAKE\_VALUE} makes a byte vector large enough to hold an object
whose type is \cvar{type}.
\code{S48\_MAKE\_SIZED\_VALUE} makes a byte vector large enough to hold an object
of \cvar{size} bytes.
\code{S48\_EXTRACT\_VALUE} returns the contents of a byte vector cast to
 \cvar{type}, \code{S48\_VALUE\_SIZE} returns its size,
and \code{S48\_EXTRACT\_VALUE\_POINTER} returns a pointer to the
contents of the byte vector.
The value returned by \code{S48\_EXTRACT\_VALUE\_POINTER} is valid only until
the next garbage collection.

\code{S48\_SET\_VALUE} stores \code{value} into the byte vector.

%There are some convenient macros for external objects that hold
% arrays:
%
%\begin{itemize}
%\item \code{S48\_MAKE\_ARRAY($b$, $s$)} returns an external object
%  which holds an array with base type $b$ and size $s$.
%\item \code{S48\_EXTRACT\_ARRAY(\cvar{value}, $b$)} returns the address of the
%  array with base type $b$ inside external object \cvar{value}.  It does not
%  check if \cvar{value} is actually an external object.  Note that the address
%  returned by \code{S48\_EXTRACT\_ARRAY} is only valid until the next
%  \link{heap allocation}[ (see
%  Sec.~\ref{sec:heap-allocation})]{sec:heap-allocation}.
%\end{itemize}

\subsection{C code and heap images}
\label{sec:hibernation}

Scheme~48 uses dumped heap images to restore a previous system state.
The Scheme~48 heap is written into a file in a machine-independent and
 operating-system-independent format.
The procedures described above may be used to create objects in the
 Scheme heap that contain information specific to the current
 machine, operating system, or process.
A heap image containing such objects may not work correctly
 when resumed.

To address this problem, a record type may be given a `resumer'
 procedure.
On startup, the resumer procedure for a type is applied to each record of
 that type in the image being restarted.
This procedure can update the record in a manner appropriate to
 the machine, operating system, or process used to resume the
 image.

\begin{protos}
\protonoresult{define-record-resumer}{ record-type procedure}
\end{protos}

\noindent{}\code{Define-record-resumer} defines \cvar{procedure},
 which should accept one argument, to be the resumer for
 \var{record-type}.
The order in which resumer procedures are called is not specified.

The \cvar{procedure} argument to \code{define-record-resumer} may
 be \code{\#f}, in which case records of the given type are
 not written out in heap images.
When writing a heap image any reference to such a record is replaced by
 the value of the record's first field, and an exception is raised
 after the image is written.

\section{Using Scheme records in C code}

External modules can create records and access their slots
 positionally.

\begin{protos}
\cgcproto{s48\_value s48\_make\_record(s48\_value)}
\cproto{int \ \ \ \ \ \ S48\_RECORD\_P(s48\_value)} 
\cproto{s48\_value S48\_RECORD\_TYPE(s48\_value)} 
\cproto{s48\_value S48\_RECORD\_REF(s48\_value, long)} 
\cproto{void \ \ \ \ \ S48\_RECORD\_SET(s48\_value, long, s48\_value)} 
\end{protos}
%
The argument to \code{s48\_make\_record} should be a shared binding
 whose value is a record type.
In C the fields of Scheme records are only accessible via offsets,
 with the first field having offset zero, the second offset one, and
 so forth.
If the order of the fields is changed in the Scheme definition of the
 record type the C code must be updated as well.

For example, given the following record-type definition
\begin{example}
(define-record-type thing :thing
  (make-thing a b)
  thing?
  (a thing-a)
  (b thing-b))
\end{example}
 the identifier \code{:thing} is bound to the record type and can
 be exported to C:
\begin{example}
(define-exported-binding "thing-record-type" :thing)
\end{example}
\code{Thing} records can then be made in C:
\begin{example}
static s48_value
  thing_record_type_binding = S48_FALSE;

void initialize_things(void)
\{
  S48_GC_PROTECT_GLOBAL(thing_record_type_binding);
  thing_record_type_binding =
     s48_get_imported_binding("thing-record-type");
\}

s48_value make_thing(s48_value a, s48_value b)
\{
  s48_value thing;
  s48_DECLARE_GC_PROTECT(2);

  S48_GC_PROTECT_2(a, b);

  thing = s48_make_record(thing_record_type_binding);
  S48_RECORD_SET(thing, 0, a);
  S48_RECORD_SET(thing, 1, b);

  S48_GC_UNPROTECT();

  return thing;
\}
\end{example}
Note that the variables \code{a} and \code{b} must be protected
 against the possibility of a garbage collection occuring during
 the call to \code{s48\_make\_record()}.

\section{Raising exceptions from external code}
\label{sec:exceptions}

The following macros explicitly raise certain errors, immediately
 returning to Scheme~48.
Raising an exception performs all
 necessary clean-up actions to properly return to Scheme~48, including
 adjusting the stack of protected variables.

The following procedures are available for raising particular
 types of exceptions.
These never return.

\begin{protos}
\cproto{s48\_assertion\_violation(const char* who, const char* message, long count, ...)}
\cproto{s48\_error(const char* who, const char* message, long count, ...)}
\cproto{s48\_os\_error(const char* who, const char* message, long count, ...)}
\cproto{s48\_out\_of\_memory\_error()}
\end{protos}

\noindent{}An assertion violation signalled via
\code{s48\_assertion\_violation} typically means that an invalid
argument (or invalid number of arguments) has been passed.  An error
signalled via \code{s48\_error} means that an environmental error
(like an I/O error) has occurred.  In both cases, \code{who} indicates
the location of the error, typically the name of the function it
occurred in.  It may be \code{NULL}, in which the system guesses a
name.  The \code{message} argument is an error message encoded in
UTF-8.  Additional arguments may be passed that become part of the
condition object that will be raised on the Scheme side: \code{count}
indicates their number, and the arguments (which must be of type
\code{s48\_value}) follow.

The \code{s48\_os\_error} function is like \code{s48\_error}, except
that the error message is inferred from an OS error code (as in
\code{strerro}).  The \code{s48\_out\_of\_memory\_error} function
signals that the system has run out of memory.

The following macros raise assertion violations if their argument does
not have the required type.  \code{S48\_CHECK\_BOOLEAN} raises an
error if its argument is neither \code{\#t} or \code{\#f}.

\begin{protos}
\cproto{void S48\_CHECK\_BOOLEAN(s48\_value)}
\cproto{void S48\_CHECK\_SYMBOL(s48\_value)}
\cproto{void S48\_CHECK\_PAIR(s48\_value)}
\cproto{void S48\_CHECK\_STRING(s48\_value)}
\cproto{void S48\_CHECK\_INTEGER(s48\_value)}
\cproto{void S48\_CHECK\_CHANNEL(s48\_value)}
\cproto{void S48\_CHECK\_BYTE\_VECTOR(s48\_value)}
\cproto{void S48\_CHECK\_RECORD(s48\_value)}
\cproto{void S48\_CHECK\_SHARED\_BINDING(s48\_value)}
\end{protos}

\section{Unsafe functions and macros}

All of the C procedures and macros described above check that their
 arguments have the appropriate types and that indexes are in range.
The following procedures and macros are identical to those described
 above, except that they do not perform type and range checks.
They are provided for the purpose of writing more efficient code;
 their general use is not recommended.

\begin{protos}
\cproto{long \ \ \ \ \ S48\_UNSAFE\_EXTRACT\_CHAR(s48\_value)}
\cproto{s48\_value S48\_UNSAFE\_ENTER\_CHAR(long)}
\cproto{long \ \ \ \ \ S48\_UNSAFE\_EXTRACT\_INTEGER(s48\_value)}
\cproto{long \ \ \ \ \ S48\_UNSAFE\_EXTRACT\_DOUBLE(s48\_value)}
\end{protos}
\begin{protos}
\cproto{long \ \ \ \ \ S48\_UNSAFE\_EXTRACT\_FIXNUM(s48\_value)}
\cproto{s48\_value S48\_UNSAFE\_ENTER\_FIXNUM(long)}
\end{protos}
\begin{protos}
\cproto{s48\_value S48\_UNSAFE\_CAR(s48\_value)}
\cproto{s48\_value S48\_UNSAFE\_CDR(s48\_value)}
\cproto{void \ \ \ \ \ S48\_UNSAFE\_SET\_CAR(s48\_value, s48\_value)}
\cproto{void \ \ \ \ \ S48\_UNSAFE\_SET\_CDR(s48\_value, s48\_value)}
\end{protos}
\begin{protos}
\cproto{long \ \ \ \ \ S48\_UNSAFE\_VECTOR\_LENGTH(s48\_value)} 
\cproto{s48\_value S48\_UNSAFE\_VECTOR\_REF(s48\_value, long)} 
\cproto{void \ \ \ \ \ S48\_UNSAFE\_VECTOR\_SET(s48\_value, long, s48\_value)} 
\end{protos}
\begin{protos}
\cproto{long \ \ \ \ \ S48\_UNSAFE\_STRING\_LENGTH(s48\_value)} 
\cproto{char \ \ \ \ \ S48\_UNSAFE\_STRING\_REF(s48\_value, long)} 
\cproto{void \ \ \ \ \ S48\_UNSAFE\_STRING\_SET(s48\_value, long, char)} 
\end{protos}
\begin{protos}
\cproto{s48\_value S48\_UNSAFE\_SYMBOL\_TO\_STRING(s48\_value)} 
\end{protos}
\begin{protos}
\cproto{long \ \ \ \ \ S48\_UNSAFE\_BYTE\_VECTOR\_LENGTH(s48\_value)} 
\cproto{char \ \ \ \ \ S48\_UNSAFE\_BYTE\_VECTOR\_REF(s48\_value, long)} 
\cproto{void \ \ \ \ \ S48\_UNSAFE\_BYTE\_VECTOR\_SET(s48\_value, long, int)} 
\end{protos}
\begin{protos}
\cproto{s48\_value S48\_UNSAFE\_SHARED\_BINDING\_REF(s48\_value s\_b)}
\cproto{int\ \ \ \ \ \ \ S48\_UNSAFE\_SHARED\_BINDING\_P(x)}
\cproto{int\ \ \ \ \ \ \ S48\_UNSAFE\_SHARED\_BINDING\_IS\_IMPORT\_P(s48\_value s\_b)}
\cproto{s48\_value S48\_UNSAFE\_SHARED\_BINDING\_NAME(s48\_value s\_b)}
\cproto{void\ \ \ \ \ \ S48\_UNSAFE\_SHARED\_BINDING\_SET(s48\_value s\_b, s48\_value value)}
\end{protos}
\begin{protos}
\cproto{s48\_value S48\_UNSAFE\_RECORD\_TYPE(s48\_value)} 
\cproto{s48\_value S48\_UNSAFE\_RECORD\_REF(s48\_value, long)} 
\cproto{void \ \ \ \ \ S48\_UNSAFE\_RECORD\_SET(s48\_value, long, s48\_value)} 
\end{protos}
\begin{protos}
\cproto{type \ \ \ \ \ S48\_UNSAFE\_EXTRACT\_VALUE(s48\_value, type)}
\cproto{type * \ \ \ S48\_UNSAFE\_EXTRACT\_VALUE\_POINTER(s48\_value, type)}
\cproto{void \ \ \ \ \ S48\_UNSAFE\_SET\_VALUE(s48\_value, type, value)}
\end{protos}

%%% Local Variables: 
%%% mode: latex
%%% TeX-master: "manual"
%%% End: 
