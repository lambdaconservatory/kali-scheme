
% Still to do:
%  destructure
%  format (?)

\chapter{Libraries}

Use the
\code{,open} command (section~\ref{module-command-guide})
 or
the module language (chapter~\ref{module-guide})
 to open the structures described below.

\section{General utilities}
\label{big-util}

%% These are missing:
%%        copy-string
%%        string->immutable-string
%%        error
%%        breakpoint

These are in the \code{big-util} structure.

\begin{protos}
\proto{atom?}{ value}{boolean}
\end{protos}
%
\code{(atom? \var{x})} is the same as \code{(not (pair? \var{x}))}.

\begin{protos}
\proto{null-list?}{ list}{boolean}
\end{protos}
%
Returns true for the empty list, false for a pair, and signals an
error otherwise.

\begin{protos}
\proto{neq?}{ value value}{boolean}
\end{protos}
\code{(neq? \var{x} \var{y})} is the same as \code{(not (eq? \var{x}
\var{y}))}.

\begin{protos}
\proto{n=}{ number number}{boolean}
\end{protos}
\code{(n= \var{x} \var{y})} is the same as \code{(not (= \var{x}
  \var{y}))}.

\begin{protos}
\proto{identity}{ value}{value}
\proto{no-op}{ value}{value}
\end{protos}
These both just return their argument.  \code{No-op} is guaranteed not to
be compiled in-line, \code{identity} may be.

\begin{protos}
\proto{memq?}{ value list}{boolean}
\end{protos}
%
Returns true if \var{value} is in \var{list}, false otherwise.

\begin{protos}
\proto{any?}{ predicate list}{boolean}
\end{protos}
Returns true if \var{predicate} is true for any element of \var{list}.

\begin{protos}
\proto{every?}{ predicate list}{boolean}
\end{protos}
  Returns true if \var{predicate} is true for every element of \var{list}.

\begin{protos}
\proto{any}{ predicate list}{value}
\proto{first}{ predicate list}{value}
\end{protos}
\code{Any} returns some element of \var{list} for which \var{predicate} is true, or
false if there are none.  \code{First} does the same except that it returns
the first element for which \var{predicate} is true.

\begin{protos}
\proto{filter}{ predicate list}{list}
\protonoindex{filter!}{ predicate list}{list}\mainschindex{filter"!}
\end{protos}
Returns a list containing all of the elements of \var{list} for which
\var{predicate} is true.  The order of the elements is preserved.
\code{Filter!} may reuse the storage of \var{list}.

\begin{protos}
\proto{filter-map}{ procedure list}{list}
\end{protos}
The same as \code{filter} except the returned list contains the results of
applying \var{procedure} instead of elements of \var{list}.  \code{(filter-map \var{p}
\var{l})} is the same as \code{(filter identity (map \var{p} \var{l}))}.

\begin{protos}
\proto{partition-list}{ predicate list}{list list}
\protonoindex{partition-list!}{ predicate list}{list list}\mainschindex{partition-list"!}
\end{protos}
The first return value contains those elements \var{list} for which
\var{predicate} is true, the second contains the remaining elements.
The order of the elements is preserved.  \code{Partition-list!} may
reuse the storage of the \var{list}.

\begin{protos}
\proto{remove-duplicates}{ list}{list}
\end{protos}
Returns its argument with all duplicate elements removed.  The first
instance of each element is preserved.

\begin{protos}
\proto{delq}{ value list}{list}
\protonoindex{delq!}{ value list}{list}\mainschindex{delq"!}
\proto{delete}{ predicate list}{list}
\end{protos}
All three of these return \var{list} with some elements removed.
\code{Delq} removes all elements \code{eq?} to \var{value}.  \code{Delq!}
does the same and may modify the list argument.  \code{Delete} removes
all elements for which \var{predicate} is true.  Both \code{delq} and
\code{delete} may reuse some of the storage in the list argument, but
won't modify it.

\begin{protos}
\protonoindex{reverse!}{ list}{list}\mainschindex{reverse"!}
\end{protos}
Destructively reverses \var{list}.

\begin{protos}
\proto{concatenate-symbol}{ value \ldots}{symbol}
\end{protos}
Returns the symbol whose name is produced by concatenating the
\code{display}ed
representations of \var{value}~\ldots.

\begin{example}
(concatenate-symbol 'abc "-" 4) \(\Longrightarrow\) 'abc-4
\end{example}

\section{Pretty-printing}

These are in the \code{pp} structure.

\begin{protos}
\protonoresult{p}{ value}
\protonoresult{p}{ value output-port}
\protonoresult{pretty-print}{ value output-port position}
\end{protos}
Pretty-print \var{value} The current output port is used if no port is
specified.  \var{Position} is the starting offset.  \var{Value} will be
pretty-printed to the right of this column.

\section{Bitwise integer operations}

These functions use the two's-complement representation for integers.
There is no limit to the number of bits in an integer.
They are in the structures \code{bitwise} and \code{big-scheme}.

\begin{protos}
\proto{bitwise-and}{ integer integer}{integer}
\proto{bitwise-ior}{ integer integer}{integer}
\proto{bitwise-xor}{ integer integer}{integer}
\proto{bitwise-not}{ integer} {integer}
\end{protos}
\noindent
These perform various logical operations on integers on a bit-by-bit
basis. `\code{ior}' is inclusive OR and `\code{xor}' is exclusive OR.

\begin{protos}
\proto{arithmetic-shift}{ integer bit-count}{integer}
\end{protos}
\noindent Shifts the integer by the given bit count, which must be an integer,
 shifting left for positive counts and right for negative ones.
Shifting preserves the integer's sign.

\begin{protos}
\proto{bit-count}{ integer}{integer}
\end{protos}
\noindent Counts the number of bits set in the integer.
If the argument is negative a bitwise NOT operation is performed
 before counting.

\section{Byte vectors}

These are homogeneous vectors of small integers ($0 \le i \le 255$).
The functions that operate on them are analogous to those for vectors.
They are in the structure \code{byte-vectors}.

\begin{protos}
\proto{byte-vector?}{ value}{boolean}
\proto{make-byte-vector}{ k fill}{byte-vector}
\proto{byte-vector}{ b \ldots}{byte-vector}
\proto{byte-vector-length}{ byte-vector}{integer}
\proto{byte-vector-ref}{ byte-vector k}{integer}
\protonoresultnoindex{byte-vector-set!}{ byte-vector k b}\mainschindex{byte-vector-set"!}
\end{protos}

\section{Sparse vectors}

These are vectors that grow as large as they need to.  That is, they
can be indexed by arbitrarily large nonnegative integers.  The
implementation allows for arbitrarily large gaps by arranging the
entries in a tree.  They are in the structure \code{sparse-vectors}.

\begin{protos}
\proto{make-sparse-vector}{}{sparse-vector}
\proto{sparse-vector-ref}{ sparse-vector k}{value}
\protonoresultnoindex{sparse-vector-set!}{ sparse-vector k value}\mainschindex{sparse-vector-set"!}
\proto{sparse-vector->list}{ sparse-vector}{list}
\end{protos}
%
\code{Make-sparse-vector}, \code{sparse-vector-ref}, and
\code{sparse-vector-set!} are analogous to \code{make-vector},
\code{vector-ref}, and \code{vector-set!}, except that the indices
passed to \code{sparse-vector-ref} and \code{sparse-vector-set!} can
be arbitrarily large.  For indices whose elements have not been set in
a sparse vector, \code{sparse-vector-ref} returns \code{\#f}.

\code{Sparse-vector->list} is for debugging: It returns a list of the
consecutive elements in a sparse vector from 0 to the highest element
that has been set.  Note that the list will also include all the
\code{\#f} elements for the unset elements.


\section{Cells}
\label{cells}

These hold a single value and are useful when a simple indirection is
 required.
The system uses these to hold the values of lexical variables that
 may be \code{set!}.

\begin{protos}
\proto{cell?}{ value}{boolean}
\proto{make-cell}{ value}{cell}
\proto{cell-ref}{ cell}{value}
\protonoresultnoindex{cell-set!}{ cell value}\mainschindex{cell-set"!}
\end{protos}

\section{Queues}

These are ordinary first-in, first-out queues.
The procedures are in structure \code{queues}.

\begin{protos}
\proto{make-queue}{}{queue}
\proto{queue?}{ value}{boolean}
\proto{queue-empty?}{ queue}{boolean}
\protonoresultnoindex{enqueue!}{ queue value}\mainschindex{enqueue"!}
\protonoindex{dequeue!}{ queue}{value}\mainschindex{dequeue"!}
\end{protos}
\noindent 
\code{Make-queue} creates an empty queue, \code{queue?} is a predicate for
 identifying queues, \code{queue-empty?} tells you if a queue is empty,
 \code{enqueue!} and \code{dequeue!} add and remove values.

\begin{protos}
\proto{queue-length}{ queue}{integer}
\proto{queue->list}{ queue}{values}
\proto{list->queue}{ values}{queue}
\protonoindex{delete-from-queue!}{ queue value}{boolean}\mainschindex{delete-from-queue"!}
\end{protos}
\noindent
\code{Queue-length} returns the number of values in \var{queue}.
\code{Queue->list} returns the values in \var{queue} as a list, in the
 order in which the values were added.
\code{List->queue} returns a queue containing \var{values}, preserving
 their order.
\code{Delete-from-queue} removes the first instance of \var{value} from
 \code{queue}, using \code{eq?} for comparisons.
\code{Delete-from-queue} returns \code{\#t} if \var{value} is found and
 \code{\#f} if it is not.

\section{Arrays}

These provide N-dimensional, zero-based arrays and
 are in the structure \code{arrays}.
The array interface is derived from one invented by Alan Bawden.

\begin{protos}
\proto{make-array}{ value dimension$_0$ \ldots}{array}
\proto{array}{ dimensions element$_0$ \ldots}{array}
\proto{copy-array}{ array}{array}
\end{protos}
\noindent
\code{Make-array} makes a new array with the given dimensions, each of which
 must be a non-negative integer.
Every element is initially set to \cvar{value}.
\code{Array} Returns a new array with the given dimensions and elements.
\cvar{Dimensions} must be a list of non-negative integers, 
The number of elements should be the equal to the product of the
 dimensions.
The elements are stored in row-major order.
\begin{example}
(make-array 'a 2 3) \evalsto \{Array 2 3\}

(array '(2 3) 'a 'b 'c 'd 'e 'f)
    \evalsto \{Array 2 3\}
\end{example}

\code{Copy-array} returns a copy of \cvar{array}.
The copy is identical to the \cvar{array} but does not share storage with it.

\begin{protos}
\proto{array?}{ value}{boolean}
\end{protos}
\noindent
Returns \code{\#t} if \cvar{value} is an array.

\begin{protos}
\proto{array-ref}{ array index$_0$ \ldots}{value}
\protonoresultnoindex{array-set!}{ array value index$_0$ \ldots}\mainschindex{array-set"!}
\proto{array->vector}{ array}{vector}
\proto{array-shape}{ array}{list}
\end{protos}
\noindent
\code{Array-ref} returns the specified array element and \code{array-set!}
 replaces the element with \cvar{value}.
\begin{example}
(let ((a (array '(2 3) 'a 'b 'c 'd 'e 'f)))
  (let ((x (array-ref a 0 1)))
    (array-set! a 'g 0 1)
    (list x (array-ref a 0 1))))
    \evalsto '(b g)
\end{example}

\code{Array->vector} returns a vector containing the elements of \cvar{array}
 in row-major order.
\code{Array-shape} returns the dimensions of
 the array as a list.

\begin{protos}
\proto{make-shared-array}{ array linear-map dimension$_0$ \ldots}{array}
\end{protos}
\noindent
\code{Make-shared-array} makes a new array that shares storage with \cvar{array}
 and uses \cvar{linear-map} to map indexes to elements.
\cvar{Linear-map} must accept as many arguments as the number of
 \cvar{dimension}s given and must return a list of non-negative integers
 that are valid indexes into \cvar{array}.
\begin{example}
(array-ref (make-shared-array a f i0 i1 ...)
           j0 j1 ...)
\end{example}
is equivalent to
\begin{example}
(apply array-ref a (f j0 j1 ...))
\end{example}

As an example, the following function makes the transpose of a two-dimensional
 array:
\begin{example}
(define (transpose array)
  (let ((shape (array-shape array)))
    (make-shared-array array
                       (lambda (x y)
                         (list y x))
                       (cadr shape)
                       (car shape))))

(array->vector
  (transpose
    (array '(2 3) 'a 'b 'c 'd 'e 'f)))
      \evalsto '(a d b e c f)
\end{example}

\section{Records}
\label{records}

New types can be constructed using the \code{define-record-type} macro
 from the \code{define-record-types} structure
The general syntax is:
\begin{example}
(define-record-type \cvar{tag} \cvar{type-name}
  (\cvar{constructor-name} \cvar{field-tag} \ldots)
  \cvar{predicate-name}
  (\cvar{field-tag} \cvar{accessor-name} [\cvar{modifier-name}])
  \ldots)
\end{example}
This makes the following definitions:
\begin{protos}
\constprotonoindex{\cvar{type-name}}{type}
\protonoindex{\cvar{constructor-name}}{ field-init \ldots}{type-name}
\protonoindex{\cvar{predicate-name}}{ value}{boolean}
\protonoindex{\cvar{accessor-name}}{ type-name}{value}
\protonoresultnoindex{\cvar{modifier-name}}{ type-name value}
\end{protos}
\noindent
\cvar{Type-name} is the record type itself, and can be used to
 specify a print method (see below).
\cvar{Constructor-name} is a constructor that accepts values
 for the fields whose tags are specified.
\cvar{Predicate-name} is a predicate that returns \code{\#t} for
 elements of the type and \code{\#f} for everything else.
The \cvar{accessor-name}s retrieve the values of fields,
 and the \cvar{modifier-name}'s update them.
\cvar{Tag} is used in printing instances of the record type and
 the \cvar{field-tag}s are used in the inspector and to match
 constructor arguments with fields.

\begin{protos}
\protonoresult{define-record-discloser}{ type discloser}
\end{protos}
\noindent
\code{Define-record-discloser} determines how
 records of type \cvar{type} are printed.
\cvar{Discloser} should be procedure which takes a single
 record of type \cvar{type} and returns a list whose car is
 a symbol.
The record will be printed as the value returned by \cvar{discloser}
 with curly braces used instead of the usual parenthesis.

For example
\begin{example}
(define-record-type pare :pare
  (kons x y)
  pare?
  (x kar set-kar!)
  (y kdr))
\end{example}
 defines \code{kons} to be a constructor, \code{kar} and \code{kdr} to be
 accessors, \code{set-kar!} to be a modifier, and \code{pare?} to be a predicate
 for a new type of object.
The type itself is named \code{:pare}.
\code{Pare} is a tag used in printing the new objects.

By default, the new objects print as \code{\#\{Pare\}}.
The print method can be modified using \code{define-record-discloser}:
\begin{example}
(define-record-discloser :pare
  (lambda (p) `(pare ,(kar p) ,(kdr p))))
\end{example}
 will cause the result of \code{(kons 1 2)} to print as
 \code{\#\{Pare 1 2\}}.

\code{Define-record-resumer} (section~\ref{sec:hibernation})
 can be used to control how records are stored in heap images.

\subsection{Low-level access to records}

Records are implemented using primitive objects exactly analogous
 to vectors.
Every record has a record type (which is another record) in the first slot.
Note that use of these procedures, especially \code{record-set!}, breaks
 the record abstraction described above; caution is advised.

These procedures are in the structure \code{records}.

\begin{protos}
\proto{make-record}{ n value}{record}
\proto{record}{ value \ldots}{record-vector}
\proto{record?}{ value}{boolean}
\proto{record-length}{ record}{integer}
\proto{record-type}{ record}{value}
\proto{record-ref}{ record i}{value}
\protonoresultnoindex{record-set!}{ record i value}\mainschindex{record-set"!}
\end{protos}
\noindent
These the same as the standard \code{vector-} procedures except that they
 operate on records.
The value returned by \code{record-length} includes the slot holding the
 record's type.
\code{(record-type \cvar{x})} is equivalent to \code{(record-ref \cvar{x} 0)}.

\subsection{Record types}

Record types are themselves records of a particular type (the first slot
 of \code{:record-type} points to itself).
A record type contains four values: the name of the record type, a list of
 the names its fields, and procedures for disclosing and resuming records
 of that type.
Procedures for manipulating them are in the structure \code{record-types}.

\begin{protos}
\proto{make-record-type}{ name field-names}{record-type}
\proto{record-type?}{ value}{boolean}
\proto{record-type-name}{ record-type}{symbol}
\proto{record-type-field-names}{ record-type}{symbols}
\end{protos}
\noindent

\begin{protos}
\proto{record-constructor}{ record-type field-names}{procedure}
\proto{record-predicate}{ record-type}{procedure}
\proto{record-accessor}{ record-type field-name}{procedure}
\proto{record-modifier}{ record-type field-name}{procedure}
\end{protos}
\noindent
These procedures construct the usual record-manipulating procedures.
\code{Record-constructor} returns a constructor that is passed the initial
 values for the fields specified and returns a new record.
\code{Record-predicate} returns a predicate that return true when passed
 a record of type \cvar{record-type} and false otherwise.
\code{Record-accessor} and \code{record-modifier} return procedures that
 reference and set the given field in records of the approriate type.

\begin{protos}
\protonoresult{define-record-discloser}{ record-type discloser}
\protonoresult{define-record-resumer}{ record-type resumer}
\end{protos}
\noindent
\noindent \code{Record-types} is the initial exporter of
 \code{define-record-discloser}
 (re-exported by \code{define-record-types} described above)
 and
 \code{define-record-resumer}
 (re-exported by
 \code{external-calls} (section~\ref{sec:hibernation})).

The procedures described in this section can be used to define new
 record-type-defining macros.
\begin{example}
(define-record-type pare :pare
  (kons x y)
  pare?
  (x kar set-kar!)
  (y kdr))
\end{example}
is (sematically) equivalent to
\begin{example}
(define :pare (make-record-type 'pare '(x y)))
(define kons (record-constructor :pare '(x y)))
(define kar (record-accessor :pare 'x))
(define set-kar! (record-modifier :pare 'x))
(define kdr (record-accessor :pare 'y))
\end{example}

The ``(semantically)'' above is because \code{define-record-type} adds
 declarations, which allows the type checker to detect some misuses of records,
 and uses more efficient definitions for the constructor, accessors, and
 modifiers.
Ignoring the declarations, which will have to wait for another edition of
 the manual, what the above example actually expands into is:
\begin{example}
(define :pare (make-record-type 'pare '(x y)))
(define (kons x y) (record :pare x y))
(define (kar r) (checked-record-ref r :pare 1))
(define (set-kar! r new)
  (checked-record-set! r :pare 1 new))
(define (kdr r) (checked-record-ref r :pare 2))
\end{example} 
\code{Checked-record-ref} and \code{Checked-record-set!} are
 low-level procedures that check the type of the
 record and access or modify it using a single VM instruction.

\section{Finite record types}

The structure \code{finite-types} has
 two macros for defining `finite' record types.
These are record types for which there are a fixed number of instances,
 all of which are created at the same time as the record type itself.
The syntax for defining an enumerated type is:
\begin{example}
(define-enumerated-type \cvar{tag} \cvar{type-name}
  \cvar{predicate-name}
  \cvar{vector-of-instances-name}
  \cvar{name-accessor}
  \cvar{index-accessor}
  (\cvar{instance-name} \ldots))
\end{example}
This defines a new record type, bound to \cvar{type-name}, with as many
 instances as there are \cvar{instance-name}'s.
\cvar{Vector-of-instances-name} is bound to a vector containing the instances
 of the type in the same order as the \cvar{instance-name} list.
\cvar{Tag} is bound to a macro that when given an \cvar{instance-name} expands
 into an expression that returns corresponding instance.
The name lookup is done at macro expansion time.
\cvar{Predicate-name} is a predicate for the new type.
\cvar{Name-accessor} and \cvar{index-accessor} are accessors for the
 name and index (in \cvar{vector-of-instances}) of instances of the type.

\begin{example}
(define-enumerated-type color :color
  color?
  colors
  color-name
  color-index
  (black white purple maroon))

(color-name (vector-ref colors 0)) \evalsto black
(color-name (color white))         \evalsto white
(color-index (color purple))       \evalsto 2
\end{example}

Finite types are enumerations that allow the user to add additional
 fields in the type.
The syntax for defining a finite type is:
\begin{example}
(define-finite-type \cvar{tag} \cvar{type-name}
  (\cvar{field-tag} \ldots)
  \cvar{predicate-name}
  \cvar{vector-of-instances-name}
  \cvar{name-accessor}
  \cvar{index-accessor}
  (\cvar{field-tag} \cvar{accessor-name} [\cvar{modifier-name}])
  \ldots
  ((\cvar{instance-name} \cvar{field-value} \ldots)
   \ldots))
\end{example}
The additional fields are specified exactly as with \code{define-record-type}.
The field arguments to the constructor are listed after the \cvar{type-name};
 these do not include the name and index fields.
The form ends with the names and the initial field values for
 the instances of the type.
The instances are constructed by applying the (unnamed) constructor to
 these initial field values.
The name must be first and 
 the remaining values must match the \cvar{field-tag}s in the constructor's
 argument list.

%This differs from \code{define-record-type} in the following ways:
%\begin{itemize}
%\item No name is specified for the constructor, but the field arguments
% to the constructor are listed.
%\item The \cvar{vector-of-instances-name} is added; it will be bound
% to a vector containing all of the instances of the type.
%These are constructed by applying the (unnamed) constructor to the
% initial field values at the end of the form.
%\item There are names for accessors for two required fields, name
% and index.
%These fields are not settable, and are not to be included
% in the argument list for the constructor.
%\item The form ends with the names and the initial field values for
% the instances of the type.
%The name must be first.
%The remaining values must match the \cvar{field-tag}s in the constructor's
% argument list.
%\item \cvar{Tag} is bound to a macro that maps \cvar{instance-name}s to the
% the corresponding instance of the vector.
%The name lookup is done at macro-expansion time.
%\end{itemize}

\begin{example}
(define-finite-type color :color
  (red green blue)
  color?
  colors
  color-name
  color-index
  (red   color-red)
  (green color-green)
  (blue  color-blue)
  ((black    0   0   0)
   (white  255 255 255)
   (purple 160  32 240)
   (maroon 176  48  96)))

(color-name (color black))         \evalsto black
(color-name (vector-ref colors 1)) \evalsto white
(color-index (color purple))       \evalsto 2
(color-red (color maroon))         \evalsto 176
\end{example}

\section{Sets over finite types}
\label{sets-finite-types}

The structure \code{enum-sets} has a macro for defining types for sets
of elements of finite types.  These work naturally with the finite
types defined by the \code{finite-types} structure, but are not tied
to them.  The syntax for defining such a type is:

\begin{example}
(define-enum-set-type \cvar{id} \cvar{type-name} \cvar{predicate} \cvar{constructor}
   \cvar{element-syntax} \cvar{element-predicate} \cvar{all-elements} \cvar{element-index-ref})
\end{example}
%
This defines \cvar{id} to be syntax for constructing sets,
\cvar{type-name} to be a value representing the type,
\cvar{predicate} to be a predicate for those sets, and
\cvar{constructor} a procedure for constructing one from a list.

\cvar{Element-syntax} must be the name of a macro for constructing set
elements from names (akin to the \cvar{tag} argument to
\code{define-enumerated-type}).  \cvar{Element-predicate} must be a
predicate for the element type, \cvar{all-elements} a vector of all
values of the element type, and \cvar{element-index-ref} must return
the index of an element within the \cvar{all-elements} vector.

\begin{protos}
\proto{enum-set->list}{ enum-set}{list}
\proto{enum-set-member?}{ enum-set enumerand}{boolean}
\proto{enum-set=?}{ enum-set enum-set}{boolean}
\proto{enum-set-subset?}{ enum-set enum-set}{boolean}
\proto{enum-set-union}{ enum-set enum-set}{enum-set}
\proto{enum-set-intersection}{ enum-set enum-set}{ enum-set}
\proto{enum-set-difference}{ enum-set enum-set}{ enum-set}
\proto{enum-set-negation}{ enum-set}{enum-set}
\end{protos}
%
\code{Enum-set->list} converts a set into a list of its elements.
\code{Enum-set-member?} tests for membership.  \code{Enum-set=?} tests
two sets of equal type for equality.  (If its arguments are not of the
same type, \code{enum-set=?} raises an exception.)
\code{Enum-set-subset?} tests, for two sets of equal type, 
if the first set is a subset of the second one.
\code{Enum-set-union} computes the union of two sets of equal type,
\code{enum-set-intersection} computes the intersection, 
\code{enum-set-difference} computes the intersection, 
and
\code{enum-set-negation} computes the complement of a set.

Here is an example.  Given an enumerated type:

\begin{example}
(define-enumerated-type color :color
  color?
  colors
  color-name
  color-index
  (red blue green))
\end{example}

we can define sets of colors:

\begin{example}
(define-enum-set-type color-set :color-set
                      color-set?
                      make-color-set
  color color? colors color-index)
\end{example}

\begin{example}
> (enum-set->list (color-set red blue))
(#\ob{}Color red\cb{} #\ob{}Color blue\cb{})
> (enum-set->list (enum-set-negation (color-set red blue)))
(#\ob{}Color green\cb{})
> (enum-set-member? (color-set red blue) (color blue))
#t
\end{example}

\section{Hash tables}

These are generic hash tables, and are in the structure \code{tables}.
Strictly speaking they are more maps than tables, as every table has a
 value for every possible key (for that type of table).
All but a finite number of those values are \code{\#f}.

\begin{protos}
\proto{make-table}{}{table}
\proto{make-symbol-table}{}{symbol-table}
\proto{make-string-table}{}{string-table}
\proto{make-integer-table}{}{integer-table}
\proto{make-table-maker}{ compare-proc hash-proc}{procedure}
\protonoresultnoindex{make-table-immutable!}{ table}\mainschindex{make-table-immutable"!}
\end{protos}
\noindent
The first four functions listed make various kinds of tables.
\code{Make-table} returns a table whose keys may be symbols, integer,
 characters, booleans, or the empty list (these are also the values
 that may be used in \code{case} expressions).
As with \code{case}, comparison is done using \code{eqv?}.
The comparison procedures used in symbol, string, and integer tables are
 \code{eq?}, \code{string=?}, and \code{=}.

\code{Make-table-maker} takes two procedures as arguments and returns
 a nullary table-making procedure.
\cvar{Compare-proc} should be a two-argument equality predicate.
\cvar{Hash-proc} should be a one argument procedure that takes a key
 and returns a non-negative integer hash value.
If \code{(\cvar{compare-proc} \cvar{x} \cvar{y})} returns true,
 then \code{(= (\cvar{hash-proc} \cvar{x}) (\cvar{hash-proc} \cvar{y}))}
 must also return true.
For example, \code{make-integer-table} could be defined
 as \code{(make-table-maker = abs)}.

\code{Make-table-immutable!} prohibits future modification to its argument.

\begin{protos}
\proto{table?}{ value}{boolean}
\proto{table-ref}{ table key}{value or {\tt \#f}}
\protonoresultnoindex{table-set!}{ table key value}\mainschindex{table-set"!}
\protonoresult{table-walk}{ procedure table}
\end{protos}
\noindent
\code{Table?} is the predicate for tables.
\code{Table-ref} and \code{table-set!} access and modify the value of \cvar{key}
 in \cvar{table}.
\code{Table-walk} applies \cvar{procedure}, which must accept two arguments,
 to every associated key and non-\code{\#f} value in \code{table}.

\begin{protos}
\proto{default-hash-function}{ value}{integer}
\proto{string-hash}{ string}{integer}
\proto{symbol-hash}{ symbol}{integer}
\end{protos}
\noindent
\code{Default-hash-function} is the hash function used in the tables
 returned by \code{make-table}, \code{string-hash} is the one used
 by \code{make-string-table}, and \code{symbol-hash} is the one used
 by \code{make-symbol-table}.

\section{Port extensions}

These procedures are in structure \code{extended-ports}.

\begin{protos}
\proto{make-string-input-port}{ string}{input-port}
\proto{make-string-output-port}{}{output-port}
\proto{string-output-port-output}{ string-output-port}{string}
\end{protos}
\noindent \code{Make-string-input-port} returns an input port that
 that reads characters from the supplied string.  An end-of-file
 object is returned if the user reads past the end of the string.
\code{Make-string-output-port} returns an output port that saves
 the characters written to it.
These are then returned as a string by \code{string-output-port-output}.

\begin{example}
(read (make-string-input-port "(a b)"))
    \evalsto '(a b)

(let ((p (make-string-output-port)))
  (write '(a b) p)
  (let ((s (string-output-port-output p)))
    (display "c" p)
    (list s (string-output-port-output p))))
    \evalsto '("(a b)" "(a b)c")
\end{example}

\begin{protos}
\protonoresult{limit-output}{ output-port n procedure}
\end{protos}
\noindent
\var{Procedure} is called on an output port.
Output written to that port is copied to \var{output-port} until \var{n}
 characters have been written, at which point \code{limit-output} returns.
If \var{procedure} returns before writing \var{n} characters, then
 \code{limit-output} also returns at that time, regardless of how many
 characters have been written.

\begin{protos}
\proto{make-tracking-input-port}{ input-port}{input-port}
\proto{make-tracking-output-port}{ output-port}{output-port}
\proto{current-row}{ port}{integer or {\tt \#f}}
\proto{current-column}{ port}{integer or {\tt \#f}}
\protonoresult{fresh-line}{ output-port}
\end{protos}
\noindent \code{Make-tracking-input-port} and \code{make-tracking-output-port}
 return ports that keep track of the current row and column and
 are otherwise identical to their arguments.
Closing a tracking port does not close the underlying port.
\code{Current-row} and \code{current-column} return
  \var{port}'s current read or write location.
They return \code{\#f} if \var{port} does not keep track of its location.
\code{Fresh-line} writes a newline character to \var{output-port} if
 \code{(current-row \cvar{port})} is not 0.

\begin{example}
(define p (open-output-port "/tmp/temp"))
(list (current-row p) (current-column p))
    \evalsto '(0 0)
(display "012" p)
(list (current-row p) (current-column p))
    \evalsto '(0 3)
(fresh-line p)
(list (current-row p) (current-column p))
    \evalsto '(1 0)
(fresh-line p)
(list (current-row p) (current-column p))
    \evalsto '(1 0)
\end{example}

\section{Fluid bindings}

These procedures implement dynamic binding and are in structure \code{fluids}.
A \cvar{fluid} is a cell whose value can be bound dynamically.
Each fluid has a top-level value that is used when the fluid
 is unbound in the current dynamic environment.

\begin{protos}
\proto{make-fluid}{ value}{fluid}
\proto{fluid}{ fluid}{value}
\proto{let-fluid}{ fluid value thunk}{value(s)}
\proto{let-fluids}{ fluid$_0$ value$_0$  fluid$_1$ value$_1$ \ldots thunk}{value(s)}
\end{protos}
\noindent
\code{Make-fluid} returns a new fluid with \cvar{value} as its initial
 top-level value.
\code{Fluid} returns \code{fluid}'s current value.
\code{Let-fluid} calls \code{thunk}, with \cvar{fluid} bound to \cvar{value}
 until \code{thunk} returns.
Using a continuation to throw out of the call to \code{thunk} causes
 \cvar{fluid} to revert to its original value, while throwing back
 in causes \cvar{fluid} to be rebound to \cvar{value}.
\code{Let-fluid} returns the value(s) returned by \cvar{thunk}.
\code{Let-fluids} is identical to \code{let-fluid} except that it binds
 an arbitrary number of fluids to new values.

\begin{example}
(let* ((f (make-fluid 'a))
       (v0 (fluid f))
       (v1 (let-fluid f 'b
             (lambda ()
               (fluid f))))
       (v2 (fluid f)))
  (list v0 v1 v2))
  \evalsto '(a b a)
\end{example}

\begin{example}
(let ((f (make-fluid 'a))
      (path '())
      (c \#f))
  (let ((add (lambda ()
               (set! path (cons (fluid f) path)))))
    (add)
    (let-fluid f 'b
      (lambda ()
        (call-with-current-continuation
          (lambda (c0)
            (set! c c0)))
        (add)))
    (add)
    (if (< (length path) 5)
        (c)
        (reverse path))))
  \evalsto '(a b a b a)
\end{example}

\section{OS strings}
\label{os-strings}

\index{OS string}
On common operating systems such as Unix and Windows, various
parameters to OS functionality---such as file names, user names,
command-line arguments etc.---appear as text in most contexts, but are
really byte sequences: On Unix, the byte sequence may be interpreted
as text through some locale-determined encoding.  On Windows, such
parameters are typically represented as sequences of UTF-16 code
units.  In both cases, not every such byte sequence has a string
equivalent: On Unix, a byte sequence encoding a file name using
Latin-1 often cannot be decoded using UTF-8.  On Windows, unpaired
UTF-16 surrogates are admissible in encodings, and no lossless text
decoding for them exists.

For representing such string-like parameters, Scheme~48 uses an
abstraction called \textit{OS strings}.  An OS string is created from
either a string or a NUL-terminated byte sequence stored in a byte
vector, and has an associated text codec (see
section~\ref{text-codecs}) that is able to convert from one
representation to the other.  The exact meaning of a NUL-terminated
byte sequence is dependent on this text codec.  However, only codecs
for encodings that are a conservative extension of ASCII (such as
ASCII itself, Latin-1, or UTF-8) should be used here, to allow a
minimal set of portable file names.  (The Windows port uses a special
synthetic encoding called UTF-8of16 compatible with UTF-8 but capable
of encoding even invalid UTF-16 internally, but uses the UTF-8 codec
at the Scheme level.)

Most procedures accepting OS strings also accept strings or byte
vectors, which are then used to construct a OS string.  In the headers
of the specifications of these procedures, such arguments occur as
\var{os-string-thing}.\index{os-string-thing@\var{os-string-thing}}
The standard Scheme procedures such as \code{open-input-file} that
take file names all accept \var{os-string-thing} arguments.  OS
strings are in the \code{os-strings} structure.

\begin{protos}
\proto{string->os-string}{ string}{os-string}
\proto{byte-vector->os-string}{ byte-vector}{os-string}
\proto{x->os-string}{ os-string-thing}{os-string}
\end{protos}
\noindent
These procedures create an OS string from a string, a byte-vector
(whose last value should be 0), and an \var{os-string-thing} argument,
respectively, always using the standard OS-string text codec (see
below).

\begin{protos}
\proto{os-string->byte-vector}{ os-string}{byte-vector}
\proto{os-string->string}{ os-string}{string}
\end{protos}
\noindent
These procedures yield the contents of an OS string.  For an OS string
created from a string, \code{os-string->string} will return a string
with the same contents; for an OS string created from a byte vector,
\code{os-string->byte-vector} will return a byte vector with the same
contents.  For the other cases, data loss as determined by the text
codec is possible.

\begin{protos}
\proto{current-os-string-text-codec}{}{text-codec}
\proto{call-with-os-string-text-codec}{ text-codec thunk}{ value(s)}
\end{protos}
\noindent
The \code{current-os-string-text-codec} returns the current text codec
used for creating new OS strings.  The initial default is determined
by the operating system.  (On Unix, this is the text codec determined
by the locale.  On Windows, this is UTF-8.)  The
\code{call-with-os-string-text-codec} procedure dynamically binds the
current text codec to \var{text-codec} during the invocation of
\var{thunk}.

\section{Shell commands}

Structure \code{c-system-function} provides access to the C \code{system()}
 function.

\begin{protos}
\proto{have-system?}{}{boolean}
\proto{system}{ os-string-thing}{integer}
\end{protos}
\noindent
\code{Have-system?} returns true if the underlying C implementation
 has a command processor.
\code{(System \cvar{string})} passes \cvar{string} to the C
 \code{system()} function and returns the result.

\begin{example}
(begin
  (system "echo foo > test-file")
  (call-with-input-file "test-file" read))
\evalsto 'foo
\end{example}

\section{Sockets}
% Richard says: add UDP documentation.

Structure \code{sockets} provides access to TCP/IP sockets for interprocess
 and network communication.

\begin{protos}
\proto{open-socket}{}{socket}
\proto{open-socket}{ port-number}{socket}
\proto{socket-port-number}{ socket}{integer}
\protonoresult{close-socket}{ socket}
\proto{socket-accept}{ socket}{input-port output-port}
\proto{get-host-name}{}{string}
\end{protos}
\noindent
\code{Open-socket} creates a new socket.
If no \cvar{port-number} is supplied the system picks one at random.
\code{Socket-port-number} returns a socket's port number.
\code{Close-socket} closes a socket, preventing any further connections.
\code{Socket-accept} accepts a single connection on \cvar{socket}, returning
 an input port and an output port for communicating with the client.
If no client is waiting \code{socket-accept} blocks until one appears.
\code{Get-host-name} returns the network name of the machine.

\begin{protos}
\proto{socket-client}{ host-name port-number}{input-port output-port}
\end{protos}
\noindent
\code{Socket-client} connects to the server at \cvar{port-number} on
 the machine named \cvar{host-name}.
\code{Socket-client} blocks until the server accepts the connection.

The following simple example shows a server and client for a centralized UID
 service.
\begin{example}
(define (id-server)
  (let ((socket (open-socket)))
    (display "Waiting on port ")
    (display (socket-port-number socket))
    (newline)
    (let loop ((next-id 0))
      (call-with-values
        (lambda ()
          (socket-accept socket))
        (lambda (in out)
          (display next-id out)
          (close-input-port in)
          (close-output-port out)
          (loop (+ next-id 1)))))))
         
(define (get-id machine port-number)
  (call-with-values
    (lambda ()
      (socket-client machine port-number))
    (lambda (in out)
      (let ((id (read in)))
        (close-input-port in)
        (close-output-port out)
        id))))
\end{example}

\section{Macros for writing loops}
% JAR says: origin? history?

\code{Iterate} and \code{reduce} are extensions of named-\code{let} for
 writing loops that walk down one or more sequences,
 such as the elements of a list or vector, the
 characters read from a port, or an arithmetic series.
Additional sequences can be defined by the user.
\code{Iterate} and \code{reduce} are in structure \code{reduce}.

\subsection{{\tt Iterate}}

The syntax of \code{iterate} is:
\begin{example}
  (iterate \cvar{loop-name}
           ((\cvar{sequence-type} \cvar{element-variable} \cvar{sequence-data} \ldots)
            \ldots)
           ((\cvar{state-variable} \cvar{initial-value})
            \ldots)
    \cvar{body-expression}
    [\cvar{final-expression}])
\end{example}

\code{Iterate} steps the \cvar{element-variable}s in parallel through the
 sequences, while each \cvar{state-variable} has the corresponding
 \cvar{initial-value} for the first iteration and have later values
 supplied by \cvar{body-expression}. 
If any sequence has reached its limit the value of the \code{iterate}
 expression is
 the value of \cvar{final-expression}, if present, or the current values of
 the \cvar{state-variable}s, returned as multiple values.
If no sequence has reached
 its limit, \cvar{body-expression} is evaluated and either calls \cvar{loop-name} with
 new values for the \cvar{state-variable}s, or returns some other value(s).

The \cvar{loop-name} and the \cvar{state-variable}s and \cvar{initial-value}s behave
exactly as in named-\code{let}.  The named-\code{let} expression
\begin{example}
  (let loop-name ((state-variable initial-value) ...)
    body ...)
\end{example}
is equivalent to an \code{iterate} expression with no sequences
 (and with an explicit
 \code{let} wrapped around the body expressions to take care of any
 internal \code{define}s):
\begin{example}
  (iterate loop-name
           ()
           ((state-variable initial-value) ...)
    (let () body ...))
\end{example}

The \cvar{sequence-type}s are keywords (they are actually macros of a particular
 form; it is easy to add additional types of sequences).
Examples are \code{list*} which walks down the elements of a list and
 \code{vector*} which does the same for vectors.
For each iteration, each \cvar{element-variable} is bound to the next
 element of the sequence.
The \cvar{sequence-data} gives the actual list or vector or whatever.

If there is a \cvar{final-expression}, it is evaluated when the end of one or more
 sequences is reached.
If the \cvar{body-expression} does not call \cvar{loop-name} the
 \cvar{final-expression} is not evaluated.
The \cvar{state-variable}s are visible in
 \cvar{final-expression} but the \cvar{sequence-variable}s are not.  

The \cvar{body-expression} and the \cvar{final-expression} are in tail-position within
 the \code{iterate}.
Unlike named-\code{let}, the behavior of a non-tail-recursive call to
 \cvar{loop-name} is unspecified (because iterating down a sequence may involve side
 effects, such as reading characters from a port).

\subsection{{\tt Reduce}}

If an \code{iterate} expression is not meant to terminate before a sequence
 has reached its end,
 \cvar{body-expression} will always end with a tail call to \cvar{loop-name}.
\code{Reduce} is a macro that makes this common case explicit.
The syntax of \code{reduce} is
 the same as that of \code{iterate}, except that there is no \cvar{loop-name}.
The \cvar{body-expression} returns new values of the \cvar{state-variable}s
 instead of passing them to \cvar{loop-name}.
Thus \cvar{body-expression} must return as many values as there are state
 variables.
By special dispensation, if there are
 no state variables then \cvar{body-expression} may return any number of values,
 all of which are ignored.

The syntax of \code{reduce} is:
\begin{example}
  (reduce ((\cvar{sequence-type} \cvar{element-variable} \cvar{sequence-data} \ldots)
            \ldots)
           ((\cvar{state-variable} \cvar{initial-value})
            \ldots)
    \cvar{body-expression}
    [\cvar{final-expression}])
\end{example}

The value(s) returned by an instance of \code{reduce} is the value(s) returned
 by the \cvar{final-expression}, if present, or the current value(s) of the state
variables when the end of one or more sequences is reached.

A \code{reduce} expression can be rewritten as an equivalent \code{iterate}
 expression by adding a \cvar{loop-var} and a wrapper for the
 \cvar{body-expression} that calls the \cvar{loop-var}.
\begin{example}
(iterate loop
         ((\cvar{sequence-type} \cvar{element-variable} \cvar{sequence-data} \ldots)
          \ldots)
         ((\cvar{state-variable} \cvar{initial-value})
          \ldots)
  (call-with-values (lambda ()
                      \cvar{body-expression})
                    loop)
  [\cvar{final-expression}])
\end{example}

\subsection{Sequence types}

The predefined sequence types are:
\begin{protos}
\syntaxprotonoresultnoindex{list*}{ \cvar{elt-var} \cvar{list}}
\syntaxprotonoresultnoindex{vector*}{ \cvar{elt-var} \cvar{vector}}
\syntaxprotonoresultnoindex{string*}{ \cvar{elt-var} \cvar{string}}
\syntaxprotonoresultnoindex{count*}
 { \cvar{elt-var} \cvar{start} [\cvar{end} [\cvar{step}]]}
\syntaxprotonoresultnoindex{input*}
 { \cvar{elt-var} \cvar{input-port} \cvar{read-procedure}}
\syntaxprotonoresultnoindex{stream*}
 { \cvar{elt-var} \cvar{procedure} \cvar{initial-data}}
\end{protos}

For lists, vectors, and strings the element variable is bound to the
 successive elements of the list or vector, or the characters in the
 string.

For \code{count*} the element variable is bound to the elements of the sequence
\begin{example}
 \cvar{start}, \cvar{start} + \cvar{step}, \cvar{start} + 2\cvar{step}, \ldots, \cvar{end}
\end{example}
 inclusive of \cvar{start} and exclusive of \cvar{end}.
The default \cvar{step} is 1.
The sequence does not terminate if no \cvar{end} is given or if there
 is no $N > 0$ such that \cvar{end} = \cvar{start} + N\cvar{step}
 (\code{=} is used to test for termination).
For example, \code{(count* i 0 -1)} doesn't terminate
 because it begins past the \cvar{end} value and \code{(count* i 0 1 2)} doesn't
 terminate because it skips over the \cvar{end} value.

For \code{input*} the elements are the results of successive applications
 of \cvar{read-procedure} to \cvar{input-port}.
The sequence ends when \cvar{read-procedure} returns an end-of-file object.

For a stream, the \cvar{procedure} takes the current data value as an argument
 and returns two values, the next value of the sequence and a new data value.
If the new data is \code{\#f} then the previous element was the last
 one.  For example,
\begin{example}
  (list* elt my-list)
\end{example}
 is the same as
\begin{example}
  (stream* elt list->stream my-list)
\end{example}
 where \code{list->stream} is
\begin{example}
  (lambda (list)
    (if (null? list)
        (values 'ignored \#f)
        (values (car list) (cdr list))))
\end{example}

\subsection{Synchronous sequences}

When using the sequence types described above, a loop terminates when any of
its sequences reaches its end.  To help detect bugs it is useful to have
sequence types that check to see if two or more sequences end on the same
iteration.  For this purpose there is second set of sequence types called
synchronous sequences.  These are identical to the ones listed above except
that they cause an error to be signalled if a loop is terminated by a
synchronous sequence and some other synchronous sequence did not reach its
end on the same iteration.

Sequences are checked for termination in order, from left to right, and
if a loop is terminated by a non-synchronous sequence no further checking
is done.

The synchronous sequences are:

\begin{protos}
\syntaxprotonoresultnoindex{list\%}{ \cvar{elt-var} \cvar{list}}
\syntaxprotonoresultnoindex{vector\%}{ \cvar{elt-var} \cvar{vector}}
\syntaxprotonoresultnoindex{string\%}{ \cvar{elt-var} \cvar{string}}
\syntaxprotonoresultnoindex{count\%}
 { \cvar{elt-var} \cvar{start} \cvar{end} [\cvar{step}]}
\syntaxprotonoresultnoindex{input\%}
 { \cvar{elt-var} \cvar{input-port} \cvar{read-procedure}}
\syntaxprotonoresultnoindex{stream\%}
 { \cvar{elt-var} \cvar{procedure} \cvar{initial-data}}
\end{protos}

Note that the synchronous \code{count\%} must have an \cvar{end}, unlike the
 nonsynchronous \code{count\%}.

\subsection{Examples}

\noindent
Gathering the indexes of list elements that answer true to some
predicate.
\begin{example}
(lambda (my-list predicate)
  (reduce ((list* elt my-list)
           (count* i 0))
          ((hits '()))
    (if (predicate elt)
        (cons i hits)
        hits)
    (reverse hits))
\end{example}

\noindent
Looking for the index of an element of a list.
\begin{example}
(lambda (my-list predicate)
  (iterate loop
           ((list* elt my-list)
            (count* i 0))
           ()                                ; no state
    (if (predicate elt)
        i
        (loop))))
\end{example}

\noindent
Reading one line.
\begin{example}
(define (read-line port)
  (iterate loop
           ((input* c port read-char))
           ((chars '()))
    (if (char=? c \#\verb2\2newline)
        (list->string (reverse chars))
        (loop (cons c chars)))
    (if (null? chars)
        (eof-object)
        ; no newline at end of file
        (list->string (reverse chars)))))
\end{example}

\noindent
Counting the lines in a file.  We can't use \code{count*} because we
need the value of the count after the loop has finished.
\begin{example}
(define (line-count name)
  (call-with-input-file name
    (lambda (in)
      (reduce ((input* l in read-line))
              ((i 0))
        (+ i 1)))))
\end{example}

\subsection{Defining sequence types}

The sequence types are object-oriented macros similar to enumerations.
A non-synchronous sequence macro needs to supply three values:
 \code{\#f} to indicate that it isn't synchronous, a list of state variables
 and their initializers, and the code for one iteration.
The first
 two methods are CPS'ed: they take another macro and argument to
 which to pass their result.
The \code{synchronized?} method gets no additional arguments.
The \code{state-vars} method is passed a list of names which
 will be bound to the arguments to the sequence.
The final method, for the step, is passed the list of names bound to
 the arguments and the list of state variables.
In addition there is
 a variable to be bound to the next element of the sequence, the
 body expression for the loop, and an expression for terminating the
 loop.

The definition of \code{list*} is
\begin{example}
(define-syntax list*
  (syntax-rules (synchronized? state-vars step)
    ((list* synchronized? (next more))
     (next \#f more))
    ((list* state-vars (start-list) (next more))
     (next ((list-var start-list)) more))
    ((list* step (start-list) (list-var)
            value-var loop-body final-exp)
     (if (null? list-var)
         final-exp
         (let ((value-var (car list-var))
               (list-var (cdr list-var)))
           loop-body)))))
\end{example}

Synchronized sequences are the same, except that they need to
 provide a termination test to be used when some other synchronized
 method terminates the loop.
\begin{example}
(define-syntax list\%
  (syntax-rules (sync done)
    ((list\% sync (next more))
     (next \#t more))
    ((list\% done (start-list) (list-var))
     (null? list-var))
    ((list\% stuff ...)
     (list* stuff ...))))
\end{example}

\subsection{Expanded code}

The expansion of 
\begin{example}
  (reduce ((list* x '(1 2 3)))
          ((r '()))
    (cons x r))
\end{example}
is
\begin{example}
  (let ((final (lambda (r) (values r)))
        (list '(1 2 3))
        (r '()))
    (let loop ((list list) (r r))
      (if (null? list)
          (final r)
          (let ((x (car list))
                (list (cdr list)))
            (let ((continue (lambda (r)
                              (loop list r))))
              (continue (cons x r)))))))
\end{example}

The only inefficiencies in this code are the \code{final} and \code{continue}
 procedures, both of which could be substituted in-line.
The macro expander could do the substitution for \code{continue} when there
 is no explicit proceed variable, as in this case, but not in general.

\section{Sorting lists and vectors}
\label{sort}

(This section, as the libraries it describes, was written mostly by
Olin Shivers for the draft of SRFI~32.)

%% Olin Shivers
%% First draft: 1998/10/19
%% Last update: 2002/7/21

%% [Todo: del-list-neighbor-dups!
%%        vector-copy -> subvector
%%        use srfi-23 for reporting errors
%%        use srfi-16 for n-aries?

The sort libraries in Scheme~48 include
%
\begin{itemize}
\item vector insert sort (stable)
\item vector heap sort
%    - Vector quick sort (with median-of-3 pivot picking)
\item vector merge sort (stable)
\item pure and destructive list merge sort (stable)
\item stable vector and list merge
\item miscellaneous sort-related procedures: vector and list merging, 
  sorted predicates, vector binary search, vector and list 
  delete-equal-neighbor procedures.
\item a general, non-algorithmic set of procedure names for general sorting
  and merging
\end{itemize}

\subsection{Design rules}

\paragraph{What vs. how}

There are two different interfaces: ``what'' (simple) and ``how'' (detailed).

\begin{description}
\item[Simple] you specify semantics: datatype (list or vector), 
  mutability, and stability.
  
\item[Detailed] you specify the actual algorithm (quick, heap,
  insert, merge). Different algorithms have different properties,
  both semantic and pragmatic, so these exports are necessary.
  
  It is necessarily the case that the specifications of these procedures
  make statements about execution ``pragmatics.'' For example, the sole
  distinction between heap sort and quick sort---both of which are
  provided by this library----is one of execution time, which is not a
  ``semantic'' distinction. Similar resource-use statements are made about
  ``iterative'' procedures, meaning that they can execute on input of
  arbitrary size in a constant number of stack frames.
\end{description}

\paragraph{Consistency across procedure signatures}

The two interfaces share common procedure signatures wherever
possible, to facilitate switching a given call from one procedure
to another.
        
\paragraph{Less-than parameter first, data parameter after}

These procedures uniformly observe the following parameter order:
the data to be sorted comes after the comparison procedure.
That is, we write

\begin{example}
  (sort \(<\) \var{list})
\end{example}

not

\begin{example}
  (sort \var{list} \(<\))
\end{example}
%

\paragraph{Ordering, comparison procedures and stability}

These routines take a $<$ comparison procedure, not a $\leq$ comparison
procedure, and they sort into increasing order. The difference between
a $<$ spec and a $\leq$ spec comes up in two places: 

\begin{itemize}
\item the definition of an ordered or sorted data set, and
\item the definition of a stable sorting algorithm.
%\item correctness of quicksort. NOTE "two" above
\end{itemize}
%

We say that a data set (a list or vector) is \textit{sorted} or
\textit{ordered} if it contains no adjacent pair of values $\ldots x,
y \ldots$ such that $y < x$.

In other words, scanning across the data never takes a ``downwards'' step.

If you use a $\leq$ procedure where these algorithms expect a $<$
procedure, you may not get the answers you expect. For example,
the \code{list-sorted?} procedure will return false if you pass it a $\leq$ comparison
procedure and an ordered list containing adjacent equal elements.

A ``stable'' sort is one that preserves the pre-existing order of equal
elements. Suppose, for example, that we sort a list of numbers by 
comparing their absolute values, i.e., using comparison procedure
%
\begin{verbatim}
(lambda (x y) (< (abs x) (abs y)))
\end{verbatim}
%
If we sort a list that contains both 3 and -3: \[\ldots 3, \ldots, -3 \ldots\]
then a stable sort is an algorithm that will not swap the order
of these two elements, that is, the answer is guaranteed to to look like
\[\ldots 3, -3 \ldots\]
not
\[\ldots -3, 3 \ldots\]
Choosing $<$ for the comparison procedure instead of $\leq$ affects
how stability is coded. Given an adjacent pair $x, y$, \code{(<
  $y$ $x$)} means ``$x$ should be moved in front of $x$''---otherwise,
leave things as they are. So using a $\leq$ procedure where a $<$
procedure is expected will \emph{invert} stability.

This is due to the definition of equality, given a $<$ comparator:
\begin{verbatim}
    (and (not (< x y))
         (not (< y x)))
\end{verbatim}
The definition is rather different, given a $\leq$ comparator:
\begin{verbatim}
    (and (<= x y)
         (<= y x))
\end{verbatim}
%
A ``stable'' merge is one that reliably favors one of its data sets
when equal items appear in both data sets. \emph{All merge operations in
this library are stable}, breaking ties between data sets in favor
of the first data set---elements of the first list come before equal 
elements in the second list.

So, if we are merging two lists of numbers ordered by absolute value,
the stable merge operation \code{list-merge}
\begin{verbatim}
    (list-merge (lambda (x y) (< (abs x) (abs y)))
                '(0 -2 4 8 -10) '(-1 3 -4 7))
\end{verbatim}
reliably places the 4 of the first list before the equal-comparing -4
  of the second list:
\begin{verbatim}
    (0 -1 -2 4 -4 7 8 -10)
\end{verbatim}
%
  Some sort algorithms will \emph{not work correctly} if given a $\leq$
  when they expect a $<$ comparison (or vice-versa).

%% For example,
%%   violating quicksort's spec may cause it to produce wrong answers,
%%   diverge, raise an error, or do some fourth thing. To see why,
%%   consider the left-scan part of the standard quicksort partition
%%   step: (let ((i (let scan ((i i)) (if (elt< (vector-ref v i) pivot)
%%   (scan (+ i 1)) i)))) ...)  Consider applying this loop to a vector
%%   of all zeroes (hence, PIVOT, as well, is zero), but erroneously
%%   using \verb|<=| for the ELT< procedure. The loop will scan right off
%%   the end of the vector, producing a vector-index error.  The
%%   guarantee that the scan loop will terminate before running off the
%%   end of the vector depends critically upon ELT< performing as a true,
%%   irreflexive $<$ relation. Running off the end of the vector is only
%%   one of a variety of possibly ways to lose---other, variant
%%   implementations of quicksort can, instead, loop forever on some data
%%   sets if ELT< is a $\leq$ predicate.

In short, if your comparison procedure $f$ answers true to \code{($f$ x x)}, then 
\begin{itemize}
\item using a stable sorting or merging algorithm will not give you a
  stable sort or merge, 
\item \code{list-sorted?} may surprise you.
% \item quicksort may fail in a variety of possible ways.
\end{itemize}
Note that  you can synthesize a $<$ procedure from a $\leq$ procedure with
\begin{verbatim}
    (lambda (x y) (not (<= y x)))
\end{verbatim}
if need be. 

Precise definitions give sharp edges to tools, but require care in use. 
``Measure twice, cut once.''

%% I have adopted the choice of $<$ from Common Lisp. One would assume
%% the definers of Common Lisp had a good reason for adopting $<$ instead
%% of $\leq$, but canvassing several of the principal actors in the
%% definition process has turned up no better reason than ``an arbitrary
%% but consistent choice.'' At minimum, then, these libraries extend the
%% coverage of that consistent choice.

\paragraph{All vector operations accept optional subrange parameters}

The vector operations specified below all take optional
\code{start}/\code{end} arguments indicating a selected subrange
of a vector's elements. If a \code{start} parameter or
\code{start}/\code{end} parameter pair is given to such a
procedure, they must be exact, non-negative integers, such that
%
\[
    0 \leq \var{start} \leq \var{end} \leq \code{(vector-length \var{vector})}
\]
%
where \var{vector} is the related vector parameter. If not specified,
they default to 0 and the length of the vector, respectively. They are
interpreted to select the range $[\var{start},\var{end})$, that
is, all elements from index \var{start} (inclusive) up to, but not
including, index \var{end}.

\paragraph{Required vs.\ allowed side-effects}

\code{List-sort!} and \code{List-stable-sort!} are allowed, but
not required, to alter their arguments' cons cells to construct the
result list. This is consistent with the what-not-how character of the
group of procedures to which they belong (the \code{sorting} structure).

The \code{list-delete-neighbor-dups!}, \code{list-merge!} and
\code{list-merge-sort!} procedures, on the other hand, provide
specific algorithms, and, as such, explicitly commit to the use of
side-effects on their input lists in order to guarantee their key
algorithmic properties (e.g., linear-time operation).

\subsection{Procedure specification}

\begin{center}
\begin{tabular}{ll}
Structure name & Functionality\\\hline
\code{sorting} & General sorting for lists and vectors\\
\code{sorted} & Sorted predicates for lists and vectors\\
\code{list-merge-sort}& List merge sort\\
\code{vector-merge-sort} & Vector merge sort\\
\code{vector-heap-sort} & Vector heap sort\\
%\code{vector-quick-sort} & Vector quick sort\\
\code{vector-insert-sort} & Vector insertion sort\\
\code{delete-neighbor-duplicates} & List and vector delete neighbor duplicates\\
\code{binary-searches} & Vector binary search
 \end{tabular}
\end{center}
%
Note that there is no ``list insert sort'' package, as you might as well always
use list merge sort. The reference implementation's destructive list merge
sort will do fewer \code{set-cdr!}s than a destructive insert sort.

\paragraph{Procedure naming and functionality}

Almost all of the procedures described below are variants of two basic
operations: sorting and merging. These procedures are consistently named
by composing a set of basic lexemes to indicate what they do.
\begin{center}

\begin{tabular}{lp{0.8\textwidth}}
Lexeme & Meaning\\\hline
\code{sort}&    The procedure sorts its input data set by some $<$ comparison procedure.
\\
\code{merge}&   The procedure merges two ordered data sets into a single ordered
          result.
\\
\code{stable} & This lexeme indicates that the sort is a stable one.
\\
\code{vector}& The procedure operates upon vectors.
\\
\code{list} &   The procedure operates upon lists.
\\
\code{!}      &  Procedures that end in \code{!} are allowed, and sometimes required, 
          to reuse their input storage to construct their answer.
\end{tabular}
\end{center}

\paragraph{Types of parameters and return values}

In the procedures specified below,
%
\begin{itemize}
\item A \code{<} or \code{=} parameter is a procedure accepting
  two arguments taken from the specified procedure's data set(s), and
  returning a boolean;
\item \code{Start} and \code{end} parameters are exact, non-negative integers that 
  serve as vector indices selecting a subrange of some associated vector.
  When specified, they must satisfy the relation
  \[
    0 \leq \var{start} \leq \var{end} \leq \code{(vector-length \var{vector})}
  \]
  where \var{vector} is the associated vector.
\end{itemize}
%
Passing values to procedures with these parameters that do not satisfy
these types is an error.

If a procedure is said to return ``unspecified,'' this means that
nothing at all is said about what the procedure returns, not even the
number of return values. Such a procedure is not even required to be
consistent from call to call in the nature or number of its return
values. It is simply required to return a value (or values) that may
be passed to a command continuation, e.g.  as the value of an
expression appearing as a non-terminal subform of a \code{begin}
expression. Note that in R$^5$RS, this restricts such a procedure to
returning a single value; non-R$^5$RS systems may not even provide this
restriction.

\subsubsection{\code{sorting}---general sorting package}

This library provides basic sorting and merging functionality suitable for
general programming. The procedures are named by their semantic properties,
i.e., what they do to the data (sort, stable sort, merge, and so forth).

\begin{protos}
  \proto{list-sorted?}{ $<$ list}{boolean} 
  \proto{list-merge}{ $<$ list$_1$ list$_2$}{list}
  \protonoindex{list-merge!}{ $<$ list$_1$ list$_2$}{list}\mainschindex{list-merge"!}
  \proto{list-sort}{ $<$ lis}{list}
  \protonoindex{list-sort!}{ $<$ lis}{list}\mainschindex{list-sort"!}
  \proto{list-stable-sort}{  $<$ list}{list}
  \protonoindex{list-stable-sort!}{ $<$ list}{list}\mainschindex{list-stable-sort"!}
  \proto{list-delete-neighbor-dups}{  $=$ list}{list}
  \proto{vector-sorted?}{ $<$ v [start [end]]}{boolean}
  \proto{vector-merge}{ $<$ v$_1$ v$_2$ [start$1$ [end$1$ [start$2$ [end$2$]]]]}{vector}
  \protonoresultnoindex{vector-merge!}{ $<$ v v$_1$ v$_2$ [start [start$1$ [end$1$ [start$2$ [end$2$]]]]]}\mainschindex{vector-merge"!}
  \proto{vector-sort}{ $<$ v [start [end]]}{vector}
  \protonoresultnoindex{vector-sort!}{ $<$ v [start [end]]}\mainschindex{vector-sort"!}
  \proto{vector-stable-sort}{ $<$ v [start [end]]}{vector}
  \protonoresultnoindex{vector-stable-sort!}{ $<$ v [start [end]]}\mainschindex{vector-stable-sort"!}
  \proto{vector-delete-neighbor-dups}{ $=$ v [start [end]]}{vector}
\end{protos}

\begin{center}
\begin{tabular}{ll}
Procedure &Suggested algorithm
\\\hline
\code{list-sort} & vector heap or quick\\
\code{list-sort!} & list merge sort\\
\code{list-stable-sort} & vector merge sort\\
\code{list-stable-sort!} & list merge sort\\
\code{vector-sort} & heap or quick sort\\
\code{vector-sort!} or quick sort\\
\code{vector-stable-sort} & vector merge sort\\
\code{vector-stable-sort!} merge sort
\end{tabular}
\end{center}
%
\code{List-Sorted?} and \code{vector-sorted?} return true if their
input list or vector is in sorted order, as determined by their \var{$<$}
comparison parameter.

All four merge operations are stable: an element of the initial list
\var{list$_1$} or vector \var{vector$_1$} will come before an
equal-comparing element in the second list \var{list$_2$} or vector
\var{vector$_2$} in the result.

The procedures
%
\begin{itemize}
\item \code{list-merge}
\item \code{list-sort}
\item \code{list-stable-sort}
\item \code{list-delete-neighbor-dups}
\end{itemize}
%
do not alter their inputs and are allowed to return a value that shares 
a common tail with a list argument.

The procedure
\begin{itemize}
\item \code{list-sort!}
\item \code{list-stable-sort!}
\end{itemize}
%
are ``linear update'' operators---they are allowed, but not required, to
alter the cons cells of their arguments to produce their results. 

On the other hand, the \code{list-merge!} procedure 
make only a single, iterative, linear-time pass over its argument
list, using \code{set-cdr!}s to rearrange the cells of the list
into the final result ---it works ``in place.'' Hence, any cons cell
appearing in the result must have originally appeared in an input. The
intent of this iterative-algorithm commitment is to allow the
programmer to be sure that if, for example, \code{list-merge!} is asked to
merge two ten-million-element lists, the operation will complete
without performing some extremely (possibly twenty-million) deep
recursion.

The vector procedures
%
\begin{itemize}
\item \code{vector-sort}
\item \code{vector-stable-sort}
\item \code{vector-delete-neighbor-dups}
\end{itemize}
%
do not alter their inputs, but allocate a fresh vector for their result,
of length $\var{end} - \var{start}$. 

The vector procedures
%
\begin{itemize}
\item \code{vector-sort!}
\item \code{vector-stable-sort!}
\end{itemize}
%
sort their data in-place. (But note that \code{vector-stable-sort!}
may allocate temporary storage proportional to the size of the
input
%%---I am not aware of $O(n \log(n))$ stable vector-sorting
%%algorithms that run in constant space
.)

\code{Vector-merge} returns a vector of length $(\var{end$_1$}-\var{start$_1$}+(\var{end$_2$}-\var{start$_2$})$.
    
\code{Vector-merge!} writes its result into vector \var{v},
beginning at index \var{start}, for indices less than \(\var{end} =
\var{start} + (\var{end$_1$}-\var{start$_1$}) +
(\var{end$_2$}-\var{start$_2$})\). The target subvector
  $\var{v}[\var{start},\var{end})$ may not overlap either source
subvector $\var{vector$_1$}[\var{start$_1$},\var{end$_1$})$ $\var{vector$_2$}[\var{start$_2$},\var{end$_2$})$.

The \code{\ldots-delete-neighbor-dups-\ldots} procedures:
These procedures delete adjacent duplicate elements from a list or a
vector, using a given element-equality procedure. The first/leftmost
element of a run of equal elements is the one that survives. The list or
vector is not otherwise disordered.
    
These procedures are linear time---much faster than the $O(n^2)$ general
duplicate-element deletors that do not assume any ``bunching'' of elements
(such as the ones provided by SRFI~1). If you want to delete duplicate
elements from a large list or vector, you can sort the elements to bring
equal items together, then use one of these procedures, for a total time
of $O(n\log(n))$.
    
The comparison procedure \(=\) passed to these procedures is always
applied
%
\code{($=$ $x$ $y$)}
%
where $x$ comes before $y$ in the containing list or vector.

\begin{itemize}
\item \code{List-delete-neighbor-dups} does not alter its input list; its answer
  may share storage with the input list.
\item \code{Vector-delete-neighbor-dups} does not alter its input vector, but
  rather allocates a fresh vector to hold the result.
\end{itemize}
%
Examples:

\begin{example}
(list-delete-neighbor-dups = '(1 1 2 7 7 7 0 -2 -2))
  \(\Longrightarrow\) (1 2 7 0 -2)

(vector-delete-neighbor-dups = '\#(1 1 2 7 7 7 0 -2 -2))
  \(\Longrightarrow\) \#(1 2 7 0 -2)

(vector-delete-neighbor-dups = '\#(1 1 2 7 7 7 0 -2 -2) 3 7)
  \(\Longrightarrow\) \#(7 0 -2)
\end{example}
         
\subsubsection{Algorithm-specific sorting packages}

These packages provide more specific sorting functionality, that is,
specific committment to particular algorithms that have particular
pragmatic consequences (such as memory locality, asymptotic running time)
beyond their semantic behaviour (sorting, stable sorting, merging, etc.).
Programmers that need a particular algorithm can use one of these packages.

\paragraph{\code{sorted}---sorted predicates}
%
\begin{protos}
  \proto{list-sorted?}{ $<$ list}{boolean}
  \proto{vector-sorted?}{ $<$ vector}{boolean}
  \proto{vector-sorted?}{ $<$ vector start}{boolean}
  \proto{vector-sorted?}{ $<$ vector start end}{boolean}
\end{protos}

Return \code{\#f} iff there is an adjacent pair \(\ldots x, y \ldots\) in the input
list or vector such that $y < x$. The optional \var{start}/\var{end} range 
arguments restrict \code{vector-sorted?} to the indicated subvector.

\paragraph{\code{list-merge-sort}---list merge sort}
%
\begin{protos}
  \proto{list-merge-sort}{ $<$ list}{list}
  \protonoindex{list-merge-sort!}{ $<$ list}{list}\mainschindex{list-merge-sort"!}
  \proto{list-merge}{ list$_1$ $<$ list$_2$}{list}
  \protonoindex{list-merge!}{ list$_1$ $<$ list$_2$}{list}\mainschindex{list-merge"!}
\end{protos}
%
The sort procedures sort their data using a list merge sort, which is
stable. (The reference implementation is, additionally, a ``natural'' sort.
See below for the properties of this algorithm.)

The \code{!} procedures are destructive---they use \code{set-cdr!}s to
rearrange the cells of the lists into the proper order. As such, they
do not allocate any extra cons cells---they are ``in place'' sorts.
%% Additionally, \code{list-merge!} is iterative---it can operate on
%% arguments of arbitrary size with a constant number of stack frames.

The merge operations are stable: an element of \var{list$_1$} will
come before an equal-comparing element in \var{list$_2$} in the result
list.

\paragraph{\code{vector-merge-sort}---vector merge sort}

\begin{protos}
  \proto{vector-merge-sort}{ $<$ vector [start [end [temp]]]}{vector}
  \protonoresultnoindex{vector-merge-sort!}{ $<$ vector [start [end [temp]]]}\mainschindex{vector-merge-sort"!}
  \proto{vector-merge}{ $<$ vector$_1$ vector$_2$ [start$_1$ [end$_1$ [start$_2$ [end$_2$]]]]}{vector}
  \protonoresultnoindex{vector-merge!}{ $<$ vector vector$_1$ vector$_2$ [start [start$_1$ [end$_1$ [start$_2$ [end$_2$]]]]]}\mainschindex{vector-merge"!}
\end{protos}
%
The sort procedures sort their data using vector merge sort, which is
stable. (The reference implementation is, additionally, a ``natural'' sort.
See below for the properties of this algorithm.)

The optional \var{start}/\var{end} arguments provide for sorting of subranges, and
default to 0 and the length of the corresponding vector.
    
Merge-sorting a vector requires the allocation of a temporary
``scratch'' work vector for the duration of the sort. This scratch
vector can be passed in by the client as the optional \var{temp}
argument; if so, the supplied vector must be of size $\leq \var{end}$,
and will not be altered outside the range [start,end). If not
supplied, the sort routines allocate one themselves.

The merge operations are stable: an element of \var{vector$_1$} will
come before an equal-comparing element in \var{vector$_2$} in the
result vector.

\begin{itemize}
\item 
\code{Vector-merge-sort!} leaves its result in
\(\var{vector}[\var{start},\var{end})\).
\item   
\code{Vector-merge-sort} returns a vector of length
\(\var{end}-\var{start}\).
\item 
\code{Vector-merge} returns a vector of length
\((\var{end$_1$}-\var{start$_1$})+(\var{end$_2$}-\var{start$_2$})\).
\item 
\code{Vector-merge!} writes its result into \var{vector}, beginning
at index \var{start},
for indices less than \(\var{end} =\var{start} +
(\var{end$_1$}-\var{start$_1$}) + (\var{end$_2$}-\var{start$_2$})\).
The target subvector
\[\var{vector}[\var{start},\var{end})\]
may not overlap either source subvector
\[\var{vector$_1$}[\var{start$_1$},\var{end$_1$}), \textrm{ or }
\var{vector$_2$}[\var{start$_2$},\var{end$_2$}).\]
\end{itemize}

\paragraph{\code{vector-heap-sort}---vector heap sort}

\begin{protos}
\proto{vector-heap-sort}{ $<$ vector [start [end]]}{vector}
\protonoresultnoindex{vector-heap-sort!}{ $<$ vector [start [end]]}\mainschindex{vector-heap-sort"!}
\end{protos}
%
These procedures sort their data using heap sort, 
which is not a stable sorting algorithm.
    
\code{Vector-heap-sort} returns a vector of length \(\var{end}-\var{start}\). 
\code{Vector-heap-sort!} is in-place, leaving its result in
\(\var{vector}[\var{start},\var{end})\).

%% vector-quick-sort-lib - vector quick sort
%%     quick-sort   < v [start end] -> vector
%%     quick-sort!  < v [start end] -> unspecified
%%     quick-sort3! c v [start end] -> unspecified

%%     These procedures sort their data using quick sort, 
%%     which is not a stable sorting algorithm.
    
%%     QUICK-SORT returns a vector of length END-START. 
%%     QUICK-SORT! is in-place, leaving its result in V[start,end).

%%     QUICK-SORT3! is a variant of quick-sort that takes a three-way
%%     comparison procedure C. C compares a pair of elements and returns
%%     an exact integer whose sign indicates their relationship:
%%       (c x y) < 0   =>   x<y
%%       (c x y) = 0   =>   x=y
%%       (c x y) > 0   =>   x>y
%%     To help remember the relationship between the sign of the result and
%%     the relation, use the procedure - as the model for C: (- x y) < 0
%%     means that x < y; (- x y) > 0 means that x > y.

%%     The extra discrimination provided by the three-way comparison can
%%     provide significant speedups when sorting data sets with many duplicates,
%%     especially when the comparison procedure is relatively expensive (e.g.,
%%     comparing long strings).

%%     WARNING: Some sort algorithms, such as insertion sort or heap sort,
%%     can tolerate being passed a <= comparison procedure when they expect a < 
%%     procedure---insertion and merge sort may simply invert stability; and
%%     heap sort will run a bit slower, but otherwise produce a correct answer.

%%     Quicksort, however, is much more critically sensitive to the distinction
%%     between a < and a <= comparison. If QUICK-SORT or QUICK-SORT! expect a <
%%     comparison procedure, and are erroneously given a <= procedure, they may,
%%     depending on implementation, produce an unsorted result, go into an 
%%     infinite loop, cause a run-time error, occasionally produce a correct
%%     result, or do some fifth thing.

%%     Implementors may wish to write QUICKSORT3! so that it (a) tests the
%%     comparison procedure (by checking that (c v[start] v[start]) produces
%%     false), or (b) is tolerant of an erroneous <= procedure, or (c) both.
%%     Clients of this procedure, however, should not count on this.

\paragraph{\code{vector-insert-sort}---vector insertion sort}

\begin{protos}
\proto{vector-insert-sort}{ $<$ vector [start [end]]}{vector}
\protonoresultnoindex{vector-insert-sort!}{ $<$ vector [start [end]]}\mainschindex{vector-insert-sort"!}
\end{protos}
%
These procedures stably sort their data using insertion sort.
%
\begin{itemize}
\item \code{Vector-insert-sort} returns a vector of length \(\var{end}-\var{start}\).
\item \code{Vector-insert-sort!} is in-place, leaving its result in
  \(\var{vector}[\var{start},\var{end})\).
\end{itemize}

\paragraph{\code{delete-neighbor-duplicates}---list and vector
  delete neighbor duplicates}

\begin{protos}
  \proto{list-delete-neighbor-dups}{ $=$ list}{list}
  \protonoindex{list-delete-neighbor-dups!}{ $=$ list}{list}\mainschindex{list-delete-neighbor-dups"!}
  \proto{vector-delete-neighbor-dups}{ $=$ vector [start [end]]}{vector}
  \protonoindex{vector-delete-neighbor-dups!}{ $=$ vector [start [end]]}{end$'$}\mainschindex{vector-delete-neighbor-dups}
\end{protos}
%
These procedures delete adjacent duplicate elements from a list or
a vector, using a given element-equality procedure $=$. The first/leftmost
element of a run of equal elements is the one that survives. The list
or vector is not otherwise disordered.

These procedures are linear time---much faster than the $O(n^2)$ general
duplicate-element deletors that do not assume any ``bunching'' of elements
(such as the ones provided by SRFI~1). If you want to delete duplicate
elements from a large list or vector, you can sort the elements to bring
equal items together, then use one of these procedures, for a total time
of $O(n\log(n))$.
    
The comparison procedure = passed to these procedures is always
applied

\begin{example}
(\(=\) \(x\) \(y\))
\end{example}

where $x$ comes before $y$ in the containing list or vector.
%
\begin{itemize}
\item 
\code{List-delete-neighbor-dups} does not alter its input list; its
answer may share storage with the input list.
\item 
\code{Vector-delete-neighbor-dups} does not alter its input vector, but
rather allocates a fresh vector to hold the result.
\item
\code{List-delete-neighbor-dups!} is permitted, but not required, to
mutate its input list in order to construct its answer.
\item
\code{Vector-delete-neighbor-dups!} reuses its input vector to hold the
answer, packing its answer into the index range
\([\var{start},\var{end$'$})\), where
\var{end$'$} is the non-negative exact integer returned as its value. It
returns \var{end$'$} as its result. The vector is not altered outside the range
\([\var{start},\var{end$'$})\).
\end{itemize}
%
Examples:

\begin{example}
(list-delete-neighbor-dups = '(1 1 2 7 7 7 0 -2 -2))
  \(\Longrightarrow\) (1 2 7 0 -2)

(vector-delete-neighbor-dups = '\#(1 1 2 7 7 7 0 -2 -2))
  \(\Longrightarrow\) \#(1 2 7 0 -2)

(vector-delete-neighbor-dups = '\#(1 1 2 7 7 7 0 -2 -2) 3 7)
  \(\Longrightarrow\) \#(7 0 -2)

;; Result left in v[3,9):
(let ((v (vector 0 0 0 1 1 2 2 3 3 4 4 5 5 6 6)))
  (cons (vector-delete-neighbor-dups! = v 3)
        v))
   \(\Longrightarrow\) (9 . \#(0 0 0 1 2 3 4 5 6 4 4 5 5 6 6))
\end{example}

\paragraph{\code{binary-searches}---vector binary search}

\begin{protos}
\proto{vector-binary-search}{ $<$ elt-$>$key key vector [start [end]]}{integer or {\tt \#f}}
\proto{vector-binary-search3}{ compare-proc vector [start [end]]}{integer or {\tt \#f}}
\end{protos}

\code{vector-binary-search} searches \var{vector} in range
\([\var{start},\var{end})\) (which default to 0 and the length of
\var{vector}, respectively) for an element whose
associated key is equal to \var{key}. The procedure \var{elt-$>$key} is used to map
an element to its associated key. The elements of the vector are assumed
to be ordered by the $<$ relation on these keys. That is, 

\begin{example}
(vector-sorted? (lambda (x y) (\(<\) (\var{elt-\(>\)key} x) (\var{elt-\(>\)key} y)))
                \var{vector} \var{start} \var{end}) \(\Longrightarrow\) true
\end{example}

An element \var{e} of \var{vector} is a match for \var{key} if it's
neither less nor greater than the key:

\begin{example}
(and (not (\(<\) (\var{elt-\(>\)key} \var{e}) \var{key}))
     (not (\(<\) \var{key} (\var{elt-\(>\)key} \var{e}))))
\end{example}

If there is such an element, the procedure returns its index in the
vector as an exact integer. If there is no such element in the searched 
range, the procedure returns false.

\begin{example}
(vector-binary-search < car 4 '\#((1 . one) (3 . three)
                                 (4 . four) (25 . twenty-five)))
\(\Longrightarrow\) 2

(vector-binary-search < car 7 '\#((1 . one) (3 . three)
                                 (4 . four) (25 . twenty-five)))
\(\Longrightarrow\) \#f
\end{example}    

\code{Vector-binary-search3} is a variant that uses a three-way comparison
procedure \var{compare-proc}. \var{Compare-proc} compares its
parameter to the search key, and returns an
exact integer whose sign indicates its relationship to the search key.
%
\[
  \begin{array}{rclcrcl}
      (\var{compare-proc}~x) &<& 0& \Rightarrow&  x &<& \var{search-key}\\
      (\var{compare-proc}~x) &=& 0& \Rightarrow&  x &=& \var{search-key}\\
      (\var{compare-proc}~x) &>& 0& \Rightarrow&  x &>& \var{search-key}
  \end{array}
\]

\begin{example}
(vector-binary-search3 (lambda (elt) (- (car elt) 4))
                       '\#((1 . one) (3 . three)
                          (4 . four) (25 . twenty-five)))
\(\Longrightarrow\) 2
\end{example}

%% Rationale:
%% %
%% \begin{itemize}
%% \item Why isn't \code{vector-binary-search}'s \var{elt-$>$key}
%%   computation simply absorbed into the $<$ procedure? It is separated
%%   out because the $<$ procedure is
%%   applied twice inside the binary-search inner loop, once with the search
%%   key for the first argument and the element key for the second argument,
%%   and once, with the reverse argument order. This is not necessary for
%%   \code{vector-binary-search3}.
%% \item When a comparison operation is able to produce a three-way
%%   discrimination, the inner loop of the binary search can trim the number
%%   of per-iteration comparisons from an average of 1.5 to a guaranteed
%%   single comparison per iteration. This can be a significant savings when
%%   searching with an expensive comparison operation (e.g., one that
%%   uses string compare, sends email, references a database, or queries
%%   a network service such as a web server).
%% \item Failure is signaled by false (rather than, say, -1) so that searches
%%   can be used in conditional forms such as
%% \begin{verbatim}
%%         (or (vector-binary-search ...) ...)
%% \end{verbatim}
%%   or
%% \begin{verbatim}
%%         (cond ((vector-binary-search ...) => index-consumer)
%%               ...)
%% \end{verbatim}
%% \end{itemize}

\subsection{Algorithmic properties}

Different sort and merge algorithms have different properties.
Choose the algorithm that matches your needs:

\begin{description}
\item[Vector insert sort]
  Stable, but only suitable for small vectors---$O(n^2)$.
%% \item[Vector quick sort]
%%   Not stable. Is fast on average---$O(n\log(n))$---but has bad worst-case
%%   behaviour. Has good memory locality for big vectors (unlike heap sort). 
%%   A clever pivot-picking trick (median of three samples) helps avoid 
%%   worst-case behaviour, but pathological cases can still blow up.
\item[Vector heap sort]
  Not stable. Guaranteed fast---$O(n\log(n))$ \emph{worst} case. Poor
  locality on large vectors. A very reliable workhorse.
\item[Vector merge sort]
    Stable. Not in-place---requires a temporary buffer of equal size. 
    Fast---$O(n\log(n))$---and has good memory locality for large vectors.
    
    The implementation of vector merge sort provided by this
    implementation is, additionally, a ``natural'' sort, meaning that it
    exploits existing order in the input data, providing $O(n)$ best case.
\item[Destructive list merge sort]
    Stable, fast and in-place (i.e., allocates no new cons cells). ``Fast''
    means $O(n\log(n))$ worse-case, and substantially better if the data
    is already mostly ordered, all the way down to linear time for
    a completely-ordered input list (i.e., it is a ``natural'' sort).

    Note that sorting lists involves chasing pointers through memory, which
    can be a loser on modern machine architectures because of poor cache and
    page locality.
%%     Pointer \emph{writing}, which is what the
%%     \code{set-cdr!}s of a destructive list-sort algorithm do, is
%%     even worse, especially if your Scheme has a generational GC---the
%%     writes will thrash the write-barrier.
    Sorting vectors has inherently better locality.
    
    This implementation's destructive list merge and merge sort
    implementations are opportunistic---they avoid redundant
    \code{set-cdr!}s, and try to take long
    already-ordered runs of list structure as-is when doing the merges.
\item[Pure list merge sort]
  Stable and fast---$O(n\log(n))$ worst-case, and possibly $O(n)$,
  depending upon the input list (see discussion above).
\end{description}

\begin{center}
  \begin{tabular}{lllll}
Algorithm &    Stable? & Worst case &  Average case &  In-place\\
\hline
Vector insert & Yes&      $O(n^2)$ &     $O(n^2)$&        Yes\\
Vector quick  & No &      $O(n^2)$  &    $O(n\log(n))$&     Yes\\
Vector heap   & No &      $O(n\log(n))$&   $O(n\log(n))$&     Yes\\
Vector merge  & Yes&      $O(n\log(n))$&   $O(n\log(n))$&     No\\
List merge    & Yes&      $O(n\log(n))$&   $O(n\log(n))$&     Either
\end{tabular}
\end{center}

\section{Regular expressions}
\label{regexp-adt}

This section describes a functional interface for building regular
 expressions and matching them against strings.
The matching is done using the POSIX regular expression package.
Regular expressions are in the structure \code{regexps}.

A regular expression is either a character set, which matches any character
in the set, or a composite expression containing one or more subexpressions.
A regular expression can be matched against a string to determine success
or failure, and to determine the substrings matched by particular subexpressions.

\begin{protos}
\proto{regexp?}{ value}{boolean}
\end{protos}
\noindent
Returns \code{\#t} if \cvar{value} is a regular expression created
using the functional interface for regular expressions, and \code{\#f}
otherwise.

\subsection{Character sets}

Character sets may be defined using a list of characters and strings,
using a range or ranges of characters, or by using set operations on
existing character sets.

\begin{protos}
\proto{set}{ character-or-string \ldots}{char-set}
\proto{range}{ low-char high-char}{char-set}
\proto{ranges}{ low-char high-char \ldots}{char-set}
\proto{ascii-range}{ low-char high-char}{char-set}
\proto{ascii-ranges}{ low-char high-char \ldots}{char-set}
\end{protos}
\noindent
\code{Set} returns a set that contains the character arguments and the
characters in any string arguments.  \code{Range} returns a character
set that contain all characters between \cvar{low-char} and \cvar{high-char},
inclusive.  \code{Ranges} returns a set that contains all characters in
the given ranges.  \code{Range} and \code{ranges} use the ordering induced by
\code{char->integer}.  \code{Ascii-range} and \code{ascii-ranges} use the
 ASCII ordering.
It is an error for a \cvar{high-char} to be less than the preceding
 \cvar{low-char} in the appropriate ordering.

\begin{protos}
\proto{negate}{ char-set}{char-set}
\proto{intersection}{ char-set char-set}{char-set}
\proto{union}{ char-set char-set}{char-set}
\proto{subtract}{ char-set char-set}{char-set}
\end{protos}
\noindent
These perform the indicated operations on character sets.

The following character sets are predefined:
\begin{center}
\texonly\begin{longtable}{ll}\endtexonly
\htmlonly\begin{tabular}{ll}\endhtmlonly
 \code{lower-case} &   \code{(set "abcdefghijklmnopqrstuvwxyz")} \\
 \code{upper-case} &   \code{(set "ABCDEFGHIJKLMNOPQRSTUVWXYZ")} \\
 \code{alphabetic} &   \code{(union lower-case upper-case)} \\
 \code{numeric} &      \code{(set "0123456789")} \\
 \code{alphanumeric} & \code{(union alphabetic numeric)} \\
 \code{punctuation} &
 \code{(set "}\verb2!\"#$%&'()*+,-./:;<=>?@[\\]^_`{|}~2\code{")} \\
 \code{graphic} &      \code{(union alphanumeric punctuation)} \\
 \code{printing} &     \code{(union graphic (set \#}\verb2\2\code{space))} \\
 \code{control} &      \code{(negate printing)} \\
 \code{blank} &
      \code{(set \#}\verb2\2\code{space (ascii->char 9))} ; 9 is tab \\
 \code{whitespace} &
      \code{(union (set \#}\verb2\2\code{space) (ascii-range 9 13))} \\
 \code{hexdigit} &     \code{(set "0123456789abcdefABCDEF")} \\
\texonly\end{longtable}\endtexonly
\htmlonly\end{tabular}\endhtmlonly
\end{center}
\noindent The above are taken from the default locale in POSIX.
The characters in \code{whitespace} are \cvar{space}, \cvar{tab},
 \cvar{newline} (= \cvar{line feed}), \cvar{vertical tab}, \cvar{form feed}, and
 \cvar{carriage return}.

\subsection{Anchoring}

\begin{protos}
\proto{string-start}{}{reg-exp}
\proto{string-end}{}{reg-exp}
\end{protos}
\noindent
\code{String-start} returns a regular expression that matches the beginning
 of the string being matched against; {string-end} returns one that matches
 the end.

\subsection{Composite expressions}

\begin{protos}
\proto{sequence}{ reg-exp \ldots}{reg-exp}
\proto{one-of}{ reg-exp \ldots}{reg-exp}
\end{protos}
\noindent
\code{Sequence} matches the concatenation of its arguments, \code{one-of} matches
any one of its arguments.

\begin{protos}
\proto{text}{ string}{reg-exp}
\end{protos}
\noindent
\code{Text} returns a regular expression that matches the characters in
 \cvar{string}, in order.

\begin{protos}
\proto{repeat}{ reg-exp}{reg-exp}
\proto{repeat}{ count reg-exp}{reg-exp}
\proto{repeat}{ min max reg-exp}{reg-exp}
\end{protos}
\noindent
\code{Repeat} returns a regular expression that matches zero or more
occurences of its \cvar{reg-exp} argument.  With no count the result
will match any number of times (\cvar{reg-exp}*).  With a single
count the returned expression will match
 \cvar{reg-exp} exactly that number of times.
The final case will match from \cvar{min} to \cvar{max}
 repetitions, inclusive.
\cvar{Max} may be \code{\#f}, in which case there
 is no maximum number of matches.
\cvar{Count} and \cvar{min} should be exact, non-negative integers;
 \cvar{max} should either be an exact non-negative integer or \code{\#f}.

\subsection{Case sensitivity}

Regular expressions are normally case-sensitive.
\begin{protos}
\proto{ignore-case}{ reg-exp}{reg-exp}
\proto{use-case}{ reg-exp}{reg-exp}
\end{protos}
\noindent
The value returned by
 \code{ignore-case} is identical its argument except that case will be
 ignored when matching.
The value returned by \code{use-case} is protected
 from future applications of \code{ignore-case}.
The expressions returned
 by \code{use-case} and \code{ignore-case} are unaffected by later uses of the
 these procedures.
By way of example, the following matches \code{"ab"} but not \code{"aB"},
 \code{"Ab"}, or \code{"AB"}.
\begin{example}
\code{(text "ab")}
\end{example}
\noindent
while
\begin{example}
\code{(ignore-case (test "ab"))}
\end{example}
\noindent
matches \code{"ab"}, \code{"aB"},
 \code{"Ab"}, and \code{"AB"} and
\begin{example}
(ignore-case (sequence (text "a")
                       (use-case (text "b"))))
\end{example}
\noindent
matches \code{"ab"} and \code{"Ab"} but not \code{"aB"} or \code{"AB"}.

\subsection{Submatches and matching}

A subexpression within a larger expression can be marked as a submatch.
When an expression is matched against a string, the success or failure
of each submatch within that expression is reported, as well as the
location of the substring matched be each successful submatch.

\begin{protos}
\proto{submatch}{ key reg-exp}{reg-exp}
\proto{no-submatches}{ reg-exp}{reg-exp}
\end{protos}
\noindent
\code{Submatch} returns a regular expression that matches its argument and
 causes the result of matching its argument to be reported by the \code{match}
 procedure.
\cvar{Key} is used to indicate the result of this particular submatch 
 in the alist of successful submatches returned by \code{match}.
 Any value may be used as a \cvar{key}.
\code{No-submatches} returns an expression identical to its
 argument, except that all submatches have been elided.

\begin{protos}
\proto{any-match?}{ reg-exp string}{boolean}
\proto{exact-match?}{ reg-exp string}{boolean}
\proto{match}{ reg-exp string}{match or {\tt \#f}}
\proto{match-start}{ match}{index}
\proto{match-end}{ match}{index}
\proto{match-submatches}{ match}{alist}
\end{protos}
\noindent
\code{Any-match?} returns \code{\#t} if \cvar{string} matches \cvar{reg-exp} or
 contains a substring that does, and \code{\#f} otherwise.
\code{Exact-match?} returns \code{\#t} if \cvar{string} matches
 \cvar{reg-exp} and \code{\#f} otherwise.

\code{Match} returns \code{\#f} if \cvar{reg-exp} does not match \cvar{string}
 and a match record if it does match.
A match record contains three values: the beginning and end of the substring
 that matched
 the pattern and an a-list of submatch keys and corresponding match records
 for any submatches that also matched.
\code{Match-start} returns the index of
 the first character in the matching substring and \code{match-end} gives index
 of the first character after the matching substring.
\code{Match-submatches} returns an alist of submatch keys and match records.
Only the top match record returned by \code{match} has a submatch alist.

Matching occurs according to POSIX.
The match returned is the one with the lowest starting index in \cvar{string}.
If there is more than one such match, the longest is returned.
Within that match the longest possible submatches are returned.

All three matching procedures cache a compiled version of \cvar{reg-exp}.
Subsequent calls with the same \cvar{reg-exp} will be more efficient.

The C interface to the POSIX regular expression code uses ASCII \code{nul}
 as an end-of-string marker.
The matching procedures will ignore any characters following an
 embedded ASCII \code{nul}s in \cvar{string}.

\begin{example}
(define pattern (text "abc"))
(any-match? pattern "abc")         \evalsto #t
(any-match? pattern "abx")         \evalsto #f
(any-match? pattern "xxabcxx")     \evalsto #t

(exact-match? pattern "abc")       \evalsto #t
(exact-match? pattern "abx")       \evalsto #f
(exact-match? pattern "xxabcxx")   \evalsto #f

(match pattern "abc")              \evalsto (#\{match 0 3\})
(match pattern "abx")              \evalsto #f
(match pattern "xxabcxx")          \evalsto (#\{match 2 5\})

(let ((x (match (sequence (text "ab")
                          (submatch 'foo (text "cd"))
                          (text "ef"))
                "xxxabcdefxx")))
  (list x (match-submatches x)))
  \evalsto (#\{match 3 9\} ((foo . #\{match 5 7\}))

(match-submatches
  (match (sequence
           (set "a")
           (one-of (submatch 'foo (text "bc"))
                   (submatch 'bar (text "BC"))))
         "xxxaBCd"))
  \evalsto ((bar . #\{match 4 6\}))
\end{example}

\section{SRFIs}

`SRFI' stands for `Scheme Request For Implementation'.
An SRFI is a description of an extension to standard Scheme.
Draft and final SRFI documents, a FAQ, and other information about SRFIs
 can be found at
\urlhd{http://srfi.schemers.org}{the SRFI web site}{\code{http://srfi.schemers.org}}.

Scheme~48 includes implementations of the following (final) SRFIs:
\begin{itemize}
\item SRFI 1 -- List Library
\item SRFI 2 -- \code{and-let*}
\item SRFI 4 -- Homogeneous numeric vector datatypes (see note below)
\item SRFI 5 -- \code{let} with signatures and rest arguments
\item SRFI 6 -- Basic string ports
\item SRFI 7 -- Program configuration
\item SRFI 8 -- \code{receive}
\item SRFI 9 -- Defining record types
\item SRFI 11 -- Syntax for receiving multiple values 
\item SRFI 13 -- String Library
\item SRFI 14 -- Character-Set Library (see note below)
\item SRFI 16 -- Syntax for procedures of variable arity
\item SRFI 17 -- Generalized \code{set!}
\item SRFI 19 -- Time Data Types and Procedures
\item SRFI 22 -- Running Scheme Scripts on Unix
\item SRFI 23 -- Error reporting mechanism
\item SRFI 25 -- Multi-dimensional Array Primitives 
\item SRFI 26 -- Notation for Specializing Parameters without Currying
\item SRFI 27 -- Sources of Random Bits
\item SRFI 28 -- Basic Format Strings
\item SRFI 31 -- A special form \texttt{rec} for recursive evaluation
\item SRFI 34 -- Exception Handling for Programs
\item SRFI 35 -- Conditions
\item SRFI 36 -- I/O Conditions
\item SRFI 37 -- args-fold: a program argument processor
\item SRFI 40 -- A Library of Streams
\item SRFI 42 -- Eager Comprehensions
\item SRFI 43 -- Vector library
\item SRFI 45 -- Primitives for Expressing Iterative Lazy Algorithms
\item SRFI 60 -- Integers as Bits
\item SRFI 61 -- A more general cond clause
\item SRFI 62 -- S-expression comments
\item SRFI 63 -- Homogeneous and Heterogeneous Arrays
\item SRFI 66 -- Octet Vectors
\item SRFI 67 -- Compare Procedures
\item SRFI 74 -- Octet-Addressed Binary Blocks
\item SRFI 78 -- Lightweight testing
\end{itemize}
Documentation on these can be found at the web site mentioned above.

SRFI~4 specifies an external representation for homogeneous numeric
vectors that is incompatible with R$^5$RS.  The Scheme~48 version of
SRFI~4 does not support this external representation.

SRFI~14 includes the procedure \code{->char-set} which is not a standard
 Scheme identifier (in R$^5$RS the only required identifier starting
 with \code{-} is \code{-} itself).
In the Scheme~48 version of SRFI~14 we have renamed \code{->char-set}
 as \code{x->char-set}.

%% With the exception of SRFI 62 (which is supported by default), the
SRFI bindings can be accessed
either by opening the appropriate structure
 (the structure \code{srfi-}\cvar{n} contains SRFI \cvar{n})
 or by loading structure \code{srfi-7} and then using
 the \code{,load-srfi-7-program} command to load an SRFI 7-style program.
The syntax for the command is
\begin{example}
\code{,load-srfi-7-program \cvar{name} \cvar{filename}}
\end{example}
This creates a new structure and associated package, binds the structure
 to \cvar{name} in the configuration package, and then loads the program
 found in \cvar{filename} into the package.

As an example, if the file \code{test.scm} contains
\begin{example}
(program (code (define x 10)))
\end{example}
this program can be loaded as follows:
\begin{example}
> ,load-package srfi-7
> ,load-srfi-7-program test test.scm
[test]
> ,in test
test> x
10
test> 
\end{example}

%%% Local Variables: 
%%% mode: latex
%%% TeX-master: "manual"
%%% End: 
