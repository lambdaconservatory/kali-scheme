% Scheme 48 documentation

\documentclass[twoside]{report}
\usepackage{hyperlatex}
\T\usepackage{longtable}
\T\usepackage{myindex}
\W\newcommand{\mainschindex}[1]{\cindex[#1]{{\tt #1}}}
\newcommand{\mainindex}[1]{\cindex[1]{{\rm #1}}}
\newcommand{\schindex}[1]{}
\newcommand{\sharpindex}[1]{}

\newcommand{\codemainindex}[1]{{\mainschindex{#1}\code{#1}}}
\newcommand{\codeindex}[1]{{\mainschindex{#1}\code{#1}}}


\makeindex

\W%% -*-latex-*-
%% Navigation menus for the Scheme 48 report.

\newcommand{\previousnextmenu}{\EmptyP{\HlxBackUrl\HlxForwUrl}{%
    \EmptyP{\HlxBackUrl}
      {Previous: \xlink{\HlxBackTitle}{\HlxBackUrl}}
      {}%
    \EmptyP{\HlxForwUrl}
      {\EmptyP{\HlxBackUrl}{{} | {}}{}%
       Next: \xlink{\HlxForwTitle}{\HlxForwUrl}}
      {}%
    }{}}

\newcommand{\toppanel}{%
    {\em Scheme 48 Manual} | {\link{Contents}{top_node}}%
    \EmptyP{\HlxUpUrl}
      {{} | In Chapter: \xlink{\HlxUpTitle}{\HlxUpUrl}}
      {}%
    \\%
    \previousnextmenu}

\newcommand{\chaptertoppanel}{%
    {\em Scheme 48 Manual} | {\link{Contents}{top_node}}%
    \EmptyP{\HlxPrevUrl}
      {{} | Previous Chapter: \xlink{\HlxPrevTitle}{\HlxPrevUrl}}
      {}%
    \EmptyP{\HlxNextUrl}
      {{} | Next Chapter: \xlink{\HlxNextTitle}{\HlxNextUrl}}
      {}%
    \\%
    \previousnextmenu{}}

\newcommand{\bottommatter}{%
  %\htmlrule
  \html{P}
  \EmptyP{\HlxAddress}
  {\par{}\html{ADDRESS}\HlxAddress\html{/ADDRESS}\\}{}}

\newcommand{\bottompanel}{\previousnextmenu}

\newcommand{\cvar}[1]{\textrm{\textit{#1}}}

%%%%%%%%%%%%%%%% Latex prototypes
\texonly

\newcommand{\evalsto}{$\rightarrow$}

\newenvironment{protos}{\list{$\bullet$}
{\leftmargin1.2em\rightmargin0pt\itemsep0pt\parsep0pt\partopsep-2pt}}
{\endlist}

% The following is for prototypes that have return types.
%    (foo int int) -> int

\newcommand{\proto}[3]{%
\protonoindex{#1}{#2}{#3}%
\mainschindex{#1}}%

\newcommand{\protonoindex}[3]{\item\noindent\unskip%
\hbox{\spaceskip=0.5em\code{({#1}{\it#2\/})} {$\rightarrow$} {\it#3}}}

\newcommand{\cproto}[1]{\item\noindent\unskip%
\hbox{\spaceskip=0.5em\code{{#1}}}}

\newcommand{\cgcproto}[1]{\item\noindent\unskip%
\hbox{\spaceskip=0.5em\code{{#1}}}\hfill\penalty 0%
\hbox{ }\nobreak\hfill\hbox{\rm (may GC)}}

\newcommand{\protonoresultnoindex}[2]{\item\noindent\unskip%
\hbox{\spaceskip=0.5em\code{(\hbox{#1}{\it#2\/})}}}

\newcommand{\protonoresult}[2]{%
\protonoresultnoindex{#1}{#2}%
\mainschindex{#1}}%

\newcommand{\protoresult}[1]{\newline\unskip%
{\hspace*{2em}\code{{$\rightarrow$} {\it#1}}\hfill}}

% Syntax prototypes

\newcommand{\syntaxprotonoresultnoindex}[2]{\item\noindent\unskip%
\hbox{\spaceskip=0.5em\code{(\hbox{#1}{#2})}}\hfill\penalty 0%
\hbox{ }\nobreak\hfill\hbox{\rm syntax}}

\newcommand{\syntaxprotonoresult}[2]{%
\syntaxprotonoresultnoindex{#1}{#2}\mainschindex{#1}}

\newcommand{\syntaxproto}[3]{\item\noindent\unskip%
\hbox{\spaceskip=0.5em\code{(\hbox{#1}{#2})}} {$\rightarrow$} {\it#3}%
\hfill\penalty 0%
\hbox{ }\nobreak\hfill\hbox{\rm syntax}}

% This can be reduced

\newcommand{\pconstproto}[2]{\item\noindent\unskip%
\hbox{\spaceskip=0.5em\code{#1}}\hfill\penalty 0%
\hbox{ }\nobreak\hfill\hbox{\rm #2}}

% Variable prototype
\newcommand{\constproto}[2]{\pconstproto{#1}{#2}\mainschindex{#1}}

\newcommand{\constprotonoindex}[2]{\pconstproto{#1}{#2}}

\endtexonly

%%%%%%%%%%%%%%%% end of Latex proto definitions

%%%%%%%%%%%%%%%% HTML prototypes
\htmlonly

\newcommand{\evalsto}{\verb| => |}

\newenvironment{protos}{\begin{itemize}}{\end{itemize}}

% The following is for prototypes that have return types.
%    (foo int int) -> int

\newcommand{\protonoindex}[3]{%
%\cindex{\code{#1}}%
\item\noindent\code{({#1}{\var{#2}\/})~-->~{\var{#3}}}}

\newcommand{\proto}[3]{%
\protonoindex{#1}{#2}{#3}%
\mainschindex{#1}}%

\newcommand{\protonoresultnoindex}[2]{%
%\cindex{\code{#1}}%
\item\noindent\code{({#1}{\var{#2}\/})}}

\newcommand{\protonoresult}[2]{%
\protonoresultnoindex{#1}{#2}%
\mainschindex{#1}}%

\newcommand{\pconstproto}[2]{%
\item\noindent\code{{#1}}\prototag{#2}}

% Variable prototype
\newcommand{\constproto}[2]{\pconstproto{#1}{#2}\mainschindex{#1}}

\newcommand{\constprotonoindex}[2]{\pconstproto{#1}{#2}}

\newcommand{\cproto}[1]{%
\item\noindent\code{{#1}}}

\newcommand{\cgcproto}[1]{%
\item\noindent\code{{#1}}\prototag{may GC}}

\newcommand{\syntaxprotonoresult}[2]{%
\syntaxprotonoresultnoindex{#1}{#2}\mainschindex{#1}}

\newcommand{\syntaxprotonoresultnoindex}[2]{%
\item\noindent\code{({#1}{#2})}\prototag{syntax}}

\newcommand{\protoresult}[1]{%
\\{}\noindent\code{\ \ \ \ -->~{\it{#1}}}}

\newcommand{\syntaxproto}[3]{%
\item\noindent\code{({#1}{#2})~-->~{\var{#3}}}\prototag{syntax}}

\newcommand{\prototag}[1]{\hfill\hbox{(#1)}}

\endhtmlonly
%%%%%%%%%%%%%%%% end of HTML proto definitions

\newcommand{\meta}[1]{{\noindent\mbox{\textrm{$\langle$#1$\rangle$}}}}
\newcommand{\hyper}[1]{\meta{#1}}
\newcommand{\hyperi}[1]{\hyper{#1$_1$}}
\newcommand{\hyperii}[1]{\hyper{#1$_2$}}
\newcommand{\hyperj}[1]{\hyper{#1$_i$}}
\newcommand{\hypern}[1]{\hyper{#1$_n$}}
\newcommand{\var}[1]{\noindent\mbox{\textnormal{\textit{#1}}}}
\newcommand{\vari}[1]{\var{#1$_1$}}
\newcommand{\varii}[1]{\var{#1$_2$}}
\newcommand{\variii}[1]{\var{#1$_3$}}
\newcommand{\variv}[1]{\var{#1$_4$}}
\newcommand{\varj}[1]{\var{#1$_j$}}
\newcommand{\varn}[1]{\var{#1$_n$}}

\newcommand{\defining}[1]{{\emph{#1}}}

\newcommand{\exprtype}{syntax}

\newcommand{\dotsfoo}{\ldots\texonly{\thinspace}\endtexonly}

\let\code=\texttt

\newcommand{\goesto}{%
\texonly{\hbox{$\longrightarrow$}}\endtexonly%
\htmlonly{\code{-->}}\endhtmlonly%
}

\newcommand{\xform}{%
\texonly{\hbox{$\Longrightarrow$}}\endtexonly%
\htmlonly{\code{--->}}\endhtmlonly%
}

\newcommand{\arbno}[1]{{{#1}$^*$}}
\newcommand{\hack}{Scheme~48}

\newcommand{\alt}{$\vert$}

\newcommand{\ok}{\discretionary{}{}{}}

\newcommand{\syn}[1]{{\noindent\mbox{\textrm{$\langle$#1$\rangle$}}}}

\newenvironment{example}{\begin{alltt}}{\end{alltt}}

%\includeonly{posix}

% Make a few big HTML files, and not a lot of small ones.
\setcounter{htmldepth}{3}

% Put the html code in its own directory.
\htmldirectory{manual}

% Set the html base name.
\htmlname{s48manual}

% Add sections to main menu
\setcounter{htmlautomenu}{2}

% White background
\htmlattributes{BODY}{BGCOLOR="#ffffff"}

\htmltitle{Scheme 48 Manual}

% Suppress navigation panel for first page.
\htmlpanel{0}

%%% End preamble

\begin{document}
\label{top_node}

\T\sloppy    % Tells TeX not to worry too much about line breaks.

\title{{\large The Incomplete} \\ Scheme 48 Reference Manual \\ {\large for release 0.54}}
%         December 25, 1994
\author{Richard Kelsey and Jonathan Rees \\
 {\normalsize with a chapter by Mike Sperber}}
\date{}
%\date{October 31, 1995}

\T\cleardoublepage\pagenumbering{roman}

\maketitle

% For some reason the tabbing doesn't work here in html mode.
\texonly{
\begin{verse}
A line may take us hours, yet if it does not seem a moment's thought \\
All our stitching and unstitching has been as nought.
\end{verse}
\begin{tabbing}
A line may take us hours, yet if it does not seem\= \kill
\> Yeats \\
\> {\em Adam's Curse}
\end{tabbing}
}

\htmlonly{
\begin{center}
\begin{tabbing}
A line may take us hours, yet if it does not seem\= a moment's thought \\
All our stitching and unstitching has been as nought. \\
\> Yeats \\
\> {\em Adam's Curse}
\end{tabbing}
\end{center}
}

\htmlpanel{1}

\chapter*{Acknowledgements}

Thanks to Scheme~48's users for their suggestions, bug reports,
  and forbearance.
Thanks also to Deborah Tatar for providing the Yeats quotation.

\T\tableofcontents

\T\cleardoublepage\pagenumbering{arabic}\setcounter{page}{1}

\setcounter{htmlautomenu}{1}



\chapter{Introduction}

Scheme~48 is an implementation of the Scheme programming language as
described in the Revised$^5$ Report on the Algorithmic Language
 Scheme~\cite{R5RS}.
It is based on a compiler and interpreter for a virtual Scheme
machine.  Scheme~48 tries to be faithful to the Revised$^5$ Scheme
Report, providing neither more nor less in the initial user
environment.  (This is not to say that more isn't available in other
environments; see below.)  Support for numbers is weak: bignums are
slow and floating point is almost nonexistent (see description of
floatnums, below).

% JAR says: replace zurich with mumble.net or ...

Scheme~48 is under continual development.
Please report bugs, especially in the VM, especially core dumps, to
scheme-48-bugs@zurich.ai.mit.edu.  Include the version number x.yy
from the "Welcome to Scheme~48 x.yy" greeting message in your bug
report.  It is a goal of this project to produce a bullet-proof
system; we want no bugs and, especially, no crashes.  (There are a few
known bugs, listed in the {\tt doc/todo.txt} file that comes with the
distribution.)

Send mail to scheme-48-request@zurich.ai.mit.edu to be put on a
mailing list for announcements, discussion, bug reports, and bug
fixes.

The name `Scheme~48' commemorates our having written the original version
 in forty-eight hours, on August 6th and 7th, 1986.



\chapter{User's guide}

This chapter details Scheme~48's user interface: its command-line arguments,
 command processor, debugger, and so forth.

\section{Command line arguments}

A few command line arguments are processed by Scheme~48 as
 it starts up.

\code{scheme48}
[\code{-i} \cvar{image}]
[\code{-h} \cvar{heapsize}]
% [\code{-s} \cvar{stacksize}]
[\code{-a} \cvar{argument \ldots}]

\begin{description}
\item[{\tt -i} \cvar{image}]
    specifies a heap image file to resume.  This defaults to a heap
    image that runs a Scheme command processor.  Heap images are
    created by the \code{,dump} and \code{,build commands}, for which see below.

\item[{\tt -h} \cvar{heapsize}]
    specifies how much space should be reserved for allocation.
    \cvar{Heapsize} is in words (where one word = 4 bytes), and covers both
    semispaces, only one of which is in use at any given time (except
    during garbage collection).  Cons cells are currently 3 words, so
    if you want to make sure you can allocate a million cons cells,
    you should specify \code{-h 6000000} (actually somewhat more than this,
    to account for the initial heap image and breathing room).
    The default heap size is 3000000 words.  The system will use a
    larger heap if the specified (or default) size is less than
    the size of the image being resumed.

%\item[{\tt -s} \cvar{stacksize}]
%    specifies how much space should be reserved for the continuation
%    and environment stack.  If this space is exhausted, continuations
%    and environments are copied to the heap.  \cvar{Stacksize} is in words
%    and defaults to 2500.

\item[{\tt -a} \cvar{argument \ldots}]
    is only useful with images built using \code{,build}.
    The arguments are passed as a list of strings to the procedure specified
    in the \code{,build} command as for example:
\begin{example}
> (define (f a) (for-each display a) (newline) 0)
> ,build f foo.image
> ,exit
\% scheme48vm -i foo.image -a mumble "foo x"
mumblefoo x
\%
\end{example}
\end{description}

The usual definition of the \code{s48} or \code{scheme48} command is actually a
 shell script that starts up the Scheme~48 virtual machine with a
 \code{-i \cvar{imagefile}}
specifying the development environment heap image and a
 \code{-o \cvar{vm-executable}} specifying the location of the virtual-machine
 executable (the executable is needed for loading external code on some
 versions of Unix; see section~\ref{dynamic-externals} for more information).
The file \code{go} in the Scheme~48 installation source directory is an example
 of such a shell script.

\section{Command processor}

When you invoke the default heap image, a command processor starts
 running.
The command processor acts as both a read-eval-print loop, reading
 expressions, evaluating them, and printing the results, and as
 an interactive debugger and data inspector.
See Chapter~\ref{chapter:command-processor} for
a description of the command processor.

\section{Editing}

We recommend running Scheme~48 under GNU Emacs or XEmacs using the
 \code{cmuscheme48} command package.
This is in the Scheme~48 distribution's \code{emacs/} subdirectory and
 is included in XEmacs's \code{scheme} package.
It is a variant of the \code{cmuscheme} library, which
 comes to us courtesy of Olin Shivers, formerly of CMU.
You might want to put the following in your Emacs init file (\code{.emacs}):
\begin{example}
(setq scheme-program-name "scheme48")
(autoload 'run-scheme
          "cmuscheme48"
          "Run an inferior Scheme process."
          t)
\end{example}
The Emacs function \code{run-scheme} can then be used to start a process
 running the program \code{scheme48} in a new buffer.
To make the \code{autoload} and \code{(require \ldots)} forms work, you will
also need
to put the directory containing \code{cmuscheme} and related files in your
emacs load-path:
\begin{example}
(setq load-path
  (append load-path '("\cvar{scheme-48-directory}/emacs")))
\end{example}
Further documentation can be found in the files \code{emacs/cmuscheme48.el} and
\code{emacs/comint.el}.

\section{Performance}
\label{section:performance}

If you want to generally have your code run faster than it normally
would, enter \code{inline-values} mode before loading anything.  Otherwise
calls to primitives (like \code{+} and \code{cons}) and in-line procedures
(like \code{not} and \code{cadr}) won't be open-coded, and programs will run
more slowly.

The system doesn't start in \code{inline-values} mode by default because the
Scheme report permits redefinitions of built-in procedures.  With
this mode set, such redefinitions don't work according to the report,
because previously compiled calls may have in-lined the old
definition, leaving no opportunity to call the new definition.

\code{Inline-values} mode is controlled by the \code{inline-values} switch.
\code{,set inline-values} and \code{,unset inline-values} turn it on and off.

\section{Disassembler}

The \code{,dis} command prints out the disassembled byte codes of a procedure.
\begin{example}
> ,dis cons
cons
  0 (protocol 2)
  2 (pop)
  3 (make-stored-object 2 pair)
  6 (return)
> 
\end{example}
The current byte codes are listed in the file \code{scheme/vm/arch.scm}.
A somewhat out-of-date description of them can be found in
\cite{Kelsey-Rees:Scheme48}.

The command argument is optional; if unsupplied it defaults to the
current focus object (\code{\#\#}).

The disassembler can also be invoked on continuations and templates.

\section{Module system}
\label{module-guide}

This section gives a brief description of modules and related entities.
For detailed information, including a description of the module
 configuration language, see 
chapter \ref{chapter:modules}.

% JAR says: this paragraph is muddy.

A {\em module} is an isolated namespace, with visibility of bindings
 controlled by module descriptions written in a special
 configuration language.
A module may be instantiated as a {\em package}, which is an environment
 in which code can be evaluated.
Most modules are instantiated only once and so have a unique package.
A {\em structure} is a subset of the bindings in a package.
Only by being included in a structure can a binding be
 made visible in other packages.
A structure has two parts, the package whose bindings are being exported
 and the set of names that are to be exported.
This set of names is called an {\em interface}.
A module then has three parts:
\begin{itemize}
\item a set of structures whose bindings are to be visible within the module
\item the source code to be evaluated within the module
\item a set of exported interfaces
\end{itemize}
Instantiating a module produces a package and a set of structures, one for
 each of the exported interfaces.

The following example uses \code{define-structure} to create a module that
 implements simple cells as pairs, instantiates this module, and binds the
 resulting structure to \code{cells}.
The syntax \code{(export \cvar{name \ldots})} creates an interface
 containing \cvar{name \ldots}.
The \code{open} clause lists structures whose bindings are visible
 within the module.
The \code{begin} clause contains source code.
\begin{example}
(define-structure cells (export make-cell
                                cell-ref
                                cell-set!)
  (open scheme)
  (begin (define (make-cell x)
           (cons 'cell x))
         (define cell-ref cdr)
         (define cell-set! set-cdr!)))
\end{example}

Cells could also have been implemented using the
record facility described in section~\ref{records}
 and available in structure \code{define-record-type}.
\begin{example}
(define-structure cells (export make-cell
                                cell-ref
                                cell-set!)
  (open scheme define-record-types)
  (begin (define-record-type cell :cell
           (make-cell value)
           cell?
           (value cell-ref cell-set!))))
\end{example}

With either definition the resulting structure can be used in other
 modules by including \code{cells} in an \code{open} clause.

The command interpreter is always operating within a particular package.
Initially this is a package in which only the standard Scheme bindings
 are visible.
The bindings of other structures can be made visible by using the 
\code{,open} command described in section~\ref{module-command-guide} below.

Note that this initial package does not include the configuration language.
Module code needs to be evaluated in the configuration package, which can
 be done by using the {\code ,config} command:
\begin{example}
> ,config (define-structure cells \ldots)
> ,open cells
> (make-cell 4)
'(cell . 4)
> (define c (make-cell 4))
> (cell-ref c)
4
\end{example}

\section{Library}

A number of useful utilities are either built in to Scheme~48 or can
be loaded from an external library.  These utilities are not visible
in the user environment by default, but can be made available with the
\code{open} command.  For example, to use the \code{tables} structure, do
\begin{example}
> ,open tables
> 
\end{example}

If the utility is not already loaded, then the \code{,open} command will
 load it.
Or, you can load something explicitly (without opening it) using the
\code{load-package} command:
\begin{example}
> ,load-package queues
> ,open queues
\end{example}

When loading a utility, the message "Note: optional optimizer not
invoked" is innocuous.  Feel free to ignore it.

See also the package system documentation, in
chapter~\ref{chapter:modules}.

Not all of the the libraries available in Scheme~48 are described in this
 manual.
All are listed in files \code{rts-packages.scm},
 \code{comp-packages.scm}, \code{env-packages.scm}, and
 \code{more-packages.scm} in the \code{scheme} directory of the distribution,
 and the bindings they
 export are listed in \code{interfaces.scm} and
 \code{more-interfaces.scm} in the same directory.

%architecture
%    Information about the virtual machine.  E.g.
%      (enum op eq?) => the integer opcode of the EQ? instruction
%
%big-scheme
%    Many generally useful features.  See doc/big-scheme.txt.
%
%bigbit
%    Extensions to the bitwise logical operators (exported by
%    the BITWISE structure) so that they operate on bignums.
%    To use these you should do
%
%        ,load-package bigbit
%        ,open bitwise
%
%conditions
%    Part of the condition system: DEFINE-CONDITION-PREDICATE and
%    routines for examining condition objects.  (See also handle,
%    signals.)
%
%defpackage
%    The module system: DEFINE-STRUCTURE and DEFINE-INTERFACE.
%
%destructuring
%    DESTRUCTURE macro.  See doc/big-scheme.txt.
%
%display-conditions
%    Displaying condition objects.
%        (DISPLAY-CONDITION condition port) \goesto{} unspecific
%          Display condition in an easily readable form.  E.g.
%\begin{example}
%          > ,open display-conditions handle conditions
%          > (display-condition
%             (call-with-current-continuation
%               (lambda (k)
%                 (with-handler (lambda (c punt)
%                                 (if (error? c)
%                                     (k c)
%                                     (punt)))
%                   (lambda () (+ 1 'a)))))
%             (current-output-port))
%
%          Error: exception
%                 (+ 1 'a)
%          > 
%\end{example}
%
%extended-ports
%    Ports for reading from and writing to strings, and related things.
%    See doc/big-scheme.txt.
%
%filenames
%    Rudimentary file name parsing and synthesis.  E.g.
%    file-name-directory and file-name-nondirectory are as in Gnu emacs.
%
%floatnums
%    Floating point numbers.  These are in a very crude state; use at
%    your own risk.  They are slow and do not read or print correctly.
%
%fluids
%    Dynamically bound "variables."
%      (MAKE-FLUID top-level-value) \goesto{} a "fluid" object
%      (FLUID fluid) \goesto{} current value of fluid object
%      (SET-FLUID! fluid value) \goesto{} unspecific; changes current value of
%        fluid object
%      (LET-FLUID fluid value thunk) \goesto{} whatever thunk returns
%        Within the dynamic extent of execution of (thunk), the fluid
%        object has value as its binding (unless changed by SET-FLUID!
%        or overridden by another LET-FLUID).
%    E.g.
%      (define f (make-fluid 7))
%      (define (baz) (+ (fluid f) 1))
%      (baz)   ;\goesto{} 8
%      (let-fluid f 4 (lambda () (+ (baz) 1)))  ;\goesto{} 6
%
%formats
%    A simple FORMAT procedure, similar to Common Lisp's or T's.
%    See doc/big-scheme.txt for documentation.
%
%handle
%    Part of the condition system.
%      (WITH-HANDLER handler thunk) \goesto{} whatever thunk returns.
%        handler is a procedure of two arguments.  The first argument
%        is a condition object, and the second is a "punt" procedure.
%        The handler should examine the condition object (using ERROR?,
%        etc. from the CONDITIONS structure).  If it decides not to do
%        anything special, it should tail-call the "punt" procedure.
%        Otherwise it should take appropriate action and perform a
%        non-local exit.  It should not just return unless it knows
%        damn well what it's doing; returns in certain situations can
%        cause VM crashes.
%
%interrupts
%    Interrupt system
%
%ports
%    A few extra port-related operations, notably FORCE-OUTPUT.
%
%pp
%    A pretty-printer.  (p \cvar{exp}) will pretty-print the result of \cvar{exp},
%    which must be an S-expression.  (Source code for procedures is not
%    retained or reconstructed.)  You can also do (p \cvar{exp} \cvar{port}) to
%    print to a specific port.
%
%    The procedure pretty-print takes three arguments: the object to be
%    printed, a port to write to, and the current horizontal cursor
%    position.  If you've just done a newline, then pass in zero for
%    the position argument.
%
%    The algorithm is very peculiar, and sometimes buggy.
%
%queues
%    FIFO queues.
%
%random
%    Not-very-random random number generator.  The \cvar{seed} should be between
%    0 and 2$^{28}$ exclusive.
%
%        > (define random (make-random \cvar{seed}))
%        > (random) \goesto{} pseudo-random number between 0 and 2$^{28}$
%
%receiving
%    Convenient interface to the call-with-values procedure, like
%    Common Lisp's multiple-value-bind macro.  See doc/big-scheme.txt.
%
%records
%    MAKE-RECORD-TYPE and friends.  See the Scheme of Things column in
%    Lisp Pointers, volume 4, number 1, for documentation.
%
%recnums
%    Complex numbers.  This should be loaded (e.g. with ,load-package)
%    but needn't be opened.
%
%search-trees
%    Balanced binary search trees.  See comments at top of
%    big/search-tree.scm. 
%
%signals
%    ERROR, WARN, and related procedures.
%
%sort
%    Online merge sort (see comment at top of file big/sort.scm).
%
%        (sort-list \cvar{list} \cvar{pred})
%        (sort-list! \cvar{list} \cvar{pred})
%
%sicp
%    Compatibility package for the Scheme dialect used in the book
%    "Structure and Interpretation of Computer Programs."
%
%sockets
%    Interface to Unix BSD sockets.  See comments at top of file
%    misc/socket.scm.
%
%threads
%    Multitasking.  See doc/threads.txt.
%
%util
%    SUBLIST, ANY, REDUCE, FILTER, and some other useful things.
%
%weak
%    Weak pointers and populations.
%        (MAKE-WEAK-POINTER thing) => weak-pointer
%        (WEAK-POINTER-REF weak-pointer) => thing or \code{\#f}
%          \code{\#f} if the thing has been gc'ed.
%
%writing
%        (RECURRING-WRITE thing port recur) => unspecific
%          This is the same as WRITE except that recursive calls invoke
%          the recur argument instead of WRITE.  For an example, see
%          the definition of LIMITED-WRITE in env/dispcond.scm, which
%          implements processing similar to common Lisp's *print-level*
%          and *print-length*.

%%% Local Variables: 
%%% mode: latex
%%% TeX-master: "manual"
%%% End: 

\chapter{Command processor}
\label{chapter:command-processor}

This chapter details Scheme~48's command processor, which incorporates
 both a read-eval-print loop and an interactive debugger.
At the \code{>} prompt, you can type either a Scheme form
 (expression or definition) or a command beginning with a comma.
In
\link*{inspection mode}[inspection mode (see section~\Ref)]{inspector}
 the prompt changes to \code{:} and commands
 no longer need to be preceded by a comma; input beginning with
 a letter or digit is assumed to be a command, not an expression.
In inspection mode the command processor prints out a
 menu of selectable components for the current object of interest.

\section{Current focus value and {\tt \#\#}}

The command processor keeps track of a current {\em focus value}.
This value is normally the last value returned by a command.
If a command returns multiple values the focus object is a list of the
 values.
The focus value is not changed if a command returns no values or 
 a distinguished `unspecific' value.
Examples of forms that return this unspecific value are definitions,
 uses of \code{set!}, and \code{(if \#f 0)}.
It prints as \code{\#\{Unspecific\}}.

The reader used by the command processor reads \code{\#\#} as a special
 expression that evaluates to the current focus object.
\begin{example}
> (list 'a 'b)
'(a b)
> (car ##)
'a
> (symbol->string ##)
"a"
> (if #f 0)
#\{Unspecific\}
> ##
"a"
> 
\end{example}

\section{Command levels}

If an error, keyboard interrupt, or other breakpoint occurs, or the
 \code{,push} command is used, the command
 processor invokes a recursive copy of itself, preserving the dynamic state of
 the program when the breakpoint occured.
The recursive invocation creates a new {\em command level}.
The command levels form a stack with the current level at the top.
The command prompt indicates the number of stopped levels below the
 current one: \code{>} or \code{:} for the
 base level and \code{\cvar{n}>} or \code{\cvar{n}:} for all other levels,
 where \cvar{n} is the command-level nesting depth.
The \code{auto-levels} switch
 described below can be used to disable the automatic pushing of new levels.

The command processor's evaluation package and the value of the
 \code{inspect-focus-value} switch are local to each command level.
They are preserved when a new level is pushed and restored when
 it is discarded.
The settings of all other switches are shared by all command levels.

\begin{description}
\item $\langle{}$eof$\rangle{}$\\
    Discards the current command level and resumes running the level down.
    $\langle{}$eof$\rangle{}$ is usually
    control-\code{D} at a Unix shell or control-\code{C} control-\code{D} using
    the Emacs \code{cmuscheme48} library.

\item \code{,pop}\\
 The same as $\langle{}$eof$\rangle{}$.

\item \code{,proceed [\cvar{exp} \ldots}]\\
 Proceed after an interrupt or error, resuming the next command
 level down, delivering the values of \cvar{exp~\ldots} to the continuation.
 Interrupt continuations discard any returned values.
 \code{,Pop} and \code{,proceed} have the same effect after an interrupt
 but behave differently after errors.
 \code{,Proceed} restarts the erroneous computation from the point where the
 error occurred (although not all errors are proceedable) while
 \code{,pop} (and $\langle{}$eof$\rangle{}$) discards it and prompts for
 a new command.

\item \code{,push}\\
 Pushes a new command level on above the current one.
 This is useful if the \code{auto-levels} switch has been used
 to disable the automatic pushing of new levels for errors and interrupts.

\item \code{,reset [\cvar{number}]}\\
  Pops down to a given level and restarts that level.
  \cvar{Number} defaults to zero, \code{,reset} restarts the command
  processor, discarding all existing levels.

\end{description}

Whenever moving to an existing level, either by sending
 an $\langle{}$eof$\rangle{}$
 or by using \code{,reset} or the other commands listed above,
 the command processor runs all of the \code{dynamic-wind} ``after'' thunks
 belonging to stopped computations on the discarded level(s).

\section{Logistical commands}
\begin{description}
\item \code{,load \cvar{filename \ldots}}\\
    Loads the named Scheme source file(s).
    Easier to type than \code{(load "\cvar{filename}")} because you don't have to
    shift to type the parentheses or quote marks.  Also, it works in
    any package, unlike \code{(load "\cvar{filename}")}, which will work only
    work in packages in which the variable \code{load} is defined appropriately.

\item \code{,exit [\cvar{exp}]}
    Exits back out to shell (or executive or whatever invoked Scheme~48
    in the first place).
    \cvar{Exp} should evaluate to an integer.  The
    integer is returned to the calling program.
    The default value of \cvar{exp} is zero, which, on Unix,
    is generally interpreted as success.
\end{description}

\section{Module commands}
\label{module-command-guide}

There are many commands related to modules.
Only the most commonly used module commands are described here;
 documentation for the
 rest can be found in \link*{the module chapter}[section~\Ref]{module-commands}.
There is also
\link*{a brief description of modules, structures, and packages}
[a brief description of modules, structures, and packages in section~\Ref]
{module-guide} below.

\begin{description}
\item \code{,open \cvar{structure \ldots}}\\
Makes the bindings in the \cvar{structure}s visible in the current package.
The packages associated with the \cvar{structure}s will be loaded if
 this has not already been done (the \code{ask-before-loading} switch
 can be used disable the automatic loading of packages).

\item \code{,config [\cvar{command}]}\\
Executes \cvar{command} in the \code{config} package, which includes
 the module configuration language.
For example, use
\begin{example}
,config ,load \cvar{filename}
\end{example}
to load a file containing module definitions.
If no \cvar{command} is given, the \code{config} package becomes the
 execution package for future commands.

\item \code{,user [\cvar{command}]} \\
    This is similar to the {\tt ,config}.  It
    moves to or executes a command in the user package (which is the
    default package when the \hack{} command processor starts).

\end{description}

\section{Debugging commands}
\label{debug-commands}

\begin{description}
\item \code{,preview}\\
    Somewhat like a backtrace, but because of tail recursion you see
    less than you might in debuggers for some other languages.
    The stack to display is chosen as follows:
\begin{enumerate}
    \item If the current focus object is a continuation or a thread,
       then that continuation or thread's stack is displayed.
    \item Otherwise, if the current command level was initiated because of
       a breakpoint in the next level down, then the stack at that
       breakpoint is displayed.
    \item Otherwise, there is no stack to display and a message is printed
       to that effect.
\end{enumerate}
    One line is printed out for each continuation on the chosen stack,
    going from top to bottom.

\item \code{,run \cvar{exp}}\\
    Evaluate \cvar{exp}, printing the result(s) and making them
    (or a list of them, if \cvar{exp} returns multiple results)
    the new focus object.
    The \code{,run} command is useful when writing
    \link*{command programs}
    [command programs, which are described in section~\Ref{} below]
    {command-programs}.

\item \code{,trace \cvar{name} \ldots}\\
    Start tracing calls to the named procedure or procedures.
    With no arguments, displays all procedures currently traced.
    This affects the binding of \cvar{name}, not the behavior of the
    procedure that is its current value.  \cvar{Name} is redefined
    to be a procedure that prints a message,
    calls the original value of \cvar{name}, prints another
    message, and finally passes along the value(s) returned by the
    original procedure.

\item \code{,untrace \cvar{name} \ldots}\\
    Stop tracing calls to the named procedure or procedures.
    With no argument, stop tracing all calls to all procedures.

\item \code{,condition}\\
    The \code{,condition} command displays the condition object
    describing the error or interrupt that initiated the current
    command level.  The condition object becomes the current focus
    value.  This is particularly useful in conjunction with
    the inspector.  For example, if a procedure is passed the wrong number of
    arguments, do \code{,condition} followed by
     \code{,inspect} to inspect the
    procedure and its arguments.

\item \code{,bound?\ \cvar{name}}\\
    Display the binding of \cvar{name}, if there is one, and otherwise
    prints `\code{Not bound}'.
 
\item \code{,expand \cvar{form}}
\T\vspace{-1em}
\item \code{,expand-all \cvar{form}}\\
    Show macro expansion of \cvar{form}, if any.
    \code{,expand} performs a single macro expansion while
    \code{,expand-all} fully expands all macros in \cvar{form}.

\item \code{,where \cvar{procedure}}\\
    Display name of file containing \cvar{procedure}'s source code.
\end{description}

\section{Switches}

There are a number of binary switches that control the behavior of the
 command processor.
They can be set using the \code{,set} and \code{,unset} commands.

\begin{description}
\item \code{,set \cvar{switch} [on | off | ?]}\\
    This sets the value of mode-switch \cvar{switch}.
    The second argument defaults to \code{on}.
    If the second argument is \code{?} the value of \cvar{switch} is
    is displayed and not changed.
    Doing \code{,set ?} will display a list of the switches and
    their current values.

\item \code{,unset \cvar{switch}}\\
    \code{,unset \cvar{switch}} is the same as
    \code{,set \cvar{switch} off}.
\end{description}

The switches are as follows:
\begin{description}
\item \code{batch}\\
    In `batch mode' any error or interrupt that comes up will cause
    Scheme~48 to exit immediately with a non-zero exit status.  Also,
    the command processor doesn't print prompts.  Batch mode is
    off by default.

% JAR says: disable auto-levels by default??

\item \code{auto-levels}\\
    Enables or disables the automatic pushing of a new command level when
    an error, interrupt, or other breakpoint occurs.
    When enabled (the default), breakpoints push a new command level,
    and $\langle{}$eof$\rangle{}$ (see above)
    or \code{,reset} is required to return to top level.  The effects of
    pushed command levels include:
\begin{itemize}
\item a longer prompt
\item retention of the continuation in effect at the point of errors
\item confusion among some newcomers
\end{itemize}
    With \code{auto-levels} disabled one must issue a
    \code{,push} command immediately
    following an error in order to retain the error continuation for
    debugging purposes; otherwise the continuation is lost as soon as
    the focus object changes.  If you don't know anything about the
    available debugging tools, then levels might as well be disabled.

\item \code{inspect-focus-value}\\
    Enable or disable `inspection' mode, which is used for inspecting
    data structures and continuations.
    \link*{Inspection mode is desribed below}
    [Inspection mode is described in section~\Ref]
    {inspector}.

\item \code{break-on-warnings}\\
    Enter a new command level when a warning is produced, just as
    when an error occurs.  Normally warnings only result in a displayed
    message and the program does not stop executing.

\item \code{ask-before-loading} \\
    If on, the system will ask before loading modules that are arguments
    to the \code{,open} command.  \code{Ask-before-loading} is off by
    default.
\begin{example}
> ,set ask-before-loading
will ask before loading modules
> ,open random
Load structure random (y/n)? y
>
\end{example}

\item \code{load-noisily}\\
    When on, the system will print out the names of modules and files
    as they are loaded.  \code{load-noisily} is off by default.
\begin{example}
> ,set load-noisily
will notify when loading modules and files
> ,open random
[random /usr/local/lib/scheme48/big/random.scm]
> 
\end{example}

\item \code{inline-values}\\
This controls whether or not the compiler is allowed to substitute
 variables' values in-line.
When \code{inline-values} mode is on,
some Scheme procedures will be substituted in-line; when it is off,
none will.
\link*{The performance section}[Section~\Ref]{section:performance}
has more information.

\end{description}

\section{Inspection mode}
\label{inspector}

There is a data inspector available via the \code{,inspect} and
 \code{,debug} commands or by setting the \code{inspect-focus-value} switch.
The inspector is particularly useful with procedures, continuations,
 and records.
The command processor can be taken out of inspection mode by
 using the \code{q} command, by unsetting the \code{inspect-focus-value} switch,
 or by going to a command level where the \code{inspect-focus-value} is not
 set.
When in inspection mode, input that begins with
  a letter or digit is read as a command, not as an expression.
To see the value of a variable or number, do \code{(begin \cvar{exp})}
  or use the \code{,run \cvar{exp}} command.

In inspection mode the command processor prints out a
 menu of selectable components for the current focus object.
To inspect a particular component, just type the corresponding number in
 the menu.
That component becomes the new focus object.
For example:
\begin{example}
> ,inspect '(a (b c) d)
(a (b c) d)

[0] a
[1] (b c)
[2] d
: 1
(b c)

[0] b
[1] c
: 
\end{example}

When a new focus object is selected the previous one is pushed onto a
 stack.
You can pop the stack, reverting to the previous object, with
 the \code{u} command, or use the \code{stack} command to move to
 an earlier object.

%\begin{description}
%\item \code{stack}\\
%    Prints the current stack out as a menu.
%    Selecting an item pops all higher values off of the stack and
%    makes that item the current focus value.
%\end{description}
%

Commands useful when in inspection mode:
\begin{itemize}
\item\code{u} (up) pop object stack
\item\code{m} (more) print more of a long menu
\item\code{(\ldots)} evaluate a form and select result
\item\code{q} quit
\item\code{template} select a closure or continuation's template
 (Templates are the static components of procedures; these are found
  inside of procedures and continuations, and contain the quoted
  constants and top-level variables referred to by byte-compiled code.)
\item\code{d} (down) move to the next continuation
 (current object must be a continuation)
\item\code{menu} print the selection menu for the focus object
\end{itemize}

Multiple selection commands (\code{u}, \code{d}, and menu indexes)
 may be put on a single line.

%\code{\#\#} is always the object currently being inspected.
%After a \code{q}
%command,
%or an error in the inspector, \code{\#\#} is the last object that was being
%inspected.

All ordinary commands are available when in inspection mode.
Similarly, the inspection commands can be used when not in inspection
 mode.
For example:
\begin{example}
> (list 'a '(b c) 'd)
'(a (b c) d)
> ,0
'(b c)
> ,menu
[0] b
[1] c
> 
\end{example}

If the current command level was initiated because of
 a breakpoint in the next level down, then 
 \code{,debug} will invoke the inspector on the
 continuation at the point of the error.
The \code{u} and \code{d} (up and down)
commands then make the inspected-value stack look like a conventional stack
debugger, with continuations playing the role of stack frames.  \code{D} goes
to older or deeper continuations (frames), and \code{u} goes back up to more
recent ones.

\section{Command programs}
\label{command-programs}

The \code{exec} package contains procedures that are used
 to execute the command processor's commands.
A command \code{,\cvar{foo}} is executed by applying the value of
 the identifier \cvar{foo} in the \code{exec} package to
 the (suitably parsed) command arguments.

\begin{description}
\item \code{,exec [\cvar{command}]}\\
   Evaluate \cvar{command} in the \code{exec} package.
   For example, use
\begin{example}
,exec ,load \cvar{filename}
\end{example}
to load a file containing commands.
If no \cvar{command} is given, the \code{exec} package becomes the
 execution package for future commands.
\end{description}

The required argument types are as follows:
\begin{itemize}
\item filenames should be strings
\item other names and identifiers should be symbols
\item expressions should be s-expressions
\item commands (as for \code{,config} and \code{,exec} itself)
 should be lists of the form
 \code{(\cvar{command-name} \cvar{argument} \cvar{...})}
 where \cvar{command-name} is a symbol.
\end{itemize}

For example, the following two commands are equivalent:
\begin{example}
,config ,load my-file.scm

,exec (config '(load "my-file.scm"))
\end{example}

The file \code{scheme/vm/load-vm.scm} in the source directory contains an
 example of an \code{exec} program.

\section{Building images}

\begin{description}
\item \code{,dump \cvar{filename} [\cvar{identification}]}\\
    Writes the current heap out to a file, which can then be run using the
    virtual machine.  The new image file includes the command processor.
    If present, \cvar{identification}
    should be a string (written with double quotes); this string will
    be part of the greeting message as the image starts up.

\item \code{,build \cvar{exp} \cvar{filename}}\\
    Like \code{,dump}, except that the image file contains the value of
    \cvar{exp}, which should be a procedure of one argument, instead of
    the command processor.  When
    \cvar{filename} is resumed, that procedure will be invoked on the VM's
    \code{-a} arguments, which are passed as a list of strings.  The
    procedure should return an integer which is
    returned to the program that invoked the VM.  The command
    processor and debugging system are not included in the image
    (unless you go to some effort to preserve them, such as retaining
    a continuation).

    Doing \code{,flush} before building an image will reduce the amount
    of debugging information in the image, making for a smaller
    image file, but if an error occurs, the error message may be less
    helpful.  Doing \code{,flush source maps} before loading any programs
    used in the image will make it still smaller.
    See \link*{the description of \code{flush}}[section~\Ref]{resource-commands}
    for more information.

\end{description}
    
\section{Resource query and control}
\label{resource-commands}.

\begin{description}
\item \code{,time \cvar{exp}}\\
    Measure execution time.

\item \code{,collect}\\
    Invoke the garbage collector.  Ordinarily this happens
    automatically, but the command tells how much space is available
    before and after the collection.

\item \code{,keep \cvar{kind}}
\T\vspace{-1em}
\item \code{,flush \cvar{kind}}\\
    These control the amount of debugging information retained after
    compiling procedures.  This information can consume a fair amount
    of space.  \cvar{kind} is one of the following:
\begin{itemize}
\item \code{maps} - environment maps (local variable names, for inspector)
\item \code{source} - source code for continuations (displayed by inspector)
\item \code{names} - procedure names (as displayed by \code{write} and in error
       messages)
\item \code{files}  - source file names
\end{itemize}
    These commands refer to future compilations only, not to procedures
    that already exist.  To have any effect, they must be done before
    programs are loaded.  The default is to keep all four types.
% JAR says: ,keep tabulate  - puts debug data in a table that can be
% independently  flushed (how? -RK) or even written out and re-read later!
% (how? -RK)

\item \code{,flush}\\
    The flush command with no argument deletes the database of names
    of initial procedures.  Doing \code{,flush} before a \code{,build} or
    \code{,dump}
    will make the resulting image significantly smaller, but will
    compromise the information content of many error
    messages.
\end{description}

\section{Threads}

Each command level has its own set of threads.  These threads are suspended
 when a new level is entered and resumed when the owning level again becomes
 the current level.
A thread that raises an error is not resumed unless
 explicitly restarted using the \code{,proceed} command.
In addition to any threads spawned by the user, each level has a thread
 that runs the command processor on that level.
A new command-processor thread is started if the current one
 dies or is terminated.
When a command level is abandoned for a lower level, or when
 a level is restarted using \code{,reset}, all of the threads on that
 level are terminated and any \code{dynamic-wind} ``after'' thunks are run.

The following commands are useful when debugging multithreaded programs:
\begin{description}
\item \code{,resume [\cvar{number}]}\\
  Pops out to a given level and resumes running all threads at that level.
  \cvar{Number} defaults to zero.

\item \code{,threads}\\
    Invokes the inspector on a list of the threads running at the
    next lower command level.

\item \code{,exit-when-done [\cvar{exp}]}\\
    Waits until all user threads have completed and then
    exits back out to shell (or executive or whatever invoked Scheme~48
    in the first place).
    \cvar{Exp} should evaluate to an integer which is then
    returned to the calling program.
% JAR says: interaction with ,build ?

%\item \code{,spawn \cvar{exp} [\cvar{name}]}\\
%    Starts a new thread running \cvar{exp} on next command level down.
%    The optional \cvar{name} is used for printing and debugging.
%
%\item \code{,suspend [\cvar{exp}]}
%\T\vspace{-1em}
%\item \code{,continue [\cvar{exp}]}
%\T\vspace{-1em}
%\item \code{,kill [\cvar{exp}]}\\
%    Suspend, unsuspend, and terminate a thread, respectively.
%    Suspended threads are not run until unsuspended, terminated
%    threads are never run again.
%    \cvar{Exp} should evaluate to a thread.
%    If \cvar{exp} is not present, the current focus object is used.
%

% example of ,threads ,suspend ...
\end{description}

\section{Quite obscure}

\begin{description}
\item \code{,go \cvar{exp}}\\
    This is like \code{,exit \cvar{exp}} except that the evaluation of \cvar{exp}
    is tail-recursive with respect to the command processor.  This
    means that the command processor itself can probably be GC'ed,
    should a garbage collection occur in the execution of \cvar{exp}.
    If an error occurs Scheme~48 will exit with a non-zero value.

\item \code{,translate \cvar{from} \cvar{to}}\\
    For \code{load} and the \code{,load} command
     (but not for \code{open-\{in|out\}put-file}), file
    names beginning with the string \cvar{from} will be changed so that the
    initial \cvar{from} is replaced by the string \cvar{to}.  E.g.
\begin{example}
\code{,translate /usr/gjc/ /zu/gjc/}
\end{example}
    will cause \code{(load "/usr/gjc/foo.scm")} to have the same effect as
    \code{(load "/zu/gjc/foo.scm")}.
% JAR says: Useful with the module system!  "virtual directories"

\item \code{,from-file \cvar{filename} \cvar{form} \ldots\ ,end}\\
    This is used by the \code{cmuscheme48} Emacs library to indicate the file
    from which the \cvar{form}s came.  \cvar{Filename} is then used by the
    command processor to determine the package in which the \cvar{form}s
    are to be evaluated.
\end{description}


\documentstyle[11pt]{article}

% Latex Macros for Lisp code in text.
% Based on macros found in C. Rich's library.

\makeatletter

% \vobeyspaces turns all spaces into non-breakable spaces.
% Note: this is like \@vobeyspaces except without spurious space in defn.

{\catcode`\ =\active\gdef\vobeyspaces{\catcode`\ =\active\let =\@xobeysp}}

% \def\vobeytabs turns all tabs into 8 non-breakable spaces

{\catcode`\^^I=\active\gdef\vobeytabs{\catcode`\^^I=\active\let^^I=\xvobeytabs}}

\def\xvobeytabs{\@xobeysp\@xobeysp\@xobeysp\@xobeysp\@xobeysp\@xobeysp\@xobeysp\@xobeysp}

% \vobeylines turns all cr's into non-breakable \par's

{\catcode`\^^M=\active\gdef\vobeylines{\catcode`\^^M=\active\let^^M=\xvobeylines}}

\def\xvobeylines{\par\penalty10000}

% \obeycrsp turns cr's into non-breakable spaces

{\catcode`\^^M=\active\gdef\obeycrsp{\catcode`\^^M=\active\let^^M=\@xobeysp}}

%% \@noligs prevents ?` and !` from being treated as ligatures
%% added 19 April 86  [copied from Latex sources]

\begingroup
\catcode``=13
\gdef\@noligs{\let`=\@lquote}
\endgroup

% Set up code environment, in which most of the common special characters
% appearing in code are treated verbatim, namely: _ # & ^ $ ~ @ " %
%  *** JAR NEEDED $ AND _ IN SOME CODE ***

% Note: \ { } are still enabled so that macros can be called in this
% environment.  Use \\, \{ and \} to use these characters verbatim
% in this environment.  

% Note: this environment allows no breaking of lines whatsoever; not
% at spaces or hypens.  To arrange for a break use the standard \- macro,
% or the \= macro which breaks, but inserts nothing.  This is useful,
% for example for allowing hypenated identifiers to be broken, e.g.
% FOO-\=BAR.

\def\setupcode{\parsep=0pt\parindent=0pt
  \tt\frenchspacing\catcode``=13\@noligs%
  \def\\{\char`\\}%
  \@makeother\#\@makeother\&\@makeother\^%\@makeother\_\@makeother\$%
  \@makeother\`\@makeother\'%
  \@makeother\~\@makeother\@\@makeother\"\@makeother\%\vobeytabs\vobeyspaces}

% Code environment as described above.  Note that blank lines are
% not preserved, and lines are not kept on one page.  Code is
% indented by the same amount as quotes.
% Note: to increase left margin, use \leftmargini=1in.
%  was  {\list{}{\parsep=0pt}\item[]\setupcode\obeylines}%
% then {\list{\parsep=0pt\listparindent=0pt\leftmargin=0pt}{}\item[]\setupcode%

\newenvironment{bigcode}%
  {\list{}{\parsep=0pt\leftmargin=0pt\labelwidth=0pt\itemindent=0pt%
\listparindent=0pt}\item[]\setupcode%
\obeylines}%
  {\endlist}

% Code is just like bigcode, but everything inside is kept on one page
% Note: This actually works by setting a huge penalty for breaking
% between lines of code.
%  was  {\list{}{\parsep=0pt}\item[]\setupcode\vobeylines}%

\newenvironment{code}%
  {\list{}{\parsep=0pt\leftmargin=0pt\labelwidth=0pt\itemindent=0pt%
\listparindent=0pt}\item[]\setupcode%
\vobeylines}%
  {\endlist}

% Reasonable separation between lines of code

\newcommand{\codeskip}{\penalty0\vspace{2ex}}

% \cd is used to build a code environment in the middle of text.
% Note: only difference from display code is that cr's are taken
% as unbreakable spaces instead of \par's.

\newcommand{\cd}{\begingroup\setupcode\obeycrsp\startcode}

\newcommand{\startcode}[1]{#1\endgroup}

%\setbox0\hbox{\@xobeysp}\hline{43\wd0}

\makeatother


% Latex macros for The Scheme of Things

\newcommand{\ev}{\hbox{$\longrightarrow$}}
\newcommand{\asterisk}{\hbox{$\ast$}}
\newcommand{\foo}{\discretionary{}{}{}}
\newcommand{\var}[1]{\hbox{\em{}#1}}
\newcommand{\piece}[1]{\subsubsection*{#1}}
\newcommand{\syn}[1]{\hbox{$\langle$\rm#1$\rangle$}}
\newcommand{\xform}{\hbox{$\Longrightarrow$}}
\newcommand{\etc}{$\ldots$}
\newcommand{\ok}{\discretionary{}{}{}}

\newcommand{\separator}{
\vspace{1ex}
\begin{center}
\noindent \asterisk\hspace{1em}\asterisk\hspace{1em}\asterisk
\end{center}
\vspace{1ex}}


% -----------------------------------------------------------------------------
  %% doframeit draws a box around it argument by manipulating boxes.  It
  %% is used in the frame environments.
  %% 
  %%  Rene' Seindal (seindal@diku.dk) Fri Feb 12 16:03:07 1988
  %%  added \fboxrule and \fboxsep to \doframeit

\def\doframeit#1{\vbox{%
  \hrule height\fboxrule
    \hbox{%
      \vrule width\fboxrule \kern\fboxsep
      \vbox{\kern\fboxsep #1\kern\fboxsep }%
      \kern\fboxsep \vrule width\fboxrule }%
    \hrule height\fboxrule }}

  %% The frameit and Frameit environments formats text within a single 
  %% Anything can be framed, including verbatim text.

\def\frameit{\smallskip \advance \linewidth by -7.5pt \setbox0=\vbox \bgroup
\strut \ignorespaces }

\def\endframeit{\ifhmode \par \nointerlineskip \fi \egroup
\doframeit{\box0}}
% -----------------------------------------------------------------------------


\newcommand{\goesto}{\hbox{$\longrightarrow$}}
\newcommand{\alt}{$\vert$}
\newcommand{\arbno}[1]{{{#1}$^*$}}
\newcommand{\hack}{Scheme~48}

\begin{document}

\begin{center}
{\Large\bf Another Module System for Scheme}

\vspace{2ex}
Jonathan Rees \\
3 January 1993 (updated 4 December 1993)
\end{center}

\vspace{3ex}

This memo describes a module system for the Scheme programming
language.  The module system is unique in the extent to which it
supports both static linking and rapid turnaround during program
development.  The design was influenced by Standard ML
modules\cite{MacQueen:Modules} and by the module system for Scheme
Xerox\cite{Curtis-Rauen:Modules}.  It has also been shaped by the
needs of \hack{}, a virtual-machine-based Scheme implementation
designed to run both on workstations and on relatively small (less
than 1 Mbyte) embedded controllers.

Except where noted, everything described here is implemented in
\hack{}, and exercised by the \hack{} implementation and a few
application programs.

Unlike the Common Lisp package system, the module system described
here controls the mapping of names to denotations, not the
mapping of strings to symbols.


\subsection*{Introduction}

The module system supports the structured division of a corpus of
Scheme software into a set of modules.  Each module has its own
isolated namespace, with visibility of bindings controlled by module
descriptions written in a special {\em configuration language.}

A module may be instantiated multiple times, producing several {\em
packages}, just as a lambda-expression can be instantiated multiple
times to produce several different procedures.  Since single
instantiation is the normal case, I will defer discussion of multiple
instantiation until a later section.  For now you can think of a
package as simply a module's internal environment mapping names to
denotations.

A module exports bindings by providing views onto the underlying
package.  Such a view is called a {\em structure} (terminology from
Standard ML).  One module may provide several different views.  A
structure is just a subset of the package's bindings.  The particular
set of names whose bindings are exported is the structure's {\em
interface}.

A module imports bindings from other modules by either {\em opening}
or {\em accessing} some structures that are built on other packages.
When a structure is opened, all of its exported bindings are visible
in the client package.  On the other hand, bindings from an accessed
structure require explicitly qualified references written with the
{\tt structure-ref} operator.

For example:
\begin{code}
    (define-package ((foo (export a c cons)))
      (open scheme)
      (begin (define a 1)
	     (define (b x) (+ a x))
	     (define (c y) (* (b a) y))))
\codeskip
    (define-package ((bar (export d)))
      (open scheme foo)
      (begin (define (d w) (+ a (c w)))))
\end{code}
This configuration defines two structures, {\tt foo} and {\tt bar}.
{\tt foo} is a view on a package in which the {\tt scheme} structure's
bindings (including {\tt define} and {\tt +}) are visible, together
with bindings for {\tt a}, {\tt b},
and {\tt c}.  {\tt foo}'s interface is {\tt (export a c cons)}, so of
the bindings in its underlying package, {\tt foo} only exports those
three.  Similarly, structure {\tt bar} consists of the binding of {\tt
d} from a package in which both {\tt scheme}'s and {\tt foo}'s
bindings are visible.  {\tt foo}'s binding of {\tt cons} is imported
from the Scheme structure and then re-exported.

A module's body, the part following {\tt begin} in the above example,
is evaluated in an isolated lexical scope completely specified by the
package definition's {\tt open} and {\tt access} clauses.  In
particular, the binding of the syntactic operator {\tt define-package}
is not visible unless it comes from some opened structure.  Similarly,
bindings from the {\tt scheme} structure aren't visible unless they
become so by {\tt scheme} (or an equivalent structure) being opened.


\subsection*{The configuration language}

The configuration language consists of top-level defining forms for
modules and interfaces.  Its syntax is given in figure~1.

\setbox0\hbox{\goesto}
\newcommand{\altz}{\hbox to 1\wd0{\hfil\alt}}

%%%%% Put the figure inside a box ?

\begin{figure}
%\begin{frameit}
\begin{tabbing}
   \syn{configuration} \=\goesto{}~\arbno{\syn{definition}} \\
   \syn{definition} \=\goesto{}~
      \tt(define-package (\arbno{(\syn{name} \syn{interface})})
          \arbno{\syn{clause}}) \\
	\>\altz{}~ \tt(define-interface \syn{name} \syn{interface}) \\
	\>\altz{}~ \tt(define-syntax \syn{name} \syn{transformer-spec}) \\
   \syn{clause} \=\goesto{}~ \tt(open \arbno{\syn{name}}) \\
	\>\altz{}~ \tt(access \arbno{\syn{name}}) \\
	\>\altz{}~ \tt(begin \syn{program}) \\
	\>\altz{}~ \tt(files \arbno{\syn{filespec}}) \\
	\>\altz{}~ \tt(optimize \arbno{\syn{optimize-spec}}) \\
   \syn{interface} \=\goesto{}~ \tt(export \arbno{\syn{item}}) \\
	\>\altz{}~ \syn{name} \\
	\>\altz{}~ \tt(compound-interface \arbno{\syn{interface}}) \\
   \syn{item} \=\goesto{}~ \syn{name} ~\alt{}~ \tt(\syn{name} \syn{type})
\end{tabbing}
\caption{The configuration language.}
%\end{frameit}
\end{figure}


A {\tt define-package} form introduces bindings of some names to some
structures.  Each structure is a view on a package which is created
according to the clauses of the {\tt define-package} form.  Each
structure has a interface that specifies which bindings in the
structure's underlying package can be seen via that structure in other
packages.

An {\tt open} clause specifies which structures will be opened up for
use inside the new package.  At least one package must be specified or
else it will be impossible to write any useful programs inside the
package, since {\tt define}, {\tt lambda}, {\tt cons}, {\tt
structure-ref}, etc.\ will be unavailable.  Typical packages to list
in the {\tt open} clause are {\tt scheme}, which exports all bindings
appropriate to Revised$^5$ Scheme, and {\tt structure-refs}, which
exports the {\tt structure-ref} operator (see below).  For building
structures that export structures, there is a {\tt defpackage} package
that exports the operators of the configuration language.  Many other
structures, such as record and hash table facilities, are also
available in the \hack{} implementation.

An {\tt access} clause specifies which bindings of names to structures
will be visible inside the package body for use in {\tt structure-ref}
forms.  {\tt structure-ref} has the following syntax:
\begin{tabbing}
\qquad \syn{expression} \goesto{}~
   \tt(structure-ref \syn{struct-name} \syn{name})
\end{tabbing}
The \syn{struct-name} must be the name of an {\tt access}ed structure,
and \syn{name} must be something that the structure exports.  Only
structures listed in an {\tt access} clause are valid in a {\tt
structure-ref}.  If a package accesses any structures, it should
probably open the {\tt structure-refs} structure so that the {\tt
structure-ref} operator itself will be available.

The package's body is specified by {\tt begin} and/or {\tt files}
clauses.  {\tt begin} and {\tt files} have the same semantics, except
that for {\tt begin} the text is given directly in the package
definition, while for {\tt files} the text is stored somewhere in the
file system.  The body consists of a Scheme program, that is, a
sequence of definitions and expressions to be evaluated in order.  In
practice, I always use {\tt files} in preference to {\tt begin}; {\tt
begin} exists mainly for expository purposes.

A name's imported binding may be lexically overridden or {\em shadowed}
by simply defining the name using a defining form such as {\tt define}
or {\tt define-syntax}.  This will create a new binding without having
any effect on the binding in the opened package.  For example, one can
do {\tt(define car 'chevy)} without affecting the binding of the name
{\tt car} in the {\tt scheme} package.

Assignments (using {\tt set!})\ to imported and undefined variables
are not allowed.  In order to {\tt set!}\ a top-level variable, the
package body must contain a {\tt define} form defining that variable.
Applied to bindings from the {\tt scheme} structure, this restriction
is compatible with the requirements of the Revised$^5$ Scheme report.

It is an error for two of a package's opened structures to both export
the same name.  However, the current implementation does not check for
this situation; the binding is taken from the structure that is comes
first within the {\tt open} clause.

File names in a {\tt files} clause can be symbols, strings, or lists
(Maclisp-style ``namelists'').  A ``{\tt.scm}'' file type suffix is
assumed.  Symbols are converted to file names by converting to upper
or lower case as appropriate for the host operating system.  A
namelist is an operating-system-indepedent way to specify a file
obtained from a subdirectory.  For example, the namelist {\tt(rts
record)} specifies the file {\tt record.scm} in the {\tt rts}
subdirectory.

If the {\tt define-package} form was itself obtained from a file, then
file names in {\tt files} clauses are interpreted relative to the
directory in which the file containing the {\tt define-package} form
was found.


\subsection*{Interfaces}

A interface can be thought of as the type of a structure.  In its
basic form it is just a list of variable names, written {\tt(export
\var{name} \etc)}.  However, in place of
a name one may write {\tt(\var{name} \var{type})}, indicating the type
of \var{name}'s binding.  Currently the type field is ignored, except
that exported macros must be indicated with type {\tt syntax}.

Interfaces may be either anonymous, as in the example in the
introduction, or they may be given names by a {\tt define-interface}
form, for example
\begin{code}
    (define-interface foo-interface (export a c cons))
    (define-package ((foo foo-interface)) \etc)
\end{code}
In principle, interfaces needn't ever be named.  If a interface
had to be given at the point of a structure's use as well as at the
point of its definition, it would be important to name interfaces in
order to avoid having to write them out twice, with risk of mismatch
should the interface ever change.  But they don't.

Still, there are several reasons to use {\tt define-interface}:
\begin{enumerate}
\item It is important to separate the interface definition from the
package definitions when there are multiple distinct structures that
have the same interface --- that is, multiple implementations of the
same abstraction.

\item It is conceptually cleaner, and useful for documentation
purposes, to separate a module's specification (interface) from its
implementation (package).

\item My experience is that configurations that are separated into
interface definitions and package definitions are easier to read; the
long lists of exported bindings just get in the way most of the time.
\end{enumerate}

The {\tt compound-interface} operator forms a interface that is the
union of two or more component interfaces.  For example,
\begin{code}
    (define-interface bar-interface
      (compound-interface foo-interface (export mumble)))
\end{code}
defines {\tt bar-interface} to be {\tt foo-interface} with the name
{\tt mumble} added.


\subsection*{Macros}

Hygienic macros, as described in
\cite{Clinger-Rees:Macros,Clinger-Rees:R4RS}, are implemented.
Structures may export macros; auxiliary names introduced into the
expansion are resolved in the environment of the macro's definition.

For example, the {\tt scheme} structure's {\tt delay} macro 
is defined by the rewrite rule
\begin{code}
    (delay \var{exp})  \xform  (make-promise (lambda () \var{exp}))\rm.
\end{code}
The variable {\tt make-promise} is defined in the {\tt scheme}
structure's underlying package, but is not exported.  A use of the
{\tt delay} macro, however, always accesses the correct definition
of {\tt make-promise}.  Similarly, the {\tt case} macro expands into
uses of {\tt cond}, {\tt eqv?}, and so on.  These names are exported
by {\tt scheme}, but their correct bindings will be found even if they
are shadowed by definitions in the client package.


\subsection*{Higher-order modules}

There are {\tt define-module} and {\tt define-structure} forms for
defining modules that are intended to be instantiated multiple times.
But these are pretty kludgey --- for example, compiled code isn't
shared between the instantiations --- so I won't describe them yet.
If you must know, figure it out from the following grammar.
\begin{tabbing}
\qquad
   \syn{definition} \=\goesto{}~
      \tt(d\=\tt{}efine-module (\syn{name} \arbno{(\syn{name} \syn{interface})}) \\
       \>  \>\arbno{\syn{definition}} \\
       \>  \>\syn{name}\tt) \\
	\>\altz{}~ \tt(define-structure \syn{name}
		        (\syn{name} \arbno{\syn{name}}))
\end{tabbing}


\subsection*{Compiling and linking}

\hack{} has a static linker that produces stand-alone heap images
from module descriptions.  One specifies a particular procedure in a
particular structure to be the image's startup procedure (entry
point), and the linker traces dependency links as given by {\tt open}
and {\tt access} clauses to determine the composition of the heap
image.

There is not currently any provision for separate compilation; the
only input to the static linker is source code.  However, it will not
be difficult to implement separate compilation.  The unit of
compilation is one module (not one file).  Any opened or accessed
structures from which macros are obtained must be processed to the
extent of extracting its macro definitions.  The compiler knows from
the interface of an opened or accessed structure which of its exports
are macros.  Except for macros, a module may be compiled without any
knowledge of the implementation of its opened and accessed structures.
However, inter-module optimization will be available as an option.

The main difficulty with separate compilation is resolution of
auxiliary bindings introduced into macro expansions.  The module
compiler must transmit to the loader or linker the search path by
which such bindings are to be resolved.  In the case of the {\tt delay}
macro's auxiliary {\tt make-promise} (see example above), the loader
or linker needs to know that the desired binding of {\tt make-promise}
is the one apparent in {\tt delay}'s defining package, not in the
package being loaded or linked.

[I need to describe structure reification.]


\subsection*{Semantics of configuration mutation}

During program development it is often desirable to make changes to
packages and interfaces.  In static languages it may be necessary to
recompile and re-link a program in order for such changes to be
reflected in a running system.  Even in interactive Common Lisp
implementations, a change to a package's exports often requires
reloading clients that have already mentioned names whose bindings
change.  Once {\tt read} resolves a use of a name to a symbol, that
resolution is fixed, so a change in the way that a name resolves to a
symbol can only be reflected by re-{\tt read}ing all such references.

The \hack{} development environment supports rapid turnaround in
modular program development by allowing mutations to a program's
configuration, and giving a clear semantics to such mutations.  The
rule is that variable bindings in a running program are always
resolved according to current structure and interface bindings, even
when these bindings change as a result of edits to the configuration.
For example, consider the following:
\begin{code}
    (define-interface foo-interface (export a c))
    (define-package ((foo foo-interface))
      (open scheme)
      (begin (define a 1)
	     (define (b x) (+ a x))
	     (define (c y) (* (b a) y))))
    (define-package ((bar (export d)))
      (open scheme foo)
      (begin (define (d w) (+ (b w) a))))
\end{code}
This program has a bug.  The variable {\tt b}, which is free in the
definition of {\tt d}, has no binding in {\tt bar}'s package.  Suppose
that {\tt b} was supposed to be exported by {\tt foo}, but was omitted
from {\tt foo-interface} by mistake.  It is not necessary to
re-process {\tt bar} or any of {\tt foo}'s other clients at this point.
One need only change {\tt foo-interface} and inform the development
system of that one change (using, say, an appropriate Emacs command),
and {\tt foo}'s binding of {\tt b} will be found when procedure {\tt
d} is called.

Similarly, it is also possible to replace a structure; clients of the
old structure will be modified so that they see bindings from the new
one.  Shadowing is also supported in the same way.  Suppose that a
client package $C$ opens a structure {\tt foo} that exports a name
{\tt x}, and {\tt foo}'s implementation obtains the binding of {\tt x}
as an import from some other structure {\tt bar}.  Then $C$ will see
the binding from {\tt bar}.  If one then alters {\tt foo} so that it
shadows {\tt bar}'s binding of {\tt x} with a definition of its own,
then procedures in $C$ that reference {\tt x} will automatically see
{\tt foo}'s definition instead of the one from {\tt bar} that they saw
earlier.

This semantics might appear to require a large amount of computation
on every variable reference: The specified behavior requires scanning
the package's list of opened structures, examining their interfaces,
on every variable reference, not just at compile time.  However, the
development environment uses caching with cache invalidation to make
variable references fast.


\subsection*{Command processor support}

While it is possible to use the \hack{} static linker for program
development, it is far more convenient to use the development
environment, which supports rapid turnaround for program changes.  The
programmer interacts with the development environment through a {\em
command processor}.  The command processor is like the usual Lisp
read-eval-print loop in that it accepts Scheme forms to evaluate.
However, all meta-level operations, such as exiting the Scheme system
or requests for trace output, are handled by {\em commands,} which are
lexically distinguished from Scheme forms.  This arrangement is
borrowed from the Symbolics Lisp Machine system, and is reminiscent of
non-Lisp debuggers.  Commands are a little easier to type than Scheme
forms (no parentheses, so you don't have to shift), but more
importantly, making them distinct from Scheme forms ensures that
programs' namespaces aren't clutterred with inappropriate bindings.
Equivalently, the command set is available for use regardless of what
bindings happen to be visible in the current program.  This is
especially important in conjunction with the module system, which puts
strict controls on visibility of bindings.

The \hack{} command processor supports the module system with a
variety of special commands.  For commands that require structure
names, these names are resolved in a designated configuration package
that is distinct from the current package for evaluating Scheme forms
given to the command processor.  The command processor interprets
Scheme forms in a particular current package, and there are commands
that move the command processor between different packages.

Commands are introduced by a comma ({\tt,}) and end at the end of
line.  The command processor's prompt consists of the name of the
current package followed by a greater-than ({\tt>}).

\begin{list}{}{}{}

\item
\begin{code}
,config
\end{code}
    The {\tt,config} command sets the command processor's current package
    to be the current configuration package.  Forms entered at this point are
    interpreted as being configuration language forms, not Scheme forms.

\item
\begin{code}
,in \var{struct-name}
\end{code}
    The {\tt ,in} command moves the command processor to a specified
    structure's underlying package.  For example:
\begin{code}
    user> ,config
    config> (define-package ((foo (export a)))
	      (open scheme))
    config> ,in foo
    foo> (define a 13)
    foo> a
    13
\end{code}
    In this example the command processor starts in a package called
    {\tt user}, but the {\tt ,config} command moves it into the
    configuration package, which has the name {\tt config}.  The {\tt
    define-package} form binds, in {\tt config}, the name {\tt foo} to
    a structure that exports {\tt a}.  Finally, the command {\tt ,in
    foo} moves the command processor into structure {\tt foo}'s
    underlying package.

    A package's body isn't executed (evaluated) until the package is
    {\em loaded}, which is accomplished by the {\tt ,load-package}
    command.

\item
\begin{code}
,load-package \var{struct-name}
\end{code}
    The {\tt,load-package} command ensures that the specified structure's
    underlying package's program has been loaded.  This 
    consists of (1) recursively ensuring that the packages of any
    opened or accessed structures are loaded, followed by (2)
    executing the package's body as specified by its definition's {\tt
    begin} and {\tt files} forms.

\item
\begin{code}
,reload-package \var{struct-name}
\end{code}
    This command re-executes the structure's package's program.  It
    is most useful if the program comes from a file or files, when
    it will update the package's bindings after mutations to its
    source file.

\item
\begin{code}
,load \var{filespec} \etc
\end{code}
    The {\tt,load} command executes forms from the specified file or
    files in the current package.  {\tt,load \var{filespec}} is similar
    to {\tt(load "\var{filespec}")}
    except that the name {\tt load} needn't be bound in the current
    package to Scheme's {\tt load} procedure.

\item
\begin{code}
,structure \var{name} \var{interface}
\end{code}
    The {\tt,structure} command defines \var{name} in the
    configuration package to be a structure with interface
    \var{interface} based on the current package.

\item
\begin{code}
,open \var{struct-name}
\end{code}
    The {\tt,open} command opens a new structure in the current
    package, as if the package definition's {\tt open} clause
    had listed \var{struct-name}.  (Not currently implemented, but
    easy to add.)
\end{list}


A number of other commands are not strictly necessary, but they
capture common idioms.

\begin{list}{}{}{}
\item
\begin{code}
,load-config \var{filespec} \etc
\end{code}
    The {\tt,load-config} command loads configuration language forms from
    the specified file or files into the current configuration package.
    Similar to {\tt,config} followed by {\tt,load} except that the current
    package is unchanged afterwards.

\item
\begin{code}
,in \var{struct-name} \var{comand}
\end{code}
    This form of the {\tt,in} command evaluates a single comand in the
    specified package without moving the command processor into that
    package.  Example:
\begin{code}
    ,in mumble (cons 1 2)
    ,in mumble ,trace foo
\end{code}

\item
\begin{code}
,new-package \var{name} \var{open} \etc
\end{code}
    The {\tt,new-package} command creates a new package.  It
    abbreviates the sequence 
\begin{code}
    > ,in config
    config> (define-package ((\var{name} (export)))
	      (open \var{open} \etc{} scheme))
    config> ,in name
    name> 
\end{code}

\end{list}



\subsection*{Configuration packages}

It is possible to set up multiple configuration packages.
Configuration packages open the {\tt defpackage} structure instead of
the {\tt scheme} structure, and in addition use structures that export
structures and interfaces.  The initial configuration package opens
the two structures {\tt built-in-structures} and {\tt
more-structures}.  The {\tt built-in-structures} structure exports
most of the basic structures, including {\tt scheme}, {\tt table},
{\tt record}, and {\tt compiler}.  {\tt more-structures} exports
structures involved in the programming environment and optional
features such as {\tt sort}, {\tt random}, and {\tt threads}.

For example:
\begin{code}
    > ,new-package foo defpackage built-in-structures more-structures
    foo> (define-package ((x (export a b)))
           (open scheme)
	   (files x))
    foo> 
\end{code}


\begin{list}{}{}{}
\item
\begin{code}
,config-package-is \var{struct-name}
\end{code}
    The {\tt,config-package-is} command designates a new configuration
    package for use by the {\tt,config} command and resolution of
    \var{struct-name}s for other commands.
\end{list}




\subsection*{Discussion}

This module system was not designed as the be-all and end-all of
Scheme module systems; it was only intended to help Richard Kelsey and
me to organize the \hack{} system.  Not only does the module system
help avoid name clashes by keeping different subsystems in different
namespaces, it has also helped us to tighten up and generalize
\hack{}'s internal interfaces.  \hack{} is unusual among Lisp
implementations in admitting many different possible modes of
operation.  Examples of such multiple modes include the following:
\begin{itemize}
    \item Linking can be either static or dynamic.

    \item The development environment (compiler, debugger, and command
    processor) can run either in the same address space as the program
    being developed or in a different address space.  The environment and
    user program may even run on different processors under different
    operating systems\cite{Rees-Donald:Program}.

    \item The virtual machine can be supported by either
    of two implementations of its implementation language, Prescheme.
\end{itemize}
The module system has been helpful in organizing these multiple modes.
By forcing us to write down interfaces and module dependencies, the
module system helps us to keep the system clean, or at least to keep
us honest about how clean or not it is.

The need to make structures and interfaces second-class instead of
first-class results from the requirements of static program analysis:
it must be possible for the compiler and linker to expand macros and
resolve variable bindings before the program is executed.  Structures
could be made first-class (as in FX\cite{Sheldon-Gifford:Static}) if a
type system were added to Scheme and the definitions of exported
macros were defined in interfaces instead of in module bodies, but
even in that case types and interfaces would remain second-class.

The prohibition on assignment to imported bindings makes substitution
a valid optimization when a module is compiled as a block.  The block
compiler first scans the entire module body, noting which variables
are assigned.  Those that aren't assigned (only {\tt define}d) may be
assumed never assigned, even if they are exported.  The optimizer can
then perform a very simple-minded analysis to determine automatically
that some procedures can and should have their calls compiled in line.

The programming style encouraged by the module system is consistent
with the unextended Scheme language.  Because module system features
do not generally show up within module bodies, an individual module
may be understood by someone who is not familiar with the module
system.  This is a great aid to code presentation and portability.  If
a few simple conditions are met (no name conflicts between packages,
no use of {\tt structure-ref}, and use of {\tt files} in preference to
{\tt begin}), then a multi-module program can be loaded into a Scheme
implementation that does not support the module system.  The \hack{}
static linker satisfies these conditions, and can therefore run in
other Scheme implementations.  \hack{}'s bootstrap process, which is
based on the static linker, is therefore nonincestuous.  This
contrasts with most other integrated programming environments, such as
Smalltalk-80, where the system can only be built using an existing
version of the system itself.

Like ML modules, but unlike Scheme Xerox modules, this module system
is compositional.  That is, structures are constructed by single
syntactic units that compose existing structures with a body of code.
In Scheme Xerox, the set of modules that can contribute to an
interface is open-ended --- any module can contribute bindings to any
interface whose name is in scope.  The module system implementation is
a cross-bar that channels definitions from modules to interfaces.  The
module system described here has simpler semantics and makes
dependencies easier to trace.  It also allows for higher-order
modules, which Scheme Xerox considers unimportant.

%[Discuss use of module system in the \hack{} implementation?  Maybe
%give an extended excerpt from \hack{}'s configuration files?]
%
%[Discuss or flush OPTIMIZE clause.]
%
%[Future work: ideas for anonymous structures and more of a module
%calculus; dealing with name conflicts; interface subtraction.]


\begin{thebibliography}{10}

\bibitem{Clinger-Rees:Macros}
William Clinger and Jonathan~Rees.
\newblock Macros that work.
\newblock {\em Principles of Programming Languages}, January 1991.

\bibitem{Clinger-Rees:R4RS}
William Clinger and Jonathan~Rees (editors).
\newblock Revised${}^4$ report on the algorithmic language {S}cheme.
\newblock {\em LISP Pointers} IV(3):1--55, July-September 1991.

\bibitem{Curtis-Rauen:Modules}
Pavel Curtis and James Rauen.
\newblock A module system for Scheme.
\newblock {\em ACM Conference on Lisp and Functional Programming,}
pages 13--19, 1990.

\bibitem{MacQueen:Modules}
David MacQueen.
\newblock Modules for Standard ML.
\newblock {\em ACM Conference on Lisp and Functional Programming,}
1984.

\bibitem{Rees-Donald:Program}
Jonathan Rees and Bruce Donald.
\newblock Program mobile robots in Scheme.
\newblock {\em International Conference on Robotics and
Automation,} IEEE, 1992. 

\bibitem{Sheldon-Gifford:Static}
Mark A.~Sheldon and David K.~Gifford.
\newblock Static dependent types for first-class modules.
\newblock {\em ACM Conference on Lisp and Functional Programming,}
pages 20--29, 1990.

\end{thebibliography}


\end{document}


% Still to do:
%  destructure
%  format (?)

\chapter{Libraries}

Use the
\code{,open} command (section~\ref{module-command-guide})
 or
the module language (chapter~\ref{module-guide})
 to open the structures described below.

\section{General utilities}
\label{big-util}

%% These are missing:
%%        copy-string
%%        string->immutable-string
%%        error
%%        breakpoint

These are in the \code{big-util} structure.

\begin{protos}
\proto{atom?}{ value}{boolean}
\end{protos}
%
\code{(atom? \var{x})} is the same as \code{(not (pair? \var{x}))}.

\begin{protos}
\proto{null-list?}{ list}{boolean}
\end{protos}
%
Returns true for the empty list, false for a pair, and signals an
error otherwise.

\begin{protos}
\proto{neq?}{ value value}{boolean}
\end{protos}
\code{(neq? \var{x} \var{y})} is the same as \code{(not (eq? \var{x}
\var{y}))}.

\begin{protos}
\proto{n=}{ number number}{boolean}
\end{protos}
\code{(n= \var{x} \var{y})} is the same as \code{(not (= \var{x}
  \var{y}))}.

\begin{protos}
\proto{identity}{ value}{value}
\proto{no-op}{ value}{value}
\end{protos}
These both just return their argument.  \code{No-op} is guaranteed not to
be compiled in-line, \code{identity} may be.

\begin{protos}
\proto{memq?}{ value list}{boolean}
\end{protos}
%
Returns true if \var{value} is in \var{list}, false otherwise.

\begin{protos}
\proto{any?}{ predicate list}{boolean}
\end{protos}
Returns true if \var{predicate} is true for any element of \var{list}.

\begin{protos}
\proto{every?}{ predicate list}{boolean}
\end{protos}
  Returns true if \var{predicate} is true for every element of \var{list}.

\begin{protos}
\proto{any}{ predicate list}{value}
\proto{first}{ predicate list}{value}
\end{protos}
\code{Any} returns some element of \var{list} for which \var{predicate} is true, or
false if there are none.  \code{First} does the same except that it returns
the first element for which \var{predicate} is true.

\begin{protos}
\proto{filter}{ predicate list}{list}
\protonoindex{filter!}{ predicate list}{list}\mainschindex{filter"!}
\end{protos}
Returns a list containing all of the elements of \var{list} for which
\var{predicate} is true.  The order of the elements is preserved.
\code{Filter!} may reuse the storage of \var{list}.

\begin{protos}
\proto{filter-map}{ procedure list}{list}
\end{protos}
The same as \code{filter} except the returned list contains the results of
applying \var{procedure} instead of elements of \var{list}.  \code{(filter-map \var{p}
\var{l})} is the same as \code{(filter identity (map \var{p} \var{l}))}.

\begin{protos}
\proto{partition-list}{ predicate list}{list list}
\protonoindex{partition-list!}{ predicate list}{list list}\mainschindex{partition-list"!}
\end{protos}
The first return value contains those elements \var{list} for which
\var{predicate} is true, the second contains the remaining elements.
The order of the elements is preserved.  \code{Partition-list!} may
reuse the storage of the \var{list}.

\begin{protos}
\proto{remove-duplicates}{ list}{list}
\end{protos}
Returns its argument with all duplicate elements removed.  The first
instance of each element is preserved.

\begin{protos}
\proto{delq}{ value list}{list}
\protonoindex{delq!}{ value list}{list}\mainschindex{delq"!}
\proto{delete}{ predicate list}{list}
\end{protos}
All three of these return \var{list} with some elements removed.
\code{Delq} removes all elements \code{eq?} to \var{value}.  \code{Delq!}
does the same and may modify the list argument.  \code{Delete} removes
all elements for which \var{predicate} is true.  Both \code{delq} and
\code{delete} may reuse some of the storage in the list argument, but
won't modify it.

\begin{protos}
\protonoindex{reverse!}{ list}{list}\mainschindex{reverse"!}
\end{protos}
Destructively reverses \var{list}.

\begin{protos}
\proto{concatenate-symbol}{ value \ldots}{symbol}
\end{protos}
Returns the symbol whose name is produced by concatenating the
\code{display}ed
representations of \var{value}~\ldots.

\begin{example}
(concatenate-symbol 'abc "-" 4) \(\Longrightarrow\) 'abc-4
\end{example}

\section{Pretty-printing}

These are in the \code{pp} structure.

\begin{protos}
\protonoresult{p}{ value}
\protonoresult{p}{ value output-port}
\protonoresult{pretty-print}{ value output-port position}
\end{protos}
Pretty-print \var{value} The current output port is used if no port is
specified.  \var{Position} is the starting offset.  \var{Value} will be
pretty-printed to the right of this column.

\section{Bitwise integer operations}

These functions use the two's-complement representation for integers.
There is no limit to the number of bits in an integer.
They are in the structures \code{bitwise} and \code{big-scheme}.

\begin{protos}
\proto{bitwise-and}{ integer integer}{integer}
\proto{bitwise-ior}{ integer integer}{integer}
\proto{bitwise-xor}{ integer integer}{integer}
\proto{bitwise-not}{ integer} {integer}
\end{protos}
\noindent
These perform various logical operations on integers on a bit-by-bit
basis. `\code{ior}' is inclusive OR and `\code{xor}' is exclusive OR.

\begin{protos}
\proto{arithmetic-shift}{ integer bit-count}{integer}
\end{protos}
\noindent Shifts the integer by the given bit count, which must be an integer,
 shifting left for positive counts and right for negative ones.
Shifting preserves the integer's sign.

\begin{protos}
\proto{bit-count}{ integer}{integer}
\end{protos}
\noindent Counts the number of bits set in the integer.
If the argument is negative a bitwise NOT operation is performed
 before counting.

\section{Byte vectors}

These are homogeneous vectors of small integers ($0 \le i \le 255$).
The functions that operate on them are analogous to those for vectors.
They are in the structure \code{byte-vectors}.

\begin{protos}
\proto{byte-vector?}{ value}{boolean}
\proto{make-byte-vector}{ k fill}{byte-vector}
\proto{byte-vector}{ b \ldots}{byte-vector}
\proto{byte-vector-length}{ byte-vector}{integer}
\proto{byte-vector-ref}{ byte-vector k}{integer}
\protonoresultnoindex{byte-vector-set!}{ byte-vector k b}\mainschindex{byte-vector-set"!}
\end{protos}

\section{Sparse vectors}

These are vectors that grow as large as they need to.  That is, they
can be indexed by arbitrarily large nonnegative integers.  The
implementation allows for arbitrarily large gaps by arranging the
entries in a tree.  They are in the structure \code{sparse-vectors}.

\begin{protos}
\proto{make-sparse-vector}{}{sparse-vector}
\proto{sparse-vector-ref}{ sparse-vector k}{value}
\protonoresultnoindex{sparse-vector-set!}{ sparse-vector k value}\mainschindex{sparse-vector-set"!}
\proto{sparse-vector->list}{ sparse-vector}{list}
\end{protos}
%
\code{Make-sparse-vector}, \code{sparse-vector-ref}, and
\code{sparse-vector-set!} are analogous to \code{make-vector},
\code{vector-ref}, and \code{vector-set!}, except that the indices
passed to \code{sparse-vector-ref} and \code{sparse-vector-set!} can
be arbitrarily large.  For indices whose elements have not been set in
a sparse vector, \code{sparse-vector-ref} returns \code{\#f}.

\code{Sparse-vector->list} is for debugging: It returns a list of the
consecutive elements in a sparse vector from 0 to the highest element
that has been set.  Note that the list will also include all the
\code{\#f} elements for the unset elements.


\section{Cells}
\label{cells}

These hold a single value and are useful when a simple indirection is
 required.
The system uses these to hold the values of lexical variables that
 may be \code{set!}.

\begin{protos}
\proto{cell?}{ value}{boolean}
\proto{make-cell}{ value}{cell}
\proto{cell-ref}{ cell}{value}
\protonoresultnoindex{cell-set!}{ cell value}\mainschindex{cell-set"!}
\end{protos}

\section{Queues}

These are ordinary first-in, first-out queues.
The procedures are in structure \code{queues}.

\begin{protos}
\proto{make-queue}{}{queue}
\proto{queue?}{ value}{boolean}
\proto{queue-empty?}{ queue}{boolean}
\protonoresultnoindex{enqueue!}{ queue value}\mainschindex{enqueue"!}
\protonoindex{dequeue!}{ queue}{value}\mainschindex{dequeue"!}
\end{protos}
\noindent 
\code{Make-queue} creates an empty queue, \code{queue?} is a predicate for
 identifying queues, \code{queue-empty?} tells you if a queue is empty,
 \code{enqueue!} and \code{dequeue!} add and remove values.

\begin{protos}
\proto{queue-length}{ queue}{integer}
\proto{queue->list}{ queue}{values}
\proto{list->queue}{ values}{queue}
\protonoindex{delete-from-queue!}{ queue value}{boolean}\mainschindex{delete-from-queue"!}
\end{protos}
\noindent
\code{Queue-length} returns the number of values in \var{queue}.
\code{Queue->list} returns the values in \var{queue} as a list, in the
 order in which the values were added.
\code{List->queue} returns a queue containing \var{values}, preserving
 their order.
\code{Delete-from-queue} removes the first instance of \var{value} from
 \code{queue}, using \code{eq?} for comparisons.
\code{Delete-from-queue} returns \code{\#t} if \var{value} is found and
 \code{\#f} if it is not.

\section{Arrays}

These provide N-dimensional, zero-based arrays and
 are in the structure \code{arrays}.
The array interface is derived from one invented by Alan Bawden.

\begin{protos}
\proto{make-array}{ value dimension$_0$ \ldots}{array}
\proto{array}{ dimensions element$_0$ \ldots}{array}
\proto{copy-array}{ array}{array}
\end{protos}
\noindent
\code{Make-array} makes a new array with the given dimensions, each of which
 must be a non-negative integer.
Every element is initially set to \cvar{value}.
\code{Array} Returns a new array with the given dimensions and elements.
\cvar{Dimensions} must be a list of non-negative integers, 
The number of elements should be the equal to the product of the
 dimensions.
The elements are stored in row-major order.
\begin{example}
(make-array 'a 2 3) \evalsto \{Array 2 3\}

(array '(2 3) 'a 'b 'c 'd 'e 'f)
    \evalsto \{Array 2 3\}
\end{example}

\code{Copy-array} returns a copy of \cvar{array}.
The copy is identical to the \cvar{array} but does not share storage with it.

\begin{protos}
\proto{array?}{ value}{boolean}
\end{protos}
\noindent
Returns \code{\#t} if \cvar{value} is an array.

\begin{protos}
\proto{array-ref}{ array index$_0$ \ldots}{value}
\protonoresultnoindex{array-set!}{ array value index$_0$ \ldots}\mainschindex{array-set"!}
\proto{array->vector}{ array}{vector}
\proto{array-dimensions}{ array}{list}
\end{protos}
\noindent
\code{Array-ref} returns the specified array element and \code{array-set!}
 replaces the element with \cvar{value}.
\begin{example}
(let ((a (array '(2 3) 'a 'b 'c 'd 'e 'f)))
  (let ((x (array-ref a 0 1)))
    (array-set! a 'g 0 1)
    (list x (array-ref a 0 1))))
    \evalsto '(b g)
\end{example}

\code{Array->vector} returns a vector containing the elements of \cvar{array}
 in row-major order.
\code{Array-dimensions} returns the dimensions of
 the array as a list.

\begin{protos}
\proto{make-shared-array}{ array linear-map dimension$_0$ \ldots}{array}
\end{protos}
\noindent
\code{Make-shared-array} makes a new array that shares storage with \cvar{array}
 and uses \cvar{linear-map} to map indexes to elements.
\cvar{Linear-map} must accept as many arguments as the number of
 \cvar{dimension}s given and must return a list of non-negative integers
 that are valid indexes into \cvar{array}.
<\begin{example}
(array-ref (make-shared-array a f i0 i1 ...)
           j0 j1 ...)
\end{example}
is equivalent to
\begin{example}
(apply array-ref a (f j0 j1 ...))
\end{example}

As an example, the following function makes the transpose of a two-dimensional
 array:
\begin{example}
(define (transpose array)
  (let ((dimensions (array-dimensions array)))
    (make-shared-array array
                       (lambda (x y)
                         (list y x))
                       (cadr dimensions)
                       (car dimensions))))

(array->vector
  (transpose
    (array '(2 3) 'a 'b 'c 'd 'e 'f)))
      \evalsto '(a d b e c f)
\end{example}

\section{Records}
\label{records}

New types can be constructed using the \code{define-record-type} macro
 from the \code{define-record-types} structure
The general syntax is:
\begin{example}
(define-record-type \cvar{tag} \cvar{type-name}
  (\cvar{constructor-name} \cvar{field-tag} \ldots)
  \cvar{predicate-name}
  (\cvar{field-tag} \cvar{accessor-name} [\cvar{modifier-name}])
  \ldots)
\end{example}
This makes the following definitions:
\begin{protos}
\constprotonoindex{\cvar{type-name}}{type}
\protonoindex{\cvar{constructor-name}}{ field-init \ldots}{type-name}
\protonoindex{\cvar{predicate-name}}{ value}{boolean}
\protonoindex{\cvar{accessor-name}}{ type-name}{value}
\protonoresultnoindex{\cvar{modifier-name}}{ type-name value}
\end{protos}
\noindent
\cvar{Type-name} is the record type itself, and can be used to
 specify a print method (see below).
\cvar{Constructor-name} is a constructor that accepts values
 for the fields whose tags are specified.
\cvar{Predicate-name} is a predicate that returns \code{\#t} for
 elements of the type and \code{\#f} for everything else.
The \cvar{accessor-name}s retrieve the values of fields,
 and the \cvar{modifier-name}'s update them.
\cvar{Tag} is used in printing instances of the record type and
 the \cvar{field-tag}s are used in the inspector and to match
 constructor arguments with fields.

\begin{protos}
\protonoresult{define-record-discloser}{ type discloser}
\end{protos}
\noindent
\code{Define-record-discloser} determines how
 records of type \cvar{type} are printed.
\cvar{Discloser} should be procedure which takes a single
 record of type \cvar{type} and returns a list whose car is
 a symbol.
The record will be printed as the value returned by \cvar{discloser}
 with curly braces used instead of the usual parenthesis.

For example
\begin{example}
(define-record-type pare :pare
  (kons x y)
  pare?
  (x kar set-kar!)
  (y kdr))
\end{example}
 defines \code{kons} to be a constructor, \code{kar} and \code{kdr} to be
 accessors, \code{set-kar!} to be a modifier, and \code{pare?} to be a predicate
 for a new type of object.
The type itself is named \code{:pare}.
\code{Pare} is a tag used in printing the new objects.

By default, the new objects print as \code{\#\{Pare\}}.
The print method can be modified using \code{define-record-discloser}:
\begin{example}
(define-record-discloser :pare
  (lambda (p) `(pare ,(kar p) ,(kdr p))))
\end{example}
 will cause the result of \code{(kons 1 2)} to print as
 \code{\#\{Pare 1 2\}}.

\code{Define-record-resumer} (section~\ref{sec:hibernation})
 can be used to control how records are stored in heap images.

\subsection{Low-level access to records}

Records are implemented using primitive objects exactly analogous
 to vectors.
Every record has a record type (which is another record) in the first slot.
Note that use of these procedures, especially \code{record-set!}, breaks
 the record abstraction described above; caution is advised.

These procedures are in the structure \code{records}.

\begin{protos}
\proto{make-record}{ n value}{record}
\proto{record}{ value \ldots}{record-vector}
\proto{record?}{ value}{boolean}
\proto{record-length}{ record}{integer}
\proto{record-type}{ record}{value}
\proto{record-ref}{ record i}{value}
\protonoresultnoindex{record-set!}{ record i value}\mainschindex{record-set"!}
\end{protos}
\noindent
These the same as the standard \code{vector-} procedures except that they
 operate on records.
The value returned by \code{record-length} includes the slot holding the
 record's type.
\code{(record-type \cvar{x})} is equivalent to \code{(record-ref \cvar{x} 0)}.

\subsection{Record types}

Record types are themselves records of a particular type (the first slot
 of \code{:record-type} points to itself).
A record type contains four values: the name of the record type, a list of
 the names its fields, and procedures for disclosing and resuming records
 of that type.
Procedures for manipulating them are in the structure \code{record-types}.

\begin{protos}
\proto{make-record-type}{ name field-names}{record-type}
\proto{record-type?}{ value}{boolean}
\proto{record-type-name}{ record-type}{symbol}
\proto{record-type-field-names}{ record-type}{symbols}
\end{protos}
\noindent

\begin{protos}
\proto{record-constructor}{ record-type field-names}{procedure}
\proto{record-predicate}{ record-type}{procedure}
\proto{record-accessor}{ record-type field-name}{procedure}
\proto{record-modifier}{ record-type field-name}{procedure}
\end{protos}
\noindent
These procedures construct the usual record-manipulating procedures.
\code{Record-constructor} returns a constructor that is passed the initial
 values for the fields specified and returns a new record.
\code{Record-predicate} returns a predicate that return true when passed
 a record of type \cvar{record-type} and false otherwise.
\code{Record-accessor} and \code{record-modifier} return procedures that
 reference and set the given field in records of the approriate type.

\begin{protos}
\protonoresult{define-record-discloser}{ record-type discloser}
\protonoresult{define-record-resumer}{ record-type resumer}
\end{protos}
\noindent
\noindent \code{Record-types} is the initial exporter of
 \code{define-record-discloser}
 (re-exported by \code{define-record-types} described above)
 and
 \code{define-record-resumer}
 (re-exported by
 \code{external-calls} (section~\ref{sec:hibernation})).

The procedures described in this section can be used to define new
 record-type-defining macros.
\begin{example}
(define-record-type pare :pare
  (kons x y)
  pare?
  (x kar set-kar!)
  (y kdr))
\end{example}
is (sematically) equivalent to
\begin{example}
(define :pare (make-record-type 'pare '(x y)))
(define kons (record-constructor :pare '(x y)))
(define kar (record-accessor :pare 'x))
(define set-kar! (record-modifier :pare 'x))
(define kdr (record-accessor :pare 'y))
\end{example}

The ``(semantically)'' above is because \code{define-record-type} adds
 declarations, which allows the type checker to detect some misuses of records,
 and uses more efficient definitions for the constructor, accessors, and
 modifiers.
Ignoring the declarations, which will have to wait for another edition of
 the manual, what the above example actually expands into is:
\begin{example}
(define :pare (make-record-type 'pare '(x y)))
(define (kons x y) (record :pare x y))
(define (kar r) (checked-record-ref r :pare 1))
(define (set-kar! r new)
  (checked-record-set! r :pare 1 new))
(define (kdr r) (checked-record-ref r :pare 2))
\end{example} 
\code{Checked-record-ref} and \code{Checked-record-set!} are
 low-level procedures that check the type of the
 record and access or modify it using a single VM instruction.

\section{Finite record types}

The structure \code{finite-types} has
 two macros for defining `finite' record types.
These are record types for which there are a fixed number of instances,
 all of which are created at the same time as the record type itself.
The syntax for defining an enumerated type is:
\begin{example}
(define-enumerated-type \cvar{tag} \cvar{type-name}
  \cvar{predicate-name}
  \cvar{vector-of-instances-name}
  \cvar{name-accessor}
  \cvar{index-accessor}
  (\cvar{instance-name} \ldots))
\end{example}
This defines a new record type, bound to \cvar{type-name}, with as many
 instances as there are \cvar{instance-name}'s.
\cvar{Vector-of-instances-name} is bound to a vector containing the instances
 of the type in the same order as the \cvar{instance-name} list.
\cvar{Tag} is bound to a macro that when given an \cvar{instance-name} expands
 into an expression that returns corresponding instance.
The name lookup is done at macro expansion time.
\cvar{Predicate-name} is a predicate for the new type.
\cvar{Name-accessor} and \cvar{index-accessor} are accessors for the
 name and index (in \cvar{vector-of-instances}) of instances of the type.

\begin{example}
(define-enumerated-type color :color
  color?
  colors
  color-name
  color-index
  (black white purple maroon))

(color-name (vector-ref colors 0)) \evalsto black
(color-name (color white))         \evalsto white
(color-index (color purple))       \evalsto 2
\end{example}

Finite types are enumerations that allow the user to add additional
 fields in the type.
The syntax for defining a finite type is:
\begin{example}
(define-finite-type \cvar{tag} \cvar{type-name}
  (\cvar{field-tag} \ldots)
  \cvar{predicate-name}
  \cvar{vector-of-instances-name}
  \cvar{name-accessor}
  \cvar{index-accessor}
  (\cvar{field-tag} \cvar{accessor-name} [\cvar{modifier-name}])
  \ldots
  ((\cvar{instance-name} \cvar{field-value} \ldots)
   \ldots))
\end{example}
The additional fields are specified exactly as with \code{define-record-type}.
The field arguments to the constructor are listed after the \cvar{type-name};
 these do not include the name and index fields.
The form ends with the names and the initial field values for
 the instances of the type.
The instances are constructed by applying the (unnamed) constructor to
 these initial field values.
The name must be first and 
 the remaining values must match the \cvar{field-tag}s in the constructor's
 argument list.

%This differs from \code{define-record-type} in the following ways:
%\begin{itemize}
%\item No name is specified for the constructor, but the field arguments
% to the constructor are listed.
%\item The \cvar{vector-of-instances-name} is added; it will be bound
% to a vector containing all of the instances of the type.
%These are constructed by applying the (unnamed) constructor to the
% initial field values at the end of the form.
%\item There are names for accessors for two required fields, name
% and index.
%These fields are not settable, and are not to be included
% in the argument list for the constructor.
%\item The form ends with the names and the initial field values for
% the instances of the type.
%The name must be first.
%The remaining values must match the \cvar{field-tag}s in the constructor's
% argument list.
%\item \cvar{Tag} is bound to a macro that maps \cvar{instance-name}s to the
% the corresponding instance of the vector.
%The name lookup is done at macro-expansion time.
%\end{itemize}

\begin{example}
(define-finite-type color :color
  (red green blue)
  color?
  colors
  color-name
  color-index
  (red   color-red)
  (green color-green)
  (blue  color-blue)
  ((black    0   0   0)
   (white  255 255 255)
   (purple 160  32 240)
   (maroon 176  48  96)))

(color-name (color black))         \evalsto black
(color-name (vector-ref colors 1)) \evalsto white
(color-index (color purple))       \evalsto 2
(color-red (color maroon))         \evalsto 176
\end{example}

\section{Sets over finite types}
\label{sets-finite-types}

The structure \code{enum-sets} has a macro for defining types for sets
of elements of finite types.  These work naturally with the finite
types defined by the \code{finite-types} structure, but are not tied
to them.  The syntax for defining such a type is:

\begin{example}
(define-enum-set-type \cvar{id} \cvar{type-name} \cvar{predicate} \cvar{constructor}
   \cvar{element-syntax} \cvar{element-predicate} \cvar{all-elements} \cvar{element-index-ref})
\end{example}
%
This defines \cvar{id} to be syntax for constructing sets,
\cvar{type-name} to be a value representing the type,
\cvar{predicate} to be a predicate for those sets, and
\cvar{constructor} a procedure for constructing one from a list.

\cvar{Element-syntax} must be the name of a macro for constructing set
elements from names (akin to the \cvar{tag} argument to
\code{define-enumerated-type}).  \cvar{Element-predicate} must be a
predicate for the element type, \cvar{all-elements} a vector of all
values of the element type, and \cvar{element-index-ref} must return
the index of an element within the \cvar{all-elements} vector.

\begin{protos}
\proto{enum-set->list}{ enum-set}{list}
\proto{enum-set-member?}{ enum-set enumerand}{boolean}
\proto{enum-set=?}{ enum-set enum-set}{boolean}
\proto{enum-set-union}{ enum-set enum-set}{enum-set}
\proto{enum-set-intersection}{ enum-set enum-set}{ enum-set}
\proto{enum-set-negation}{ enum-set}{enum-set}
\end{protos}
%
\code{Enum-set->list} converts a set into a list of its elements.
\code{Enum-set-member?} tests for membership.  \code{Enum-set=?} tests
two sets of equal type for equality.  (If its arguments are not of the
same type, \code{enum-set=?} raises an exception.)
\code{Enum-set-union} computes the union of two sets of equal type,
\code{enum-set-intersection} computes the intersection, and
\code{enum-set-negation} computes the complement of a set.

Here is an example.  Given an enumerated type:

\begin{example}
(define-enumerated-type color :color
  color?
  colors
  color-name
  color-index
  (red blue green))
\end{example}

we can define sets of colors:

\begin{example}
(define-enum-set-type color-set :color-set
                      color-set?
                      make-color-set
  color color? colors color-index)
\end{example}

\begin{example}
> (enum-set->list (color-set red blue))
(#\ob{}Color red\cb{} #\ob{}Color blue\cb{})
> (enum-set->list (enum-set-negation (color-set red blue)))
(#\ob{}Color green\cb{})
> (enum-set-member? (color-set red blue) (color blue))
#t
\end{example}

\section{Hash tables}

These are generic hash tables, and are in the structure \code{tables}.
Strictly speaking they are more maps than tables, as every table has a
 value for every possible key (for that type of table).
All but a finite number of those values are \code{\#f}.

\begin{protos}
\proto{make-table}{}{table}
\proto{make-symbol-table}{}{symbol-table}
\proto{make-string-table}{}{string-table}
\proto{make-integer-table}{}{integer-table}
\proto{make-table-maker}{ compare-proc hash-proc}{procedure}
\protonoresultnoindex{make-table-immutable!}{ table}\mainschindex{make-table-immutable"!}
\end{protos}
\noindent
The first four functions listed make various kinds of tables.
\code{Make-table} returns a table whose keys may be symbols, integer,
 characters, booleans, or the empty list (these are also the values
 that may be used in \code{case} expressions).
As with \code{case}, comparison is done using \code{eqv?}.
The comparison procedures used in symbol, string, and integer tables are
 \code{eq?}, \code{string=?}, and \code{=}.

\code{Make-table-maker} takes two procedures as arguments and returns
 a nullary table-making procedure.
\cvar{Compare-proc} should be a two-argument equality predicate.
\cvar{Hash-proc} should be a one argument procedure that takes a key
 and returns a non-negative integer hash value.
If \code{(\cvar{compare-proc} \cvar{x} \cvar{y})} returns true,
 then \code{(= (\cvar{hash-proc} \cvar{x}) (\cvar{hash-proc} \cvar{y}))}
 must also return true.
For example, \code{make-integer-table} could be defined
 as \code{(make-table-maker = abs)}.

\code{Make-table-immutable!} prohibits future modification to its argument.

\begin{protos}
\proto{table?}{ value}{boolean}
\proto{table-ref}{ table key}{value or {\tt \#f}}
\protonoresultnoindex{table-set!}{ table key value}\mainschindex{table-set"!}
\protonoresult{table-walk}{ procedure table}
\end{protos}
\noindent
\code{Table?} is the predicate for tables.
\code{Table-ref} and \code{table-set!} access and modify the value of \cvar{key}
 in \cvar{table}.
\code{Table-walk} applies \cvar{procedure}, which must accept two arguments,
 to every associated key and non-\code{\#f} value in \code{table}.

\begin{protos}
\proto{default-hash-function}{ value}{integer}
\proto{string-hash}{ string}{integer}
\end{protos}
\noindent
\code{Default-hash-function} is the hash function used in the tables
 returned by \code{make-table}, and \code{string-hash} it the one used
 by \code{make-string-table}.

\section{Port extensions}

These procedures are in structure \code{extended-ports}.

\begin{protos}
\proto{make-string-input-port}{ string}{input-port}
\proto{make-string-output-port}{}{output-port}
\proto{string-output-port-output}{ string-output-port}{string}
\end{protos}
\noindent \code{Make-string-input-port} returns an input port that
 that reads characters from the supplied string.  An end-of-file
 object is returned if the user reads past the end of the string.
\code{Make-string-output-port} returns an output port that saves
 the characters written to it.
These are then returned as a string by \code{string-output-port-output}.

\begin{example}
(read (make-string-input-port "(a b)"))
    \evalsto '(a b)

(let ((p (make-string-output-port)))
  (write '(a b) p)
  (let ((s (string-output-port-output p)))
    (display "c" p)
    (list s (string-output-port-output p))))
    \evalsto '("(a b)" "(a b)c")
\end{example}

\begin{protos}
\protonoresult{limit-output}{ output-port n procedure}
\end{protos}
\noindent
\var{Procedure} is called on an output port.
Output written to that port is copied to \var{output-port} until \var{n}
 characters have been written, at which point \code{limit-output} returns.
If \var{procedure} returns before writing \var{n} characters, then
 \code{limit-output} also returns at that time, regardless of how many
 characters have been written.

\begin{protos}
\proto{make-tracking-input-port}{ input-port}{input-port}
\proto{make-tracking-output-port}{ output-port}{output-port}
\proto{current-row}{ port}{integer or {\tt \#f}}
\proto{current-column}{ port}{integer or {\tt \#f}}
\protonoresult{fresh-line}{ output-port}
\end{protos}
\noindent \code{Make-tracking-input-port} and \code{make-tracking-output-port}
 return ports that keep track of the current row and column and
 are otherwise identical to their arguments.
Closing a tracking port does not close the underlying port.
\code{Current-row} and \code{current-column} return
  \var{port}'s current read or write location.
They return \code{\#f} if \var{port} does not keep track of its location.
\code{Fresh-line} writes a newline character to \var{output-port} if
 \code{(current-row \cvar{port})} is not 0.

\begin{example}
(define p (open-output-port "/tmp/temp"))
(list (current-row p) (current-column p))
    \evalsto '(0 0)
(display "012" p)
(list (current-row p) (current-column p))
    \evalsto '(0 3)
(fresh-line p)
(list (current-row p) (current-column p))
    \evalsto '(1 0)
(fresh-line p)
(list (current-row p) (current-column p))
    \evalsto '(1 0)
\end{example}

\section{Fluid bindings}

These procedures implement dynamic binding and are in structure \code{fluids}.
A \cvar{fluid} is a cell whose value can be bound dynamically.
Each fluid has a top-level value that is used when the fluid
 is unbound in the current dynamic environment.

\begin{protos}
\proto{make-fluid}{ value}{fluid}
\proto{fluid}{ fluid}{value}
\proto{let-fluid}{ fluid value thunk}{value(s)}
\proto{let-fluids}{ fluid$_0$ value$_0$  fluid$_1$ value$_1$ \ldots thunk}{value(s)}
\end{protos}
\noindent
\code{Make-fluid} returns a new fluid with \cvar{value} as its initial
 top-level value.
\code{Fluid} returns \code{fluid}'s current value.
\code{Let-fluid} calls \code{thunk}, with \cvar{fluid} bound to \cvar{value}
 until \code{thunk} returns.
Using a continuation to throw out of the call to \code{thunk} causes
 \cvar{fluid} to revert to its original value, while throwing back
 in causes \cvar{fluid} to be rebound to \cvar{value}.
\code{Let-fluid} returns the value(s) returned by \cvar{thunk}.
\code{Let-fluids} is identical to \code{let-fluid} except that it binds
 an arbitrary number of fluids to new values.

\begin{example}
(let* ((f (make-fluid 'a))
       (v0 (fluid f))
       (v1 (let-fluid f 'b
             (lambda ()
               (fluid f))))
       (v2 (fluid f)))
  (list v0 v1 v2))
  \evalsto '(a b a)
\end{example}

\begin{example}
(let ((f (make-fluid 'a))
      (path '())
      (c \#f))
  (let ((add (lambda ()
               (set! path (cons (fluid f) path)))))
    (add)
    (let-fluid f 'b
      (lambda ()
        (call-with-current-continuation
          (lambda (c0)
            (set! c c0)))
        (add)))
    (add)
    (if (< (length path) 5)
        (c)
        (reverse path))))
  \evalsto '(a b a b a)
\end{example}

\section{Shell commands}

Structure \code{c-system-function} provides access to the C \code{system()}
 function.

\begin{protos}
\proto{have-system?}{}{boolean}
\proto{system}{ string}{integer}
\end{protos}
\noindent
\code{Have-system?} returns true if the underlying C implementation
 has a command processor.
\code{(System \cvar{string})} passes \cvar{string} to the C
 \code{system()} function and returns the result.

\begin{example}
(begin
  (system "echo foo > test-file")
  (call-with-input-file "test-file" read))
\evalsto 'foo
\end{example}

\section{Sockets}
% Richard says: add UDP documentation.

Structure \code{sockets} provides access to TCP/IP sockets for interprocess
 and network communication.

\begin{protos}
\proto{open-socket}{}{socket}
\proto{open-socket}{ port-number}{socket}
\proto{socket-port-number}{ socket}{integer}
\protonoresult{close-socket}{ socket}
\proto{socket-accept}{ socket}{input-port output-port}
\proto{get-host-name}{}{string}
\end{protos}
\noindent
\code{Open-socket} creates a new socket.
If no \cvar{port-number} is supplied the system picks one at random.
\code{Socket-port-number} returns a socket's port number.
\code{Close-socket} closes a socket, preventing any further connections.
\code{Socket-accept} accepts a single connection on \cvar{socket}, returning
 an input port and an output port for communicating with the client.
If no client is waiting \code{socket-accept} blocks until one appears.
\code{Get-host-name} returns the network name of the machine.

\begin{protos}
\proto{socket-client}{ host-name port-number}{input-port output-port}
\end{protos}
\noindent
\code{Socket-client} connects to the server at \cvar{port-number} on
 the machine named \cvar{host-name}.
\code{Socket-client} blocks until the server accepts the connection.

The following simple example shows a server and client for a centralized UID
 service.
\begin{example}
(define (id-server)
  (let ((socket (open-socket)))
    (display "Waiting on port ")
    (display (socket-port-number socket))
    (newline)
    (let loop ((next-id 0))
      (call-with-values
        (lambda ()
          (socket-accept socket))
        (lambda (in out)
          (display next-id out)
          (close-input-port in)
          (close-output-port out)
          (loop (+ next-id 1)))))))
         
(define (get-id machine port-number)
  (call-with-values
    (lambda ()
      (socket-client machine port-number))
    (lambda (in out)
      (let ((id (read in)))
        (close-input-port in)
        (close-output-port out)
        id))))
\end{example}

\section{Macros for writing loops}
% JAR says: origin? history?

\code{Iterate} and \code{reduce} are extensions of named-\code{let} for
 writing loops that walk down one or more sequences,
 such as the elements of a list or vector, the
 characters read from a port, or an arithmetic series.
Additional sequences can be defined by the user.
\code{Iterate} and \code{reduce} are in structure \code{reduce}.

\subsection{{\tt Iterate}}

The syntax of \code{iterate} is:
\begin{example}
  (iterate \cvar{loop-name}
           ((\cvar{sequence-type} \cvar{element-variable} \cvar{sequence-data} \ldots)
            \ldots)
           ((\cvar{state-variable} \cvar{initial-value})
            \ldots)
    \cvar{body-expression}
    [\cvar{final-expression}])
\end{example}

\code{Iterate} steps the \cvar{element-variable}s in parallel through the
 sequences, while each \cvar{state-variable} has the corresponding
 \cvar{initial-value} for the first iteration and have later values
 supplied by \cvar{body-expression}. 
If any sequence has reached its limit the value of the \code{iterate}
 expression is
 the value of \cvar{final-expression}, if present, or the current values of
 the \cvar{state-variable}s, returned as multiple values.
If no sequence has reached
 its limit, \cvar{body-expression} is evaluated and either calls \cvar{loop-name} with
 new values for the \cvar{state-variable}s, or returns some other value(s).

The \cvar{loop-name} and the \cvar{state-variable}s and \cvar{initial-value}s behave
exactly as in named-\code{let}.  The named-\code{let} expression
\begin{example}
  (let loop-name ((state-variable initial-value) ...)
    body ...)
\end{example}
is equivalent to an \code{iterate} expression with no sequences
 (and with an explicit
 \code{let} wrapped around the body expressions to take care of any
 internal \code{define}s):
\begin{example}
  (iterate loop-name
           ()
           ((state-variable initial-value) ...)
    (let () body ...))
\end{example}

The \cvar{sequence-type}s are keywords (they are actually macros of a particular
 form; it is easy to add additional types of sequences).
Examples are \code{list*} which walks down the elements of a list and
 \code{vector*} which does the same for vectors.
For each iteration, each \cvar{element-variable} is bound to the next
 element of the sequence.
The \cvar{sequence-data} gives the actual list or vector or whatever.

If there is a \cvar{final-expression}, it is evaluated when the end of one or more
 sequences is reached.
If the \cvar{body-expression} does not call \cvar{loop-name} the
 \cvar{final-expression} is not evaluated.
The \cvar{state-variable}s are visible in
 \cvar{final-expression} but the \cvar{sequence-variable}s are not.  

The \cvar{body-expression} and the \cvar{final-expression} are in tail-position within
 the \code{iterate}.
Unlike named-\code{let}, the behavior of a non-tail-recursive call to
 \cvar{loop-name} is unspecified (because iterating down a sequence may involve side
 effects, such as reading characters from a port).

\subsection{{\tt Reduce}}

If an \code{iterate} expression is not meant to terminate before a sequence
 has reached its end,
 \cvar{body-expression} will always end with a tail call to \cvar{loop-name}.
\code{Reduce} is a macro that makes this common case explicit.
The syntax of \code{reduce} is
 the same as that of \code{iterate}, except that there is no \cvar{loop-name}.
The \cvar{body-expression} returns new values of the \cvar{state-variable}s
 instead of passing them to \cvar{loop-name}.
Thus \cvar{body-expression} must return as many values as there are state
 variables.
By special dispensation, if there are
 no state variables then \cvar{body-expression} may return any number of values,
 all of which are ignored.

The syntax of \code{reduce} is:
\begin{example}
  (reduce ((\cvar{sequence-type} \cvar{element-variable} \cvar{sequence-data} \ldots)
            \ldots)
           ((\cvar{state-variable} \cvar{initial-value})
            \ldots)
    \cvar{body-expression}
    [\cvar{final-expression}])
\end{example}

The value(s) returned by an instance of \code{reduce} is the value(s) returned
 by the \cvar{final-expression}, if present, or the current value(s) of the state
variables when the end of one or more sequences is reached.

A \code{reduce} expression can be rewritten as an equivalent \code{iterate}
 expression by adding a \cvar{loop-var} and a wrapper for the
 \cvar{body-expression} that calls the \cvar{loop-var}.
\begin{example}
(iterate loop
         ((\cvar{sequence-type} \cvar{element-variable} \cvar{sequence-data} \ldots)
          \ldots)
         ((\cvar{state-variable} \cvar{initial-value})
          \ldots)
  (call-with-values (lambda ()
                      \cvar{body-expression})
                    loop)
  [\cvar{final-expression}])
\end{example}

\subsection{Sequence types}

The predefined sequence types are:
\begin{protos}
\syntaxprotonoresultnoindex{list*}{ \cvar{elt-var} \cvar{list}}
\syntaxprotonoresultnoindex{vector*}{ \cvar{elt-var} \cvar{vector}}
\syntaxprotonoresultnoindex{string*}{ \cvar{elt-var} \cvar{string}}
\syntaxprotonoresultnoindex{count*}
 { \cvar{elt-var} \cvar{start} [\cvar{end} [\cvar{step}]]}
\syntaxprotonoresultnoindex{input*}
 { \cvar{elt-var} \cvar{input-port} \cvar{read-procedure}}
\syntaxprotonoresultnoindex{stream*}
 { \cvar{elt-var} \cvar{procedure} \cvar{initial-data}}
\end{protos}

For lists, vectors, and strings the element variable is bound to the
 successive elements of the list or vector, or the characters in the
 string.

For \code{count*} the element variable is bound to the elements of the sequence
\begin{example}
 \cvar{start}, \cvar{start} + \cvar{step}, \cvar{start} + 2\cvar{step}, \ldots, \cvar{end}
\end{example}
 inclusive of \cvar{start} and exclusive of \cvar{end}.
The default \cvar{step} is 1.
The sequence does not terminate if no \cvar{end} is given or if there
 is no $N > 0$ such that \cvar{end} = \cvar{start} + N\cvar{step}
 (\code{=} is used to test for termination).
For example, \code{(count* i 0 -1)} doesn't terminate
 because it begins past the \cvar{end} value and \code{(count* i 0 1 2)} doesn't
 terminate because it skips over the \cvar{end} value.

For \code{input*} the elements are the results of successive applications
 of \cvar{read-procedure} to \cvar{input-port}.
The sequence ends when \cvar{read-procedure} returns an end-of-file object.

For a stream, the \cvar{procedure} takes the current data value as an argument
 and returns two values, the next value of the sequence and a new data value.
If the new data is \code{\#f} then the previous element was the last
 one.  For example,
\begin{example}
  (list* elt my-list)
\end{example}
 is the same as
\begin{example}
  (stream* elt list->stream my-list)
\end{example}
 where \code{list->stream} is
\begin{example}
  (lambda (list)
    (if (null? list)
        (values 'ignored \#f)
        (values (car list) (cdr list))))
\end{example}

\subsection{Synchronous sequences}

When using the sequence types described above, a loop terminates when any of
its sequences reaches its end.  To help detect bugs it is useful to have
sequence types that check to see if two or more sequences end on the same
iteration.  For this purpose there is second set of sequence types called
synchronous sequences.  These are identical to the ones listed above except
that they cause an error to be signalled if a loop is terminated by a
synchronous sequence and some other synchronous sequence did not reach its
end on the same iteration.

Sequences are checked for termination in order, from left to right, and
if a loop is terminated by a non-synchronous sequence no further checking
is done.

The synchronous sequences are:

\begin{protos}
\syntaxprotonoresultnoindex{list\%}{ \cvar{elt-var} \cvar{list}}
\syntaxprotonoresultnoindex{vector\%}{ \cvar{elt-var} \cvar{vector}}
\syntaxprotonoresultnoindex{string\%}{ \cvar{elt-var} \cvar{string}}
\syntaxprotonoresultnoindex{count\%}
 { \cvar{elt-var} \cvar{start} \cvar{end} [\cvar{step}]}
\syntaxprotonoresultnoindex{input\%}
 { \cvar{elt-var} \cvar{input-port} \cvar{read-procedure}}
\syntaxprotonoresultnoindex{stream\%}
 { \cvar{elt-var} \cvar{procedure} \cvar{initial-data}}
\end{protos}

Note that the synchronous \code{count\%} must have an \cvar{end}, unlike the
 nonsynchronous \code{count\%}.

\subsection{Examples}

\noindent
Gathering the indexes of list elements that answer true to some
predicate.
\begin{example}
(lambda (my-list predicate)
  (reduce ((list* elt my-list)
           (count* i 0))
          ((hits '()))
    (if (predicate elt)
        (cons i hits)
        hits)
    (reverse hits))
\end{example}

\noindent
Looking for the index of an element of a list.
\begin{example}
(lambda (my-list predicate)
  (iterate loop
           ((list* elt my-list)
            (count* i 0))
           ()                                ; no state
    (if (predicate elt)
        i
        (loop))))
\end{example}

\noindent
Reading one line.
\begin{example}
(define (read-line port)
  (iterate loop
           ((input* c port read-char))
           ((chars '()))
    (if (char=? c \#\verb2\2newline)
        (list->string (reverse chars))
        (loop (cons c chars)))
    (if (null? chars)
        (eof-object)
        ; no newline at end of file
        (list->string (reverse chars)))))
\end{example}

\noindent
Counting the lines in a file.  We can't use \code{count*} because we
need the value of the count after the loop has finished.
\begin{example}
(define (line-count name)
  (call-with-input-file name
    (lambda (in)
      (reduce ((input* l in read-line))
              ((i 0))
        (+ i 1)))))
\end{example}

\subsection{Defining sequence types}

The sequence types are object-oriented macros similar to enumerations.
A non-synchronous sequence macro needs to supply three values:
 \code{\#f} to indicate that it isn't synchronous, a list of state variables
 and their initializers, and the code for one iteration.
The first
 two methods are CPS'ed: they take another macro and argument to
 which to pass their result.
The \code{synchronized?} method gets no additional arguments.
The \code{state-vars} method is passed a list of names which
 will be bound to the arguments to the sequence.
The final method, for the step, is passed the list of names bound to
 the arguments and the list of state variables.
In addition there is
 a variable to be bound to the next element of the sequence, the
 body expression for the loop, and an expression for terminating the
 loop.

The definition of \code{list*} is
\begin{example}
(define-syntax list*
  (syntax-rules (synchronized? state-vars step)
    ((list* synchronized? (next more))
     (next \#f more))
    ((list* state-vars (start-list) (next more))
     (next ((list-var start-list)) more))
    ((list* step (start-list) (list-var)
            value-var loop-body final-exp)
     (if (null? list-var)
         final-exp
         (let ((value-var (car list-var))
               (list-var (cdr list-var)))
           loop-body)))))
\end{example}

Synchronized sequences are the same, except that they need to
 provide a termination test to be used when some other synchronized
 method terminates the loop.
\begin{example}
(define-syntax list\%
  (syntax-rules (sync done)
    ((list\% sync (next more))
     (next \#t more))
    ((list\% done (start-list) (list-var))
     (null? list-var))
    ((list\% stuff ...)
     (list* stuff ...))))
\end{example}

\subsection{Expanded code}

The expansion of 
\begin{example}
  (reduce ((list* x '(1 2 3)))
          ((r '()))
    (cons x r))
\end{example}
is
\begin{example}
  (let ((final (lambda (r) (values r)))
        (list '(1 2 3))
        (r '()))
    (let loop ((list list) (r r))
      (if (null? list)
          (final r)
          (let ((x (car list))
                (list (cdr list)))
            (let ((continue (lambda (r)
                              (loop list r))))
              (continue (cons x r)))))))
\end{example}

The only inefficiencies in this code are the \code{final} and \code{continue}
 procedures, both of which could be substituted in-line.
The macro expander could do the substitution for \code{continue} when there
 is no explicit proceed variable, as in this case, but not in general.

\section{Sorting lists and vectors}
\label{sort}

(This section, as the libraries it describes, was written mostly by
Olin Shivers for the draft of SRFI~32.)

%% Olin Shivers
%% First draft: 1998/10/19
%% Last update: 2002/7/21

%% [Todo: del-list-neighbor-dups!
%%        vector-copy -> subvector
%%        use srfi-23 for reporting errors
%%        use srfi-16 for n-aries?

The sort libraries in Scheme~48 include
%
\begin{itemize}
\item vector insert sort (stable)
\item vector heap sort
%    - Vector quick sort (with median-of-3 pivot picking)
\item vector merge sort (stable)
\item pure and destructive list merge sort (stable)
\item stable vector and list merge
\item miscellaneous sort-related procedures: vector and list merging, 
  sorted predicates, vector binary search, vector and list 
  delete-equal-neighbor procedures.
\item a general, non-algorithmic set of procedure names for general sorting
  and merging
\end{itemize}

\subsection{Design rules}

\paragraph{What vs. how}

There are two different interfaces: ``what'' (simple) and ``how'' (detailed).

\begin{description}
\item[Simple] you specify semantics: datatype (list or vector), 
  mutability, and stability.
  
\item[Detailed] you specify the actual algorithm (quick, heap,
  insert, merge). Different algorithms have different properties,
  both semantic and pragmatic, so these exports are necessary.
  
  It is necessarily the case that the specifications of these procedures
  make statements about execution ``pragmatics.'' For example, the sole
  distinction between heap sort and quick sort---both of which are
  provided by this library----is one of execution time, which is not a
  ``semantic'' distinction. Similar resource-use statements are made about
  ``iterative'' procedures, meaning that they can execute on input of
  arbitrary size in a constant number of stack frames.
\end{description}

\paragraph{Consistency across procedure signatures}

The two interfaces share common procedure signatures wherever
possible, to facilitate switching a given call from one procedure
to another.
        
\paragraph{Less-than parameter first, data parameter after}

These procedures uniformly observe the following parameter order:
the data to be sorted comes after the comparison procedure.
That is, we write

\begin{example}
  (sort \(<\) \var{list})
\end{example}

not

\begin{example}
  (sort \var{list} \(<\))
\end{example}
%

\paragraph{Ordering, comparison procedures and stability}

These routines take a $<$ comparison procedure, not a $\leq$ comparison
procedure, and they sort into increasing order. The difference between
a $<$ spec and a $\leq$ spec comes up in two places: 

\begin{itemize}
\item the definition of an ordered or sorted data set, and
\item the definition of a stable sorting algorithm.
%\item correctness of quicksort. NOTE "two" above
\end{itemize}
%

We say that a data set (a list or vector) is \textit{sorted} or
\textit{ordered} if it contains no adjacent pair of values $\ldots x,
y \ldots$ such that $y < x$.

In other words, scanning across the data never takes a ``downwards'' step.

If you use a $\leq$ procedure where these algorithms expect a $<$
procedure, you may not get the answers you expect. For example,
the \code{list-sorted?} procedure will return false if you pass it a $\leq$ comparison
procedure and an ordered list containing adjacent equal elements.

A ``stable'' sort is one that preserves the pre-existing order of equal
elements. Suppose, for example, that we sort a list of numbers by 
comparing their absolute values, i.e., using comparison procedure
%
\begin{verbatim}
(lambda (x y) (< (abs x) (abs y)))
\end{verbatim}
%
If we sort a list that contains both 3 and -3: \[\ldots 3, \ldots, -3 \ldots\]
then a stable sort is an algorithm that will not swap the order
of these two elements, that is, the answer is guaranteed to to look like
\[\ldots 3, -3 \ldots\]
not
\[\ldots -3, 3 \ldots\]
Choosing $<$ for the comparison procedure instead of $\leq$ affects
how stability is coded. Given an adjacent pair $x, y$, \code{(<
  $y$ $x$)} means ``$x$ should be moved in front of $x$''---otherwise,
leave things as they are. So using a $\leq$ procedure where a $<$
procedure is expected will \emph{invert} stability.

This is due to the definition of equality, given a $<$ comparator:
\begin{verbatim}
    (and (not (< x y))
         (not (< y x)))
\end{verbatim}
The definition is rather different, given a $\leq$ comparator:
\begin{verbatim}
    (and (<= x y)
         (<= y x))
\end{verbatim}
%
A ``stable'' merge is one that reliably favors one of its data sets
when equal items appear in both data sets. \emph{All merge operations in
this library are stable}, breaking ties between data sets in favor
of the first data set---elements of the first list come before equal 
elements in the second list.

So, if we are merging two lists of numbers ordered by absolute value,
the stable merge operation \code{list-merge}
\begin{verbatim}
    (list-merge (lambda (x y) (< (abs x) (abs y)))
                '(0 -2 4 8 -10) '(-1 3 -4 7))
\end{verbatim}
reliably places the 4 of the first list before the equal-comparing -4
  of the second list:
\begin{verbatim}
    (0 -1 -2 4 -4 7 8 -10)
\end{verbatim}
%
  Some sort algorithms will \emph{not work correctly} if given a $\leq$
  when they expect a $<$ comparison (or vice-versa).

%% For example,
%%   violating quicksort's spec may cause it to produce wrong answers,
%%   diverge, raise an error, or do some fourth thing. To see why,
%%   consider the left-scan part of the standard quicksort partition
%%   step: (let ((i (let scan ((i i)) (if (elt< (vector-ref v i) pivot)
%%   (scan (+ i 1)) i)))) ...)  Consider applying this loop to a vector
%%   of all zeroes (hence, PIVOT, as well, is zero), but erroneously
%%   using \verb|<=| for the ELT< procedure. The loop will scan right off
%%   the end of the vector, producing a vector-index error.  The
%%   guarantee that the scan loop will terminate before running off the
%%   end of the vector depends critically upon ELT< performing as a true,
%%   irreflexive $<$ relation. Running off the end of the vector is only
%%   one of a variety of possibly ways to lose---other, variant
%%   implementations of quicksort can, instead, loop forever on some data
%%   sets if ELT< is a $\leq$ predicate.

In short, if your comparison procedure $f$ answers true to \code{($f$ x x)}, then 
\begin{itemize}
\item using a stable sorting or merging algorithm will not give you a
  stable sort or merge, 
\item \code{list-sorted?} may surprise you.
% \item quicksort may fail in a variety of possible ways.
\end{itemize}
Note that  you can synthesize a $<$ procedure from a $\leq$ procedure with
\begin{verbatim}
    (lambda (x y) (not (<= y x)))
\end{verbatim}
if need be. 

Precise definitions give sharp edges to tools, but require care in use. 
``Measure twice, cut once.''

%% I have adopted the choice of $<$ from Common Lisp. One would assume
%% the definers of Common Lisp had a good reason for adopting $<$ instead
%% of $\leq$, but canvassing several of the principal actors in the
%% definition process has turned up no better reason than ``an arbitrary
%% but consistent choice.'' At minimum, then, these libraries extend the
%% coverage of that consistent choice.

\paragraph{All vector operations accept optional subrange parameters}

The vector operations specified below all take optional
\code{start}/\code{end} arguments indicating a selected subrange
of a vector's elements. If a \code{start} parameter or
\code{start}/\code{end} parameter pair is given to such a
procedure, they must be exact, non-negative integers, such that
%
\[
    0 \leq \var{start} \leq \var{end} \leq \code{(vector-length \var{vector})}
\]
%
where \var{vector} is the related vector parameter. If not specified,
they default to 0 and the length of the vector, respectively. They are
interpreted to select the range $[\var{start},\var{end})$, that
is, all elements from index \var{start} (inclusive) up to, but not
including, index \var{end}.

\paragraph{Required vs.\ allowed side-effects}

\code{List-sort!} and \code{List-stable-sort!} are allowed, but
not required, to alter their arguments' cons cells to construct the
result list. This is consistent with the what-not-how character of the
group of procedures to which they belong (the \code{sorting} structure).

The \code{list-delete-neighbor-dups!}, \code{list-merge!} and
\code{list-merge-sort!} procedures, on the other hand, provide
specific algorithms, and, as such, explicitly commit to the use of
side-effects on their input lists in order to guarantee their key
algorithmic properties (e.g., linear-time operation).

\subsection{Procedure specification}

\begin{center}
\begin{tabular}{ll}
Structure name & Functionality\\\hline
\code{sorting} & General sorting for lists and vectors\\
\code{sorted} & Sorted predicates for lists and vectors\\
\code{list-merge-sort}& List merge sort\\
\code{vector-merge-sort} & Vector merge sort\\
\code{vector-heap-sort} & Vector heap sort\\
%\code{vector-quick-sort} & Vector quick sort\\
\code{vector-insert-sort} & Vector insertion sort\\
\code{delete-neighbor-duplicates} & List and vector delete neighbor duplicates\\
\code{binary-searches} & Vector binary search
 \end{tabular}
\end{center}
%
Note that there is no ``list insert sort'' package, as you might as well always
use list merge sort. The reference implementation's destructive list merge
sort will do fewer \code{set-cdr!}s than a destructive insert sort.

\paragraph{Procedure naming and functionality}

Almost all of the procedures described below are variants of two basic
operations: sorting and merging. These procedures are consistently named
by composing a set of basic lexemes to indicate what they do.
\begin{center}

\begin{tabular}{lp{0.8\textwidth}}
Lexeme & Meaning\\\hline
\code{sort}&    The procedure sorts its input data set by some $<$ comparison procedure.
\\
\code{merge}&   The procedure merges two ordered data sets into a single ordered
          result.
\\
\code{stable} & This lexeme indicates that the sort is a stable one.
\\
\code{vector}& The procedure operates upon vectors.
\\
\code{list} &   The procedure operates upon lists.
\\
\code{!}      &  Procedures that end in \code{!} are allowed, and sometimes required, 
          to reuse their input storage to construct their answer.
\end{tabular}
\end{center}

\paragraph{Types of parameters and return values}

In the procedures specified below,
%
\begin{itemize}
\item A \code{<} or \code{=} parameter is a procedure accepting
  two arguments taken from the specified procedure's data set(s), and
  returning a boolean;
\item \code{Start} and \code{end} parameters are exact, non-negative integers that 
  serve as vector indices selecting a subrange of some associated vector.
  When specified, they must satisfy the relation
  \[
    0 \leq \var{start} \leq \var{end} \leq \code{(vector-length \var{vector})}
  \]
  where \var{vector} is the associated vector.
\end{itemize}
%
Passing values to procedures with these parameters that do not satisfy
these types is an error.

If a procedure is said to return ``unspecified,'' this means that
nothing at all is said about what the procedure returns, not even the
number of return values. Such a procedure is not even required to be
consistent from call to call in the nature or number of its return
values. It is simply required to return a value (or values) that may
be passed to a command continuation, e.g.  as the value of an
expression appearing as a non-terminal subform of a \code{begin}
expression. Note that in R$^5$RS, this restricts such a procedure to
returning a single value; non-R$^5$RS systems may not even provide this
restriction.

\subsubsection{\code{sorting}---general sorting package}

This library provides basic sorting and merging functionality suitable for
general programming. The procedures are named by their semantic properties,
i.e., what they do to the data (sort, stable sort, merge, and so forth).

\begin{protos}
  \proto{list-sorted?}{ $<$ list}{boolean} 
  \proto{list-merge}{ $<$ list$_1$ list$_2$}{list}
  \protonoindex{list-merge!}{ $<$ list$_1$ list$_2$}{list}\mainschindex{list-merge"!}
  \proto{list-sort}{ $<$ lis}{list}
  \protonoindex{list-sort!}{ $<$ lis}{list}\mainschindex{list-sort"!}
  \proto{list-stable-sort}{  $<$ list}{list}
  \protonoindex{list-stable-sort!}{ $<$ list}{list}\mainschindex{list-stable-sort"!}
  \proto{list-delete-neighbor-dups}{  $=$ list}{list}
  \proto{vector-sorted?}{ $<$ v [start [end]]}{boolean}
  \proto{vector-merge}{ $<$ v$_1$ v$_2$ [start$1$ [end$1$ [start$2$ [end$2$]]]]}{vector}
  \protonoresultnoindex{vector-merge!}{ $<$ v v$_1$ v$_2$ [start [start$1$ [end$1$ [start$2$ [end$2$]]]]]}\mainschindex{vector-merge"!}
  \proto{vector-sort}{ $<$ v [start [end]]}{vector}
  \protonoresultnoindex{vector-sort!}{ $<$ v [start [end]]}\mainschindex{vector-sort"!}
  \proto{vector-stable-sort}{ $<$ v [start [end]]}{vector}
  \protonoresultnoindex{vector-stable-sort!}{ $<$ v [start [end]]}\mainschindex{vector-stable-sort"!}
  \proto{vector-delete-neighbor-dups}{ $=$ v [start [end]]}{vector}
\end{protos}

\begin{center}
\begin{tabular}{ll}
Procedure &Suggested algorithm
\\\hline
\code{list-sort} & vector heap or quick\\
\code{list-sort!} & list merge sort\\
\code{list-stable-sort} & vector merge sort\\
\code{list-stable-sort!} & list merge sort\\
\code{vector-sort} & heap or quick sort\\
\code{vector-sort!} or quick sort\\
\code{vector-stable-sort} & vector merge sort\\
\code{vector-stable-sort!} merge sort
\end{tabular}
\end{center}
%
\code{List-Sorted?} and \code{vector-sorted?} return true if their
input list or vector is in sorted order, as determined by their \var{$<$}
comparison parameter.

All four merge operations are stable: an element of the initial list
\var{list$_1$} or vector \var{vector$_1$} will come before an
equal-comparing element in the second list \var{list$_2$} or vector
\var{vector$_2$} in the result.

The procedures
%
\begin{itemize}
\item \code{list-merge}
\item \code{list-sort}
\item \code{list-stable-sort}
\item \code{list-delete-neighbor-dups}
\end{itemize}
%
do not alter their inputs and are allowed to return a value that shares 
a common tail with a list argument.

The procedure
\begin{itemize}
\item \code{list-sort!}
\item \code{list-stable-sort!}
\end{itemize}
%
are ``linear update'' operators---they are allowed, but not required, to
alter the cons cells of their arguments to produce their results. 

On the other hand, the \code{list-merge!} procedure 
make only a single, iterative, linear-time pass over its argument
list, using \code{set-cdr!}s to rearrange the cells of the list
into the final result ---it works ``in place.'' Hence, any cons cell
appearing in the result must have originally appeared in an input. The
intent of this iterative-algorithm commitment is to allow the
programmer to be sure that if, for example, \code{list-merge!} is asked to
merge two ten-million-element lists, the operation will complete
without performing some extremely (possibly twenty-million) deep
recursion.

The vector procedures
%
\begin{itemize}
\item \code{vector-sort}
\item \code{vector-stable-sort}
\item \code{vector-delete-neighbor-dups}
\end{itemize}
%
do not alter their inputs, but allocate a fresh vector for their result,
of length $\var{end} - \var{start}$. 

The vector procedures
%
\begin{itemize}
\item \code{vector-sort!}
\item \code{vector-stable-sort!}
\end{itemize}
%
sort their data in-place. (But note that \code{vector-stable-sort!}
may allocate temporary storage proportional to the size of the
input
%%---I am not aware of $O(n \log(n))$ stable vector-sorting
%%algorithms that run in constant space
.)

\code{Vector-merge} returns a vector of length $(\var{end$_1$}-\var{start$_1$}+(\var{end$_2$}-\var{start$_2$})$.
    
\code{Vector-merge!} writes its result into vector \var{v},
beginning at index \var{start}, for indices less than \(\var{end} =
\var{start} + (\var{end$_1$}-\var{start$_1$}) +
(\var{end$_2$}-\var{start$_2$})\). The target subvector
  $\var{v}[\var{start},\var{end})$ may not overlap either source
subvector $\var{vector$_1$}[\var{start$_1$},\var{end$_1$})$ $\var{vector$_2$}[\var{start$_2$},\var{end$_2$})$.

The \code{\ldots-delete-neighbor-dups-\ldots} procedures:
These procedures delete adjacent duplicate elements from a list or a
vector, using a given element-equality procedure. The first/leftmost
element of a run of equal elements is the one that survives. The list or
vector is not otherwise disordered.
    
These procedures are linear time---much faster than the $O(n^2)$ general
duplicate-element deletors that do not assume any ``bunching'' of elements
(such as the ones provided by SRFI~1). If you want to delete duplicate
elements from a large list or vector, you can sort the elements to bring
equal items together, then use one of these procedures, for a total time
of $O(n\log(n))$.
    
The comparison procedure \(=\) passed to these procedures is always
applied
%
\code{($=$ $x$ $y$)}
%
where $x$ comes before $y$ in the containing list or vector.

\begin{itemize}
\item \code{List-delete-neighbor-dups} does not alter its input list; its answer
  may share storage with the input list.
\item \code{Vector-delete-neighbor-dups} does not alter its input vector, but
  rather allocates a fresh vector to hold the result.
\end{itemize}
%
Examples:

\begin{example}
(list-delete-neighbor-dups = '(1 1 2 7 7 7 0 -2 -2))
  \(\Longrightarrow\) (1 2 7 0 -2)

(vector-delete-neighbor-dups = '\#(1 1 2 7 7 7 0 -2 -2))
  \(\Longrightarrow\) \#(1 2 7 0 -2)

(vector-delete-neighbor-dups = '\#(1 1 2 7 7 7 0 -2 -2) 3 7)
  \(\Longrightarrow\) \#(7 0 -2)
\end{example}
         
\subsubsection{Algorithm-specific sorting packages}

These packages provide more specific sorting functionality, that is,
specific committment to particular algorithms that have particular
pragmatic consequences (such as memory locality, asymptotic running time)
beyond their semantic behaviour (sorting, stable sorting, merging, etc.).
Programmers that need a particular algorithm can use one of these packages.

\paragraph{\code{sorted}---sorted predicates}
%
\begin{protos}
  \proto{list-sorted?}{ $<$ list}{boolean}
  \proto{vector-sorted?}{ $<$ vector}{boolean}
  \proto{vector-sorted?}{ $<$ vector start}{boolean}
  \proto{vector-sorted?}{ $<$ vector start end}{boolean}
\end{protos}

Return \code{\#f} iff there is an adjacent pair \(\ldots x, y \ldots\) in the input
list or vector such that $y < x$. The optional \var{start}/\var{end} range 
arguments restrict \code{vector-sorted?} to the indicated subvector.

\paragraph{\code{list-merge-sort}---list merge sort}
%
\begin{protos}
  \proto{list-merge-sort}{ $<$ list}{list}
  \protonoindex{list-merge-sort!}{ $<$ list}{list}\mainschindex{list-merge-sort"!}
  \proto{list-merge}{ list$_1$ $<$ list$_2$}{list}
  \protonoindex{list-merge!}{ list$_1$ $<$ list$_2$}{list}\mainschindex{list-merge"!}
\end{protos}
%
The sort procedures sort their data using a list merge sort, which is
stable. (The reference implementation is, additionally, a ``natural'' sort.
See below for the properties of this algorithm.)

The \code{!} procedures are destructive---they use \code{set-cdr!}s to
rearrange the cells of the lists into the proper order. As such, they
do not allocate any extra cons cells---they are ``in place'' sorts.
%% Additionally, \code{list-merge!} is iterative---it can operate on
%% arguments of arbitrary size with a constant number of stack frames.

The merge operations are stable: an element of \var{list$_1$} will
come before an equal-comparing element in \var{list$_2$} in the result
list.

\paragraph{\code{vector-merge-sort}---vector merge sort}

\begin{protos}
  \proto{vector-merge-sort}{ $<$ vector [start [end [temp]]]}{vector}
  \protonoresultnoindex{vector-merge-sort!}{ $<$ vector [start [end [temp]]]}\mainschindex{vector-merge-sort"!}
  \proto{vector-merge}{ $<$ vector$_1$ vector$_2$ [start$_1$ [end$_1$ [start$_2$ [end$_2$]]]]}{vector}
  \protonoresultnoindex{vector-merge!}{ $<$ vector vector$_1$ vector$_2$ [start [start$_1$ [end$_1$ [start$_2$ [end$_2$]]]]]}\mainschindex{vector-merge"!}
\end{protos}
%
The sort procedures sort their data using vector merge sort, which is
stable. (The reference implementation is, additionally, a ``natural'' sort.
See below for the properties of this algorithm.)

The optional \var{start}/\var{end} arguments provide for sorting of subranges, and
default to 0 and the length of the corresponding vector.
    
Merge-sorting a vector requires the allocation of a temporary
``scratch'' work vector for the duration of the sort. This scratch
vector can be passed in by the client as the optional \var{temp}
argument; if so, the supplied vector must be of size $\leq \var{end}$,
and will not be altered outside the range [start,end). If not
supplied, the sort routines allocate one themselves.

The merge operations are stable: an element of \var{vector$_1$} will
come before an equal-comparing element in \var{vector$_2$} in the
result vector.

\begin{itemize}
\item 
\code{Vector-merge-sort!} leaves its result in
\(\var{vector}[\var{start},\var{end})\).
\item   
\code{Vector-merge-sort} returns a vector of length
\(\var{end}-\var{start}\).
\item 
\code{Vector-merge} returns a vector of length
\((\var{end$_1$}-\var{start$_1$})+(\var{end$_2$}-\var{start$_2$})\).
\item 
\code{Vector-merge!} writes its result into \var{vector}, beginning
at index \var{start},
for indices less than \(\var{end} =\var{start} +
(\var{end$_1$}-\var{start$_1$}) + (\var{end$_2$}-\var{start$_2$})\).
The target subvector
\[\var{vector}[\var{start},\var{end})\]
may not overlap either source subvector
\[\var{vector$_1$}[\var{start$_1$},\var{end$_1$}), \textrm{ or }
\var{vector$_2$}[\var{start$_2$},\var{end$_2$}).\]
\end{itemize}

\paragraph{\code{vector-heap-sort}---vector heap sort}

\begin{protos}
\proto{vector-heap-sort}{ $<$ vector [start [end]]}{vector}
\protonoresultnoindex{vector-heap-sort!}{ $<$ vector [start [end]]}\mainschindex{vector-heap-sort"!}
\end{protos}
%
These procedures sort their data using heap sort, 
which is not a stable sorting algorithm.
    
\code{Vector-heap-sort} returns a vector of length \(\var{end}-\var{start}\). 
\code{Vector-heap-sort!} is in-place, leaving its result in
\(\var{vector}[\var{start},\var{end})\).

%% vector-quick-sort-lib - vector quick sort
%%     quick-sort   < v [start end] -> vector
%%     quick-sort!  < v [start end] -> unspecified
%%     quick-sort3! c v [start end] -> unspecified

%%     These procedures sort their data using quick sort, 
%%     which is not a stable sorting algorithm.
    
%%     QUICK-SORT returns a vector of length END-START. 
%%     QUICK-SORT! is in-place, leaving its result in V[start,end).

%%     QUICK-SORT3! is a variant of quick-sort that takes a three-way
%%     comparison procedure C. C compares a pair of elements and returns
%%     an exact integer whose sign indicates their relationship:
%%       (c x y) < 0   =>   x<y
%%       (c x y) = 0   =>   x=y
%%       (c x y) > 0   =>   x>y
%%     To help remember the relationship between the sign of the result and
%%     the relation, use the procedure - as the model for C: (- x y) < 0
%%     means that x < y; (- x y) > 0 means that x > y.

%%     The extra discrimination provided by the three-way comparison can
%%     provide significant speedups when sorting data sets with many duplicates,
%%     especially when the comparison procedure is relatively expensive (e.g.,
%%     comparing long strings).

%%     WARNING: Some sort algorithms, such as insertion sort or heap sort,
%%     can tolerate being passed a <= comparison procedure when they expect a < 
%%     procedure---insertion and merge sort may simply invert stability; and
%%     heap sort will run a bit slower, but otherwise produce a correct answer.

%%     Quicksort, however, is much more critically sensitive to the distinction
%%     between a < and a <= comparison. If QUICK-SORT or QUICK-SORT! expect a <
%%     comparison procedure, and are erroneously given a <= procedure, they may,
%%     depending on implementation, produce an unsorted result, go into an 
%%     infinite loop, cause a run-time error, occasionally produce a correct
%%     result, or do some fifth thing.

%%     Implementors may wish to write QUICKSORT3! so that it (a) tests the
%%     comparison procedure (by checking that (c v[start] v[start]) produces
%%     false), or (b) is tolerant of an erroneous <= procedure, or (c) both.
%%     Clients of this procedure, however, should not count on this.

\paragraph{\code{vector-insert-sort}---vector insertion sort}

\begin{protos}
\proto{vector-insert-sort}{ $<$ vector [start [end]]}{vector}
\protonoresultnoindex{vector-insert-sort!}{ $<$ vector [start [end]]}\mainschindex{vector-insert-sort"!}
\end{protos}
%
These procedures stably sort their data using insertion sort.
%
\begin{itemize}
\item \code{Vector-insert-sort} returns a vector of length \(\var{end}-\var{start}\).
\item \code{Vector-insert-sort!} is in-place, leaving its result in
  \(\var{vector}[\var{start},\var{end})\).
\end{itemize}

\paragraph{\code{delete-neighbor-duplicates}---list and vector
  delete neighbor duplicates}

\begin{protos}
  \proto{list-delete-neighbor-dups}{ $=$ list}{list}
  \protonoindex{list-delete-neighbor-dups!}{ $=$ list}{list}\mainschindex{list-delete-neighbor-dups"!}
  \proto{vector-delete-neighbor-dups}{ $=$ vector [start [end]]}{vector}
  \protonoindex{vector-delete-neighbor-dups!}{ $=$ vector [start [end]]}{end$'$}\mainschindex{vector-delete-neighbor-dups}
\end{protos}
%
These procedures delete adjacent duplicate elements from a list or
a vector, using a given element-equality procedure $=$. The first/leftmost
element of a run of equal elements is the one that survives. The list
or vector is not otherwise disordered.

These procedures are linear time---much faster than the $O(n^2)$ general
duplicate-element deletors that do not assume any ``bunching'' of elements
(such as the ones provided by SRFI~1). If you want to delete duplicate
elements from a large list or vector, you can sort the elements to bring
equal items together, then use one of these procedures, for a total time
of $O(n\log(n))$.
    
The comparison procedure = passed to these procedures is always
applied

\begin{example}
(\(=\) \(x\) \(y\))
\end{example}

where $x$ comes before $y$ in the containing list or vector.
%
\begin{itemize}
\item 
\code{List-delete-neighbor-dups} does not alter its input list; its
answer may share storage with the input list.
\item 
\code{Vector-delete-neighbor-dups} does not alter its input vector, but
rather allocates a fresh vector to hold the result.
\item
\code{List-delete-neighbor-dups!} is permitted, but not required, to
mutate its input list in order to construct its answer.
\item
\code{Vector-delete-neighbor-dups!} reuses its input vector to hold the
answer, packing its answer into the index range
\([\var{start},\var{end$'$})\), where
\var{end$'$} is the non-negative exact integer returned as its value. It
returns \var{end$'$} as its result. The vector is not altered outside the range
\([\var{start},\var{end$'$})\).
\end{itemize}
%
Examples:

\begin{example}
(list-delete-neighbor-dups = '(1 1 2 7 7 7 0 -2 -2))
  \(\Longrightarrow\) (1 2 7 0 -2)

(vector-delete-neighbor-dups = '\#(1 1 2 7 7 7 0 -2 -2))
  \(\Longrightarrow\) \#(1 2 7 0 -2)

(vector-delete-neighbor-dups = '\#(1 1 2 7 7 7 0 -2 -2) 3 7)
  \(\Longrightarrow\) \#(7 0 -2)

;; Result left in v[3,9):
(let ((v (vector 0 0 0 1 1 2 2 3 3 4 4 5 5 6 6)))
  (cons (vector-delete-neighbor-dups! = v 3)
        v))
   \(\Longrightarrow\) (9 . \#(0 0 0 1 2 3 4 5 6 4 4 5 5 6 6))
\end{example}

\paragraph{\code{binary-searches}---vector binary search}

\begin{protos}
\proto{vector-binary-search}{ $<$ elt-$>$key key vector [start [end]]}{integer or {\tt \#f}}
\proto{vector-binary-search3}{ compare-proc vector [start [end]]}{integer or {\tt \#f}}
\end{protos}

\code{vector-binary-search} searches \var{vector} in range
\([\var{start},\var{end})\) (which default to 0 and the length of
\var{vector}, respectively) for an element whose
associated key is equal to \var{key}. The procedure \var{elt-$>$key} is used to map
an element to its associated key. The elements of the vector are assumed
to be ordered by the $<$ relation on these keys. That is, 

\begin{example}
(vector-sorted? (lambda (x y) (\(<\) (\var{elt-\(>\)key} x) (\var{elt-\(>\)key} y)))
                \var{vector} \var{start} \var{end}) \(\Longrightarrow\) true
\end{example}

An element \var{e} of \var{vector} is a match for \var{key} if it's
neither less nor greater than the key:

\begin{example}
(and (not (\(<\) (\var{elt-\(>\)key} \var{e}) \var{key}))
     (not (\(<\) \var{key} (\var{elt-\(>\)key} \var{e}))))
\end{example}

If there is such an element, the procedure returns its index in the
vector as an exact integer. If there is no such element in the searched 
range, the procedure returns false.

\begin{example}
(vector-binary-search < car 4 '\#((1 . one) (3 . three)
                                 (4 . four) (25 . twenty-five)))
\(\Longrightarrow\) 2

(vector-binary-search < car 7 '\#((1 . one) (3 . three)
                                 (4 . four) (25 . twenty-five)))
\(\Longrightarrow\) \#f
\end{example}    

\code{Vector-binary-search3} is a variant that uses a three-way comparison
procedure \var{compare-proc}. \var{Compare-proc} compares its
parameter to the search key, and returns an
exact integer whose sign indicates its relationship to the search key.
%
\[
  \begin{array}{rclcrcl}
      (\var{compare-proc}~x) &<& 0& \Rightarrow&  x &<& \var{search-key}\\
      (\var{compare-proc}~x) &=& 0& \Rightarrow&  x &=& \var{search-key}\\
      (\var{compare-proc}~x) &>& 0& \Rightarrow&  x &>& \var{search-key}
  \end{array}
\]

\begin{example}
(vector-binary-search3 (lambda (elt) (- (car elt) 4))
                       '\#((1 . one) (3 . three)
                          (4 . four) (25 . twenty-five)))
\(\Longrightarrow\) 2
\end{example}

%% Rationale:
%% %
%% \begin{itemize}
%% \item Why isn't \code{vector-binary-search}'s \var{elt-$>$key}
%%   computation simply absorbed into the $<$ procedure? It is separated
%%   out because the $<$ procedure is
%%   applied twice inside the binary-search inner loop, once with the search
%%   key for the first argument and the element key for the second argument,
%%   and once, with the reverse argument order. This is not necessary for
%%   \code{vector-binary-search3}.
%% \item When a comparison operation is able to produce a three-way
%%   discrimination, the inner loop of the binary search can trim the number
%%   of per-iteration comparisons from an average of 1.5 to a guaranteed
%%   single comparison per iteration. This can be a significant savings when
%%   searching with an expensive comparison operation (e.g., one that
%%   uses string compare, sends email, references a database, or queries
%%   a network service such as a web server).
%% \item Failure is signaled by false (rather than, say, -1) so that searches
%%   can be used in conditional forms such as
%% \begin{verbatim}
%%         (or (vector-binary-search ...) ...)
%% \end{verbatim}
%%   or
%% \begin{verbatim}
%%         (cond ((vector-binary-search ...) => index-consumer)
%%               ...)
%% \end{verbatim}
%% \end{itemize}

\subsection{Algorithmic properties}

Different sort and merge algorithms have different properties.
Choose the algorithm that matches your needs:

\begin{description}
\item[Vector insert sort]
  Stable, but only suitable for small vectors---$O(n^2)$.
%% \item[Vector quick sort]
%%   Not stable. Is fast on average---$O(n\log(n))$---but has bad worst-case
%%   behaviour. Has good memory locality for big vectors (unlike heap sort). 
%%   A clever pivot-picking trick (median of three samples) helps avoid 
%%   worst-case behaviour, but pathological cases can still blow up.
\item[Vector heap sort]
  Not stable. Guaranteed fast---$O(n\log(n))$ \emph{worst} case. Poor
  locality on large vectors. A very reliable workhorse.
\item[Vector merge sort]
    Stable. Not in-place---requires a temporary buffer of equal size. 
    Fast---$O(n\log(n))$---and has good memory locality for large vectors.
    
    The implementation of vector merge sort provided by this
    implementation is, additionally, a ``natural'' sort, meaning that it
    exploits existing order in the input data, providing $O(n)$ best case.
\item[Destructive list merge sort]
    Stable, fast and in-place (i.e., allocates no new cons cells). ``Fast''
    means $O(n\log(n))$ worse-case, and substantially better if the data
    is already mostly ordered, all the way down to linear time for
    a completely-ordered input list (i.e., it is a ``natural'' sort).

    Note that sorting lists involves chasing pointers through memory, which
    can be a loser on modern machine architectures because of poor cache and
    page locality.
%%     Pointer \emph{writing}, which is what the
%%     \code{set-cdr!}s of a destructive list-sort algorithm do, is
%%     even worse, especially if your Scheme has a generational GC---the
%%     writes will thrash the write-barrier.
    Sorting vectors has inherently better locality.
    
    This implementation's destructive list merge and merge sort
    implementations are opportunistic---they avoid redundant
    \code{set-cdr!}s, and try to take long
    already-ordered runs of list structure as-is when doing the merges.
\item[Pure list merge sort]
  Stable and fast---$O(n\log(n))$ worst-case, and possibly $O(n)$,
  depending upon the input list (see discussion above).
\end{description}

\begin{center}
  \begin{tabular}{lllll}
Algorithm &    Stable? & Worst case &  Average case &  In-place\\
\hline
Vector insert & Yes&      $O(n^2)$ &     $O(n^2)$&        Yes\\
Vector quick  & No &      $O(n^2)$  &    $O(n\log(n))$&     Yes\\
Vector heap   & No &      $O(n\log(n))$&   $O(n\log(n))$&     Yes\\
Vector merge  & Yes&      $O(n\log(n))$&   $O(n\log(n))$&     No\\
List merge    & Yes&      $O(n\log(n))$&   $O(n\log(n))$&     Either
\end{tabular}
\end{center}

\section{Regular expressions}
\label{regexp-adt}

This section describes a functional interface for building regular
 expressions and matching them against strings.
The matching is done using the POSIX regular expression package.
Regular expressions are in the structure \code{regexps}.

A regular expression is either a character set, which matches any character
in the set, or a composite expression containing one or more subexpressions.
A regular expression can be matched against a string to determine success
or failure, and to determine the substrings matched by particular subexpressions.

\begin{protos}
\proto{regexp?}{ value}{boolean}
\end{protos}
\noindent
Returns \code{\#t} if \cvar{value} is a regular expression created
using the functional interface for regular expressions, and \code{\#f}
otherwise.

\subsection{Character sets}

Character sets may be defined using a list of characters and strings,
using a range or ranges of characters, or by using set operations on
existing character sets.

\begin{protos}
\proto{set}{ character-or-string \ldots}{char-set}
\proto{range}{ low-char high-char}{char-set}
\proto{ranges}{ low-char high-char \ldots}{char-set}
\proto{ascii-range}{ low-char high-char}{char-set}
\proto{ascii-ranges}{ low-char high-char \ldots}{char-set}
\end{protos}
\noindent
\code{Set} returns a set that contains the character arguments and the
characters in any string arguments.  \code{Range} returns a character
set that contain all characters between \cvar{low-char} and \cvar{high-char},
inclusive.  \code{Ranges} returns a set that contains all characters in
the given ranges.  \code{Range} and \code{ranges} use the ordering induced by
\code{char->integer}.  \code{Ascii-range} and \code{ascii-ranges} use the
 ASCII ordering.
It is an error for a \cvar{high-char} to be less than the preceding
 \cvar{low-char} in the appropriate ordering.

\begin{protos}
\proto{negate}{ char-set}{char-set}
\proto{intersection}{ char-set char-set}{char-set}
\proto{union}{ char-set char-set}{char-set}
\proto{subtract}{ char-set char-set}{char-set}
\end{protos}
\noindent
These perform the indicated operations on character sets.

The following character sets are predefined:
\begin{center}
\texonly\begin{longtable}{ll}\endtexonly
\htmlonly\begin{tabular}{ll}\endhtmlonly
 \code{lower-case} &   \code{(set "abcdefghijklmnopqrstuvwxyz")} \\
 \code{upper-case} &   \code{(set "ABCDEFGHIJKLMNOPQRSTUVWXYZ")} \\
 \code{alphabetic} &   \code{(union lower-case upper-case)} \\
 \code{numeric} &      \code{(set "0123456789")} \\
 \code{alphanumeric} & \code{(union alphabetic numeric)} \\
 \code{punctuation} &
 \code{(set "}\verb2!\"#$%&'()*+,-./:;<=>?@[\\]^_`{|}~2\code{")} \\
 \code{graphic} &      \code{(union alphanumeric punctuation)} \\
 \code{printing} &     \code{(union graphic (set \#}\verb2\2\code{space))} \\
 \code{control} &      \code{(negate printing)} \\
 \code{blank} &
      \code{(set \#}\verb2\2\code{space (ascii->char 9))} ; 9 is tab \\
 \code{whitespace} &
      \code{(union (set \#}\verb2\2\code{space) (ascii-range 9 13))} \\
 \code{hexdigit} &     \code{(set "0123456789abcdefABCDEF")} \\
\texonly\end{longtable}\endtexonly
\htmlonly\end{tabular}\endhtmlonly
\end{center}
\noindent The above are taken from the default locale in POSIX.
The characters in \code{whitespace} are \cvar{space}, \cvar{tab},
 \cvar{newline} (= \cvar{line feed}), \cvar{vertical tab}, \cvar{form feed}, and
 \cvar{carriage return}.

\subsection{Anchoring}

\begin{protos}
\proto{string-start}{}{reg-exp}
\proto{string-end}{}{reg-exp}
\end{protos}
\noindent
\code{String-start} returns a regular expression that matches the beginning
 of the string being matched against; {string-end} returns one that matches
 the end.

\subsection{Composite expressions}

\begin{protos}
\proto{sequence}{ reg-exp \ldots}{reg-exp}
\proto{one-of}{ reg-exp \ldots}{reg-exp}
\end{protos}
\noindent
\code{Sequence} matches the concatenation of its arguments, \code{one-of} matches
any one of its arguments.

\begin{protos}
\proto{text}{ string}{reg-exp}
\end{protos}
\noindent
\code{Text} returns a regular expression that matches the characters in
 \cvar{string}, in order.

\begin{protos}
\proto{repeat}{ reg-exp}{reg-exp}
\proto{repeat}{ count reg-exp}{reg-exp}
\proto{repeat}{ min max reg-exp}{reg-exp}
\end{protos}
\noindent
\code{Repeat} returns a regular expression that matches zero or more
occurences of its \cvar{reg-exp} argument.  With no count the result
will match any number of times (\cvar{reg-exp}*).  With a single
count the returned expression will match
 \cvar{reg-exp} exactly that number of times.
The final case will match from \cvar{min} to \cvar{max}
 repetitions, inclusive.
\cvar{Max} may be \code{\#f}, in which case there
 is no maximum number of matches.
\cvar{Count} and \cvar{min} should be exact, non-negative integers;
 \cvar{max} should either be an exact non-negative integer or \code{\#f}.

\subsection{Case sensitivity}

Regular expressions are normally case-sensitive.
\begin{protos}
\proto{ignore-case}{ reg-exp}{reg-exp}
\proto{use-case}{ reg-exp}{reg-exp}
\end{protos}
\noindent
The value returned by
 \code{ignore-case} is identical its argument except that case will be
 ignored when matching.
The value returned by \code{use-case} is protected
 from future applications of \code{ignore-case}.
The expressions returned
 by \code{use-case} and \code{ignore-case} are unaffected by later uses of the
 these procedures.
By way of example, the following matches \code{"ab"} but not \code{"aB"},
 \code{"Ab"}, or \code{"AB"}.
\begin{example}
\code{(text "ab")}
\end{example}
\noindent
while
\begin{example}
\code{(ignore-case (test "ab"))}
\end{example}
\noindent
matches \code{"ab"}, \code{"aB"},
 \code{"Ab"}, and \code{"AB"} and
\begin{example}
(ignore-case (sequence (text "a")
                       (use-case (text "b"))))
\end{example}
\noindent
matches \code{"ab"} and \code{"Ab"} but not \code{"aB"} or \code{"AB"}.

\subsection{Submatches and matching}

A subexpression within a larger expression can be marked as a submatch.
When an expression is matched against a string, the success or failure
of each submatch within that expression is reported, as well as the
location of the substring matched be each successful submatch.

\begin{protos}
\proto{submatch}{ key reg-exp}{reg-exp}
\proto{no-submatches}{ reg-exp}{reg-exp}
\end{protos}
\noindent
\code{Submatch} returns a regular expression that matches its argument and
 causes the result of matching its argument to be reported by the \code{match}
 procedure.
\cvar{Key} is used to indicate the result of this particular submatch 
 in the alist of successful submatches returned by \code{match}.
 Any value may be used as a \cvar{key}.
\code{No-submatches} returns an expression identical to its
 argument, except that all submatches have been elided.

\begin{protos}
\proto{any-match?}{ reg-exp string}{boolean}
\proto{exact-match?}{ reg-exp string}{boolean}
\proto{match}{ reg-exp string}{match or {\tt \#f}}
\proto{match-start}{ match}{index}
\proto{match-end}{ match}{index}
\proto{match-submatches}{ match}{alist}
\end{protos}
\noindent
\code{Any-match?} returns \code{\#t} if \cvar{string} matches \cvar{reg-exp} or
 contains a substring that does, and \code{\#f} otherwise.
\code{Exact-match?} returns \code{\#t} if \cvar{string} matches
 \cvar{reg-exp} and \code{\#f} otherwise.

\code{Match} returns \code{\#f} if \cvar{reg-exp} does not match \cvar{string}
 and a match record if it does match.
A match record contains three values: the beginning and end of the substring
 that matched
 the pattern and an a-list of submatch keys and corresponding match records
 for any submatches that also matched.
\code{Match-start} returns the index of
 the first character in the matching substring and \code{match-end} gives index
 of the first character after the matching substring.
\code{Match-submatches} returns an alist of submatch keys and match records.
Only the top match record returned by \code{match} has a submatch alist.

Matching occurs according to POSIX.
The match returned is the one with the lowest starting index in \cvar{string}.
If there is more than one such match, the longest is returned.
Within that match the longest possible submatches are returned.

All three matching procedures cache a compiled version of \cvar{reg-exp}.
Subsequent calls with the same \cvar{reg-exp} will be more efficient.

The C interface to the POSIX regular expression code uses ASCII \code{nul}
 as an end-of-string marker.
The matching procedures will ignore any characters following an
 embedded ASCII \code{nul}s in \cvar{string}.

\begin{example}
(define pattern (text "abc"))
(any-match? pattern "abc")         \evalsto #t
(any-match? pattern "abx")         \evalsto #f
(any-match? pattern "xxabcxx")     \evalsto #t

(exact-match? pattern "abc")       \evalsto #t
(exact-match? pattern "abx")       \evalsto #f
(exact-match? pattern "xxabcxx")   \evalsto #f

(match pattern "abc")              \evalsto (#\{match 0 3\})
(match pattern "abx")              \evalsto #f
(match pattern "xxabcxx")          \evalsto (#\{match 2 5\})

(let ((x (match (sequence (text "ab")
                          (submatch 'foo (text "cd"))
                          (text "ef"))
                "xxxabcdefxx")))
  (list x (match-submatches x)))
  \evalsto (#\{match 3 9\} ((foo . #\{match 5 7\}))

(match-submatches
  (match (sequence
           (set "a")
           (one-of (submatch 'foo (text "bc"))
                   (submatch 'bar (text "BC"))))
         "xxxaBCd"))
  \evalsto ((bar . #\{match 4 6\}))
\end{example}

\section{SRFIs}

`SRFI' stands for `Scheme Request For Implementation'.
An SRFI is a description of an extension to standard Scheme.
Draft and final SRFI documents, a FAQ, and other information about SRFIs
 can be found at
\urlhd{http://srfi.schemers.org}{the SRFI web site}{\code{http://srfi.schemers.org}}.

Scheme~48 includes implementations of the following (final) SRFIs:
\begin{itemize}
\item SRFI 1 -- List Library
\item SRFI 2 -- \code{and-let*}
\item SRFI 5 -- \code{let} with signatures and rest arguments
\item SRFI 6 -- Basic string ports
\item SRFI 7 -- Program configuration
\item SRFI 8 -- \code{receive}
\item SRFI 9 -- Defining record types
\item SRFI 11 -- Syntax for receiving multiple values 
\item SRFI 13 -- String Library
\item SRFI 14 -- Character-Set Library (see note below)
\item SRFI 16 -- Syntax for procedures of variable arity
\item SRFI 17 -- Generalized \code{set!}
\item SRFI 22 -- Running Scheme Scripts on Unix
\item SRFI 23 -- Error reporting mechanism
\item SRFI 25 -- Multi-dimensional Array Primitives 
\item SRFI 26 -- Notation for Specializing Parameters without Currying
\item SRFI 27 -- Sources of Random Bits
\item SRFI 28 -- Basic Format Strings
\item SRFI 31 -- A special form \texttt{rec} for recursive evaluation
\item SRFI 34 -- Exception Handling for Programs
\item SRFI 35 -- Conditions
\item SRFI 36 -- I/O Conditions
\item SRFI 37 -- args-fold: a program argument processor
\item SRFI 40 -- A Library of Streams
\item SRFI 42 -- Eager Comprehensions
\item SRFI 43 -- Vector library
\item SRFI 45 -- Primitives for Expressing Iterative Lazy Algorithms
\item SRFI 60 -- Integers as Bits
\item SRFI 61 -- A more general cond clause
\end{itemize}
Documentation on these can be found at the web site mentioned above.

SRFI~14 includes the procedure \code{->char-set} which is not a standard
 Scheme identifier (in R$^5$RS the only required identifier starting
 with \code{-} is \code{-} itself).
In the Scheme~48 version of SRFI~14 we have renamed \code{->char-set}
 as \code{x->char-set}.

With the exception of SRFI 62 (which is supported by default), the
SRFI bindings can be accessed
either by opening the appropriate structure
 (the structure \code{srfi-}\cvar{n} contains SRFI \cvar{n})
 or by loading structure \code{srfi-7} and then using
 the \code{,load-srfi-7-program} command to load an SRFI 7-style program.
The syntax for the command is
\begin{example}
\code{,load-srfi-7-program \cvar{name} \cvar{filename}}
\end{example}
This creates a new structure and associated package, binds the structure
 to \cvar{name} in the configuration package, and then loads the program
 found in \cvar{filename} into the package.

As an example, if the file \code{test.scm} contains
\begin{example}
(program (code (define x 10)))
\end{example}
this program can be loaded as follows:
\begin{example}
> ,load-package srfi-7
> ,load-srfi-7-program test test.scm
[test]
> ,in test
test> x
10
test> 
\end{example}

%%% Local Variables: 
%%% mode: latex
%%% TeX-master: "manual"
%%% End: 

\chapter{Mixing Scheme 48 and C}
\label{external-chapter}

This chapter describes an interface for calling C functions
 from Scheme, calling Scheme functions from C, and allocating
 storage in the Scheme heap..
Scheme~48 manages stub functions in C that
 negotiate between the calling conventions of Scheme and C and the
 memory allocation policies of both worlds.
No stub generator is available yet, but writing stubs is a straightforward task.

\section{Available facilities}
\label{sec:facilities}

The following facilities are available for interfacing between
 Scheme~48 and C:
%
\begin{itemize}
\item Scheme code can call C functions.
\item The external interface provides full introspection for all
  Scheme objects.  External code may inspect, modify, and allocate
  Scheme objects arbitrarily.
\item External code may raise exceptions back to Scheme~48 to
  signal errors.
\item External code may call back into Scheme.  Scheme~48
  correctly unrolls the process stack on non-local exits.
\item External modules may register bindings of names to values with a 
  central registry accessible from
  Scheme.  Conversely, Scheme code can register shared
  bindings for access by C code.
\end{itemize}
%
%This section has three parts: the first describes how bindings are
% moved from Scheme to C and vice versa, the second tells how to call
% C functions from Scheme, and the third covers the C interface
% to Scheme objects, including calling Scheme procedures, using the
% Scheme heap, and so forth.

\subsection{Scheme structures}

The structure \code{external-calls} has 
 most of the Scheme functions described here.
The others are in 
 \code{dynamic-externals}, which has the functions for dynamic loading and
 name lookup from
Section~\ref{dynamic-externals},
 and \code{shared-bindings}, which has the additional shared-binding functions
 described in
section~\ref{more-shared-bindings}.

\subsection{C naming conventions}

The names of all of Scheme~48's visible C bindings begin 
 with `\code{s48\_}' (for procedures and variables) or 
 `\code{S48\_}' (for macros).
Whenever a C name is derived from a Scheme identifier, we
 replace `\code{-}' with `\code{\_}' and convert letters to lowercase
 for procedures and uppercase for macros.
A final `\code{?}'  converted to `\code{\_p}' (`\code{\_P}' in C macro names).
A final `\code{!}' is dropped.
Thus the C macro for Scheme's \code{pair?} is \code{S48\_PAIR\_P} and
 the one for \code{set-car!} is \code{S48\_SET\_CAR}.
Procedures and macros that do not check the types of their arguments
 have `\code{unsafe}' in their names.

All of the C functions and macros described have prototypes or definitions
 in the file \code{c/scheme48.h}.
The C type for Scheme values is defined there to be \code{s48\_value}.

\subsection{Garbage collection}

Scheme~48 uses a copying garbage collector.
The collector must be able to locate all references
 to objects allocated in the Scheme~48 heap in order to ensure that
 storage is not reclaimed prematurely and to update references to objects
 moved by the collector.
The garbage collector may run whenever an object is allocated in the heap.
C variables whose values are Scheme~48 objects and which are live across
 heap allocation calls need to be registered with
 the
garbage collector.  See section~\ref{gc} for more information.

\section{Shared bindings}
\label{sec:shared-bindings}

Shared bindings are the means by which named values are shared between Scheme
 code and C code.
There are two separate tables of shared bindings, one for values defined in
 Scheme and accessed from C and the other for values going the other way.
Shared bindings actually bind names to cells, to allow a name to be looked
 up before it has been assigned.
This is necessary because C initialization code may be run before or after
 the corresponding Scheme code, depending on whether the Scheme code is in
 the resumed image or is run in the current session.

\subsection{Exporting Scheme values to C}

\begin{protos}
\proto{define-exported-binding}{ name value}{shared-binding}
\end{protos}

\begin{protos}
\cproto{s48\_value s48\_get\_imported\_binding(char *name)}
\cproto{s48\_value S48\_SHARED\_BINDING\_REF(s48\_value shared\_binding)}
\end{protos}

\noindent\code{Define-exported-binding} makes \cvar{value} available to C code
 under as \cvar{name} which must be a \cvar{string}, creating a new shared
 binding if necessary.
The C function \code{s48\_get\_imported\_binding} returns the shared binding
 defined for \code{name}, again creating it if necessary.
The C macro \code{S48\_SHARED\_BINDING\_REF} dereferences a shared binding,
 returning its current value.

\subsection{Exporting C values to Scheme}

\begin{protos}
\cproto{void s48\_define\_exported\_binding(char *name, s48\_value v)}
\end{protos}

\begin{protos}
\proto{lookup-imported-binding}{ string}{shared-binding}
\proto{shared-binding-ref}{ shared-binding}{value}
\end{protos}

\noindent These are used to define shared bindings from C and to access them
 from Scheme.
Again, if a name is looked up before it has been defined, a new binding is
 created for it.

The common case of exporting a C function to Scheme can be done using
 the macro \code{S48\_EXPORT\_FUNCTION(\cvar{name})}.
This expands into
\begin{example}
s48\_define\_exported\_binding("\cvar{name}",
                               s48\_enter\_pointer(\cvar{name}))
\end{example}

\noindent which boxes the function into a Scheme byte vector and then
 exports it.
Note that \code{s48\_enter\_pointer} allocates space in the Scheme heap
 and might trigger a 
garbage collection; see Section~\ref{gc}.

\begin{protos}
\syntaxprotonoresult{import-definition}{ \cvar{name}}
\syntaxprotonoresultnoindex{import-definition}{ \cvar{name c-name}}
\end{protos}
These macros simplify importing definitions from C to Scheme.
They expand into

\code{(define \cvar{name} (lookup-imported-binding \cvar{c-name}))}

\noindent{}where \cvar{c-name} is as supplied for the second form.
For the first form \cvar{c-name} is derived from \cvar{name} by
 replacing `\code{-}' with `\code{\_}' and converting letters to lowercase.
For example, \code{(import-definition my-foo)} expands into

\code{(define my-foo (lookup-imported-binding "my\_foo"))}

\subsection{Complete shared binding interface}
\label{more-shared-bindings}

There are a number of other Scheme functions related to shared bindings;
 these are in the structure \code{shared-bindings}.

\begin{protos}
\proto{shared-binding?}{ x}{boolean}
\proto{shared-binding-name}{ shared-binding}{string}
\proto{shared-binding-is-import?}{ shared-binding}{boolean}
\protonoresultnoindex{shared-binding-set!}{ shared-binding value}\mainschindex{shared-binding-set"!}
\protonoresult{define-imported-binding}{ string value}
\protonoresult{lookup-exported-binding}{ string}
\protonoresult{undefine-imported-binding}{ string}{}
\protonoresult{undefine-exported-binding}{ string}{}
\end{protos}

\noindent\code{Shared-binding?} is the predicate for shared-bindings.
\code{Shared-binding-name} returns the name of a binding.
\code{Shared-binding-is-import?} is true if the binding was defined from C.
\code{Shared-binding-set!} changes the value of a binding.
\code{Define-imported-binding} and \code{lookup-exported-binding} are
 Scheme versions of \code{s48\_define\_exported\_binding}
 and \code{s48\_lookup\_imported\_binding}.
The two \code{undefine-} procedures remove bindings from the two tables.
They do nothing if the name is not found in the table.

The following C macros correspond to the Scheme functions above.

\begin{protos}
\cproto{int\ \ \ \ \ \ \ S48\_SHARED\_BINDING\_P(x)}
\cproto{int\ \ \ \ \ \ \ S48\_SHARED\_BINDING\_IS\_IMPORT\_P(s48\_value s\_b)}
\cproto{s48\_value S48\_SHARED\_BINDING\_NAME(s48\_value s\_b)}
\cproto{void\ \ \ \ \ \ S48\_SHARED\_BINDING\_SET(s48\_value s\_b, s48\_value v)}
\end{protos}

\section{Calling C functions from Scheme}
\label{sec:external-call}

There are three different ways to call C functions from Scheme, depending on
 how the C function was obtained.

\begin{protos}
\proto{call-imported-binding}{ binding arg$_0$ \ldots}{value}
\end{protos}
\noindent
Each of these applies its first argument, a C function, to the rest of
 the arguments.
For \code{call-imported-binding} the function argument must be an
 imported binding.

For all of these, the C function is passed the \cvar{arg$_i$} values
 and the value returned is that returned by C procedure.
No automatic representation conversion occurs for either arguments or
 return values.
Up to twelve arguments may be passed.
There is no method supplied for returning multiple values to
 Scheme from C (or vice versa) (mainly because C does not have multiple return
 values).

Keyboard interrupts that occur during a call to a C function are ignored
 until the function returns to Scheme (this is clearly a
 problem; we are working on a solution).

\begin{protos}
\syntaxprotonoresult{import-lambda-definition}
{ \cvar{name} (\cvar{formal} \ldots)}
\syntaxprotonoresultnoindex{import-lambda-definition}
{ \cvar{name} (\cvar{formal} \ldots)\ \cvar{c-name}}
\end{protos}
\noindent{}These macros simplify importing functions from C.
They define \cvar{name} to be a function with the given formals that
 applies those formals to the corresponding C binding.
\cvar{C-name}, if supplied, should be a string.
These expand into

\begin{example}
(define temp (lookup-imported-binding \cvar{c-name}))
(define \cvar{name}
  (lambda (\cvar{formal} \ldots)
    (call-imported-binding temp \cvar{formal} \ldots)))
\end{example}

\noindent{}
If \cvar{c-name} is not supplied, it is derived from \cvar{name} by converting
 all letters to lowercase and replacing `\code{-}' with `\code{\_}'.

\section{Dynamic loading}
\label{dynamic-externals}

External code can be loaded into a running Scheme~48---at least on
most variants of Unix and on Windows.  The required Scheme functions
are in the structure \code{load-dynamic-externals}.

To be suitable for dynamic loading, the externals code must reside in
a shared object.  The shared object must define a function:
%
\begin{protos}
  \cproto{void s48\_on\_load(void)}
\end{protos}
%
The \code{s48\_on\_load} is run upon loading the shared objects.  It
typically contains invocations of \code{S48\_EXPORT\_FUNCTION} to
make the functionality defined by the shared object known to
Scheme~48. 

The shared object may also define either or both of the following
functions:
%
\begin{protos}
   \cproto{void s48\_on\_unload(void)}
   \cproto{void s48\_on\_reload(void)}
\end{protos}
Scheme~48 calls \code{s48\_on\_unload} just before it unloads the
shared object.  If \code{s48\_on\_reload} is present, Scheme~48 calls
it when it loads the shared object for the second time, or some new
version thereof.  If it is not present, Scheme~48 calls
\code{s48\_on\_load} instead.  (More on that later.)

For Linux, the following commands compile \code{foo.c} into a file
\code{foo.so} that can be loaded dynamically.
\begin{example}
\% gcc -c -o foo.o foo.c
\% ld -shared -o foo.so foo.o
\end{example}
%
The following procedures provide the basic functionality for loading
shared objects containing dynamic externals:
%
\begin{protos}
\proto{load-dynamic-externals}{ string plete?
  rrepeat? rresume?}{dynamic-externals}
\proto{unload-dynamic-externals}{ string}{ dynamic-externals}
\protonoresult{reload-dynamic-externals}{ dynamic-externals}
\end{protos}
%
\code{Load-dynamic-externals} loads the named shared objects.  The
\cvar{plete?} argument determines whether Scheme~48 appends the
OS-specific suffix (typically \code{.so} for Unix, and \code{.dll} for
Windows) to the name.  The \cvar{rrepeat?}  argument determines how
\code{load-dynamic-externals} behaves if it is called again with the
same argument: If this is true, it reloads the shared object (and
calls its \code{s48\_on\_unload} on unloading if present, and, after
reloading, \code{s48\_on\_reload} if present or \code{s48\_on\_load}
if not), otherwise, it will not do anything.  The \cvar{rresume?}
argument determines if an image subsequently dumped will try to load
the shared object again automatically.  (The shared objects will be
loaded before any record resumers run.)  \code{Load-dynamic-externals}
returns a handle identifying the shared object just loaded.

\code{Unload-dynamic-externals} unloads the shared object associated
with the handle passed as its argument, previously calling its
\code{s48\_on\_unload} function if present.  Note that this invalidates
all external bindings associated with the shared object; referring to
any of them will probably crash the program.

\code{Reload-dynamic-externals} will reload the shared object named by
its argument and call its \code{s48\_on\_unload} function before
unloading, and, after reloading, \code{s48\_on\_reload} if present or
\code{s48\_on\_load} if not.

\begin{protos}
\proto{import-dynamic-externals}{ string}{dynamic-externals}  
\end{protos}
%
This procedure represents the expected most usage for loading
dynamic-externals.  It is best explained by its definition:
%
\begin{example}
(define (import-dynamic-externals name)
  (load-dynamic-externals name #t #f #t))
\end{example}

\section{Accessing Scheme data from C}
\label{sec:scheme-data}

The C header file \code{scheme48.h} provides
 access to Scheme~48 data structures.
The type \code{s48\_value} is used for Scheme values.
When the type of a value is known, such as the integer returned
 by \code{vector-length} or the boolean returned by \code{pair?},
 the corresponding C procedure returns a C value of the appropriate
 type, and not a \code{s48\_value}.
Predicates return \code{1} for true and \code{0} for false.

\subsection{Constants}
\label{sec:constants}

The following macros denote Scheme constants:
%
\begin{itemize}
\item \code{S48\_FALSE} is \verb|#f|.
\item \code{S48\_TRUE} is \verb|#t|.
\item \code{S48\_NULL} is the empty list.
\item \code{S48\_UNSPECIFIC} is a value used for functions which have no
  meaningful return value
 (in Scheme~48 this value returned by the nullary procedure \code{unspecific}
 in the structure \code{util}).
\item \code{S48\_EOF} is the end-of-file object
 (in Scheme~48 this value is returned by the nullary procedure \code{eof-object}
 in the structure \code{i/o-internal}).
\end{itemize}

\subsection{Converting values}

The following macros and functions convert values between Scheme and C
 representations.
The `extract' ones convert from Scheme to C and the `enter's go the other
 way.

\begin{protos}
\cproto{int \ \ \ \ \ \ S48\_EXTRACT\_BOOLEAN(s48\_value)}
\cproto{long\ \ \ \ \ \ s48\_extract\_char(s48\_value)}
\cproto{char * \ \ \ s48\_extract\_string(s48\_value)}
\cproto{char * \ \ \ s48\_extract\_byte\_vector(s48\_value)}
\cproto{long \ \ \ \ \ s48\_extract\_integer(s48\_value)}
\cproto{double \ \ \ s48\_extract\_double(s48\_value)}
\cproto{s48\_value S48\_ENTER\_BOOLEAN(int)}
\cproto{s48\_value s48\_enter\_char(long)}
\cgcproto{s48\_value s48\_enter\_byte\_vector(char *, long)}
\cgcproto{s48\_value s48\_enter\_integer(long)}
\cgcproto{s48\_value s48\_enter\_double(double)}
\end{protos}

\noindent{}\code{S48\_EXTRACT\_BOOLEAN} is false if its argument is
 \code{\#f} and true otherwise.
 \code{S48\_ENTER\_BOOLEAN} is \code{\#f} if its argument is zero
  and \code{\#t} otherwise.

The \code{s48\_extract\_char} function extracts the scalar value from
a Scheme character as a C \code{long}.  Conversely,
\code{s48\_enter\_char} creates a Scheme character from a scalar
value.  (Note that ASCII values are also scalar values.)

 The \code{s48\_extract\_byte\_vector} function returns a
 pointer to the actual
 storage used by the byte vector.
 These pointers are valid only until the next GC; see Section~\ref{gc}.

The second argument to \code{s48\_enter\_byte\_vector} is the length of
 byte vector.

\code{s48\_enter\_integer()} needs to allocate storage when
 its argument is too large to fit in a Scheme~48 fixnum.
In cases where the number is known to fit within a fixnum (currently 30 bits
 including the sign), the following procedures can be used.
These have the disadvantage of only having a limited range, but
 the advantage of never causing a garbage collection.
\code{S48\_FIXNUM\_P} is a macro that true if its argument is a fixnum
 and false otherwise.

\begin{protos}
\cproto{int \ \ \ \ \ \ S48\_TRUE\_P(s48\_value)}
\cproto{int \ \ \ \ \ \ S48\_FALSE\_P(s48\_value)}
\end{protos}

\noindent \code{S48\_TRUE\_P} is true if its argument is \code{S48\_TRUE}
 and \code{S48\_FALSE\_P} is true if its argument is \code{S48\_FALSE}.

\begin{protos}
\cproto{int \ \ \ \ \ \ S48\_FIXNUM\_P(s48\_value)}
\cproto{long \ \ \ \ \ s48\_extract\_fixnum(s48\_value)}
\cproto{s48\_value s48\_enter\_fixnum(long)}
\cproto{long \ \ \ \ \ S48\_MAX\_FIXNUM\_VALUE}
\cproto{long \ \ \ \ \ S48\_MIN\_FIXNUM\_VALUE}
\end{protos}

\noindent An error is signalled if \code{s48\_extract\_fixnum}'s argument
 is not a fixnum or if the argument to \code{s48\_enter\_fixnum} is less than
 \code{S48\_MIN\_FIXNUM\_VALUE} or greater than \code{S48\_MAX\_FIXNUM\_VALUE}
 ($-2^{29}$ and $2^{29}-1$ in the current system).

\begin{protos}
\cgcproto{s48\_value s48\_enter\_string\_latin\_1(char*);}
\cgcproto{s48\_value s48\_enter\_string\_latin\_1\_n(char*, long);}
\cproto{void\ \ \ \ \ \ s48\_copy\_latin\_1\_to\_string(char*, s48\_value);}
\cproto{void\ \ \ \ \ \ s48\_copy\_latin\_1\_to\_string\_n(char*, long, s48\_value);}
\cproto{void\ \ \ \ \ \ s48\_copy\_string\_to\_latin\_1(s48\_value, char*);}
\cproto{void\ \ \ \ \ \ s48\_copy\_string\_to\_latin\_1\_n(s48\_value, long, long, char*);}
\cgcproto{s48\_value s48\_enter\_string\_utf\_8(char*);}
\cgcproto{s48\_value s48\_enter\_string\_utf\_8\_n(char*, long);}
\cproto{long\ \ \ \ \ \ s48\_string\_utf\_8\_length(s48\_value);}
\cproto{long\ \ \ \ \ \ s48\_string\_utf\_8\_length\_n(s48\_value, long, long);}
\cproto{void\ \ \ \ \ \ s48\_copy\_string\_to\_utf\_8(s48\_value, char*);}
\cproto{void\ \ \ \ \ \ s48\_copy\_string\_to\_utf\_8\_n(s48\_value, long, long, char*);}
\end{protos}
%
The \code{s48\_enter\_string\_latin\_1} function creates a Scheme
string, initializing its contents from its NUL-terminated,
Latin-1-encoded argument.  The \code{s48\_enter\_string\_latin\_1\_n}
function does the same, but allows specifying the length explicitly---no NUL
terminator is necessary.

The \code{s48\_copy\_latin\_1\_to\_string} function copies
Latin-1-encoded characters from its first NUL-terminated argument to
the Scheme string that is its second argument.  The
\code{s48\_copy\_latin\_1\_to\_string\_n} does the same, but allows
specifying the number of characters explicitly.  The
\code{s48\_copy\_string\_to\_latin\_1} function converts the
characters of the Scheme string specified as the first argument into
Latin-1 and writes them into the string specified as the second
argument.  (Note that it does not NUL-terminate the result.)  The
\code{s48\_copy\_string\_to\_latin\_1\_n} function does the same, but
allows specifying a starting index and a character count into the
source string.

The \code{s48\_enter\_string\_utf\_8} function creates a Scheme
string, initializing its contents from its NUL-terminated,
UTF-8-encoded argument.  The \code{s48\_enter\_string\_utf\_8\_n}
function does the same, but allows specifying the length
explicitly---no NUL terminator is necessary.

The \code{s48\_string\_utf\_8\_length} function computes the length
that the UTF-8 encoding of its argument (a Scheme string) would
occupy, not including NUL termination.  The
\code{s48\_string\_utf\_8\_length} function does the same, but allows
specifying a starting index and a count into the input string.

The \code{s48\_copy\_string\_to\_utf\_8} function converts the
characters of the Scheme string specified as the first argument into
UTF-8 and writes them into the string specified as the second
argument.  (Note that it does not NUL-terminate the result.)  The
\code{s48\_copy\_string\_to\_utf\_8\_n} function does the same, but
allows specifying a starting index and a character count into the
source string.

\subsection{C versions of Scheme procedures}

The following macros and procedures are C versions of Scheme procedures.
The names were derived by replacing `\code{-}' with `\code{\_}',
 `\code{?}' with `\code{\_P}', and dropping `\code{!}.

\begin{protos}
\cproto{int \ \ \ \ \ \ S48\_EQ\_P(s48\_value, s48\_VALUE)}
\cproto{int \ \ \ \ \ \ S48\_CHAR\_P(s48\_value)}
\end{protos}
\begin{protos}
\cproto{int \ \ \ \ \ \ S48\_PAIR\_P(s48\_value)}
\cproto{s48\_value S48\_CAR(s48\_value)}
\cproto{s48\_value S48\_CDR(s48\_value)}
\cproto{void \ \ \ \ \ S48\_SET\_CAR(s48\_value, s48\_value)}
\cproto{void \ \ \ \ \ S48\_SET\_CDR(s48\_value, s48\_value)}
\cgcproto{s48\_value s48\_cons(s48\_value, s48\_value)}
\cproto{long \ \ \ \ \ s48\_length(s48\_value)} 
\end{protos}
\begin{protos}
\cproto{int \ \ \ \ \ \ S48\_VECTOR\_P(s48\_value)} 
\cproto{long \ \ \ \ \ S48\_VECTOR\_LENGTH(s48\_value)} 
\cproto{s48\_value S48\_VECTOR\_REF(s48\_value, long)} 
\cproto{void \ \ \ \ \ S48\_VECTOR\_SET(s48\_value, long, s48\_value)} 
\cgcproto{s48\_value s48\_make\_vector(long, s48\_value)}
\end{protos}
\begin{protos}
\cproto{int \ \ \ \ \ \ S48\_STRING\_P(s48\_value)} 
\cproto{long \ \ \ \ \ S48\_STRING\_LENGTH(s48\_value)} 
\cproto{long \ \ \ \ \ S48\_STRING\_REF(s48\_value, long)} 
\cproto{void \ \ \ \ \ S48\_STRING\_SET(s48\_value, long, long)} 
\cgcproto{s48\_value s48\_make\_string(long, char)}
\end{protos}
\begin{protos}
\cproto{int \ \ \ \ \ \ S48\_SYMBOL\_P(s48\_value)} 
\cproto{s48\_value s48\_SYMBOL\_TO\_STRING(s48\_value)} 
\end{protos}
\begin{protos}
\cproto{int \ \ \ \ \ \ S48\_BYTE\_VECTOR\_P(s48\_value)} 
\cproto{long \ \ \ \ \ S48\_BYTE\_VECTOR\_LENGTH(s48\_value)} 
\cproto{char \ \ \ \ \ S48\_BYTE\_VECTOR\_REF(s48\_value, long)} 
\cproto{void \ \ \ \ \ S48\_BYTE\_VECTOR\_SET(s48\_value, long, int)} 
\cgcproto{s48\_value s48\_make\_byte\_vector(long, int)}
\end{protos}

\section{Calling Scheme functions from C}
\label{sec:external-callback}

External code that has been called from Scheme can call back to Scheme
 procedures using the following function.

\begin{protos}
\cproto{s48\_value s48\_call\_scheme(s48\_value p, long nargs, \ldots)}
\end{protos}
\noindent{}This calls the Scheme procedure \code{p} on \code{nargs}
 arguments, which are passed as additional arguments to \code{s48\_call\_scheme}.
There may be at most twelve arguments.
The value returned by the Scheme procedure is returned by the C procedure.
Invoking any Scheme procedure may potentially cause a garbage collection.

There are some complications that occur when mixing calls from C to Scheme
 with continuations and threads.
C only supports downward continuations (via \code{longjmp()}).
Scheme continuations that capture a portion of the C stack have to follow the
 same restriction.
For example, suppose Scheme procedure \code{s0} captures continuation \code{a}
 and then calls C procedure \code{c0}, which in turn calls Scheme procedure
 \code{s1}.
Procedure \code{s1} can safely call the continuation \code{a}, because that
 is a downward use.
When \code{a} is called Scheme~48 will remove the portion of the C stack used
 by the call to \code{c0}.
On the other hand, if \code{s1} captures a continuation, that continuation
 cannot be used from \code{s0}, because by the time control returns to
 \code{s0} the C stack used by \code{c0} will no longer be valid.
An attempt to invoke an upward continuation that is closed over a portion
 of the C stack will raise an exception.

In Scheme~48 threads are implemented using continuations, so the downward
 restriction applies to them as well.
An attempt to return from Scheme to C at a time when the appropriate
 C frame is not on top of the C stack will cause the current thread to
 block until the frame is available.
For example, suppose thread \code{t0} calls a C procedure which calls back
 to Scheme, at which point control switches to thread \code{t1}, which also
 calls C and then back to Scheme.
At this point both \code{t0} and \code{t1} have active calls to C on the
 C stack, with \code{t1}'s C frame above \code{t0}'s.
If thread \code{t0} attempts to return from Scheme to C it will block,
 as its frame is not accessible.
Once \code{t1} has returned to C and from there to Scheme, \code{t0} will
 be able to resume.
The return to Scheme is required because context switches can only occur while
 Scheme code is running.
\code{T0} will also be able to resume if \code{t1} uses a continuation to
 throw past its call to C.

\section{Interacting with the Scheme heap}
\label{sec:heap-allocation}
\label{gc}

Scheme~48 uses a copying, precise garbage collector.
Any procedure that allocates objects within the Scheme~48 heap may trigger
 a garbage collection.
Variables bound to values in the Scheme~48 heap need to be registered with
 the garbage collector so that the value will be retained and so that the
 variables will be updated if the garbage collector moves the object.
The garbage collector has no facility for updating pointers to the interiors
 of objects, so such pointers, for example the ones returned by
 \code{s48\_extract\_byte\_vector}, will likely become invalid when a garbage collection
 occurs.

\subsection{Registering objects with the GC}
\label{sec:gc-register}

A set of macros are used to manage the registration of local variables with the
 garbage collector.

\begin{protos}
\cproto{S48\_DECLARE\_GC\_PROTECT($n$)}
\cproto{void S48\_GC\_PROTECT\_$n$(s48\_value$_1$, $\ldots$, s48\_value$_n$)}
\cproto{void S48\_GC\_UNPROTECT()}
\end{protos}

\code{S48\_DECLARE\_GC\_PROTECT($n$)}, where  $1\leq n\leq 9$, allocates
 storage for registering $n$ variables.
% JAR says: what is a block?  (How to describe it? -RK)
At most one use of \code{S48\_DECLARE\_GC\_PROTECT} may occur in a
 block.
\code{S48\_GC\_PROTECT\_$n$($v_1$, $\ldots$, $v_n$)} registers the
 $n$ variables (l-values) with the garbage collector.
It must be within scope of a \code{S48\_DECLARE\_GC\_PROTECT($n$)}
 and be before any code which can cause a GC.
\code{S48\_GC\_UNPROTECT} removes the block's protected variables from
 the garbage collector's list.
It must be called at the end of the block after 
  any code which may cause a garbage collection.
Omitting any of the three may cause serious and
 hard-to-debug problems.
Notably, the garbage collector may relocate an object and
 invalidate \code{s48\_value} variables which are not protected.

A \code{gc-protection-mismatch} exception is raised if, when a C
 procedure returns to Scheme, the calls
 to \code{S48\_GC\_PROTECT()} have not been matched by an equal number of
 calls to \code{S48\_GC\_UNPROTECT()}.

Global variables may also be registered with the garbage collector.

\begin{protos}
\cproto{void * S48\_GC\_PROTECT\_GLOBAL(\cvar{value})}
\cproto{void S48\_GC\_UNPROTECT\_GLOBAL(void * handle)}
\end{protos}

\noindent{}\code{S48\_GC\_PROTECT\_GLOBAL} permanently registers the
variable \cvar{value} (an l-value of type \code{s48\_value}) with the
garbage collector.  It returns a handle pointer for use as an argument
to \code{S48\_GC\_UNPROTECT\_GLOBAL}, which unregisters the variable
again.

\subsection{Keeping C data structures in the Scheme heap}
\label{sec:external-data}

C data structures can be kept in the Scheme heap by embedding them
 inside byte vectors.
The following macros can be used to create and access embedded C objects.

\begin{protos}
\cgcproto{s48\_value S48\_MAKE\_VALUE(type)}
\cproto{type \ \ \ \ \ S48\_EXTRACT\_VALUE(s48\_value, type)}
\cproto{type * \ \ \ S48\_EXTRACT\_VALUE\_POINTER(s48\_value, type)}
\cproto{void \ \ \ \ \ S48\_SET\_VALUE(s48\_value, type, value)}
\end{protos}

\noindent{}
\code{S48\_MAKE\_VALUE} makes a byte vector large enough to hold an object
 whose type is \cvar{type}.
\code{S48\_EXTRACT\_VALUE} returns the contents of a byte vector cast to
 \cvar{type}, and \code{S48\_EXTRACT\_VALUE\_POINTER} returns a pointer
 to the contents of the byte vector.
The value returned by \code{S48\_EXTRACT\_VALUE\_POINTER} is valid only until
 the next garbage collection.

\code{S48\_SET\_VALUE} stores \code{value} into the byte vector.

%There are some convenient macros for external objects that hold
% arrays:
%
%\begin{itemize}
%\item \code{S48\_MAKE\_ARRAY($b$, $s$)} returns an external object
%  which holds an array with base type $b$ and size $s$.
%\item \code{S48\_EXTRACT\_ARRAY(\cvar{value}, $b$)} returns the address of the
%  array with base type $b$ inside external object \cvar{value}.  It does not
%  check if \cvar{value} is actually an external object.  Note that the address
%  returned by \code{S48\_EXTRACT\_ARRAY} is only valid until the next
%  \link{heap allocation}[ (see
%  Sec.~\ref{sec:heap-allocation})]{sec:heap-allocation}.
%\end{itemize}

\subsection{C code and heap images}
\label{sec:hibernation}

Scheme~48 uses dumped heap images to restore a previous system state.
The Scheme~48 heap is written into a file in a machine-independent and
 operating-system-independent format.
The procedures described above may be used to create objects in the
 Scheme heap that contain information specific to the current
 machine, operating system, or process.
A heap image containing such objects may not work correctly
 when resumed.

To address this problem, a record type may be given a `resumer'
 procedure.
On startup, the resumer procedure for a type is applied to each record of
 that type in the image being restarted.
This procedure can update the record in a manner appropriate to
 the machine, operating system, or process used to resume the
 image.

\begin{protos}
\protonoresult{define-record-resumer}{ record-type procedure}
\end{protos}

\noindent{}\code{Define-record-resumer} defines \cvar{procedure},
 which should accept one argument, to be the resumer for
 \var{record-type}.
The order in which resumer procedures are called is not specified.

The \cvar{procedure} argument to \code{define-record-resumer} may
 be \code{\#f}, in which case records of the given type are
 not written out in heap images.
When writing a heap image any reference to such a record is replaced by
 the value of the record's first field, and an exception is raised
 after the image is written.

\section{Using Scheme records in C code}

External modules can create records and access their slots
 positionally.

\begin{protos}
\cgcproto{s48\_value s48\_make\_record(s48\_value)}
\cproto{int \ \ \ \ \ \ S48\_RECORD\_P(s48\_value)} 
\cproto{s48\_value S48\_RECORD\_TYPE(s48\_value)} 
\cproto{s48\_value S48\_RECORD\_REF(s48\_value, long)} 
\cproto{void \ \ \ \ \ S48\_RECORD\_SET(s48\_value, long, s48\_value)} 
\end{protos}
%
The argument to \code{s48\_make\_record} should be a shared binding
 whose value is a record type.
In C the fields of Scheme records are only accessible via offsets,
 with the first field having offset zero, the second offset one, and
 so forth.
If the order of the fields is changed in the Scheme definition of the
 record type the C code must be updated as well.

For example, given the following record-type definition
\begin{example}
(define-record-type thing :thing
  (make-thing a b)
  thing?
  (a thing-a)
  (b thing-b))
\end{example}
 the identifier \code{:thing} is bound to the record type and can
 be exported to C:
\begin{example}
(define-exported-binding "thing-record-type" :thing)
\end{example}
\code{Thing} records can then be made in C:
\begin{example}
static s48_value
  thing_record_type_binding = S48_FALSE;

void initialize_things(void)
\{
  S48_GC_PROTECT_GLOBAL(thing_record_type_binding);
  thing_record_type_binding =
     s48_get_imported_binding("thing-record-type");
\}

s48_value make_thing(s48_value a, s48_value b)
\{
  s48_value thing;
  s48_DECLARE_GC_PROTECT(2);

  S48_GC_PROTECT_2(a, b);

  thing = s48_make_record(thing_record_type_binding);
  S48_RECORD_SET(thing, 0, a);
  S48_RECORD_SET(thing, 1, b);

  S48_GC_UNPROTECT();

  return thing;
\}
\end{example}
Note that the variables \code{a} and \code{b} must be protected
 against the possibility of a garbage collection occuring during
 the call to \code{s48\_make\_record()}.

\section{Raising exceptions from external code}
\label{sec:exceptions}

The following macros explicitly raise certain errors, immediately
 returning to Scheme~48.
Raising an exception performs all
 necessary clean-up actions to properly return to Scheme~48, including
 adjusting the stack of protected variables.

The following procedures are available for raising particular
 types of exceptions.
These never return.

\begin{protos}
\cproto{s48\_assertion\_violation(const char* who, const char* message, long count, ...)}
\cproto{s48\_error(const char* who, const char* message, long count, ...)}
\cproto{s48\_os\_error(const char* who, const char* message, long count, ...)}
\cproto{s48\_out\_of\_memory\_error()}
\end{protos}

\noindent{}An assertion violation signalled via
\code{s48\_assertion\_violation} typically means that an invalid
argument (or invalid number of arguments) has been passed.  An error
signalled via \code{s48\_error} means that an environmental error
(like an I/O error) has occurred.  In both cases, \code{who} indicates
the location of the error, typically the name of the function it
occurred in.  It may be \code{NULL}, in which the system guesses a
name.  The \code{message} argument is an error message encoded in
UTF-8.  Additional arguments may be passed that become part of the
condition object that will be raised on the Scheme side: \code{count}
indicates their number, and the arguments (which must be of type
\code{s48\_value}) follow.

The \code{s48\_os\_error} function is like \code{s48\_error}, except
that the error message is inferred from an OS error code (as in
\code{strerro}).  The \code{s48\_out\_of\_memory\_error} function
signals that the system has run out of memory.

The following macros raise assertion violations if their argument does
not have the required type.  \code{S48\_CHECK\_BOOLEAN} raises an
error if its argument is neither \code{\#t} or \code{\#f}.

\begin{protos}
\cproto{void S48\_CHECK\_BOOLEAN(s48\_value)}
\cproto{void S48\_CHECK\_SYMBOL(s48\_value)}
\cproto{void S48\_CHECK\_PAIR(s48\_value)}
\cproto{void S48\_CHECK\_STRING(s48\_value)}
\cproto{void S48\_CHECK\_INTEGER(s48\_value)}
\cproto{void S48\_CHECK\_CHANNEL(s48\_value)}
\cproto{void S48\_CHECK\_BYTE\_VECTOR(s48\_value)}
\cproto{void S48\_CHECK\_RECORD(s48\_value)}
\cproto{void S48\_CHECK\_SHARED\_BINDING(s48\_value)}
\end{protos}

\section{Unsafe functions and macros}

All of the C procedures and macros described above check that their
 arguments have the appropriate types and that indexes are in range.
The following procedures and macros are identical to those described
 above, except that they do not perform type and range checks.
They are provided for the purpose of writing more efficient code;
 their general use is not recommended.

\begin{protos}
\cproto{long \ \ \ \ \ S48\_UNSAFE\_EXTRACT\_CHAR(s48\_value)}
\cproto{s48\_value S48\_UNSAFE\_ENTER\_CHAR(long)}
\cproto{long \ \ \ \ \ S48\_UNSAFE\_EXTRACT\_INTEGER(s48\_value)}
\cproto{long \ \ \ \ \ S48\_UNSAFE\_EXTRACT\_DOUBLE(s48\_value)}
\end{protos}
\begin{protos}
\cproto{long \ \ \ \ \ S48\_UNSAFE\_EXTRACT\_FIXNUM(s48\_value)}
\cproto{s48\_value S48\_UNSAFE\_ENTER\_FIXNUM(long)}
\end{protos}
\begin{protos}
\cproto{s48\_value S48\_UNSAFE\_CAR(s48\_value)}
\cproto{s48\_value S48\_UNSAFE\_CDR(s48\_value)}
\cproto{void \ \ \ \ \ S48\_UNSAFE\_SET\_CAR(s48\_value, s48\_value)}
\cproto{void \ \ \ \ \ S48\_UNSAFE\_SET\_CDR(s48\_value, s48\_value)}
\end{protos}
\begin{protos}
\cproto{long \ \ \ \ \ S48\_UNSAFE\_VECTOR\_LENGTH(s48\_value)} 
\cproto{s48\_value S48\_UNSAFE\_VECTOR\_REF(s48\_value, long)} 
\cproto{void \ \ \ \ \ S48\_UNSAFE\_VECTOR\_SET(s48\_value, long, s48\_value)} 
\end{protos}
\begin{protos}
\cproto{long \ \ \ \ \ S48\_UNSAFE\_STRING\_LENGTH(s48\_value)} 
\cproto{char \ \ \ \ \ S48\_UNSAFE\_STRING\_REF(s48\_value, long)} 
\cproto{void \ \ \ \ \ S48\_UNSAFE\_STRING\_SET(s48\_value, long, char)} 
\end{protos}
\begin{protos}
\cproto{s48\_value S48\_UNSAFE\_SYMBOL\_TO\_STRING(s48\_value)} 
\end{protos}
\begin{protos}
\cproto{long \ \ \ \ \ S48\_UNSAFE\_BYTE\_VECTOR\_LENGTH(s48\_value)} 
\cproto{char \ \ \ \ \ S48\_UNSAFE\_BYTE\_VECTOR\_REF(s48\_value, long)} 
\cproto{void \ \ \ \ \ S48\_UNSAFE\_BYTE\_VECTOR\_SET(s48\_value, long, int)} 
\end{protos}
\begin{protos}
\cproto{s48\_value S48\_UNSAFE\_SHARED\_BINDING\_REF(s48\_value s\_b)}
\cproto{int\ \ \ \ \ \ \ S48\_UNSAFE\_SHARED\_BINDING\_P(x)}
\cproto{int\ \ \ \ \ \ \ S48\_UNSAFE\_SHARED\_BINDING\_IS\_IMPORT\_P(s48\_value s\_b)}
\cproto{s48\_value S48\_UNSAFE\_SHARED\_BINDING\_NAME(s48\_value s\_b)}
\cproto{void\ \ \ \ \ \ S48\_UNSAFE\_SHARED\_BINDING\_SET(s48\_value s\_b, s48\_value value)}
\end{protos}
\begin{protos}
\cproto{s48\_value S48\_UNSAFE\_RECORD\_TYPE(s48\_value)} 
\cproto{s48\_value S48\_UNSAFE\_RECORD\_REF(s48\_value, long)} 
\cproto{void \ \ \ \ \ S48\_UNSAFE\_RECORD\_SET(s48\_value, long, s48\_value)} 
\end{protos}
\begin{protos}
\cproto{type \ \ \ \ \ S48\_UNSAFE\_EXTRACT\_VALUE(s48\_value, type)}
\cproto{type * \ \ \ S48\_UNSAFE\_EXTRACT\_VALUE\_POINTER(s48\_value, type)}
\cproto{void \ \ \ \ \ S48\_UNSAFE\_SET\_VALUE(s48\_value, type, value)}
\end{protos}

%%% Local Variables: 
%%% mode: latex
%%% TeX-master: "manual"
%%% End: 


\chapter{Access to POSIX}

This chapter describes Scheme~48's interface to the POSIX C calls
 \cite{POSIX}.
Scheme versions of most of the functions in POSIX are provided.
Both the interface and implementation are new and are likely to
 change in future releases.
Section~\ref{function-correspondence} lists which Scheme functions
 call which C functions.

Scheme~48's POSIX interface will likely change significantly in the
 future.
The implementation is new and may have significant bugs.

The POSIX bindings are available in several structures:

\begin{center}
\begin{tabular}{ll}
 \code{posix-processes} & fork, exec, and friends \\
 \code{posix-process-data} & information about processes \\
 \code{posix-files} & files and directories \\
 \code{posix-i/o} & operations on ports \\
 \code{posix-time} & time functions \\
 \code{posix-users} & users and groups \\
 \code{posix-regexps} & regular expression matching \\
 \code{posix} & all of the above
\end{tabular}
\end{center}

Scheme~48's POSIX interface differs from
 Scsh's \cite{Shivers:Scsh-manual,Shivers:Scsh96} in several ways.
The interface here lacks Scsh's high-level constructs and utilities,
 such as the process notation, \code{awk} procedure, and parsing
 utilities.
Scheme~48 uses distinct types for some values that Scsh leaves
 as symbols or unboxed integers; these include file types, file modes,
 and user and group ids.
Many of the names and other interface details are different, as well.

\section{Process primitives}

The procedures described in this section control the creation of processes
 and the execution of programs.
They are in the structures \code{posix-process} and \code{posix}.

\subsection{Process creation and termination}
\label{processes}

\begin{protos}
\proto{fork}{}{process-id or {\tt \#f}}
\protonoresult{fork-and-forget}{ thunk}
% Handler versions?
\end{protos}
\noindent
\code{Fork} creates a new child process and returns the child's process-id in
 the parent and \code{\#f} in the child.
\code{Fork-and-forget} calls \cvar{thunk} in a new process; no process-id
 is returned.
\code{Fork-and-forget} uses an intermediate process to avoid creating
 a zombie process.

\begin{protos}
\proto{process-id?}{ x}{boolean}
\proto{process-id=?}{ process-id0 process-id1}{boolean}
\proto{process-id->integer}{ process-id}{integer}
\proto{integer->process-id}{ integer}{process-id}
\end{protos}
\noindent
%A \cvar{process-id} is a  around the actual integer id.
\code{Process-id?} is a predicate for process-ids,
\code{process-id=?} compares two to see if they are the same,
 and \code{process-id-uid} returns the actual Unix id.
\code{Process-id->integer} and \code{integer->process-id}
 convert process ids to and from integers.

\begin{protos}
\proto{process-id-exit-status}{ process-id}{integer or {\tt \#f}}
\proto{process-id-terminating-signal}{ process-id}{signal or {\tt \#f}}
\protonoresult{wait-for-child-process}{ process-id}
\end{protos}
\noindent
If a process terminates normally
 \code{process-id-exit-status} will return its exit status.
If the process is still running or was terminated by a signal then
 \code{process-id-exit-status} will return \code{\#f}.
Similarly, if a child process was terminated by a signal
 \code{process-id-terminating-signal} will return that signal and
 will return \code{\#f} if the process is still running or terminated
 normally.
\code{Wait-for-child-process} blocks until the child process terminates.
Scheme~48 may reap child processes before the user requests their
 exit status, but it does not always do so.

\begin{protos}
\protonoresult{exit}{ status}
\end{protos}
\noindent
Terminates the current process with the integer \cvar{status}
 as its exit status.

\subsection{{\tt Exec}}

\begin{protos}
\protonoresult{exec}{ program-name arg0 \ldots}
\protonoresult{exec-with-environment}{ program-name env arg0 \ldots}
\protonoresult{exec-file}{ filename arg0 \ldots}
\protonoresult{exec-file-with-environment}{ filename env arg0 \ldots}
\protonoresult{exec-with-alias}{ name lookup? maybe-env arguments}
\end{protos}
\noindent
All of these replace the current program with a new one.
They differ in how the new program is found, what its environment is,
 and what arguments it is passed.
\code{Exec} and \code{exec-with-environment}
 look up the new program in the search path,
 while \code{exec-file} and \code{exec-file-with-environment}
 execute a particular file.
The environment is either inherited from the current process
 (\code{exec} and \code{exec-file}) or given as an argument
 (\code{\ldots{}-with-environment}).
\cvar{Program-name} and \cvar{filename} and any \cvar{arg$_i$} should be strings.
 \cvar{Env} should be a list of strings of the form
 \code{"\cvar{name}=\cvar{value}"}.
The first four procedures add their first argument, \cvar{program-name} or
 \cvar{filename}, before the \cvar{arg0 \ldots} arguments.

\code{Exec-with-alias} is an omnibus procedure that subsumes the other
 four.
\cvar{Name} is looked up in the search path if \cvar{lookup?} is true
 and is used as a filename otherwise.
\cvar{Maybe-env} is either a list of strings for the environment of the
 new program or \code{\#f} in which case the new program inherits its
 environment from the current one.
\cvar{Arguments} should be a list of strings; unlike with the other four
 procedures, \cvar{name} is not added to this list (hence \code{-with-alias}).

\section{Signals}

There are two varieties of signals available, {\em named} and {\em anonymous}.
A named signal is one for which we have a symbolic name, such as \code{kill}
 or \code{pipe}.  
Anonymous signals, for which we only have the current operating system's
 signal number, have no meaning in other operating systems.
Named signals preserve their meaning in image files.
Not all named signals are available from all OS's and
 there may be multiple names for a single OS signal number.

\begin{protos}
\proto{name->signal}{ symbol}{signal or {\tt \#f}}
\proto{integer->signal}{ integer}{signal}
\proto{signal?}{ x}{boolean}
\proto{signal-name}{ signal}{symbol or {\tt \#f}}
\proto{signal-os-number}{ signal}{integer}
\proto{signal=?}{ signal0 signal1}{boolean}
\end{protos}
\noindent
\code{Name->signal} returns a (named) signal or \code{\#f} if the
 the signal \cvar{name} is not supported by the operating system.
The signal returned by \code{integer->signal} is a named signal if
 \cvar{integer} corresponds to a named signal in the current operating
 system; otherwise it returns an anonymous signal.
\code{Signal-name} returns a symbol if \cvar{signal} is named and
 \code{\#f} if it is anonymous. 
\code{Signal=?} returns \code{\#t} if \cvar{signal0} and \cvar{signal1}
 have the same operating system number and \code{\#f} if they do not.

\subsection{POSIX signals}
The following lists the names of the POSIX signals.
\begin{center}
\begin{tabular}{ll}
\code{abrt} & abort - abnormal termination (as by abort()) \\
\code{alrm} & alarm - timeout signal (as by alarm()) \\
\code{fpe } & floating point exception \\
\code{hup } & hangup - hangup on controlling terminal or death of
  controlling process \\
\code{ill } & illegal instruction \\
\code{int } & interrupt - interaction attention \\
\code{kill} & kill - termination signal, cannot be caught or ignored \\
\code{pipe} & pipe - write on a pipe with no readers \\
\code{quit} & quit - interaction termination \\
\code{segv} & segmentation violation - invalid memory reference \\
\code{term} & termination - termination signal \\
\code{usr1} & user1 - for use by applications \\
\code{usr2} & user2 - for use by applications \\
\code{chld} & child - child process stopped or terminated \\
\code{cont} & continue - continue if stopped \\
\code{stop} & stop - cannot be caught or ignored \\
\code{tstp} & interactive stop \\
\code{ttin} & read from control terminal attempted by background process \\
\code{ttou} & write to control terminal attempted by background process \\
\code{bus } & bus error - access to undefined portion of memory \\
\end{tabular}
\end{center}

\subsection{Other signals}
The following lists the names of the non-POSIX signals that the system is
 currently aware of.
\begin{center}
\W\begin{tabular}{ll}
\T\setlongtables
\T\begin{longtable}{ll}
\code{trap  } & trace or breakpoint trap \\
\code{iot   } & IOT trap - a synonym for ABRT \\
\code{emt   } & \\
\code{sys   } & bad argument to routine	(SVID) \\
\code{stkflt} & stack fault on coprocessor \\
\code{urg   } & urgent condition on socket	(4.2 BSD) \\
\code{io    } & I/O now possible		(4.2 BSD) \\
\code{poll  } & A synonym for SIGIO		(System V) \\
\code{cld   } & A synonym for SIGCHLD \\
\code{xcpu  } & CPU time limit exceeded	(4.2 BSD) \\
\code{xfsz  } & File size limit exceeded	(4.2 BSD) \\
\code{vtalrm} & Virtual alarm clock		(4.2 BSD) \\
\code{prof  } & Profile alarm clock \\
\code{pwr   } & Power failure			(System V) \\
\code{info  } & A synonym for SIGPWR \\
\code{lost  } & File lock lost \\
\code{winch } & Window resize signal		(4.3 BSD, Sun) \\
\code{unused} & Unused signal \\
\W\end{tabular}
\T\end{longtable}
\end{center}

\subsection{Sending signals}

\begin{protos}
\protonoresult{signal-process}{ process-id signal}
\end{protos}
\noindent
Send \cvar{signal} to the process corresponding to \cvar{process-id}.

\subsection{Receiving signals}

Signals received by the Scheme process can be obtained via one or more
 signal-queues.
Each signal queue has a list of monitored signals and a queue of
 received signals that have yet to be read from the signal-queue.
When the Scheme process receives a signal that signal is added to the
 received-signal queues of all signal-queues which are currently monitoring
 that particular signal.

\begin{protos}
\proto{make-signal-queue}{ signals}{signal-queue}
\proto{signal-queue?}{ x}{boolean}
\proto{signal-queue-monitored-signals}{ signal-queue}{list of signals}
\proto{dequeue-signal!}{ signal-queue}{signal}
\proto{maybe-dequeue-signal!}{ queue-queue}{signal or {\tt \#f}}
\end{protos}
\noindent
\code{Make-signal-queue} returns a new signal-queue that will monitor
 the signals in the list \cvar{signals}.
\code{Signal-queue?} is a predicate for signal queues.
\code{Signal-queue-monitored-signals} returns a list of the signals
 currently monitored by \cvar{signal-queue}.
\code{Dequeue-signal!} and \code{maybe-dequeue-signal} both return
 the next received-but-unread signal from \cvar{signal-queue}.
If \cvar{signal-queue}'s queue of signals is empty \code{dequeue-signal!}
 blocks until an appropriate signal is received.
\code{Maybe-dequeue-signal!} does not block; it returns \code{\#f} instead.

There is a bug in the current system that causes an erroneous deadlock
 error if threads are blocked waiting for signals and no other threads
 are available to run.
A work around is to create a thread that sleeps for a long time, which
 prevents any deadlock errors (including real ones):
\begin{example}
> ,open threads
> (spawn (lambda ()
           ; Sleep for a year
           (sleep (* 1000 60 60 24 365))))
\end{example}

\begin{protos}
\protonoresult{add-signal-queue-signal!}{ signal-queue signal}
\protonoresult{remove-signal-queue-signal!}{ signal-queue signal}
\end{protos}
\noindent
These two procedures can be used to add or remove signals from a
 signal-queue's list of monitored signals.
When a signal is removed from a signal-queue's list of monitored signals
 any occurances of the signal are removed from that signal-queue's pending
 signals.
In other words, \code{dequeue-signal!} and \code{maybe-dequeue-signal!}
 will only return signals that are currently on the signal-queue's list
 of signals.

\section{Process environment}

These are in structures \code{posix-process-data} and \code{posix}.

\subsection{Process identification}

\begin{protos}
\proto{get-process-id}{}{ process-id}
\proto{get-parent-process-id}{}{ process-id}
\end{protos}
\noindent
These return the process ids of the current process and its parent.
See section~\ref{processes} for operations on process ids.

\begin{protos}
\proto{get-user-id}{}{ user-id}
\proto{get-effective-user-id}{}{ user-id}
\protonoresult{set-user-id!}{ user-id}
\end{protos}

\begin{protos}
\proto{get-group-id}{}{ group-id}
\proto{get-effective-group-id}{}{ group-id}
\protonoresult{set-group-id!}{ group-id}
\end{protos}
\noindent
Every process has both the original and effective user id and group id.
The effective values may be set, but not the original ones.

\begin{protos}
\proto{get-groups}{}{ group-ids}
\proto{get-login-name}{}{ string}
\end{protos}
\noindent
\code{Get-groups} returns a list of the supplementary groups of the
 current process.
\code{Get-login-name} returns a user name for the current process.

% 4.5 Get Process Times

% 4.6 Environment Variables

\subsection{Environment variables}

\begin{protos}
\proto{lookup-environment-variable}{ string}{string or {\tt \#f}}
\proto{environment-alist}{}{alist}
\end{protos}
\noindent
\code{Lookup-environment-variable} looks up its argument in the
 environment list and returns the corresponding value or \code{\#f}
 if there is none.
\code{Environment-alist} returns the entire environment as a list of
 \code{(\cvar{name-string} . \cvar{value-string})} pairs.

\section{Users and groups}

\cvar{User-id}\/s and \cvar{group-id}\/s are boxed integers representing
 Unix users and groups.
The procedures in this section are in structures \code{posix-users} and
 \code{posix}.

\begin{protos}
\proto{user-id?}{ x}{boolean}
\proto{user-id=?}{ user-id0 user-id1}{boolean}
\proto{user-id->integer}{ user-id}{integer}
\proto{integer->user-id}{ integer}{user-id}
\end{protos}

\begin{protos}
\proto{group-id?}{ x}{boolean}
\proto{group-id=?}{ group-id0 group-id1}{boolean}
\proto{group-id->integer}{ group-id}{integer}
\proto{integer->group-id}{ integer}{group-id}
\end{protos}
\noindent
User-ids and group-ids have their own
 own predicates and comparison, boxing, and unboxing functions.

\begin{protos}
\proto{user-id->user-info}{ user-id}{user-info}
\proto{name->user-info}{ string}{user-info}
\end{protos}
\noindent
These return the user info for a user identified by user-id or name.

\begin{protos}
\proto{user-info?}{ x}{ boolean}
\proto{user-info-name}{ user-info}{ string}
\proto{user-info-id}{ user-info}{ user-id}
\proto{user-info-group}{ user-info}{ group-id}
\proto{user-info-home-directory}{ user-info}{ string}
\proto{user-info-shell}{ user-info}{ string}
\end{protos}
\noindent
A \code{user-info} contains information about a user.
Available are the user's name, id, group, home directory, and shell.

\begin{protos}
\proto{group-id->group-info}{ group-id}{group-info}
\proto{name->group-info}{ string}{group-info}
\end{protos}
\noindent
These return the group info for a group identified by group-id or name.

\begin{protos}
\proto{group-info?}{ x}{ boolean}
\proto{group-info-name}{ group-info}{ string}
\proto{group-info-id}{ group-info}{ group-id}
\proto{group-info-members}{ group-info}{ user-ids}
\end{protos}
\noindent
A \code{group-info} contains information about a group.
Available are the group's name, id, and a list of members.

\section{OS and machine identification}

These procedures return strings that are supposed to identify the current
 OS and machine.
The POSIX standard does not indicate the format of the strings.
The procedures are in structures \code{posix-platform-names} and \code{posix}.

\begin{protos}
\proto{os-name}{}{string}
\proto{os-node-name}{}{string}
\proto{os-release-name}{}{string}
\proto{os-version-name}{}{string}
\proto{machine-name}{}{string}
\end{protos}

% 4.8 Configurable System Variables

\section{Files and directories}

These procedures are in structures \code{posix-files} and \code{posix}.

\subsection{Directory streams}

Directory streams are like input ports, with each read operation
 returning the next name in the directory.

\begin{protos}
\proto{open-directory-stream}{ name}{directory}
\proto{directory-stream?}{ x}{boolean}
\proto{read-directory-stream}{ directory}{name or {\tt \#f}}
\protonoresult{close-directory-stream}{ directory}
\end{protos}
\noindent
\code{Open-directory-stream} opens a new directory stream.
\code{Directory-stream?} is a predicate that recognizes directory streams.
\code{Read-directory-stream} returns the next name in the directory or
 \code{\#f} if all names have been read.
\code{Close-directory-stream} closes a directory stream.

\begin{protos}
\proto{list-directory}{ name}{list of strings}
\end{protos}
\noindent
This is the obvious utility; it returns a list of the names in directory
 \cvar{name}.

% 5.2 Working Directory
\subsection{Working directory}

\begin{protos}
\proto{working-directory}{}{string}
\protonoresult{set-working-directory!}{ string}
\end{protos}
\noindent
These return and set the working directory.

% 5.3 File Creation
\subsection{File creation and removal}

\begin{protos}
\proto{open-file}{ path file-options}{port}
\proto{open-file}{ path file-options file-mode}{port}
\end{protos}
\noindent
\code{Open-file} opens a port to the file named by string \cvar{path}.
The \cvar{file-options} argument determines various aspects of the 
 returned port.
The optional \cvar{file-mode} argument is used only if the file to be opened
 does not already exist.
The returned port is an input port if \cvar{file-options} includes
 \code{read-only}; otherwise it returns an output port.
\code{Dup-switching-mode} can be used to open an input port for
 output ports opened with the \code{read/write} option.

\begin{protos}
\syntaxproto{file-options}{ \cvar{file-option-name} \ldots}{file-options}
\proto{file-options-on?}{ file-options file-options}{boolean}
\end{protos}
\noindent
The syntax \code{file-options} returns a file-option with the
 indicated options set.
\code{File-options-on?} returns true if its first argument includes all of
 the options listed in the second argument.
The following file options may be used with \code{open-file}.

\begin{center}
\begin{tabular}{llp{2.8in}}
 & \code{create} & create file if it does not already exist; a file-mode argument
 is required with this option \\
 & \code{exclusive} & an error will be raised if this option and \code{create}
 are both set and the file already exists \\
 & \code{no-controlling-tty} & if \cvar{path} is a terminal device this option
 causes the terminal to not become the controlling terminal of the process\\
 & \code{truncate} & file is truncated \\
 & \code{append} & writes are appended to existing contents \\
 & \code{nonblocking} & read and write operations do not block \\
 & \code{read-only} & port may not be written \\
 & \code{read-write} & file descriptor may be read or written \\
 & \code{write-only} & port may not be read
\end{tabular}
\end{center}
\noindent Only one of the last three options may be used.

For example
\begin{example}
(open-file "some-file.txt"
           (file-options create write-only)
           (file-mode read owner-write))
\end{example}
 returns an output port that writes to a newly-created file that can be
 read by anyone and written only by the owner.
Once the file exists,
\begin{example}
(open-file "some-file.txt"
           (file-options append write-only))
\end{example}
will open an output port that appends to the file.

The \code{append} and \code{nonblocking} options and the read/write nature of
 the port can be read using \code{i/o-flags}.
The \code{append} and \code{nonblocking} options can be set
 using \code{set-i/o-flags!}.

To keep port operations from blocking the Scheme~48 process, output
 ports are set to be nonblocking at the time of creation (input ports
 are managed using \code{select()}).
You can use \code{set-i/o-flags!} to make an output port blocking, for
 example just before a fork, but care should be exercised.
The Scheme~48 runtime code may get confused if an I/O operation blocks.

\begin{protos}
\protonoresult{set-file-creation-mask!}{ file-mode}
\end{protos}
\noindent
Sets the file creation mask to be \cvar{file-mode}.
Bits set in \cvar{file-mode} are cleared in the modes of any files or
 directories created by the current process.

\begin{protos}
\protonoresult{link}{ existing new}
\end{protos}
\noindent
\code{Link} makes path \cvar{new} be a new link to the file
 pointed to by path \cvar{existing}.
The two paths must be in the same file system.

% 5.4 Special File Creation

\begin{protos}
\protonoresult{make-directory}{ name file-mode}
\protonoresult{make-fifo}{ file-mode}
\end{protos}
\noindent
These two procedures make new directories and fifo files.

% 5.5 File Removal

\begin{protos}
\protonoresult{unlink}{ path}
\protonoresult{remove-directory}{ path}
\protonoresult{rename}{ old-path new-path}
\end{protos}
\noindent
\code{Unlink} removes the link indicated by \cvar{path}.
\code{Remove-directory} removes the indicated (empty) directory.
\code{Rename} moves the file pointed to by \cvar{old-path} to the
 location pointed to by \cvar{new-path} (the two paths must be in
 the same file system).
Any other links to the file remain unchanged.

% 5.6.3 (out of POSIX order)

\begin{protos}
\proto{accessible?}{ path access-mode . more-modes}{boolean}
\syntaxproto{access-mode}{ \cvar{mode-name}}{access-mode}
\end{protos}
\noindent
\code{Accessible?} returns true if \cvar{path} is a file that
 can be accessed in the listed mode.
If more than one mode is specified \code{accessible?} returns true
 if all of the specified modes are permitted.
The \cvar{mode-name}\/s are: \code{read}, \code{write}, \code{execute},
 \code{exists}.

% 5.6 File Characteristics
\subsection{File information}

\begin{protos}
\proto{get-file-info}{ name}{file-info}
\proto{get-file/link-info}{ name}{file-info}
\proto{get-port-info}{ fd-port}{file-info}
\end{protos}
\noindent
\code{Get-file-info} and \code{get-file/link-info} both return
 a file info record for the named file.
\code{Get-file-info} follows symbolic links while \code{get-file/link-info}
 does not.
\code{Get-port-info} returns a file info record for the file
 which \cvar{port} reads from or writes to.
An error is raised if \cvar{fd-port} does not read from or write to a
 file descriptor.

\begin{protos}
\proto{file-info?}{ x}{boolean}
\proto{file-info-name}{ file-info}{string}
\end{protos}
\noindent
\code{File-info?} is a predicate for file-info records.
\code{File-info-name} is the name which was used to get \code{file-info},
 either as passed to \code{get-file-info} or \code{get-file/link-info},
 or used to open the port passed to \code{get-port-info}.

\begin{protos}
\proto{file-info-type}{ file-info}{file-type}
\proto{file-type?}{ x}{boolean}
\proto{file-type-name}{ file-type}{symbol}
\syntaxproto{file-type}{ \cvar{type}}{file-type}
\end{protos}
\noindent
%I need to add the rest - the procedure version of file-type, enum-case(?).
\code{File-info-type} returns the type of the file, as a file-type object
File types may be compared using \code{eq?}.
The valid file types are:

\begin{center}
\begin{tabular}{l}
 \code{regular} \\
 \code{directory} \\
 \code{character-device} \\
 \code{block-device} \\
 \code{fifo} \\
 \code{symbolic-link} \\
 \code{socket} \\
 \code{other}
\end{tabular}
\end{center}

\noindent
\code{Symbolic-link} and \code{socket} are not required by POSIX.

\begin{protos}
\proto{file-info-device}{ file-info}{integer}
\proto{file-info-inode}{ file-info}{integer}
\end{protos}
\noindent
The device and inode numbers uniquely determine a file.

\begin{protos}
\proto{file-info-link-count}{ file-info}{integer}
\proto{file-info-size}{ file-info}{integer}
\end{protos}
\noindent
These return the number of links to a file and the file size in bytes.
The size is only meaningful for regular files.

\begin{protos}
\proto{file-info-owner}{ file-info}{user-id}
\proto{file-info-group}{ file-info}{group-id}
\proto{file-info-mode}{ file-info}{file-mode}
\end{protos}
\noindent
These return the owner, group, and access mode of a file.

\begin{protos}
\proto{file-info-last-access}{ file-info}{time}
\proto{file-info-last-modification}{ file-info}{time}
\proto{file-info-last-info-change}{ file-info}{time}
\end{protos}
\noindent
These return the time the file was last read, modified, or had its
 status modified

% Modes
\subsection{File modes}

A file mode is a boxed integer representing a file protection mask.

\begin{protos}
\syntaxproto{file-mode}{ permission-name \ldots}{file-mode}
\proto{file-mode?}{ x}{boolean}
\proto{file-mode+}{ file-mode \ldots}{file-mode}
\proto{file-mode-}{ file-mode0 file-mode1}{file-mode}
\end{protos}
\noindent
\code{File-mode} is syntax for creating file modes.
The mode-names are listed below.
\code{File-mode?} is a predicate for file modes.
\code{File-mode+} returns a mode that contains all of permissions of
 its arguments.
\code{File-mode-} returns a mode that has all of the permissions of
 \cvar{file-mode0} that are not in \cvar{file-mode1}.

\begin{protos}
\proto{file-mode=?}{ file-mode0 file-mode1}{boolean}
\proto{file-mode<=?}{ file-mode0 file-mode1}{boolean}
\proto{file-mode>=?}{ file-mode0 file-mode1}{boolean}
\end{protos}
\noindent
\code{File-mode=?} returns true if the two modes are exactly the same.
\code{File-mode<=?} returns true if \cvar{file-mode0} has a subset
 of the permissions of \cvar{file-mode1}.
\code{File-mode>=?} is \code{file-mode<=?} with the arguments reversed.

\begin{protos}
\proto{file-mode->integer}{ file-mode}{integer}
\proto{integer->file-mode}{ integer}{file-mode}
\end{protos}
\noindent
\code{Integer->file-mode} and \code{file-mode->integer} translate file modes
 to and from the classic Unix file mode masks.
These may not be the masks used by the underlying OS.

\begin{center}
\begin{tabular}{lll}
Permission name & Bit mask \\
\code{set-uid} & \code{\#o4000} & set user id when executing \\
\code{set-gid} & \code{\#o2000} & set group id when executing \\
\code{owner-read} & \code{\#o0400} & read by owner \\
\code{owner-write} & \code{\#o0200} & write by owner \\
\code{owner-exec} & \code{\#o0100} & execute (or search) by owner \\
\code{group-read} & \code{\#o0040} & read by group \\
\code{group-write} & \code{\#o0020} & write by group \\
\code{group-exec} & \code{\#o0010} & execute (or search) by group \\
\code{other-read} & \code{\#o0004} & read by others \\
\code{other-write} & \code{\#o0002} & write by others \\
\code{other-exec} & \code{\#o0001} & execute (or search) by others \\
\end{tabular}
\end{center}

\begin{center}
\begin{tabular}{lll}
\multicolumn{3}{l}{Names for sets of permissions} \\
\code{owner} & \code{\#o0700} & read, write, and execute by owner \\
\code{group} & \code{\#o0070} & read, write, and execute by group \\
\code{other} & \code{\#o0007} & read, write, and execute by others \\
\code{read} & \code{\#o0444} & read by anyone \\
\code{write} & \code{\#o0222} & write by anyone \\
\code{exec} & \code{\#o0111} & execute by anyone \\
\code{all} & \code{\#o0777} & anything by anyone
\end{tabular}
\end{center}

%----------------
% Time - seconds since the epoch.
\section{Time}

These procedures are in structures \code{posix-time} and \code{posix}.

\begin{protos}
\proto{make-time}{ integer}{time}
\proto{current-time}{}{time}
\proto{time?}{ x}{boolean}
\proto{time-seconds}{ time}{integer}
\end{protos}
\noindent
A \code{time} record contains an integer that represents time as
 the number of second since the Unix epoch (00:00:00 GMT, January 1, 1970).
\code{Make-time} and \code{current-time} return \code{time}\/s, with
 \code{make-time}'s using its argument while \code{current-time}'s has
 the current time.
\code{Time?} is a predicate that recognizes \code{time}\/s and
 \code{time-seconds} returns the number of seconds \cvar{time} represents.

\begin{protos}
\proto{time=?}{ time time}{boolean}
\proto{time<?}{ time time}{boolean}
\proto{time<=?}{ time time}{boolean}
\proto{time>?}{ time time}{boolean}
\proto{time>=?}{ time time}{boolean}
\end{protos}
\noindent
These perform various comparison operations on the \code{time}\/s.

\begin{protos}
\proto{time->string}{ time}{string}
\end{protos}
\noindent
\code{Time->string} returns a string representation of \cvar{time} in the
 following form.
\begin{example}
"Wed Jun 30 21:49:08 1993
"
\end{example}

%----------------
% Dates - what a mess.
%
% This is not yet working.

\section{I/O}

These procedures are in structures \code{posix-i/o} and \code{posix}.

% Rest of 5.6

% int chmod(char *path, mode_t mode)
% int fchmod(int fd, mode_t mode)

% int chown(char *path, uid_t owner, gid_t group)
% int utime(char *path, struct utimbuf * times)

% int ftruncate(int fd, off_t length)

%----------------
% 5.7 Configurable Pathname Variables
%
% long pathconf(char *path, int name)
% long fpathconf(int fd, int name)

%----------------
% 6.1 Pipes

\begin{protos}
\proto{open-pipe}{}{input-port + output-port}
\end{protos}
\noindent
\code{Open-pipe} creates a new pipe and returns the two ends as an
 input port and an output port.

% 6.2 File descriptor manipulation.

A {\em file descriptor} port (or \cvar{fd-port})
 is a port that reads to or writes from an OS file descriptor.
Fd-ports are returned by \code{open-input-file}, \code{open-output-file},
 \code{open-file}, \code{open-pipe}, and other procedures.
% Richard says: add an example of a non-fd port once the extended ports
% are described in the manual.

\begin{protos}
\proto{fd-port?}{ port}{boolean}
\proto{port->fd}{ port}{integer or {\tt \#f}}
\end{protos}
\noindent
\code{Fd-port?} returns true if its argument is an fd-port.
\code{Port->fd} returns the file descriptor associated with 
 or \code{\#f} if \cvar{port} is not an fd-port.

\begin{protos}
\protonoresult{remap-file-descriptors}{ fd-spec \ldots}
\end{protos}
\noindent
\code{Remap-file-descriptors} reassigns file descriptors to ports.
The \cvar{fd-specs} indicate which port is to be mapped to each
 file descriptor: the first gets file descriptor \code{0}, the second gets
 \code{1}, and so forth.
A \cvar{fd-spec} is either a port that reads from or writes to
 a file descriptor,
 or \code{\#f}, with \code{\#f} indicating that the corresponding file
 descriptor is not used.
Any open ports not listed are marked `close-on-exec'.
The same port may be moved to multiple new file descriptors.

For example,
\begin{example}
(remap-file-descriptors (current-output-port)
                        \#f
                        (current-input-port))
\end{example}
moves the current output port to file descriptor \code{0} and the
current input port to file descriptor \code{2}.

\begin{protos}
\proto{dup}{ fd-port}{fd-port}
\proto{dup-switching-mode}{ fd-port}{fd-port}
\proto{dup2}{ fd-port file-descriptor}{fd-port}
\end{protos}
\noindent
These change \cvar{fd-port}'s file descriptor and return a new port
 that uses \cvar{ports}'s old file descriptor.
\code{Dup} uses the lowest unused file descriptor and \code{dup2} uses the
 one provided.
\code{Dup-switching-mode} is the same as \code{dup} except that the returned
 port is an input port if the argument was an output port and vice
 versa.
If any existing port uses the file descriptor passed to \code{dup2}, that
 port is closed.

\begin{protos}
\protonoresult{close-all-but}{ port \ldots}
\end{protos}
\noindent
\code{Close-all-but} closes all file descriptors whose associated ports
 are not passed to it as arguments.

% 6.3 File Descriptor Reassignment
%
% int close(int fd)     % Use close-{input|output}-{port|channel}
%
% 6.4 Input and Output
%
% read() and write()    ; Already available in various forms.

% 6.5 Control Operations on Files

%; fcntl(fd, F_DUPFD, target_fd)            ; Use DUP instead.
%
%; Descriptor flags
%; fcntl(fd, F_GETFD)
%; fcntl(fd, F_SETFD, flags)
%;
%; The only POSIX flag is FD_CLOEXEC, so that's all we do.

\begin{protos}
\proto{close-on-exec?}{ port}{boolean}
\protonoresult{set-close-on-exec?!}{ port boolean}
\end{protos}
\noindent
\code{Close-on-exec?} returns true if \code{port} will be closed
 when a new program is exec'ed.
\code{Set-close-on-exec?!} sets \code{port}'s close-on-exec flag.

%; Status flags
%; fcntl(fd, F_GETFL)
%; fcntl(fd, F_SETFL, flags)

\begin{protos}
\proto{i/o-flags}{ port}{file-options}
\protonoresult{set-i/o-flags!}{ port file-options}
\end{protos}
\noindent
These two procedures read and write various options for \code{port}.
The options that can be read are \code{append}, \code{nonblocking},
 \code{read-only}, \code{write-only}, and \code{read/write}.
Only the \code{append} and \code{nonblocking} can be written.

% off_t lseek(int fd, off_t offset, int whence)

%\subsection{Terminals}

\begin{protos}
\proto{port-is-a-terminal?}{ port}{boolean}
\proto{port-terminal-name}{ port}{string}
\end{protos}
\noindent
\code{Port-is-a-terminal?} returns true if \cvar{port} has an underlying
 file descriptor that is associated with a terminal.
For such ports \code{port-terminal-name} returns the name of the
 terminal, for all others it returns \code{\#f}.

% 6. File Synchronization
%
% int fsync(int fd)              % optional
% int fdatasync(int fd)          % optional
%
% 7. Asynchronous Input and Output
%
% All optional

\section{Regular expressions}
% Regexps - where is the C interface defined? (In part 2, the shell book)

The procedures in this section provide access to POSIX regular expression
 matching.
The regular expression syntax and semantics are far too complex to
 be described here.
Because the C interface uses zero bytes for marking the ends of strings,
 patterns and strings that contain zero bytes will not work correctly.

These procedures are in structures \code{posix-regexps} and \code{posix}.

An abstract data type for creating POSIX regular expressions is
 described in section~\ref{regexp-adt}.

\begin{protos}
\proto{make-regexp}{ string . regexp-options}{regexp}
\syntaxproto{regexp-option}{ \cvar{option-name}}{regexp-option}
\end{protos}
\noindent
\code{Make-regexp} makes a new regular expression, using \cvar{string}
 as the pattern.
The possible option names are:

\begin{center}
\begin{tabular}{ll}
\code{extended} & use the extended patterns \\
\code{ignore-case} & ignore case when matching \\
\code{submatches} & report submatches \\
\code{newline} & treat newlines specially 
\end{tabular}
\end{center}

The regular expression is not compiled until it matched against a string,
 so any errors in the pattern string will not be reported until that
 point.

\begin{protos}
\proto{regexp?}{ x}{boolean}
\end{protos}
\noindent
This is a predicate for regular expressions.

\begin{protos}
\protonoresult{regexp-match}{ regexp string submatches? starts-line? ends-line?}
\protoresult{boolean or list of matches}
\proto{match?}{ x}{boolean}
\proto{match-start}{ match}{integer}
\proto{match-end}{ match}{integer}
\end{protos}
\noindent
\code{Regexp-match} matches the regular expression against the characters
 in \cvar{string}.
If the string does not match the regular expression, \code{regexp-match}
 returns \code{\#f}.
If the string does match, then a list of match records is returned
 if \cvar{submatches?} is true, or \code{\#t} is returned if it is not.
Each match record contains the index of the character at the beginning
 of the match and one more than the index of the character at the end.
The first match record gives the location of the substring that matched
 \cvar{regexp}.
If the pattern in \cvar{regexp} contained submatches, then the results
 of these are returned in order, with a match records reporting submatches
 that succeeded and \code{\#f} in place of those that did not.

\cvar{Starts-line?} should be true if \cvar{string} starts at the beginning
 of a line and \cvar{ends-line?} should be true if it ends one.

\section{C to Scheme correspondence}
\label{function-correspondence}

The following table lists the Scheme procedures that correspond to
 particular C procedures.
Not all of the Scheme procedures listed are part of the POSIX interface.

\W\begin{tabular}{ll}
\T\setlongtables
\T\begin{longtable}{ll}
C procedure & Scheme procedure(s) \\
\T\endfirsthead
\T C procedure & Scheme procedure(s) \\
\T\endhead
\code{access}&\code{accessible?}\\
\code{chdir}&\code{set-working-directory!}\\
\code{close}&\code{close-input-port, close-output-port,}\\
 &\code{\ \ close-channel, close-socket}\\
\code{closedir}&\code{close-directory-stream}\\
\code{creat}&\code{open-file}\\
\code{ctime}&\code{time->string}\\
\code{dup}&\code{dup, dup-switching-mode}\\
\code{dup2}&\code{dup2}\\
\code{exec[l|v][e|p|$\epsilon$]}
 &\code{exec, exec-with-environment,} \\
 &\code{\ \ exec-file, exec-file-with-environment,} \\
 &\code{\ \ exec-with-alias}\\
\code{\_exit}&\code{exit}\\
\code{fcntl}&\code{io-flags, set-io-flags!,}\\
 &\code{\ \ close-on-exec, set-close-on-exec!}\\
\code{fork}&\code{fork, fork-and-forget}\\
\code{fstat}&\code{get-port-info}\\
\code{getcwd}&\code{working-directory}\\
\code{getegid}&\code{get-effective-group-id}\\
\code{getenv}&\code{lookup-environment-variable,} \\
 &\code{\ \ environment-alist}\\
\code{geteuid}&\code{get-effective-user-id}\\
\code{getgid}&\code{get-group-id}\\
\code{getgroups}&\code{get-groups}\\
\code{getlogin}&\code{get-login-name}\\
\code{getpid}&\code{get-process-id}\\
\code{getppid}&\code{get-parent-process-id}\\
\code{getuid}&\code{get-user-id}\\
\code{isatty}&\code{port-is-a-terminal?}\\
\code{link}&\code{link}\\
\code{lstat}&\code{get-file/link-info}\\
\code{mkdir}&\code{make-directory}\\
\code{mkfifo}&\code{make-fifo}\\
\code{open}&\code{open-file}\\
\code{opendir}&\code{open-directory-stream}\\
\code{pipe}&\code{open-pipe}\\
\code{read}&\code{read-char, read-block}\\
\code{readdir}&\code{read-directory-stream}\\
\code{rename}&\code{rename}\\
\code{rmdir}&\code{remove-directory}\\
\code{setgid}&\code{set-group-id!}\\
\code{setuid}&\code{set-user-id!}\\
\code{stat}&\code{get-file-info}\\
\code{time}&\code{current-time}\\
\code{ttyname}&\code{port-terminal-name}\\
\code{umask}&\code{set-file-creation-mask!}\\
\code{uname}&\code{os-name, os-node-name,} \\
 &\code{\ \ os-release-name, os-version-name,} \\
 &\code{\ \ machine-name}\\
\code{unlink}&\code{unlink}\\
\code{waitpid}&\code{wait-for-child-process}\\
\code{write}&\code{write-char, write-block}\\
% The ones that aren't there yet.
%wait \\
%kill \\
%sigemptyset, sigfillset, sigaddset, digdelset, sigismember \\
%sigaction \\
%sigsuspend \\
%alarm \\
%pause \\
%sleep \\
%getpgrp \\
%setsid \\
%times \\
%ctermid \\
%sysconf \\
%rewinddir \\
%chmod \\
%chown \\
%utime \\
%ftruncate \\
%pathconf, fpathconf \\
%lseek \\
%fsync \\
\W\end{tabular}
\T\end{longtable}


\begin{thebibliography}{10}
\addtocounter{chapter}{1}
\T\addcontentsline{toc}{chapter}{\protect\numberline{\thechapter}{Bibliography}}

\bibitem{POSIX}
Information technology -- Portable Operating System Interface (POSIX).
\newblock ISO/IEC 9945-1 ANSI/IEEE Std 1003.1.
\newblock 2nd Ed., 1996.

\bibitem{Clinger-Rees:Macros}
William Clinger and Jonathan~Rees.
\newblock Macros that work.
\newblock {\em Principles of Programming Languages}, January 1991.

\bibitem{Clinger-Rees:R4RS}
William Clinger and Jonathan~Rees (editors).
\newblock Revised${}^4$ report on the algorithmic language {S}cheme.
\newblock {\em LISP Pointers} IV(3):1--55, July-September 1991.

\bibitem{Curtis-Rauen:Modules}
Pavel Curtis and James Rauen.
\newblock A module system for Scheme.
\newblock {\em ACM Conference on Lisp and Functional Programming,}
pages 13--19, 1990.

\bibitem{Kelsey-Rees:Scheme48}
Richard~Kelsey and Jonathan~Rees.
\newblock A Tractable {Scheme} Implementation.
\newblock {\em Lisp and Symbolic Computation} 7:315--335 1994.

\bibitem{R5RS}
Richard Kelsey, Will Clinger, Jonathan Rees (editors).
\newblock Revised$^5$ Report on the Algorithmic Language Scheme.
\newblock {\em Higher-Order and Symbolic Computation,} Vol. 11, No. 1,
 September, 1998.
\newblock and {\em ACM SIGPLAN Notices}, Vol. 33, No. 9, October, 1998.

\bibitem{MacQueen:Modules}
David MacQueen.
\newblock Modules for Standard ML.
\newblock {\em ACM Conference on Lisp and Functional Programming,}
1984.

\bibitem{Rees-Donald:Program}
Jonathan Rees and Bruce Donald.
\newblock Program mobile robots in Scheme.
\newblock {\em International Conference on Robotics and
Automation,} IEEE, 1992. 

\bibitem{Sheldon-Gifford:Static}
Mark A.~Sheldon and David K.~Gifford.
\newblock Static dependent types for first-class modules.
\newblock {\em ACM Conference on Lisp and Functional Programming,}
pages 20--29, 1990.

\bibitem{Shivers:Scsh-manual}
Olin Shivers, Brian D.~Carlstrom, Martin Gasbichler and Mike Sperber.
\newblock Scsh Reference Manual, scsh release 0.6.6.
\newblock Available at URL \texttt{http://www.scsh.net/}.

\bibitem{Shivers:Scsh96}
Olin Shivers.
\newblock A universal scripting framework, 
 or Lambda: the ultimate ``little language''.
\newblock {\em Concurrency and Parallelism, Programming, Networking, and
 Security,} pages 254--265, Springer 1996.
\newblock Joxan Jaffar and Roland H.~C.~Yap, editors.

\end{thebibliography}


\texonly{
\begin{myindex}
\addtocounter{chapter}{1}
\addcontentsline{toc}{chapter}{\protect\numberline{\thechapter}{Index}}

The principal entry for each term, procedure, or keyword is listed
first, separated from the other entries by a semicolon.

\bigskip

\indexspace
\item{\tt{accessible?}}{\hskip .75em}71
\item{\tt{add-signal-queue-signal!}}{\hskip .75em}67
\item{\tt{any-match?}}{\hskip .75em}47
\item{\tt{arithmetic-shift}}{\hskip .75em}30
\item{\tt{array}}{\hskip .75em}31
\item{\tt{array->vector}}{\hskip .75em}32
\item{\tt{array-dimensions}}{\hskip .75em}32
\item{\tt{array-ref}}{\hskip .75em}32
\item{\tt{array-set!}}{\hskip .75em}32
\item{\tt{array?}}{\hskip .75em}32
\item{\tt{ascii->char}}{\hskip .75em}30
\item{\tt{ascii-limit}}{\hskip .75em}30
\item{\tt{ascii-range}}{\hskip .75em}44
\item{\tt{ascii-ranges}}{\hskip .75em}44
\item{\tt{ascii-whitespaces}}{\hskip .75em}30
\indexspace
\item{\tt{bit-count}}{\hskip .75em}31
\item{\tt{bitwise-and}}{\hskip .75em}30
\item{\tt{bitwise-ior}}{\hskip .75em}30
\item{\tt{bitwise-not}}{\hskip .75em}30
\item{\tt{bitwise-xor}}{\hskip .75em}30
\item{\tt{byte-vector}}{\hskip .75em}31
\item{\tt{byte-vector-length}}{\hskip .75em}31
\item{\tt{byte-vector-ref}}{\hskip .75em}31
\item{\tt{byte-vector-set!}}{\hskip .75em}31
\item{\tt{byte-vector?}}{\hskip .75em}31
\indexspace
\item{\tt{call-external}}{\hskip .75em}54; 53
\item{\tt{call-external-value}}{\hskip .75em}53
\item{\tt{call-imported-binding}}{\hskip .75em}53
\item{\tt{cell-ref}}{\hskip .75em}31
\item{\tt{cell-set!}}{\hskip .75em}31
\item{\tt{cell?}}{\hskip .75em}31
\item{\tt{char->ascii}}{\hskip .75em}30
\item{\tt{close-all-but}}{\hskip .75em}75
\item{\tt{close-directory-stream}}{\hskip .75em}69
\item{\tt{close-on-exec?}}{\hskip .75em}75
\item{\tt{close-socket}}{\hskip .75em}37
\item{\tt{compound-interface}}{\hskip .75em}21
\item{\tt{copy-array}}{\hskip .75em}31
\item{\tt{current-time}}{\hskip .75em}74
\indexspace
\item{\tt{default-hash-function}}{\hskip .75em}36
\item{\tt{define-exported-binding}}{\hskip .75em}51
\item{\tt{define-imported-binding}}{\hskip .75em}52
\item{\tt{define-interface}}{\hskip .75em}21
\item{\tt{define-record-discloser}}{\hskip .75em}33
\item{\tt{define-record-resumer}}{\hskip .75em}59
\item{\tt{define-structure}}{\hskip .75em}18
\item{\tt{dequeue-signal!}}{\hskip .75em}67
\item{\tt{directory-stream?}}{\hskip .75em}69
\item{\tt{dup}}{\hskip .75em}75
\item{\tt{dup-switching-mode}}{\hskip .75em}75
\item{\tt{dup2}}{\hskip .75em}75
\item{\tt{dynamic-load}}{\hskip .75em}54
\indexspace
\item{\tt{environment-alist}}{\hskip .75em}68
\item{\tt{exact-match?}}{\hskip .75em}47
\item{\tt{exec}}{\hskip .75em}64
\item{\tt{exec-file}}{\hskip .75em}64
\item{\tt{exec-file-with-environment}}{\hskip .75em}64
\item{\tt{exec-with-alias}}{\hskip .75em}64
\item{\tt{exec-with-environment}}{\hskip .75em}64
\item{\tt{exit}}{\hskip .75em}64
\item{\tt{external-name}}{\hskip .75em}54
\item{\tt{external-value}}{\hskip .75em}54
\item{\tt{external?}}{\hskip .75em}54
\indexspace
\item{\tt{fd-port?}}{\hskip .75em}74
\item{\tt{file-info-device}}{\hskip .75em}72
\item{\tt{file-info-group}}{\hskip .75em}72
\item{\tt{file-info-inode}}{\hskip .75em}72
\item{\tt{file-info-last-access}}{\hskip .75em}72
\item{\tt{file-info-last-info-change}}{\hskip .75em}72
\item{\tt{file-info-last-modification}}{\hskip .75em}72
\item{\tt{file-info-link-count}}{\hskip .75em}72
\item{\tt{file-info-mode}}{\hskip .75em}72
\item{\tt{file-info-name}}{\hskip .75em}71
\item{\tt{file-info-owner}}{\hskip .75em}72
\item{\tt{file-info-size}}{\hskip .75em}72
\item{\tt{file-info-type}}{\hskip .75em}72
\item{\tt{file-info?}}{\hskip .75em}71
\item{\tt{file-mode+}}{\hskip .75em}73
\item{\tt{file-mode-}}{\hskip .75em}73
\item{\tt{file-mode->integer}}{\hskip .75em}73
\item{\tt{file-mode<=?}}{\hskip .75em}73
\item{\tt{file-mode=?}}{\hskip .75em}73
\item{\tt{file-mode>=?}}{\hskip .75em}73
\item{\tt{file-mode?}}{\hskip .75em}72
\item{\tt{file-options-on?}}{\hskip .75em}70
\item{\tt{file-type-name}}{\hskip .75em}72
\item{\tt{file-type?}}{\hskip .75em}72
\item{\tt{fluid}}{\hskip .75em}36
\item{\tt{fork}}{\hskip .75em}64
\item{\tt{fork-and-forget}}{\hskip .75em}64
\indexspace
\item{\tt{get-effective-group-id}}{\hskip .75em}68
\item{\tt{get-effective-user-id}}{\hskip .75em}68
\item{\tt{get-external}}{\hskip .75em}54
\item{\tt{get-file-info}}{\hskip .75em}71
\item{\tt{get-file/link-info}}{\hskip .75em}71
\item{\tt{get-group-id}}{\hskip .75em}68
\item{\tt{get-groups}}{\hskip .75em}68
\item{\tt{get-host-name}}{\hskip .75em}37
\item{\tt{get-login-name}}{\hskip .75em}68
\item{\tt{get-parent-process-id}}{\hskip .75em}67
\item{\tt{get-port-info}}{\hskip .75em}71
\item{\tt{get-process-id}}{\hskip .75em}67
\item{\tt{get-user-id}}{\hskip .75em}68
\item{\tt{group-id->group-info}}{\hskip .75em}69
\item{\tt{group-id->integer}}{\hskip .75em}68
\item{\tt{group-id=?}}{\hskip .75em}68
\item{\tt{group-id?}}{\hskip .75em}68
\item{\tt{group-info-id}}{\hskip .75em}69
\item{\tt{group-info-members}}{\hskip .75em}69
\item{\tt{group-info-name}}{\hskip .75em}69
\item{\tt{group-info?}}{\hskip .75em}69
\indexspace
\item{\tt{i/o-flags}}{\hskip .75em}75
\item{\tt{ignore-case}}{\hskip .75em}46
\item{\tt{integer->file-mode}}{\hskip .75em}73
\item{\tt{integer->group-id}}{\hskip .75em}68
\item{\tt{integer->process-id}}{\hskip .75em}64
\item{\tt{integer->signal}}{\hskip .75em}65
\item{\tt{integer->user-id}}{\hskip .75em}68
\item{\tt{intersection}}{\hskip .75em}45
\item{\tt{is-a-terminal?}}{\hskip .75em}75
\indexspace
\item{\tt{let-fluid}}{\hskip .75em}36
\item{\tt{let-fluids}}{\hskip .75em}36
\item{\tt{link}}{\hskip .75em}71
\item{\tt{list-directory}}{\hskip .75em}70
\item{\tt{lookup-all-externals}}{\hskip .75em}54
\item{\tt{lookup-environment-variable}}{\hskip .75em}68
\item{\tt{lookup-exported-binding}}{\hskip .75em}52
\item{\tt{lookup-external}}{\hskip .75em}54
\item{\tt{lookup-imported-binding}}{\hskip .75em}52
\indexspace
\item{\tt{machine-name}}{\hskip .75em}69
\item{\tt{make-array}}{\hskip .75em}31
\item{\tt{make-byte-vector}}{\hskip .75em}31
\item{\tt{make-cell}}{\hskip .75em}31
\item{\tt{make-directory}}{\hskip .75em}71
\item{\tt{make-fifo}}{\hskip .75em}71
\item{\tt{make-fluid}}{\hskip .75em}36
\item{\tt{make-integer-table}}{\hskip .75em}35
\item{\tt{make-regexp}}{\hskip .75em}76
\item{\tt{make-shared-array}}{\hskip .75em}32
\item{\tt{make-signal-queue}}{\hskip .75em}67
\item{\tt{make-string-table}}{\hskip .75em}35
\item{\tt{make-symbol-table}}{\hskip .75em}35
\item{\tt{make-table}}{\hskip .75em}35
\item{\tt{make-table-immutable!}}{\hskip .75em}35
\item{\tt{make-table-maker}}{\hskip .75em}35
\item{\tt{make-time}}{\hskip .75em}74
\item{\tt{match}}{\hskip .75em}47
\item{\tt{match-end}}{\hskip .75em}76
\item{\tt{match-end match}}{\hskip .75em}47
\item{\tt{match-start}}{\hskip .75em}76
\item{\tt{match-start match}}{\hskip .75em}47
\item{\tt{match-submatches match}}{\hskip .75em}47
\item{\tt{match?}}{\hskip .75em}76
\item{\tt{maybe-dequeue-signal!}}{\hskip .75em}67
\item{\tt{modify}}{\hskip .75em}19
\indexspace
\item{\tt{name->group-info}}{\hskip .75em}69
\item{\tt{name->signal}}{\hskip .75em}65
\item{\tt{name->user-info}}{\hskip .75em}68
\item{\tt{negate}}{\hskip .75em}45
\item{\tt{no-submatches}}{\hskip .75em}46
\indexspace
\item{\tt{one-of}}{\hskip .75em}45
\item{\tt{open-directory-stream}}{\hskip .75em}69
\item{\tt{open-file}}{\hskip .75em}70
\item{\tt{open-pipe}}{\hskip .75em}74
\item{\tt{open-socket}}{\hskip .75em}37
\item{\tt{os-name}}{\hskip .75em}69
\item{\tt{os-node-name}}{\hskip .75em}69
\item{\tt{os-release-name}}{\hskip .75em}69
\item{\tt{os-version-name}}{\hskip .75em}69
\indexspace
\item{\tt{port->fd}}{\hskip .75em}74
\item{\tt{prefix}}{\hskip .75em}19
\item{\tt{process-id->integer}}{\hskip .75em}64
\item{\tt{process-id-exit-status}}{\hskip .75em}64
\item{\tt{process-id-terminating-signal}}{\hskip .75em}64
\item{\tt{process-id=?}}{\hskip .75em}64
\item{\tt{process-id?}}{\hskip .75em}64
\indexspace
\item{\tt{range}}{\hskip .75em}44
\item{\tt{ranges}}{\hskip .75em}44
\item{\tt{read-directory-stream}}{\hskip .75em}69
\item{\tt{regexp-match}}{\hskip .75em}76
\item{\tt{regexp?}}{\hskip .75em}76
\item{\tt{remap-file-descriptors}}{\hskip .75em}74
\item{\tt{remove-directory}}{\hskip .75em}71
\item{\tt{remove-signal-queue-signal!}}{\hskip .75em}67
\item{\tt{rename}}{\hskip .75em}71
\item{\tt{repeat}}{\hskip .75em}46
\indexspace
\item{\tt{sequence}}{\hskip .75em}45
\item{\tt{set}}{\hskip .75em}44
\item{\tt{set-close-on-exec?!}}{\hskip .75em}75
\item{\tt{set-file-creation-mask!}}{\hskip .75em}71
\item{\tt{set-fluid!}}{\hskip .75em}36
\item{\tt{set-group-id!}}{\hskip .75em}68
\item{\tt{set-i/o-flags!}}{\hskip .75em}75
\item{\tt{set-user-id!}}{\hskip .75em}68
\item{\tt{set-working-directory!}}{\hskip .75em}70
\item{\tt{shared-binding-is-import?}}{\hskip .75em}52
\item{\tt{shared-binding-name}}{\hskip .75em}52
\item{\tt{shared-binding-ref}}{\hskip .75em}52
\item{\tt{shared-binding-set!}}{\hskip .75em}52
\item{\tt{shared-binding?}}{\hskip .75em}52
\item{\tt{signal-name}}{\hskip .75em}65
\item{\tt{signal-os-number}}{\hskip .75em}65
\item{\tt{signal-process}}{\hskip .75em}67
\item{\tt{signal-queue-monitored-signals}}{\hskip .75em}67
\item{\tt{signal-queue?}}{\hskip .75em}67
\item{\tt{signal=?}}{\hskip .75em}65
\item{\tt{signal?}}{\hskip .75em}65
\item{\tt{socket-accept}}{\hskip .75em}37
\item{\tt{socket-client}}{\hskip .75em}38
\item{\tt{socket-port-number}}{\hskip .75em}37
\item{\tt{string-end}}{\hskip .75em}45
\item{\tt{string-hash}}{\hskip .75em}36
\item{\tt{string-start}}{\hskip .75em}45
\item{\tt{submatch}}{\hskip .75em}46
\item{\tt{subset}}{\hskip .75em}19
\item{\tt{subtract}}{\hskip .75em}45
\indexspace
\item{\tt{table-ref}}{\hskip .75em}36
\item{\tt{table-set!}}{\hskip .75em}36
\item{\tt{table-walk}}{\hskip .75em}36
\item{\tt{table?}}{\hskip .75em}36
\item{\tt{terminal-name}}{\hskip .75em}75
\item{\tt{text}}{\hskip .75em}45
\item{\tt{time->string}}{\hskip .75em}74
\item{\tt{time-seconds}}{\hskip .75em}74
\item{\tt{time<=?}}{\hskip .75em}74
\item{\tt{time<?}}{\hskip .75em}74
\item{\tt{time=?}}{\hskip .75em}74
\item{\tt{time>=?}}{\hskip .75em}74
\item{\tt{time>?}}{\hskip .75em}74
\item{\tt{time?}}{\hskip .75em}74
\indexspace
\item{\tt{undefine-exported-binding}}{\hskip .75em}52
\item{\tt{undefine-imported-binding}}{\hskip .75em}52
\item{\tt{union}}{\hskip .75em}45
\item{\tt{unlink}}{\hskip .75em}71
\item{\tt{use-case}}{\hskip .75em}46
\item{\tt{user-id->integer}}{\hskip .75em}68
\item{\tt{user-id->user-info}}{\hskip .75em}68
\item{\tt{user-id=?}}{\hskip .75em}68
\item{\tt{user-id?}}{\hskip .75em}68
\item{\tt{user-info-group}}{\hskip .75em}69
\item{\tt{user-info-home-directory}}{\hskip .75em}69
\item{\tt{user-info-id}}{\hskip .75em}69
\item{\tt{user-info-name}}{\hskip .75em}68
\item{\tt{user-info-shell}}{\hskip .75em}69
\item{\tt{user-info?}}{\hskip .75em}68
\indexspace
\item{\tt{wait-for-child-process}}{\hskip .75em}64
\item{\tt{working-directory}}{\hskip .75em}70

\end{myindex}
}

\htmlonly{
\chapter*{Index}
\htmlprintindex
}

\end{document}    
